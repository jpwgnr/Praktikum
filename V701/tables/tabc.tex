\begin{table}\caption{Die Pulse wurden zur Analyse der Statistik des radioaktiven Zerfalls bestimmt.}
\label{tabc}
\centering
\sisetup{round-mode = places, round-precision=1, round-integer-to-decimal=true}
\begin{tabular}{c c c c c} 
\toprule
{Pulse} & {Pulse} & {Pulse} & {Pulse} & {Pulse}\\
\midrule
4361,0 & 4319,0 & 4551,0 & 4577,0 & 4804,0\\
4679,0 & 4444,0 & 4442,0 & 4447,0 & 4691,0\\
4669,0 & 4650,0 & 4338,0 & 4640,0 & 4468,0\\
4723,0 & 4701,0 & 4373,0 & 4478,0 & 4799,0\\
4790,0 & 4310,0 & 4402,0 & 4359,0 & 4863,0\\
4274,0 & 4769,0 & 4833,0 & 4722,0 & 4422,0\\
4438,0 & 4464,0 & 4537,0 & 4709,0 & 4419,0\\
4676,0 & 4506,0 & 4319,0 & 4624,0 & 4426,0\\
4484,0 & 4409,0 & 4606,0 & 4644,0 & 4463,0\\
4398,0 & 4509,0 & 4591,0 & 4624,0 & 4421,0\\
4402,0 & 4428,0 & 4709,0 & 4708,0 & 4277,0\\
4837,0 & 4494,0 & 4671,0 & 4506,0 & 4186,0\\
4693,0 & 4336,0 & 4617,0 & 4495,0 & 4261,0\\
4526,0 & 4449,0 & 4606,0 & 4609,0 & 4249,0\\
4811,0 & 4712,0 & 4659,0 & 4308,0 & 4537,0\\
4331,0 & 4749,0 & 4574,0 & 4761,0 & 4601,0\\
4436,0 & 4450,0 & 4772,0 & 4671,0 & 4397,0\\
4661,0 & 4667,0 & 4355,0 & 4589,0 & 4626,0\\
4506,0 & 4578,0 & 4636,0 & 4461,0 & 4297,0\\
\bottomrule
\end{tabular}\end{table}
