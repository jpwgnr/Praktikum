\section{Auswertung}
\label{sec:Auswertung}

Für die Auswertung wird Python und im Speziellen Matplotlib \cite{matplotlib}, SciPy \cite{scipy}, Uncertainties \cite{uncertainties} und NumPy \cite{numpy} verwendet.

\subsection{Bestimmung des Energieverlustes von Alphastrahlung in Luft}

\subsubsection{Erster Abstand}
Die gemessenen Pulse und Positionen der Energiemaxima bei den verschiedenen Drücken sind für den Abstand $d_1 = \SI{2.7}{\centi\meter}$ in Tab. \ref{taba} zu sehen. Die Pulse, Positionen und zugehörigen Reichweiten befinden sich in Tab. \ref{tab1}. 

\begin{table}\caption{Die Anzahl der Impulse, der Startwert auf der Mikrometerschraube und der Endwert auf der Mikrometerschraube.}
\label{taba}
\centering
\sisetup{round-mode = places, round-precision=2, round-integer-to-decimal=true}
\begin{tabular}{S[]S[]S[]} 
\toprule
{Anzahl} & {$d_\text{Start} / \si{\milli\meter}$} & {$d_\text{Start} / \si{\milli\meter}$}\\
\midrule
3001.0 & 6.73 & 2.0\\
3002.0 & 6.73 & 2.0\\
3000.0 & 1.82 & 6.5\\
3000.0 & 6.74 & 2.0\\
3000.0 & 1.83 & 6.5\\
3000.0 & 6.74 & 2.0\\
3001.0 & 1.84 & 6.5\\
3000.0 & 2.83 & 7.5\\
3001.0 & 7.77 & 3.0\\
3002.0 & 2.75 & 7.5\\
\bottomrule
\end{tabular}\end{table}
\begin{table}\caption{Erste Messung.}
\label{tab1}
\centering
\sisetup{round-mode = places, round-precision=1, round-integer-to-decimal=true}
\begin{tabular}{S[]S[]S[]} 
\toprule
{$g / \si{\centi\meter}$} & {$b / \si{\centi\meter}$} & {$B / \si{\centi\meter}$}\\
\midrule
12.700000000000003 & 38.9 & 8.4\\
13.700000000000003 & 32.2 & 6.5\\
14.700000000000003 & 28.200000000000003 & 5.3\\
15.700000000000003 & 25.299999999999997 & 4.4\\
16.700000000000003 & 22.5 & 3.7\\
17.700000000000003 & 21.299999999999997 & 3.3\\
18.700000000000003 & 19.799999999999997 & 3.0\\
19.700000000000003 & 18.5 & 2.6\\
20.700000000000003 & 17.799999999999997 & 2.4\\
21.700000000000003 & 17.400000000000006 & 2.2\\
\bottomrule
\end{tabular}\end{table}

%Zählrate als Funktion der effektiven Länge für den ersten Abstand
\noindent Die Zählrate ist in Abb. \ref{zaehlrate1} gegen die mit Gleichung \eqref{eqn:abstand} bestimmte effektive Länge aufgetragen.
\begin{figure}
    \centering
    \includegraphics[width=12cm, height=9cm]{build/plot.pdf}
    \caption{}
    \label{fig:zaehlrate1}
\end{figure}

\noindent Dadurch lässt sich die mittlere Reichweite der $\alpha$-Teilchen zu dem Wert %wie genau?
\begin{equation*}
    R_\text{m,1} = \SI{}{\milli\meter}
\end{equation*}
bestimmen.

\noindent Das entspricht einer Energie von %wie genau?
\begin{equation*}
    E_1 = \SI{}{}.
\end{equation*}

%Energie als Funktion der effektiven Länge für den ersten Abstand
Die Energie ist in Abb. \ref{fig:energie1} gegen die effektive Länge aufgetragen.
\begin{figure}
    \centering
    \includegraphics[width=12cm, height=9cm]{build/plot.pdf}
    \caption{}
    \label{fig:energie1}
\end{figure}

\noindent Daraus lässt sich anhand der Steigung der Energieverlust der Strahlung bestimmen. Dieser entspricht %wie genau?
\begin{equation*}
    - \left( \frac{dE}{dx} \right)_1 = \SI{}{}.
\end{equation*}


\subsubsection{Zweiter Abstand}
Die gemessenen Pulse und Positionen der Energiemaxima bei den verschiedenen Drücken sind für den Abstand $d_2 = \SI{2}{\centi\meter}$ in Tab. \ref{tabb} zu sehen. Die Pulse, Positionen und die zugehörigen Reichweiten befinden sich in Tab. \ref{tab2}.

\begin{table}\caption{Die Frequenzen der Sägezahnspannung.}
\label{tabb}
\centering
\sisetup{round-mode = places, round-precision=2, round-integer-to-decimal=true}
\begin{tabular}{S[]S[]} 
\toprule
{Index} & {$\nu_\text{Sä} / \si{\hertz}$}\\
\midrule
1.0 & 25.02\\
2.0 & 49.95\\
3.0 & 99.99\\
4.0 & 149.97\\
\bottomrule
\end{tabular}\end{table}
\begin{table}\caption{Die Spannung, die Stromstärke, die Anzahl der Impulse, die transportierte Ladungsmenge und die transporte Ladungsmenge in Einheiten der Elementarladung.}
\label{tab1}
\centering
\sisetup{round-mode = places, round-precision=2, round-integer-to-decimal=true}
\begin{tabular}{S[]S[] S[]@{${}\pm{}$}S[] S[]@{${}\pm{}$} S[] S[]@{${}\pm{}$} S[]} 
\toprule
{U / \si{\volt}} & {I / \si{\ampere}} & \multicolumn{2}{c}{N/second} &  \multicolumn{2}{c}{$\Delta Q / \si{\coulomb}$} &  \multicolumn{2}{c}{$\Delta Q \si{\elementarycharge}$}\\
\midrule
320.0 & 0.1     & 86.91 & 0.07 &  8.975  &  0.007  & 5.602   &  0.005e+19\\
400.0 & 0.2     & 90.92 & 0.07 & 17.157  &  0.014  & 1.0709  &  0.0009e+20\\
480.0 & 0.3     & 93.35 & 0.07 & 25.068  &  0.020  & 1.5646  &  0.0012e+20\\
540.0 & 0.35    & 94.62 & 0.07 & 28.851  &  0.023  & 1.8008  &  0.0014e+20\\
560.0 & 0.4     & 92.83 & 0.07 & 33.610  &  0.027  & 2.0977  &  0.0017e+20\\
600.0 & 0.45    & 95.03 & 0.07 & 36.935  &  0.029  & 2.3053  &  0.0018e+20\\
640.0 & 0.5     & 95.41 & 0.08 & 40.877  &  0.032  & 2.5514  &  0.0020e+20\\
660.0 & 0.55    & 96.21 & 0.08 & 44.591  &  0.035  & 2.7832  &  0.0022e+20\\
680.0 & 0.6     & 97.38 & 0.08 & 48.06   &  0.04   & 2.9997  &  0.0023e+20\\
\bottomrule
\end{tabular}\end{table}

%Zählrate als Funktion der effektiven Länge für den zweiten Abstand
\noindent Die Zählrate ist in Abb. \ref{zaehlrate2} gegen die mit Gleichung \eqref{eqn:abstand} bestimmte effektive Länge aufgetragen.
\begin{figure}
    \centering
    \includegraphics[width=12cm, height=9cm]{build/plot.pdf}
    \caption{}
    \label{fig:zaehlrate2}
\end{figure}

\noindent Dadurch lässt sich die mittlere Reichweite der $\alpha$-Teilchen zu dem Wert %wie genau?
\begin{equation*}
    R_\text{m,2} = \SI{}{\milli\meter}
\end{equation*}
bestimmen.

\noindent Das entspricht einer Energie von %wie genau?
\begin{equation*}
    E_2 = \SI{}{}.
\end{equation*}

%Energie als Funktion der effektiven Länge für den zweiten Abstand
Die Energie ist in Abb. \ref{fig:energie2} gegen die effektive Länge aufgetragen.
\begin{figure}
    \centering
    \includegraphics[width=12cm, height=9cm]{build/plot.pdf}
    \caption{}
    \label{fig:energie2}
\end{figure}

\noindent Daraus lässt sich anhand der Steigung der Energieverlust der Strahlung bestimmen. Dieser entspricht %wie genau?
\begin{equation*}
    - \left( \frac{dE}{dx} \right)_2 = \SI{}{}.
\end{equation*}

\subsection{Untersuchung der Statistik des radioaktiven Zerfalls}

Die Anzahl der Pulse, die jeweils in $\SI{10}{\second}$ gemessen wurde, sind in Tab. \ref{tabc} eingetragen.

\begin{table}\caption{Der magnetische Fluss $B$ des gemessenen Magnetfelds gegen den Strom $I$ des erzeugenden Magnetfelds, Neukurve.}
\label{tabc}
\centering
\sisetup{round-mode = places, round-precision=1, round-integer-to-decimal=true}
\begin{tabular}{S[]S[]} 
\toprule
{$B$/ \si{\milli\tesla}} & {$I$/ \si{\ampere}}\\
\midrule
0.0 & 0.0\\
111.19999999999999 & 1.0\\
273.5 & 2.0\\
397.8 & 3.0\\
479.9 & 4.0\\
537.9000000000001 & 5.0\\
585.0999999999999 & 6.0\\
621.8000000000001 & 7.0\\
653.1 & 8.0\\
679.9 & 9.0\\
704.3000000000001 & 10.0\\
\bottomrule
\end{tabular}\end{table}

%Zerfallsraten in Histogramm
Die Zerfallsraten sind in Abb. \ref{fig:histogramm} in einem Histogramm aufgetragen. Außerdem ist eine Gauß- und eine Poissonverteilung eingetragen.
\begin{figure}
    \centering
    \includegraphics[width=12cm, height=9cm]{build/plot.pdf}
    \caption{}
    \label{fig:histogramm}
\end{figure}

%Mittelwert und Varianz
\noindent Aus den gemessenen Zählraten lassen sich der  Mittelwert und die Varianz bestimmen: %Gleichungen erwähnen, Varianz ergänzen?
\begin{align*}
    \bar{x} &= \num{} \\
    var &= \num{}.
\end{align*}