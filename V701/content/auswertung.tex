\section{Auswertung}
\label{sec:Auswertung}

Für die Auswertung wird Python und im Speziellen Matplotlib \cite{matplotlib}, SciPy \cite{scipy}, Uncertainties \cite{uncertainties} und NumPy \cite{numpy} verwendet.

\subsection{Bestimmung des Energieverlustes von Alphastrahlung in Luft}

\subsubsection{Erster Abstand}
Die gemessenen Pulse und Positionen der Energiemaxima bei den verschiedenen Drücken sind für den Abstand $d_1 = \SI{2.7}{\centi\meter}$ in Tab. \ref{taba} zu sehen. Die Drücke, die Anzahl der Pulse und die Position des jeweiligen Maximums befinden sich in Tab. \ref{taba}. 

\begin{table}\caption{Die Anzahl der Impulse, der Startwert auf der Mikrometerschraube und der Endwert auf der Mikrometerschraube.}
\label{taba}
\centering
\sisetup{round-mode = places, round-precision=2, round-integer-to-decimal=true}
\begin{tabular}{S[]S[]S[]} 
\toprule
{Anzahl} & {$d_\text{Start} / \si{\milli\meter}$} & {$d_\text{Start} / \si{\milli\meter}$}\\
\midrule
3001.0 & 6.73 & 2.0\\
3002.0 & 6.73 & 2.0\\
3000.0 & 1.82 & 6.5\\
3000.0 & 6.74 & 2.0\\
3000.0 & 1.83 & 6.5\\
3000.0 & 6.74 & 2.0\\
3001.0 & 1.84 & 6.5\\
3000.0 & 2.83 & 7.5\\
3001.0 & 7.77 & 3.0\\
3002.0 & 2.75 & 7.5\\
\bottomrule
\end{tabular}\end{table}

%Zählrate als Funktion der effektiven Länge für den ersten Abstand
Die mit Gleichung \eqref{eqn:x} ermittelten Abstände, die Anzahl der Pulse (Zählrate) und die mit Gleichung \eqref{eqn:energie} ermittelten Energien befinden sich in Tab. \ref{tab1}. 

\noindent Die Zählrate ist in Abb. \ref{zaehlrate1} gegen die mit Gleichung \eqref{eqn:abstand} bestimmte effektive Länge aufgetragen.
\begin{figure}
    \centering
    \includegraphics[width=12cm, height=9cm]{build/plota.pdf}
    \caption{}
    \label{fig:zaehlrate1}
\end{figure}

\noindent Die Fitparameter der linearen Regression ergeben sich dadurch zu 
\begin{align*}
    m &= \num{-11300952.96(324920)} \frac{N}{\SI{120}{\second} \si{\meter}} \\
    n &= \num{289663.5} \frac{N}{\SI{120}{\second}} .
\end{align*}


\noindent Mit dem Umformen dieser linearen Gleichung ergibt sich bei $y = \frac{1}{2} max$ die mittlere Reichweite der $\alpha$-Teilchen zu dem Wert %wie genau?
\begin{equation*}
    R_\text{m,1} = \SI{22.7}{\milli\meter}
\end{equation*}
bestimmen.

\noindent Das entspricht  nach Ablesen in Tab. \ref{tab1} einer Energie von %wie genau?
\begin{equation*}
    E_1 = \SI{12.7}{\mega\electronvolt}.
\end{equation*}

%Energie als Funktion der effektiven Länge für den ersten Abstand
Die Energie ist in Abb. \ref{fig:energie1} gegen die effektive Länge aufgetragen.
\begin{figure}
    \centering
    \includegraphics[width=12cm, height=9cm]{build/plotb.pdf}
    \caption{}
    \label{fig:energie1}
\end{figure}

\noindent Die Fitparameter der linearen Regression ergeben sich dadurch zu 
\begin{align*}
    m &= \SI{-114512929.24(426762263)}{\mega\electronvolt\per\meter} \\
    n &= \SI{4132682.60}{\electronvolt} .
\end{align*}

\noindent Daraus lässt sich anhand der Steigung der Energieverlust der Strahlung bestimmen %wie genau?
\begin{equation*}
    - \left( \frac{dE}{dx} \right)_1 = - \SI{11.45}{\mega\electronvolt}.
\end{equation*}


\subsubsection{Zweiter Abstand}
Die gemessenen Pulse und Positionen der Energiemaxima bei den verschiedenen Drücken sind für den Abstand $d_2 = \SI{2}{\centi\meter}$ in Tab. \ref{tabb} zu sehen. Die Drücke, die Anzahl der Pulse und die Position des jeweiligen Maximums befinden sich in Tab. \ref{tabb}. 

\begin{table}\caption{Die Frequenzen der Sägezahnspannung.}
\label{tabb}
\centering
\sisetup{round-mode = places, round-precision=2, round-integer-to-decimal=true}
\begin{tabular}{S[]S[]} 
\toprule
{Index} & {$\nu_\text{Sä} / \si{\hertz}$}\\
\midrule
1.0 & 25.02\\
2.0 & 49.95\\
3.0 & 99.99\\
4.0 & 149.97\\
\bottomrule
\end{tabular}\end{table}

%Zählrate als Funktion der effektiven Länge für den ersten Abstand
Die mit Gleichung \eqref{eqn:x} ermittelten Abstände, die Anzahl der Pulse (Zählrate) und die mit Gleichung \eqref{eqn:energie} ermittelten Energien befinden sich in Tab. \ref{tab2}. 

\noindent Die Zählrate ist in Abb. \ref{fig:zaehlrate2} gegen die mit Gleichung \eqref{eqn:abstand} bestimmte effektive Länge aufgetragen.
\begin{figure}
    \centering
    \includegraphics[width=12cm, height=9cm]{build/plotc.pdf}
    \caption{}
    \label{fig:zaehlrate2}
\end{figure}

\noindent Die Fitparameter der linearen Regression ergeben sich dadurch zu 
\begin{align*}
    m &= \num{-11300952.96(324920)} \frac{N}{\SI{120}{\second} \si{\meter}} \\
    n &= \num{289663.5} \frac{N}{\SI{120}{\second}} .
\end{align*}


\noindent Mit dem Umformen dieser linearen Gleichung ergibt sich bei $y = \frac{1}{2} max$ die mittlere Reichweite der $\alpha$-Teilchen zu dem Wert %wie genau?
\begin{equation*}
    R_\text{m,2} = \SI{<++>}{\milli\meter}
\end{equation*}
bestimmen.

\noindent Das entspricht  nach Ablesen in Tab. \ref{tab2} einer Energie von %wie genau?
\begin{equation*}
    E_2 = \SI{<++>}{\mega\electronvolt}.
\end{equation*}

%Energie als Funktion der effektiven Länge für den ersten Abstand
Die Energie ist in Abb. \ref{fig:energie2} gegen die effektive Länge aufgetragen.
\begin{figure}
    \centering
    \includegraphics[width=12cm, height=9cm]{build/plotd.pdf}
    \caption{}
    \label{fig:energie2}
\end{figure}

\noindent Die Fitparameter der linearen Regression ergeben sich dadurch zu 
\begin{align*}
    m &= \SI{<++>}{\<++>} \\
    n &= \SI{<++>}{\<++>} .
\end{align*}

\noindent Daraus lässt sich anhand der Steigung der Energieverlust der Strahlung bestimmen %wie genau?
\begin{equation*}
    - \left( \frac{dE}{dx} \right)_1 = \SI{}{}.
\end{equation*}


\subsection{Untersuchung der Statistik des radioaktiven Zerfalls}

Die Anzahl der Pulse, die jeweils in $\SI{10}{\second}$ gemessen wurde, sind in Tab. \ref{tabc} eingetragen.

\begin{table}\caption{Der magnetische Fluss $B$ des gemessenen Magnetfelds gegen den Strom $I$ des erzeugenden Magnetfelds, Neukurve.}
\label{tabc}
\centering
\sisetup{round-mode = places, round-precision=1, round-integer-to-decimal=true}
\begin{tabular}{S[]S[]} 
\toprule
{$B$/ \si{\milli\tesla}} & {$I$/ \si{\ampere}}\\
\midrule
0.0 & 0.0\\
111.19999999999999 & 1.0\\
273.5 & 2.0\\
397.8 & 3.0\\
479.9 & 4.0\\
537.9000000000001 & 5.0\\
585.0999999999999 & 6.0\\
621.8000000000001 & 7.0\\
653.1 & 8.0\\
679.9 & 9.0\\
704.3000000000001 & 10.0\\
\bottomrule
\end{tabular}\end{table}

%Zerfallsraten in Histogramm
Die Zerfallsraten sind in Abb. \ref{fig:histogramm} in einem Histogramm aufgetragen. Außerdem ist eine Gauß- und eine Poissonverteilung eingetragen. Bei der Erzeugung der beiden Verteilungen wurde ein Seed von 42 benutzt, um die Auswertung deterministisch zu machen. 
\begin{figure}
    \centering
    \includegraphics[width=12cm, height=9cm]{build/plot.pdf}
    \caption{}
    \label{fig:histogramm}
\end{figure}

%Mittelwert und Varianz
\noindent Aus den gemessenen Zählraten lassen sich der  Mittelwert und die Varianz bestimmen: %Gleichungen erwähnen, Varianz ergänzen?
\begin{align*}
    \bar{x} &= \num{4541.6} \\
    var &= \num{25065}.
\end{align*}
