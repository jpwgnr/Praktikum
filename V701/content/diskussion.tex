\section{Diskussion}
\label{sec:Diskussion}

\subsection{Erster Abstand}

Bei der ersten Messung mit dem Abstand \SI{2.7}{\centi\meter} ergab sich für die Steigung der linearen Ausgleichsrechnung ein relativer Fehler von \SI{0.03}{\percent}. Die mittlere Reichweite hat damit einen relativen Fehler von \SI{0.04}{\percent } und die somit ermittelte Energie hat dann einen relativen Fehler von \SI{0.03}{\percent}. 

\noindent Für die Steigung der linearen Regression bei dem Plot mit der Energie ergibt sich ein relativer Fehler von \SI{3.67}{\percent}. Somit entspricht dies auch dem relativen Fehler des Energieverlusts der Strahlung. 

\subsection{Zweiter Abstand}

Bei der zweiten Messung mit dem Abstand \SI{2.0}{\centi\meter} ergab sich für die Steigung der linearen Ausgleichsrechnung ein relativer Fehler von \SI{3.67}{\percent}. Die mittlere Reichweite hat damit einen relativen Fehler von \SI{3.65}{\percent } und die somit ermittelte Energie hat also einen relativen Fehler von \SI{2.48}{\percent}. 

\noindent Für die Steigung der linearen Regression bei dem Plot mit der Energie ergibt sich ein relativer Fehler von \SI{2.33}{\percent}. Somit entspricht dies auch dem relativen Fehler des Energieverlusts der Strahlung. 

\subsection{Statistik des radioaktiven Zerfalls}

Der Fehler des Mittelwerts liegt bei \SI{3.49}{\percent}. 
Die zufällig erzeugten Gauß- und Poisson-Verteilungen sehen der gemessenen Verteilung nicht wirklich ähnlich. Sie sind alle auf eine Höhe normiert. Was aber auffällt, ist, dass in der Mitte der gemessenen Werte eine Lücke vorhanden ist, die genau von der Poisson-Verteilung ausgefüllt wird. Insofern passt die Poisson-Verteilung nicht gut auf das gemessene Ergebnis. Die Gauß-Verteilung passt schon besser, hat aber auch nicht die charakteristische Lücke, die unsere Verteilung in der Nähe des Erwartungswertes aufweist. 
