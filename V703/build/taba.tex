\begin{table}\caption{Die angelegte Spannung des elektrischen Feldes innerhalb des Geiger-Müller-Zählrohrs  und die Anzahl der jeweils gemessenen Impulse.}
\label{taba}
\centering
\sisetup{round-mode = places, round-precision=1, round-integer-to-decimal=true}
\begin{tabular}{S[]S[]} 
\toprule
{U / \si{\volt}} & {N}\\
\midrule
320.0 & 11298.0\\
340.0 & 11674.0\\
360.0 & 11921.0\\
380.0 & 11839.0\\
400.0 & 11820.0\\
420.0 & 12087.0\\
440.0 & 11940.0\\
460.0 & 12259.0\\
480.0 & 12135.0\\
500.0 & 12013.0\\
520.0 & 12099.0\\
540.0 & 12301.0\\
560.0 & 12068.0\\
580.0 & 12097.0\\
600.0 & 12354.0\\
620.0 & 12289.0\\
640.0 & 12403.0\\
660.0 & 12507.0\\
680.0 & 12659.0\\
\bottomrule
\end{tabular}\end{table}