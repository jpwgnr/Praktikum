\begin{table}\caption{Die angelegte Spannung des elektrischen Feldes innerhalb des Geiger-Müller-Zählrohrs, die Anzahl der jeweils gemessenen Impulse und der Strom innerhalb des Geiger-Müller-Zählrohrs.}
\label{tabb}
\centering
\sisetup{round-mode = places, round-precision=2, round-integer-to-decimal=true}
\begin{tabular}{S[]S[]S[]} 
\toprule
{U / \si{\volt}} & {N} & {I / \si{\ampere}}\\
\midrule
320.0 & 11298.0 & 0.1\\
400.0 & 11820.0 & 0.2\\
480.0 & 12135.0 & 0.3\\
540.0 & 12301.0 & 0.35\\
560.0 & 12068.0 & 0.4\\
600.0 & 12354.0 & 0.45\\
640.0 & 12403.0 & 0.5\\
660.0 & 12507.0 & 0.55\\
680.0 & 12659.0 & 0.6\\
\bottomrule
\end{tabular}\end{table}