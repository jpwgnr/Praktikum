\section{Diskussion}
\label{sec:Diskussion}

\subsection{Zählrohr-Charakteristik}

Die Steigung liegt mit ungefähr einem Prozent pro \SI{100}{\volt} in einem Bereich, der ungefähr dem entspricht, was zu erwarten war. Im Vergleich liegt der Wert bei einem handelsüblichen Geiger-Müller-Zählrohr \cite{vergleich} bei \SI{0.04}{\percent\per\volt}. Somit liegt die relative Abweichung bei \SI{75}{\percent}. Daher liegt der berechnete Wert unterhalb des Literaturwertes, was auf eine gute Qualität des Zählrohrs hindeutet. 

\subsection{Totzeitbestimmung}

Die Totzeit durch die Messung mit dem Oszilloskop ergibt einen Wert, der bei einer relativen Abweichung von \SI{54.74}{\percent} zu der Messung durch die Zwei-Quellen-Methode liegt. 
Zumindest die Größenordnung scheint ungefähr gleich zu sein. 

\noindent Der relative Fehler der ersten Totzeitmessung liegt bei \SI{18.52}{\percent}. Die Messung mit der Zwei-Quellen-Methode ergibt einen relativen Fehler von \SI{2.60}{\percent}.

%Vergleich mit T_lit
\noindent Die angegebene Totzeit bei einem handelsüblichen Geiger-Müller-Zählrohr \cite{vergleich} beträgt
\begin{equation*}
    T_\text{lit} = \SI{90}{\micro\second}.
\end{equation*}
Die ermittelten Totzeiten weichen davon für $T_{Tot,1}$ um \SI{40}{\percent} bzw. für $T_{Tot,2}$ um \SI{32.56}{\percent} ab.

\subsection{Transportierte Ladungsmenge}

Die transportierten Ladungsmengen haben einen relativen Fehler, der sehr klein ist, also unter \SI{0.1}{\percent}. Die Messpunkte liegen wie erwartet ungefähr auf einer Geraden. 
