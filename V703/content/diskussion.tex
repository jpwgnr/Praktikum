\section{Diskussion}
\label{sec:Diskussion}

\subsection{Zählrohr-Charakteristik}

Die Steigung liegt mit ungefähr einem Prozent pro \SI{100}{\volt} in einem Bereich, der dem entspricht, was zu erwarten war. %Quelle? Andere Vergleichswerte 

\subsection{Totzeitbestimmung}

Die Totzeit durch die Messung mit dem Oszilloskop ergibt einen Wert, der bei einer relativen Abweichung von \SI{54.74}{\percent} von der Messung durch die Zwei-Quellen-Methode liegt. 
Zumindest die Größenordnung scheint ungefähr gleich zu sein. 

\noindent Der relative Fehler der ersten Totzeitmessung liegt bei \SI{18.52}{\percent}. Die Messung mit der Zwei-Quellen-Methode ergibt einen relativen Fehler von \SI{2.60}{\percent}.

%Vergleich mit T_lit
\noindent Die angegebene Totzeit bei einem handelsüblichen Geiger-Müller-Zählrohr \cite{vergleich} beträgt
\begin{equation*}
    T_\text{lit} = \SI{90}{\micro\second}.
\end{equation*}
Die ermittelten Totzeiten weichen davon um \SI{40.0}{\percent} bzw. \SI{32.5}{\percent} ab.

\subsection{Transportierte Ladungsmenge}

Die transportierten Ladungsmengen haben einen relativen Fehler in einem Bereich zwischen \SI{0}{\percent} und \SI{0}{\percent}. %