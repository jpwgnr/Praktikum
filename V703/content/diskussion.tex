\section{Diskussion}
\label{sec:Diskussion}

\subsection{Zählrohr-Charakteristik}

Die Steigung liegt mit ungefähr einem Prozent pro \SI{100}{\volt} in einem Bereich, der dem entspricht, was zu erwarten war. %Quelle? Andere Vergleichswerte 

\subsection{Totzeitbestimmung}

Die Totzeit durch die Messung mit dem Oszilloskop ergibt einen Wert, der bei einer relativen Abweichung von \SI{54.74}{\percent} von der Messung durch die zwei Quellen Methode liegt. 
Zumindest die Größenordnung scheint dann ungefähr gleich zu sein. 

Der relative Fehler der ersten Totzeit Messung liegt bei \SI{18.52}{\percent}. Die Messung mit der zwei Quellenmethode ergibt einen relativen Fehler von \SI{2.60}{\percent}. 

\subsection{Transportierte Ladungsmenge}

Die transportierten Ladungsmengen haben einen relativen Fehler in einem Bereich zwischen \SI{0}{\percent} und \SI{0}{\percent}.
