\section{Auswertung}
\label{sec:Auswertung}


\subsection{Zählrohr-Charakteristik}

Im Folgenden wird die Zählrohr-Charakteristik eines Halogenzählrohrs bestimmt. Die Werte für die angelegte Spannung $U$ und die Anzahl der gemessenen Impulse sind in Tab. \ref{taba} dargestellt. 

\begin{table}\caption{Die Anzahl der Impulse, der Startwert auf der Mikrometerschraube und der Endwert auf der Mikrometerschraube.}
\label{taba}
\centering
\sisetup{round-mode = places, round-precision=2, round-integer-to-decimal=true}
\begin{tabular}{S[]S[]S[]} 
\toprule
{Anzahl} & {$d_\text{Start} / \si{\milli\meter}$} & {$d_\text{Start} / \si{\milli\meter}$}\\
\midrule
3001.0 & 6.73 & 2.0\\
3002.0 & 6.73 & 2.0\\
3000.0 & 1.82 & 6.5\\
3000.0 & 6.74 & 2.0\\
3000.0 & 1.83 & 6.5\\
3000.0 & 6.74 & 2.0\\
3001.0 & 1.84 & 6.5\\
3000.0 & 2.83 & 7.5\\
3001.0 & 7.77 & 3.0\\
3002.0 & 2.75 & 7.5\\
\bottomrule
\end{tabular}\end{table}

\noindnet Die Anzahl der Impulse wird in einer Zeit von 
\begin{equation*}
    \Delta t = \SI{130}{\second}
\end{equation*}
gemessen. 
Die normierten und fehlerbehafteten Werte für die Anzahl der Impulse befinden sich in Tab. \ref{tab1}. 

\begin{table}\caption{Erste Messung.}
\label{tab1}
\centering
\sisetup{round-mode = places, round-precision=1, round-integer-to-decimal=true}
\begin{tabular}{S[]S[]S[]} 
\toprule
{$g / \si{\centi\meter}$} & {$b / \si{\centi\meter}$} & {$B / \si{\centi\meter}$}\\
\midrule
12.700000000000003 & 38.9 & 8.4\\
13.700000000000003 & 32.2 & 6.5\\
14.700000000000003 & 28.200000000000003 & 5.3\\
15.700000000000003 & 25.299999999999997 & 4.4\\
16.700000000000003 & 22.5 & 3.7\\
17.700000000000003 & 21.299999999999997 & 3.3\\
18.700000000000003 & 19.799999999999997 & 3.0\\
19.700000000000003 & 18.5 & 2.6\\
20.700000000000003 & 17.799999999999997 & 2.4\\
21.700000000000003 & 17.400000000000006 & 2.2\\
\bottomrule
\end{tabular}\end{table}

\noindent In Abb. \ref{fig1} sind diese Werte gegeneinander aufgetragen und es wird eine Ausgleichsgerade im Bereich von $\num{360}$ bis $\num{620}$ mit der allgemeinen Gleichung \eqref{linReg} für die lineare Regressionen bestimmt. 

\begin{figure}

\end{figure}

Die lineare Regression ergibt als Parameter

\begin{align*} 
   m &= 0.012 \\
   n &= 86.88.
\end{align*}

Mit diesen Parametern lässt sich dann aus den Werten für die Anzahl bei \num{450} und \num{550} die Steigung des Plateaus mit Gleichung \eqref{steigung}.
Die Steigung des Plateaus beträgt 

\begin{align*} 
    m_{Plateau} = 1.33 \frac{\si{\percent}}{\SI{100}{\volt}}.
\end{align*}

\subsection{Totzeitbestimmung}

Bei einer Spannung von $\SI{500}{\volt}$ ergibt sich durch das Ablesen im Oszilloskop eine Totzeit von 

\begin{equation*}
    t_\text{Tot,1} = \SI{54(10)e-6}{\second}.
\end{equation*}

% Einfügen des Fotos

Die Totzeit lässt sich auch durch die 2-Quellen-Methode berechnen. 
Bei der Messung haben sich für die erste und zweite Quelle, sowie für die Kombination aus beiden folgende Werte ergeben
\begin{align*} 
   N_1 &= 9730 \frac{N}{\SI{60}{\second}}\\
   N_2 &= 11918 \frac{N}{\SI{60}{\second}} \\
   N_{1,2} &= \frac{N}{\SI{60}{\second}} 21187.
\end{align*}

Diese Werte werden durch die Messdauer von $\SI{60}{\second}$ geteilt und ein Fehler entsteht ebenfalls dadurch. 

\begin{align*} 
   N_1 &= \num{162.17(10)} \frac{N}{\SI{60}{\second}}\\
   N_2 &= \num{198.63(11)} \frac{N}{\SI{60}{\second}} \\
   N_{1,2} &=\num{353.12(14)} \frac{N}{\SI{60}{\second}}.
\end{align*}

Daraus ergibt sich dann für die Totzeit mit Gleichung \eqref{totzeit}
ein Wert von 
\begin{equation*} 
    t_\text{Tot,2} = \SI{119.3(31)}{\second}.
\end{equation*} 

\subsection{Transportierte Ladungsmenge}
In Tab. \ref{tabb} stehen die Spannung, die Anzahl der gemessenen Impulse und die Stromstärke bei verschiedenen angelegten Spannungen.
In Abb. \ref{fig2} sind die Spannung und die Stromstärke gegeneinander aufgetragen. 

# plot

Aus den Werten für die Stromspannung und der Anzahl der Impulse ergibt sich mit Gleichung \eqref{eqn:ladung} die transportierte Ladungsmenge $\Delta Q$ und daraus wiederum die transportierte Ladungsmenge in Abhängigkeit von der Elementarladung $e$. Die Ergebnisse sind in Tab. \ref{tab2} aufgetragen. 

\begin{table}\caption{Die Spannung, die Stromstärke, die Anzahl der Impulse, die transportierte Ladungsmenge und die transporte Ladungsmenge in Einheiten der Elementarladung.}
\label{tab1}
\centering
\sisetup{round-mode = places, round-precision=2, round-integer-to-decimal=true}
\begin{tabular}{S[]S[] S[]@{${}\pm{}$}S[] S[]@{${}\pm{}$} S[] S[]@{${}\pm{}$} S[]} 
\toprule
{U / \si{\volt}} & {I / \si{\ampere}} & \multicolumn{2}{c}{N/second} &  \multicolumn{2}{c}{$\Delta Q / \si{\coulomb}$} &  \multicolumn{2}{c}{$\Delta Q \si{\elementarycharge}$}\\
\midrule
320.0 & 0.1     & 86.91 & 0.07 &  8.975  &  0.007  & 5.602   &  0.005e+19\\
400.0 & 0.2     & 90.92 & 0.07 & 17.157  &  0.014  & 1.0709  &  0.0009e+20\\
480.0 & 0.3     & 93.35 & 0.07 & 25.068  &  0.020  & 1.5646  &  0.0012e+20\\
540.0 & 0.35    & 94.62 & 0.07 & 28.851  &  0.023  & 1.8008  &  0.0014e+20\\
560.0 & 0.4     & 92.83 & 0.07 & 33.610  &  0.027  & 2.0977  &  0.0017e+20\\
600.0 & 0.45    & 95.03 & 0.07 & 36.935  &  0.029  & 2.3053  &  0.0018e+20\\
640.0 & 0.5     & 95.41 & 0.08 & 40.877  &  0.032  & 2.5514  &  0.0020e+20\\
660.0 & 0.55    & 96.21 & 0.08 & 44.591  &  0.035  & 2.7832  &  0.0022e+20\\
680.0 & 0.6     & 97.38 & 0.08 & 48.06   &  0.04   & 2.9997  &  0.0023e+20\\
\bottomrule
\end{tabular}\end{table}


