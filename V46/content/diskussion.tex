\section{Diskussion}
\label{sec:Diskussion}

Anfangs wurde die Magnetfeldstärke gemessen. Dabei wurde ein Maximalwert von fast $\SI{0.4}{\tesla}$ gemessen. Dieser Wert fiel aber stark ab auf kurzer Entfernung von dem Maximum. Falls die Probe nicht ganz an diesem Maximum gesessen haben sollte, könnte dies eine potentielle Fehlerquelle sein. 

In Abb. \ref{fig:alle} ist zu erkennen, dass die skalierten Winkel vor allem für die beiden n-dotierten Proben sehr stark streuen für größere Wellenlängen. 
Die effektive Masse für GaAs beträgt nach dem Literaturwert $m_{\Gamma} = \num{0.063} \, m_\text{e}$. 
Die effektive Masse von Probe 1 weicht um \SI{20.6}{\percent} von dem Literaturwert ab, Probe 2 liegt genau auf dem Wert. 
Insofern scheinen die Messungen trotz der großen Streuung zumindest in der passenden Größenordnung zu liegen. 

Gründe für die Abweichung könnten neben der bereits erwähnten Justierung des Magnetfelds auch die Justierung der Gesamtapparatur sein.


