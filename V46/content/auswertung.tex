\section{Auswertung}
\label{sec:Auswertung}

\subsection{Kraftflussdichte}
B(z) gegen z auftragen, maximale Kraftflussdichte ermitteln (am Ort der Probe)

\subsection{Faraday-Rotation}
Messwerte: für neun verschiedene Wellenlängen jeweils die beiden Winkel pro Probe (hochrein und n-dotiert)

Damit Proben vergleichbar sind: Drehwinkel der Faraday-Rotation auf Probenlänge normieren (sind unterschiedlich lang).
\begin{equation}
    \theta = \frac{\theta_1 - \theta_2}{2L}
\end{equation}

\subsection{Effektive Masse}
Der Drehwinkel wird gegen $\lambda^2$ aufgetragen. Der Proportionalitätsfaktor zwischen den Größen wird bestimmt. Daraus wird die effektive Masse mit Formel \ref{eq:theta} berechnet.