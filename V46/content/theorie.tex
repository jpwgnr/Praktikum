\section{Theorie}
\label{sec:Theorie}
In diesem Versuch soll die effektive Masse der Leitungselektronen von n-dotiertem Galliumarsenid mithilfe des Faraday-Effekts bestimmt werden. 
\subsection{Halbleiter}
Das Bändermodell beschreibt die Energiezustände in einem Kristall. Da viele Zustände dicht beieinander liegen, werden sie als Energieband angesehen. Das Valenzband ist das letzte vollbesetzte Band. Das Leitungsband ist durch eine Energie- bzw. Bandlücke vom Valenzband getrennt. Bei Halbleitern ist diese Lücke geringer als bei Isolatoren. 

Ein Elektron, das ins Leitungsband gelangt, lässt im Valenzband ein Loch zurück. Diese fehlende negative Ladung wird als positive Ladung gesehen. \cite{demtroeder}

Dotierte Halbleiter sind reine Halbleiter, in die ein Fremdatom eingesetzt wird, z.B. indem ein Kristall in Dampf von Fremdatomen gebracht wird. Die Fremdatome wirken in dem Halbleiter als Störstellen.
Es gibt zwei Arten von Dotierung. Bei der n-Dotierung, welche in diesem Versuch vorliegt, ist die Wertigkeit der Fremdatome höher als die der Kristallatome, wodurch ein Hüllenelektron über viele Gitterplätze delokalisiert ist und deshalb als frei angesehen werden kann. Die Fremdatome werden daher Donatoren genannt. Bei der p-Dotierung bleibt ein freier positiv geladener Platz zwischen den Atomen, in den Elektronen eingefangen werden können. Die Fremdatome heißen Akzeptoren. \cite{demtroeder}

\subsection{Effektive Masse}
Die effektive Masse wird eingeführt, damit Elektronen wie freie Ladungsträger behandelt werden können. Im Kristall erfahren Elektronen ein Potential. In einem äußeren elektrischen Feld ändert sich deswegen nicht nur die kinetische Energie des Elektrons, sondern auch die potentielle Energie. Bei der effektiven Masse ist das Potential berücksichtigt. \cite{demtroeder}

Die effektive Masse von Elektronen im Leitungsband bzw. von Löchern im Valenzband ist gegeben durch
\begin{equation*}
    m_\text{eff} = \hbar^2 \cdot \left( \frac{d^2 E}{dk_i \, dk_j} \right)^{-1}.
\end{equation*}
Die effektive Masse hat Tensorcharakter und gibt die inverse Krümmung der Dispersionsrelation $E(k)$ an. \cite{demtroeder}

\subsection{Faraday-Effekt}
Die Polarisationsebene von linear polarisiertem Licht in einem Medium wird gedreht, wenn das Licht parallel zu einem Magnetfeld durch das Material läuft. \cite{heintze}
Eine schematische Darstellung des Faraday-Effekts ist in Abb. \ref{fig:FaradayEffekt} zu sehen.
%Optisch inaktive Materie dreht beim Anlegen eines äußeren Magnetfeldes die Polarisationsebene des Lichts, das parallel zur Magnetfeldrichtung einfällt. \cite{V46}

Damit der Kristall die Polarisationsebene des linear polarisierten Lichts bei der Transmission drehen kann (zirkulare Doppelbrechung), muss die Phasengeschwindigkeit für links- und rechtszirkular polarisiertes Licht in dem Medium verschieden sein. Das bedeutet, dass die Brechungsindizes verschieden sind. \cite{V46}

Die Faraday-Rotation pro Einheitslänge wird durch folgende Formel (\cite{V46}) beschrieben
\begin{equation}
    \theta_\text{frei} = \frac{{e_0}^3}{8 \, \pi^2 \, \epsilon_0 \, c^3 \, {m_\text{eff}}^2} \frac{N \, B}{n} \lambda^2 = \frac{\theta}{d},
    \label{eq:theta}
\end{equation}
wobei $e_0$ die Elementarladung ist, $\epsilon_0$ die elektrische Feldkonstante, $c$ die Vakuum-Lichtgeschwindigkeit, $m_\text{eff}$ die effektive Masse, $N$ die Donatorenkonzentration, $B$ die Magnetfeldstärke, $n$ der Brechungsindex, $d$ die Probendicke und $\lambda$ die Wellenlänge.
%Die Einheit der Faraday-Rotation ist $rad/m$.

Bei der in diesem Versuch verwendeten Messmethode werden zwei Winkel aufgenommen. Aus diesen lässt sich die auf die Länge skalierte Faraday-Rotation durch
\begin{equation}
    \theta = \frac{\theta_1 - \theta_2}{2d}
    \label{eq:rotation}
\end{equation}
bestimmen.

\begin{figure}
    \centering
    \includegraphics[width=0.7\textwidth]{fotos/FaradayEffekt.png}
    \caption{Zu sehen ist der Faraday-Effekt. Die rote Linie beschreibt das linear polarisierte Licht. In gelb ist die Polarisationsebene hervorgehoben. Das Medium der Länge $d$ ist rötlich dargestellt. Hier bezeichnet $\beta$ die Drehung der Polarisationsebene. \cite{Faraday}}
    \label{fig:FaradayEffekt}    
\end{figure}