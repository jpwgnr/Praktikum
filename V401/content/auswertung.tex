\section{Auswertung}
\label{sec:Auswertung}

\noindent Es soll mit Hilfe eines Michelson-Interferometers die Wellenlänge eines Dioden-Lasers bestimmt werden. 
Dafür werden die Anzahl der ausgelösten Impulse der Photodiode und die Start- und Endwerte auf der Mikrometerschraube in Tab. \ref{taba} dargestellt.  

\begin{table}\caption{Die Anzahl der Impulse, der Startwert auf der Mikrometerschraube und der Endwert auf der Mikrometerschraube.}
\label{taba}
\centering
\sisetup{round-mode = places, round-precision=2, round-integer-to-decimal=true}
\begin{tabular}{S[]S[]S[]} 
\toprule
{Anzahl} & {$d_\text{Start} / \si{\milli\meter}$} & {$d_\text{Start} / \si{\milli\meter}$}\\
\midrule
3001.0 & 6.73 & 2.0\\
3002.0 & 6.73 & 2.0\\
3000.0 & 1.82 & 6.5\\
3000.0 & 6.74 & 2.0\\
3000.0 & 1.83 & 6.5\\
3000.0 & 6.74 & 2.0\\
3001.0 & 1.84 & 6.5\\
3000.0 & 2.83 & 7.5\\
3001.0 & 7.77 & 3.0\\
3002.0 & 2.75 & 7.5\\
\bottomrule
\end{tabular}\end{table}

\noindent Aus dem Betrag der Differenz der Start- und Endwerte auf der Mikrometerschraube ergibt sich ein $\Delta d$. Dieses muss dann noch mit der Hebelübersetzung $1:5,017$  multipliziert werden. 

\noindent Als Mittelwert der Anzahl der Impulse $\z_1$ und der korrekt berechneten $\Delta d$ ergeben sich die Werte 

\begin{align*} 
   z_1 &= \num{3000.7(8)} \\
   \Delta d &= \SI{940(8)}{\micro\meter}.
\end{align*}

\noindent Daraus ergibt sich dann mit Gleichung \eqref{eqn:lambda} und der Fehlerformel \eqref{eqn:errlambda} die Wellengleichung des Lasers mit einem Wert von 

\begin{equation*}
    \lambda = \SI{626(5)}{\nano\meter}.
\end{equation*}

\noindent Der angegebene Literaturwert der Wellenlänge liegt bei 

\begin{equation*}
    \lambda = \SI{635}{\nano\meter}.
\end{equation*}


\noindent Im zweiten Teil wird aus der Anzahl der Impulse, die durch eine Druckveränderung auf der Strecke $b = \SI{5}{\milli\meter}$ des Lichtstrahls ausgelöst werden, der Brechungsindex von Luft bestimmt. 
Die Werte der Anzahl sind in \ref{tabb} dargestellt mit der jeweiligen Druckdifferenz. 

\begin{table}\caption{Die Frequenzen der Sägezahnspannung.}
\label{tabb}
\centering
\sisetup{round-mode = places, round-precision=2, round-integer-to-decimal=true}
\begin{tabular}{S[]S[]} 
\toprule
{Index} & {$\nu_\text{Sä} / \si{\hertz}$}\\
\midrule
1.0 & 25.02\\
2.0 & 49.95\\
3.0 & 99.99\\
4.0 & 149.97\\
\bottomrule
\end{tabular}\end{table}

\noindent Die Anzahl $z_2$ ergibt sich zu 

\begin{equation*}
    z_2 = \num{24.0(6)}.
\end{equation*}

\noindent Daraus ergibt sich eine Abweichung $\Delta n$ mit den fehlerbehafteten Größen $z_2$ und $\lambda$ und der Gleichung \eqref{eqn:deltan} und der Fehlerformel \eqref{eqn:errdeltan} zu 
\begin{equation*}
    \Delta n = \num{150(4)}.
\end{equation*}

\noindent Unter Normalbedingungen ist Temperatur $T_0 = \SI{273.15}{\kelvin}$ und der Druck $ p_0 = \SI{1.0132}{\bar}.$ Die Druckdifferenz wurde auf dem Messgerät abgelesen und ist $\Delta p= \SI{0.6}{\bar}$.

\noindent Mit Gleichung \eqref{eqn:n} und der Fehlerformel \eqref{eqn:errn} ergibt sich dann als Brechungsindex

\begin{equation*}
    n = \num{1.000272(8)}.
\end{equation*}

Der Literaturwert liegt \cite{luft}

\begin{equation*}
    n = \num{1.00029}.
\end{equation*}

