\section{Auswertung}
\label{sec:Auswertung}

\subsection{Bestimmung der Wellenlänge}
Es soll mithilfe eines Michelson-Interferometers die Wellenlänge eines Dioden-Lasers bestimmt werden. 
Dafür werden die Anzahl der ausgelösten Impulse der Photodiode und die Start- und Endwerte auf der Mikrometerschraube in Tab. \ref{taba} dargestellt.  

\begin{table}\caption{Die Länge der Zylinder und die Spannung mit den jeweiligen Zeitenpunkten der Ausschläge.}
\label{taba}
\centering
\sisetup{round-mode = places, round-precision=2, round-integer-to-decimal=true}
\begin{tabular}{S[]S[]S[]S[]S[]} 
\toprule
{$l/ \si{\milli\meter}$} & {$U_1/ \si{\volt}$} & {$t_1/ \si{\micro\second}$} & {$U_2/ \si{\volt}$} & {$t_2/ \si{\micro\second}$}\\
\midrule
120.8 & 1.29 & 0.6 & 0.17 & 88.7\\
102.3 & 1.27 & 0.5 & 0.2 & 76.5\\
80.5 & 1.33 & 0.6 & 0.76 & 59.8\\
40.4 & 1.33 & 0.5 & 1.34 & 30.2\\
31.1 & 1.29 & 0.5 & 1.37 & 23.8\\
\bottomrule
\end{tabular}\end{table}

\noindent Aus dem Betrag der Differenz der Start- und Endwerte auf der Mikrometerschraube ergibt sich ein $\Delta d$. Dieses muss noch mit der Hebelübersetzung $1:5,017$  multipliziert werden. 

\noindent Als Mittelwert der Anzahl der Impulse $\z_1$ und der korrekt berechneten $\Delta d$ ergeben sich die Werte 

\begin{align*} 
   z_1 &= \num{3000.7(8)} \\
   \Delta d &= \SI{940(8)}{\micro\meter}.
\end{align*}

\noindent Daraus ergibt sich mit Gleichung \eqref{eqn:lambda} und der Fehlerformel \eqref{eqn:errlambda} die Wellenlänge des Lasers zu einem Wert von 

\begin{equation*}
    \lambda_\text{exp} = \SI{626(5)}{\nano\meter}.
\end{equation*}

\noindent Der angegebene Literaturwert der Wellenlänge liegt bei 

\begin{equation*}
    \lambda_\text{lit} = \SI{635}{\nano\meter}.
\end{equation*}

\subsection{Bestimmung des Brechungsindex}
Im zweiten Teil wird aus der Anzahl der Impulse, die durch eine Druckveränderung auf der Strecke $b = \SI{5}{\centi\meter}$ des Lichtstrahls ausgelöst werden, der Brechungsindex von Luft bestimmt. 
Die Werte der Anzahl sind in \ref{tabb} mit der jeweiligen Druckdifferenz dargestellt.

\begin{table}\caption{Die angelegte Spannung des elektrischen Feldes innerhalb des Geiger-Müller-Zählrohrs, die Anzahl der jeweils gemessenen Impulse und der Strom innerhalb des Geiger-Müller-Zählrohrs.}
\label{tabb}
\centering
\sisetup{round-mode = places, round-precision=2, round-integer-to-decimal=true}
\begin{tabular}{c c S[]} 
\toprule
{$U / \si{\volt}$} & {$\frac{N}{\SI{130}{\second}}$} & {$I / \si{\ampere}$}\\
\midrule
320 & 11298 & 0.1\\
400 & 11820 & 0.2\\
480 & 12135 & 0.3\\
540 & 12301 & 0.35\\
560 & 12068 & 0.4\\
600 & 12354 & 0.45\\
640 & 12403 & 0.5\\
660 & 12507 & 0.55\\
680 & 12659 & 0.6\\
\bottomrule
\end{tabular}\end{table}

\noindent Der Mittelwert der Anzahl $z_2$ ergibt sich zu 

\begin{equation*}
    z_2 = \num{24.0(6)}.
\end{equation*}

\noindent Daraus ergibt sich eine Abweichung $\Delta n$ mit den fehlerbehafteten Größen $z_2$ und $\lambda$, mit der Gleichung \eqref{eqn:deltan} und der Fehlerformel \eqref{eqn:errdeltan} zu 
\begin{equation*}
    \Delta n = \num{150(4)}.
\end{equation*}

\noindent Unter Normalbedingungen ist die Temperatur $T_0 = \SI{273.15}{\kelvin}$ und der Druck $ p_0 = \SI{1.0132}{\bar}.$ Die Druckdifferenz wurde auf dem Messgerät abgelesen und ist $\Delta p= \SI{0.6}{\bar}$.

\noindent Mit Gleichung \eqref{eqn:n} und der Fehlerformel \eqref{eqn:errn} ergibt sich als Brechungsindex

\begin{equation*}
    n_\text{exp} = \num{1.000272(8)}.
\end{equation*}

\noindent Der Literaturwert \cite{luft} liegt bei

\begin{equation*}
    n_\text{lit} = \num{1.00029}.
\end{equation*}

