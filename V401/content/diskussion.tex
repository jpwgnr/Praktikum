\section{Diskussion}
\label{sec:Diskussion}

\subsection{Bestimmung der Wellenlänge}
Der relative Fehler des $\Delta d$ ergibt sich zu einem Wert von \SI{0.85}{\percent}. 
\newline
Die Anzahl der Impulse hat einen relativen Fehler von \SI{0.03}{\percent}. 
\newline
Der relative Fehler der Wellenlänge ergibt sich somit zu \SI{0.80}{\percent}. 
\newline
Die relative Abweichung zum Literaturwert liegt bei 
\SI{1.42}{\percent}.

\subsection{Bestimmung des Brechungsindex}
Die Anzahl der Impulse hat einen relativen Fehler von \SI{2.5}{\percent}. 
\newline
Der relative Fehler von $\Delta n$ ergibt sich zu \SI{2.67}{\percent}.
\newline
Wenn die Eins abgezogen wird, ergibt sich für $n$ ein relativer Fehler von \SI{2.94}{\percent}. Damit ist die relative Abweichung zum Literaturwert \SI{6.2}{\percent}. 
