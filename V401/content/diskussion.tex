\section{Diskussion}
\label{sec:Diskussion}

\subsection{Bestimmung der Wellenlänge}
Der relative Fehler des $\Delta \, d$ ergibt sich zu einem Wert von \SI{0.85}{\percent}. Damit scheint die Messung relativ exakt gewesen zu sein.  
Die Anzahl der Impulse hat einen relativen Fehler von \SI{0.03}{\percent}. Somit ist auch dieser Fehler sehr gering und der Wert scheint sehr exakt gemessen worden zu sein.  
Der relative Fehler der Wellenlänge ergibt sich somit zu \SI{0.80}{\percent} 
Die relative Abweichung zum Literaturwert liegt bei 
\SI{1.42}{\percent}.
Als Fazit ergibt sich, dass die Messung im ersten Teil ziemlich exakt war und kaum Abweichung zur Literatur zu erkennen ist. 

\subsection{Bestimmung des Brechungsindex}
Die Anzahl der Impulse hat einen relativen Fehler von \SI{2.5}{\percent}. 
Der relative Fehler von $\Delta n$ ergibt sich zu \SI{2.67}{\percent}.
Wenn die Eins abgezogen wird, ergibt sich für $n$ ein relativer Fehler von \SI{2.94}{\percent}. Damit ist die relative Abweichung zum Literaturwert \SI{6.2}{\percent}. Somit scheint das Ergebnis auch noch relativ exakt zu sein, aber etwas weniger exakt als die Bestimmung der Wellenlänge. 
