\section{Fehlerrechnung}

Der Mittelwert einer Stichprobe von $N$ Werten wird durch
\begin{equation}
    \overline{x} = \frac{1}{N} \sum_{i=1}^N x_i
    \label{eqn:mittelwert}
\end{equation}
bestimmt.
\newline
Die Standardabweichung der Stichprobe wird berechnet mit
\begin{equation*}
    \sigma_x = \sqrt{\frac{1}{N-1} \sum_{i=1}^N (x_i - \overline{x})^2}.
    \label{eqn:standard}
\end{equation*}
\newline
Die realtive Abweichung zwischen zwei Werten kann durch
\begin{equation*}
    f = \frac{x_\text{a} - x_\text{r}}{x_\text{r}}
\end{equation*}
bestimmt werden.
\newline

Die Fehlerfortpflanzung für Gleichung \eqref{eqn:lambda} ergibt sich zu 
\begin{equation}
    \sigma_{\Delta d} = \sqrt{\frac{\lambda^{2} \sigma_{z}^{2}}{4} + \frac{z^{2} \sigma_{\lambda}^{2}}{4}}.
    \label{eqn:errlambda}
\end{equation}

Für die Gleichung \eqref{eqn:deltan} ergibt sich die Fehlerfortpflanzung als 
\begin{equation}
    \sigma_{\Delta n} =\sqrt{\frac{\lambda^{2} \sigma_{z}^{2}}{4 b^{2}} + \frac{z^{2} \sigma_{\lambda}^{2}}{4 b^{2}}}
    \label{eqn:errdeltan}
\end{equation}

Für die Gleichung \eqref{eqn:n} ergibt sich die Fehlerfortpflanzung als 
\begin{equation}
    \sigma_{n} =  \sqrt{\frac{T^{2} \Delta n^{2} \sigma_{\Delta p}^{2} p_{0}^{2}}{T_{0}^{2} \Delta p^{4}} + \frac{T^{2} \sigma_{\Delta n}^{2} p_{0}^{2}}{T_{0}^{2} \Delta p^{2}} + \frac{\Delta n^{2} \sigma_{T}^{2} p_{0}^{2}}{T_{0}^{2} \Delta p^{2}}}.
\label{eqn:errn}
\end{equation}
