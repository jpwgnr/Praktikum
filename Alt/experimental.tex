Prüfer: Bayer, Note: 1.0

Wir begannen mit den klassischen Experimenten der Quantenmechanik. Als erstes sollte ich den Photoeffekt erklären. 
Ich skizzierte den Aufbau, schrieb die Formel auf und skizzierte ebenso den Plot, der dabei überlicherweise gemessen wird.
Ab und an fragte Herr Bayer dann auch das ein oder andere Detail, aber im Großen und Ganzen ließ er mich einfach reden und erklären. 
Das Hanbury Brown - Twiss Experiment nannte ich noch als Alternative, da Herr Bayer häufig erzählt hatte, dass Lamb (übrigens der gleiche Lamb, nach dem der Lamb-Shift benannt wurde) und noch irgendwer anders den Photoeffekt auch klassisch erklärt hatten, indem der Detektor als quantisiert betrachtet wird. 

\noindent Anschließend sollte ich noch auf ein Experiment eingehen, bei dem man den Wellencharakter eines Teilchens sehen kann. Also erklärte ich den Doppelspaltversuch und erzählte von Möllenstedt-Düker und Davisson-Germer. Auch hier fragte Herr Bayer recht wenig. Im Gegenteil eigentlich. Er erzählte gefühlt mehr als ich selbst. Zu diesem Zeitpunkt fühlte es sich gar nicht mehr nach einer Prüfung an, sondern viel mehr wie ein entspanntes Gespräch beim Kaffeetrinken. Die Prüfungsatmosphäre war echt genial und ist sehr zu empfehlen. %kann man eine Prüfungsatmosphäre empfehlen? Vielleicht eher die Prüfung bei Bayer
Er erklärte dann, dass das ja alles richtig sei, was ich gesagt hätte, aber dass, wenn er ehrlich ist, ich Davisson-Germer eigentlich noch gar nicht verstehen kann, solange ich nicht mehr über Festkörperphysik gelernt hätte. Dann ging es also weiter. 

\noindent Nun kamen wir aber zum theoretischen Teil der Quantenmechanik, wobei eigentlich hauptsächlich die Basics abgefragt wurden. \enquote{Nennen Sie bitte die Schrödingergleichung. Wie sieht die stationäre Schrödingergleichung aus und was bedeutet sie? Was bedeutet denn das Betragsquadrat der Wellenfunktion? ...} Davon ausgehend erzählte ich dann etwas über die Kontinuitätsgleichung und die Wahrscheinlichkeitsstromdichte und schrieb die Formel dafür hin. Er fragte, wie die Wellenfunktion eines freien Teilchens aussieht. Dann fragte er noch nach dem Potentialtopf und dem harmonischen Oszillator. Ich schrieb zu beidem die Energieeigenwerte auf und skizzierte die Energieniveaus und erklärte noch, dass beim Potentialtopf das $n$ bei \num{1} anfängt und beim harmonischen Oszillator bei \num{0} und nannte in dem Zusammenhang noch den Begriff der Vakuumfluktuation und dann war es das auch schon. Er schickte mich raus und rief mich kurz danach wieder rein und meinte dann, dass das ein klares Ding gewesen sei und ich bekam meine Note.

\noindent Als Fazit: Es schien vor allem wichtig zu sein, die Versuche recht gut erklären zu können. Beim zweiten Teil reichte es die Basics zu beherrschen und dabei sicher zu sein. Insgesamt konnte man das Gespräch an vielen Stellen in die Richtungen lenken, in die man wollte und hatte viele Freiheiten, solange man keinen Quatsch erzählte. Eine sehr spaßige Prüfung! 
