\section{Note und Fazit}

\subsection{Note}

Als ich dann meine Note bekam meinte er, dass sich das relativ einfach zusammensetzt. Den ersten Teil würde er mit einer \num{1.0} bewerten, den QM Teil aber leider nur mit einer \num{3.0}, was sich dann insgesamt zu einer \num{2.0} ergibt. 

\subsection{Fazit}

\noindent Als Fazit zu der Prüfung: Man hat nicht so viel Spielraum wie bei anderen Prüfer*Innen. Es gibt Prüfer*Innen, die lassen dich viel reden und merken dann, dass du schon ein großes Verständnis hast, aber bei Uhrig ist es eher so, dass du ihm wirklich auf eine exakt gestellte Frage auch eine exakte Antwort liefern musst, ansonsten ist er nicht zufrieden und hakt die ganze Zeit nach. Das hat mich persönlich leider sehr verunsichert, sodass ich dann nach der Frage mit den zeitabhängigen Zuständen echt sehr durcheinander war. Wäre ich besser vorbereitet gewesen, hätte man das aber bestimmt gut beantworten können.
Auf jeden Fall sollte man eine ganze Menge Formeln wirklich gut verstanden haben und auswendig können. Es ist sehr häufig der Satz: "Dann schreiben Sie das doch einmal hin." vorgekommen, also dass ich das verbal zwar beantwortet hatte, er aber explizit Formeln oder Skizzen zu verschiedenen Dingen sehen wollte. 

\noindent \textbf{Viel Erfolg!} 
