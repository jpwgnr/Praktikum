\section{Durchführung und Fragen}

\noindent Wir begannen die Prüfung mit der analytischen Mechanik aus dem 3. Semester. 
Als erstes sollte ich das Prinzip von d'Alembert erklären und dann wollte er recht schnell auf die Lagrange Funktion und die ELG hinaus. 
Die nächsten Fragen waren dann sehr zielgerichtet auf den Hamiltonformalismus und darauf, dass der Hamilton die Legendre Trafo der Lagrange-Funktion ist.
Nachdem ich das dann allgemein hingeschrieben hatte, fragte er nochmal nach und ich sollte noch die Indizes an den richtigen Stellen ergänzen. 
Anschließend wollte er die Hamilton DGL'n sehen und die Vorteile zwischen den Formalismen hören. Explizit fragte er dann nach den Punkttrafos bei Lagrange und nach den kanonischen Trafos beim Hamiltonformalismus und von was es mehr gäbe. Da wusste ich zum Glück noch, dass die Punkttrafos eine Teilmenge der kanonischen Trafos waren, was man sich dann auch ganz schnell überlegen kann, wenn man über die Legendre-Transformation nachdenkt. 
Ein bisschen was sollte ich dann zum Phasenraum erklären, aber nur sehr oberflächlich -was für Achsen man da betrachtet und sowas. 
Als letztes wollte er dann -zum Glück nur verbal- erklärt bekommen, was der Hamilton-Jacobi Formalismus ist und wie man diesen umsetzt. 
Er hätte gern noch etwas zum KAM-Theorem gefragt, aber das hatten wir bei Frau Hiller nicht besprochen, also war er somit zufrieden bzgl. der analytischen Mechanik.

\noindent Im nächsten Teil war ich nicht so stark. Er machte nun den Sprung zur Quantenmechanik. Die Basics, Schrödingergleichung etc. liefen noch ganz gut. Er wollte dann sehen, wie ich von einem Zustand auf eine Wellenfunktion komme, wie man zu einem stationären Zustand kommt und wie so ein allgemeiner Erwartungswert aussieht. Das war auch alles gut. 
Dann fragte er mich, ob ein Erwartungswert zeitabhängig sei. Mit ein bisschen Hilfe kam ich darauf, dass ein einzelner Zustand seine Zeitabhängigkeit immer verliert, aber war schon recht verunsichert. Dann wollte er von mir hören, ob ein Erwartungswert im allgemeinen immer zeitunabhängig ist. Das wusste ich nicht. Er bestand aber darauf, sodass ich dann versuchte es mir herzuleiten, aber schaffte es nicht. Im Endeffekt löste er es dann auf. Man müsste einfach nur Zustände überlagern, dann käme man auch auf Erwartungswerte mit Zeitabhängigkeit vorausgesetzt das passt mit der Entartung.

\noindent Das nächste was mich dann überforderte, waren Fragen zum harmonischen Oszillator. Ich konnte zwar Energieeigenwerte aufschreiben und auch sagen, dass die Eigenfunktionen durch Hermite-Polynome beschrieben werden können, aber dann sollte ich ihm die Eigenfunktionen aufzeichnen und da war ich dann wieder raus. 

\noindent Im letzten Teil fragte er mich dann noch zur Störungstheorie und wollte da den gestörten Energiewert erster Ordnung hören, ich hatte aber nur noch den zweiten im Kopf, wobei ich da dann auch noch Probleme mit den Indizes hatte und somit war er mit Recht sehr unzufrieden mit dem zweiten Teil der Prüfung.

