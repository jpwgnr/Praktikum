\section{Theorie}

\begin{frame}{Inhaltsverzeichnis}
    \tableofcontents[currentsection, hidesubsections]
\end{frame} 

\subsection{Geschwindigkeit}
\begin{frame}[t]{Theorie}
    \begin{block}{Geschwindigkeit}    
    \begin{equation*}
        v= \frac{s}{t}
    \end{equation*}
    
    \begin{equation*}
        v = \omega \cdot r
    \end{equation*}
\end{block}
\end{frame}

\subsection{Dreh- und Trägheitsmoment}

\begin{frame}[t]{Theorie}
    \begin{block}{Drehmoment}
    \begin{equation}
        \abs{\vec{M}} = \abs{\,\vec{r} \cross \vec{F}\, } = I \cdot \dot{\omega} = r\, m\, g
        \label{eqn:drehmoment}
    \end{equation}
    \end{block}

    \begin{block}{Trägheitsmoment}
    \begin{equation}
        I_\text{S}= \frac{1}{2}\, m R^2 
        \label{eqn:trägheit}
    \end{equation}

    \begin{equation}
        I_\text{A} = I_\text{S}+ m \cdot r^2
        \label{eqn:steiner}
    \end{equation}
    \end{block}
\end{frame}


\subsection{Beschleunigung}
\begin{frame}[t]{Theorie}
    \begin{block}{Beschleunigung}
    Mit Gleichung \eqref{eqn:drehmoment}, Gleichung \eqref{eqn:trägheit} und Gleichung \eqref{eqn:steiner} ergibt sich
    \begin{equation*}
       \left(\frac{R^2}{2 r^2} +1\right) \cdot \dot{v} = g
    \end{equation*}
   und damit ist  
    \begin{equation*}
        \ddot{s} = \frac{1}{1 + \frac{R^2}{2r^2}} \cdot g.
    \end{equation*}

    Die tatsächliche Beschleunigung ergibt sich zu 
    
    \begin{equation*}
        a = \frac{2s}{t^2}.
    \end{equation*}
    \end{block}
\end{frame}

\begin{frame}[t]{Theorie}
    \begin{block}{Radius}
    Der Radius $r$ ergibt sich mit der abgerollten Länge von 10 Umdrehungen $\Delta s$ zu

    \begin{equation*}
        r = \frac{\Delta s}{10 \cdot 2\pi}.
    \end{equation*}
    \end{block}
\end{frame}


\subsection{Energien}

\begin{frame}[t]{Theorie}
    \begin{block}{Potentielle Energie}
    \begin{equation}
    E_\text{pot} = m\, g\, h 
    \end{equation}
    \end{block}

    \begin{block}{Rotationsenergie}
    \begin{equation}
        E_\text{rot} = \frac{1}{2}\, I_\text{A} \cdot \omega^2
    \end{equation}
    \end{block}
\end{frame}

