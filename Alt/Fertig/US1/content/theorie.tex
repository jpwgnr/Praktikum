\section{Ziel}
Die grundlegenden Eigenschaften und Begriffe der
Ultraschallechographie sollen kennengelernt und angewandt
werden. Dabei werden die Schallgeschwindigkeit und der Absorptionskoeffizient in Acryl gemessen.
Außerdem wird die Länge von unterschiedlichen Acrylzylindern mit verschiedenen Verfahren bestimmt.
Die Längen in einem Augenmodell werden bestimmt. %Zwei, drei Zeilen mehr 

\section{Theorie}
\label{sec:Theorie}
\subsection{Allgemeines} %bessere Überschrift?
%Frequenzbereich
Der Frequenzbereich des Ultraschalls liegt zwischen 
\SI{20}{\kilo\hertz} und ca. \SI{1}{\giga\hertz}. 
%Anwendung
Die Ultraschalltechnik findet Anwendung bei 
zerstörfreier Materialprüfung und in der Medizin. 

%Def. Schall
\noindent Der Schall ist eine longitudinale Druckwelle und wird 
beschrieben durch
\begin{equation*}
    p(x,t) = p_0 + v_0 \, Z \, cos(\omega t - kx).
\end{equation*}

%Akustische Impedanz
\noindent Dabei ist $Z$ die akustische Impedanz, die sich aus der 
Schallgeschwindigkeit in diesem Material und dessen Dichte 
zusammensetzt. Es gilt 
\begin{equation*}
    Z = c \cdot \rho.
\end{equation*}

%Eigenschaften
\noindent Die Welle besitzt ähnliche Eigenschaften wie 
elektromagnetische Wellen. Die Phasengeschwindigkeit 
ist aber materialabhängig. 
%Schallgeschwindigkeit
In Gasen und Flüssigkeiten breitet 
sich der Schall immer longitudinal aus. In Flüssigkeiten ist 
die Geschwindigkeit abhängig von der Kompressibilität 
$\kappa$ und der Dichte $\rho$. 
Sie ergibt sich zu 
\begin{equation*}
    c_{\text{Fl}}= \sqrt{\frac{1}{\kappa \cdot \rho}}.
\end{equation*}
Bei einem Festkörper ergibt sich die von dem Elastizitätsmodul
$E$ abhängige Geschwindigkeit zu 
\begin{equation*}
    c_{\text{Fe}}= \sqrt{\frac{E}{\rho}}.
\end{equation*}
Die Geschwindigkeit unterscheidet 
sich im Festkörper für die longitudinale und
transversale Ausbreitung. Die Geschwindigkeit ist richtungsabhängig.

%Intensität
\noindent Ein Teil der Energie bei der Ausbreitung von Schall geht 
durch Absorption verloren. Die Intensität $I_0$ nimmt 
exponentiell auf der Strecke $x$ ab:
\begin{equation}
    I(x)= I_0 \cdot e^{-\alpha x}.
    \label{eqn:I}
\end{equation}
Der Faktor $\alpha$ 
ist der Absorptionskoeffizient.
In Luft wird Schall stark absorbiert, weshalb zwischen 
Schallgeber und zu untersuchendem Material ein Kontaktmittel 
verwendet wird.
\newline
Eine Schallwelle, die auf eine Grenzfläche trifft, wird 
reflektiert. Der Reflexionskoeffizient ergibt sich mit den 
Impedanzen beider Materialien zu 
\begin{equation*}
    R = \left(\frac{Z_1-Z_2}{Z_1-Z_2}\right)^2. %höhere Klammern?
\end{equation*}
Der Transmissionsanteil wird mittels $T= 1-R$ bestimmt.

\noindent Wenn die Länge $s$ und die Durchlaufzeit $\Delta t = t_2 - t_1$
bekannt sind, lässt sich aus der bekannten Gleichung
\begin{equation}
    s=\frac{1}{2}c \Delta t
    \label{eqn:s}
\end{equation}
die Schallgeschwindigkeit mittels
\begin{equation}
    c = \frac{2 s}{\Delta t}
    \label{eqn:c}
\end{equation}
berechnen.

\subsection{Erzeugung von Ultraschall}
\noindent Die Erzeugung von Ultraschall funktioniert auf verschiedene 
Arten. Eine Art ist die Verwendung des reziproken 
piezo-elektrischen Effekts. Dafür bringt man einen 
piezoelektrischen Kristall in ein elektrisches, sich
wechselndes Feld, sodass der Kristall, wenn eine Achse in 
Richtung des Feldes gerichtet ist, zu 
schwingen beginnt. Beim Schwingen strahlt er Ultraschallwellen ab. 
Stimmen Anregungsfrequenz und Eigenfrequenz überein, 
entstehen große Amplituden, sodass hohe Schallenergiedichten 
genutzt werden. Der Kristall kann auch als Schallempfänger 
verwendet werden. Dabei wird auch wieder der piezo-elektrische Effekt genutzt, bloß andersherum. 
Durch die Schallwellen wird der Kristall in Schwingung versetzt und erzeugt so einen messbares Feld. %Wiefunktioniert er als Schallempfänger 
Quarze sind dabei die meist benutzten 
Piezokristalle, da sie konstante Eigenschaften besitzten. 
Der piezoelektrische Effekt ist bei diesen aber relativ schwach. 
%"sich wechselnd"?

\subsection{Verfahren}
In der Ultraschalltechnik werden zwei Verfahren verwendet - 
das Duchschallungs-Verfahren und das Impuls-Echo-Verfahren. 
\newline
Das Durchschallungs-Verfahren funktioniert so, dass mit einem 
Ultraschallsender ein kurzzeitiger Schallimpuls gesendet wird 
und am anderen Ende der Probe ein Empfänger steht. Ein 
abgeschwächtes Signal gibt Auskunft darüber, dass eine 
Fehlstelle vorhanden ist. Dabei kann aber nicht bestimmt 
werden, wo sich die Fehlstelle befindet.
\newline
Beim Impuls-Echo-Verfahren ist der Schallsender auch der 
Empfänger. Der Ultraschallpuls wird hierbei an Grenzflächen 
reflektiert und nach der Rückkehr vom Empfänger aufgenommen. 
Bei Fehlstellen kann dann deren Größe durch die Höhe des Echos
bestimmt werden. 

\noindent Laufzeitdiagramme können als A-Scan, B-Scan oder TM-Scan 
durchgeführt werden. Ein A-Scan ist eine Amplitudenmessung. Die gemessene Amplitude wird also gegen die Zeit 
aufgetragen. Der B-Scan ist die Helligkeitsmessung. Dabei wird ein 2D-Bild, an dem man Ort und Intensität anhand 
der Farben in dem Bild erkennen kann, erstellt. Der TM-Scan ist die Time-Motion Darstellung des Bildes. 
Dabei werden mehrere Amplituden in gewissen Zeitabständen losgeschickt. 
Die Amplitude ist auf der vertikalen Achse angegeben und die Echozüge werden auf der horizontalen Achse dargestellt, 
stellen also die Zeitachse dar. Bewegungen des Gewebes lassen sich damit bespielsweise gut erkennen. 
