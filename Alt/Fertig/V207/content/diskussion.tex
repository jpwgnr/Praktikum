\section{Diskussion}
\label{sec:Diskussion}

\subsection{Bestimmung der Apparaturkonstante für die große Kugel}
Bei der Messung der Viskosität für die kleine Kugel ergab sich eine Viskosität von \SI{773}{\micro\pascal\second} mit 
einem realtiven Fehler von \SI{0.91}{\percent}.
Daraus ergab sich wiederum eine Apparaturkonstante von \SI{11.4}{\nano\pascal\cubic\meter\per\kilo\gram} mit einem 
realtiven Fehler von \SI{0.96}{\percent}. 
Es gibt keine Literaturwerte, somit kann über die Richtigkeit dieser Ergebnisse keine Aussage getroffen werden. Fehlerquellen könnten dabei zum Beispiel sein, dass dabei die Apparatur nicht richtig eingerastet war und dass eventuell kleine Blasen die Werte beeinflusst haben. 

\subsection{Bestimmung der Temperaturabhängigkeit der Viskosität von destilliertem Wasser}
Die Messung der Temperaturabhängigkeiten führte bei uns ein wenig zu Problemen, da wir mit \SI{25}{\degreeCelsius} 
begonnen hatten, sich aber dann bei ca. \SI{50}{\degreeCelsius} ziemlich starke Blasen gebildet haben und wir diese 
daraufhin entfernt haben. Das erneute Erwärmen des zugeführten Wassers dauerte aber länger als gedacht und führte 
dazu, dass wir unsere Messung nochmal neu begannen bei einer Temperatur von \SI{53}{\degreeCelsius} bzw. 
\SI{326.15}{\degreeCelsius}.

\noindent Die ermittelten Konstanten A und B aus der Andradeschen Gleichung ergeben sich für die erste Messung zu 
\SI{56.6}{\nano\pascal\second} mit einem relativen Fehler von \SI{90.97}{\percent} und \SI{3012.12}{\kelvin} mit 
einem relativen Fehler von \SI{10.1}{\percent}. Für die zweite Messung ergibt sich für A ein Wert von 
\SI{37.5}{\nano\pascal\second} mit einem relativen Fehler, der bei \SI{66.76}{\percent} liegt. Für B ergibt sich 
\SI{3140.87}{\kelvin} mit einem relativen Fehler von \SI{7.12}{\percent}. 

\subsection{Graphische Auswertung}
Bei der graphischen Auswertung fällt auf, dass die ersten zwei Werte noch stark von der Ausgleichsgeraden abweichen, 
die darauf folgenden Werte aber nicht. Dies liegt vermutlich an dem zugeführten Wasser. Die Temperatur in dem 
Behälter wird vermutlich noch nicht ganz der Temperatur auf der Anzeige entsprochen haben. Somit sind die Werte ein 
wenig verschoben und müssten eigentlich noch weiter links auf der Temperaturskala stehen. Dieses Ergebnis ergibt also 
im Kontext unserer Durchführung Sinn und entspricht dem, was zu erwarten war. 

\subsection{Bestimmung der Reynoldszahlen}
Die Reynoldszahl für den Fall der kleinen Kugel beträgt \num{172.7} und der relative Fehler liegt bei 
\SI{1.80}{\percent}. Für die große Kugel ergab sich ein Wert von \num{31.27} mit einem relativen Fehler von 
\SI{0.99}{\percent}. Da Flüssigkeiten mit einem Wert von unter \num{2300} bei der Reynoldszahl noch als laminare 
Flüssigkeit geleten, haben wir es in unserer Messung auch mit einer laminaren Flüssigkeit zu tun gehabt. 

\noindent Die in der Tabelle \ref{tab7} bestimmten Werte zeigen, dass mit zunehmender Temperatur auch die jeweilige 
Reynoldszahl steigt. Bei einer Zunahme von \SI{17}{\degreeCelsius} nimmt die Reynoldszahl um mehr als \num{100} zu. 
Das bedeutet, dass die Flüssigkeit mit zunehmender Temperatur immer turbulenter wird. Dies entspricht auch dem zu 
erwartenden Ergebnis, da sich Moleküle schneller bewegen, je wärmer das jeweilige Material ist, was einem 
turbulenteren Verhalten entspricht.
