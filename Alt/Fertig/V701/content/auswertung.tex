\section{Auswertung}
\label{sec:Auswertung}

Für die Auswertung wird Python und im Speziellen Matplotlib \cite{matplotlib}, SciPy \cite{scipy}, Uncertainties \cite{uncertainties} und NumPy \cite{numpy} verwendet.

\subsection{Bestimmung des Energieverlustes von Alphastrahlung in Luft}

\subsubsection{Energieverlust bei einem Abstand von d = 2,7 cm} %Verbesserung
Die Drücke, die mit Gleichung \eqref{eqn:x} ermittelten Abstände $x_1$, die Anzahl der Pulse, die Position des jeweiligen Maximums für den Abstand $d_1 = \SI{2.7}{\centi\meter}$ und die sich daraus mit Gleichung \eqref{eqn:energie} ergebenden Energien sind in Tab. \ref{taba} zu sehen.
 
\begin{table}\caption{Die Anzahl der Impulse, der Startwert auf der Mikrometerschraube und der Endwert auf der Mikrometerschraube.}
\label{taba}
\centering
\sisetup{round-mode = places, round-precision=2, round-integer-to-decimal=true}
\begin{tabular}{S[]S[]S[]} 
\toprule
{Anzahl} & {$d_\text{Start} / \si{\milli\meter}$} & {$d_\text{Start} / \si{\milli\meter}$}\\
\midrule
3001.0 & 6.73 & 2.0\\
3002.0 & 6.73 & 2.0\\
3000.0 & 1.82 & 6.5\\
3000.0 & 6.74 & 2.0\\
3000.0 & 1.83 & 6.5\\
3000.0 & 6.74 & 2.0\\
3001.0 & 1.84 & 6.5\\
3000.0 & 2.83 & 7.5\\
3001.0 & 7.77 & 3.0\\
3002.0 & 2.75 & 7.5\\
\bottomrule
\end{tabular}\end{table}

\noindent Die Zählrate ist in Abb. \ref{zaehlrate1} gegen die mit Gleichung \eqref{eqn:x} bestimmte effektive Länge aufgetragen.

\begin{figure}
    \centering
    \includegraphics[width=15cm, height=9cm]{build/plota.pdf}
    \caption{Die Zählrate pro $\num{120}$ Sekunden Messzeit ist gegen den effektiven Abstand $x_1$ aufgetragen. Dabei wurde dieses Ergebnis durch die Abhängigkeit vom jeweiligen Druck ermittelt.}
    \label{zaehlrate1}
\end{figure}

\noindent Die Fitparameter der linearen Regression ergeben sich dadurch zu 
\begin{align*}
    m &= - \num{11300.95(325)} \,\frac{\text{1}}{\SI{120}{\second}\, \si{\milli\meter}}, \\
    n &= \num{289663.5} \, \frac{\text{1}}{\SI{120}{\second}} .
\end{align*}


\noindent Mit dem Umformen dieser linearen Gleichung ergibt sich bei $y = \frac{1}{2} N_\text{max}$ die mittlere Reichweite der $\alpha$-Teilchen zu %wie genau?
\begin{equation*}
    R_\text{m,1} = \SI{22.71(1)}{\milli\meter}.
\end{equation*}

\noindent Das entspricht nach Formel \eqref{eqn:Rm} einer Energie von %wie genau?
\begin{equation*}
    E_1 = \SI{3.772(1)}{\mega\electronvolt}.
\end{equation*}

%Energie als Funktion der effektiven Länge für den ersten Abstand
\noindent Die Energie aus Tab. \ref{taba} ist in Abb. \ref{fig:energie1} gegen die effektive Länge aufgetragen.
\begin{figure}
    \centering
    \includegraphics[width=15cm, height=9cm]{build/plotb.pdf}
    \caption{Die Energie ist gegen den effektiven Abstand $x_1$ aufgetragen.}
    \label{fig:energie1}
\end{figure}

\noindent Die Fitparameter der linearen Regression mit einer allgemeinen Gleichung \eqref{linReg} ergeben sich dadurch zu 
\begin{align*}
    m &= - \SI{0.1145(42)}{\mega\electronvolt\per\milli\per\meter}, \\
    n &= \SI{4.070}{\mega\electronvolt} .
\end{align*}

\noindent Daraus lässt sich anhand der Steigung der Energieverlust der Strahlung bestimmen %wie genau?
\begin{equation*}
    - \left( \frac{dE}{dx} \right)_1 = - \SI{0.1145(45)}{\mega\electronvolt\per\milli\per\meter}.
\end{equation*}

\noindent Die Energie, die sich aus der Steigung und dem y-Abschnitt ergibt, ist  

\begin{equation*}
    E_\text{\alpha} (R_{m,1}) = m \cdot R_{m,1} + n = \SI{3.298(7)}{\MeV}.
\end{equation*}


\subsubsection{Energieverlust bei einem Abstand von d = 2 cm} %Verbesserung
Die Drücke, die mit Gleichung \eqref{eqn:x} ermittelten Abstände $x_2$, die Anzahl der Pulse, die Position des jeweiligen Maximums für den Abstand $d_2 = \SI{2.0}{\centi\meter}$ und die sich daraus mit Gleichung \eqref{eqn:energie} ergebenden Energien sind in Tab. \ref{tabb} zu sehen.
 
 \begin{table}\caption{Die Frequenzen der Sägezahnspannung.}
\label{tabb}
\centering
\sisetup{round-mode = places, round-precision=2, round-integer-to-decimal=true}
\begin{tabular}{S[]S[]} 
\toprule
{Index} & {$\nu_\text{Sä} / \si{\hertz}$}\\
\midrule
1.0 & 25.02\\
2.0 & 49.95\\
3.0 & 99.99\\
4.0 & 149.97\\
\bottomrule
\end{tabular}\end{table}

\noindent Die Zählrate ist in Abb. \ref{zaehlrate2} gegen die mit Gleichung \eqref{eqn:x} bestimmte effektive Länge aufgetragen.

\begin{figure}
    \centering
    \includegraphics[width=15cm, height=9cm]{build/plotc.pdf}
    \caption{Die Zählrate pro $\num{120}$ Sekunden Messzeit ist gegen den effektiven Abstand $x_2$ aufgetragen. Dabei wurde dieses Ergebnis durch die Abhängigkeit vom jeweiligen Druck ermittelt.}
    \label{zaehlrate2}
\end{figure}

\noindent Die Fitparameter der linearen Regression ergeben sich dadurch mit einer allgemeinen Gleichung \eqref{linReg} zu 
\begin{align*}
    m &= -\num{839.54(3085)}\, \frac{\text{1}}{\SI{120}{\second} \, \si{\milli\meter}}, \\
    n &= \num{108514.077} \, \frac{\text{1}}{\SI{120}{\second}} .
\end{align*}


\noindent Mit dem Umformen dieser linearen Gleichung ergibt sich bei $y = \frac{1}{2} N_\text{max}$ die mittlere Reichweite der $\alpha$-Teilchen zu %wie genau?
\begin{equation*}
    R_\text{m,2} = \SI{65.8(24)}{\milli\meter}.
\end{equation*}


\noindent Das entspricht nach Formel \eqref{eqn:Rm} einer Energie von %wie genau?
\begin{equation*}
    E_2 = \SI{7.67(19)}{\mega\electronvolt}.
\end{equation*}

%Energie als Funktion der effektiven Länge für den ersten Abstand
\noindent Die Energie aus Tab. \ref{tabb} ist in Abb. \ref{fig:energie2} gegen die effektive Länge aufgetragen.
\begin{figure}
    \centering
    \includegraphics[width=15cm, height=9cm]{build/plotd.pdf}
    \caption{Die Energie ist gegen den effektiven Abstand $x_2$ aufgetragen.}
    \label{fig:energie2}
\end{figure}

\noindent Die Fitparameter der linearen Regression ergeben sich dadurch zu 
\begin{align*}
    m &= -\SI{0.1117(26)}{\mega\electronvolt\per\milli\per\meter}, \\
    n &= \SI{4.1327}{\mega\electronvolt}.
\end{align*}

\noindent Daraus lässt sich anhand der Steigung der Energieverlust der Strahlung bestimmen %wie genau?
\begin{equation*}
    - \left( \frac{dE}{dx} \right)_2 = - \SI{0.1117(26)}{\mega\electronvolt\per\milli\per\meter}.
\end{equation*}

\noindent Die Energie, die sich aus der Steigung und dem y-Abschnitt ergibt, ist  

\begin{equation*}
    E_\text{\alpha} (R_{m,2}) = m \cdot R_{m,2} + n = \SI{5.33(20)}{\MeV}.
\end{equation*}

\subsection{Untersuchung der Statistik des radioaktiven Zerfalls}

Die Anzahl der Pulse, die jeweils in $\SI{10}{\second}$ gemessen wurde, sind in Tab. \ref{tabc} eingetragen.

\begin{table}\caption{Der magnetische Fluss $B$ des gemessenen Magnetfelds gegen den Strom $I$ des erzeugenden Magnetfelds, Neukurve.}
\label{tabc}
\centering
\sisetup{round-mode = places, round-precision=1, round-integer-to-decimal=true}
\begin{tabular}{S[]S[]} 
\toprule
{$B$/ \si{\milli\tesla}} & {$I$/ \si{\ampere}}\\
\midrule
0.0 & 0.0\\
111.19999999999999 & 1.0\\
273.5 & 2.0\\
397.8 & 3.0\\
479.9 & 4.0\\
537.9000000000001 & 5.0\\
585.0999999999999 & 6.0\\
621.8000000000001 & 7.0\\
653.1 & 8.0\\
679.9 & 9.0\\
704.3000000000001 & 10.0\\
\bottomrule
\end{tabular}\end{table}

%Zerfallsraten in Histogramm
\noindent Die Zerfallsraten sind in Abb. \ref{fig:histogramm1} und Abb. \ref{fig:histogramm2} in einem Histogramm aufgetragen. Außerdem ist eine Gauß- und eine Poissonverteilung nach Gleichung \eqref{eqn:gauss} und \eqref{eqn:poisson} aufgetragen. Bei der Erzeugung der beiden Verteilungen wurde ein Seed von 42 benutzt, um die Auswertung deterministisch zu machen. 
\begin{figure}
    \centering
    \includegraphics[width=15cm, height=9cm]{build/plotf.pdf}
    \caption{Die Daten aus Tab. \ref{tabc} werden skaliert und eine Gaussverteilung und die Daten werden histogrammiert.}
    \label{fig:histogramm1}
\end{figure}

\begin{figure}
    \centering
    \includegraphics[width=15cm, height=9cm]{build/plote.pdf}
    \caption{Die Daten aus Tab. \ref{tabc} werden skaliert und eine Poissonverteilung und die Daten werden histogrammiert.}
    \label{fig:histogramm2}
\end{figure}
%Mittelwert und Varianz
\noindent Aus den gemessenen Zählraten lassen sich der  Mittelwert $\mu$ und die Varianz $\sigma^2$ bestimmen: %Gleichungen erwähnen, Varianz ergänzen?
\begin{align*}
    \mu &= \num{4541.6} \\
    \sigma^2 &= \num{25065.36}.
\end{align*}

\noindent Die Werte werden skaliert indem der niedrigste gemessene Wert subtrahiert wird und sie durch \num{100} geteilt werden. 

\noindent Der neue Mittelwert $\mu_\text{Skaliert}$ und die Varianz $\sigma^2_\text{Skaliert}$ bestimmen: %Gleichungen erwähnen, Varianz ergänzen?
\begin{align*}
    \mu_\text{Skaliert} &= \num{3.556} \\
    \sigma^2_\text{Skaliert} &= \num{2.507}.
\end{align*}

