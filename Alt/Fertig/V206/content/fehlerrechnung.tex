\section{Fehlerrechnung}
%Mittelwert, Standardabweichung, relative Fehler Formeln:
Der Mittelwert einer Stichprobe von $N$ Werten wird durch
\begin{equation*}
    \overline{x} = \frac{1}{N} \sum_{i=1}^N x_i
    \label{eqn:mittelwert}
\end{equation*}
bestimmt.
\newline
Die Standardabweichung der Stichprobe wird berechnet mit:
\begin{equation*}
    \sigma_x = \sqrt{\frac{1}{N-1} \sum_{i=1}^N (x_i - \overline{x})^2}.
    \label{eqn:standard}
\end{equation*}
\newline
Der realtive Fehler zwischen zwei Werten kann durch
\begin{equation*}
    \frac{a-b}{a}
\end{equation*}
bestimmt werden.
\newline
%c) Fehler der Ableitung
Die Formel der Gauß'schen Fehlerfortpflanzung für die Ableitung eines Polynoms dritten Grades mit den fehlerbehafteten Größen 
 $a$, $b$, $c$ und $t$ lautet:
\begin{equation}
    \sigma_\text{T'(t)} = \sqrt{9 \sigma_{a}^{2} t^{4} + 4 \sigma_{b}^{2} t^{2} + \sigma_{c}^{2} + \sigma_{t}^{2} \left(6 a t + 2 b\right)^{2}}.
    \label{eqn:pol3}
\end{equation}

%d) Fehler der Güte
\noindent Die Formel der Gauß'schen Fehlerfortpflanzung für die Formel \ref{eqn:güteziffer}, mit der die Güteziffer bestimmt wird, ergibt sich mit den fehlerbehafteten Größen 
$m_1$, $N$, $\frac{\Delta T_1}{\Delta t}$ zu:
\begin{equation}
    \begin{split}
        \sigma_{\nu} = {} &\Biggl(\left(\frac{\Delta T_1}{\Delta t}\right)^{2}\frac{c_{w}^{2} \sigma_{m_{1}}^{2}}{N^{2}}  
    + \frac{\sigma_{\frac{\Delta T_1}{\Delta t}}^{2}}{N^{2}} \left(c_{k} m_{k} + c_{w} m_{1}\right)^{2}  \\
    &+\left(\frac{\Delta T_1}{\Delta t}\right)^{2} \frac{\sigma_{N}^{2}}{N^{4}} \left(c_{k} m_{k} + c_{w} m_{1}\right)^{2}\Biggr)^{\frac{1}{2}}.
    \end{split}
\label{eq:gütefehler}
\end{equation}

%d) Fehler der idealen Güte
\noindent Die Formel der Gauß'schen Fehlerfortpflanzung für die Formel \ref{eqn:ideal}, mit der die ideale Güteziffer bestimmt wird, ergibt sich mit den fehlerbehafteten Größen 
$T_1$ und $T_2$ zu:
\begin{equation}
\sqrt{\frac{T_{1}^{2} \sigma_{T_{2}}^{2}}{\left(T_{1} - T_{2}\right)^{4}} + \sigma_{T_{1}}^{2} \left(- \frac{T_{1}}{\left(T_{1} - T_{2}\right)^{2}} + \frac{1}{T_{1} - T_{2}}\right)^{2}}.
    \label{eq:idealFehler}
\end{equation}

%e) Fehler des Massendurchsatzes (Delta m_1 / Delta t)
\noindent Die Formel der Gauß'schen Fehlerfortpflanzung für die Formel \ref{eqn:massendurchsatz}, mit der der Massendurchsatz bestimmt wird, ergibt sich mit den fehlerbehafteten Größen 
$m_2$, $N$, $\frac{\Delta T_2}{\Delta t}$ zu:
\begin{equation}
    \begin{split}
        \sigma_{\frac{\Delta m}{\Delta t}} = {} & \Biggl(\left(\frac{\Delta T_2}{\Delta t}\right)^{2}\frac{c_{w}^{2} \sigma_{m_{2}}^{2}}{N^{2}} 
    + \frac{\sigma_{\frac{\Delta T_2}{\Delta t}}^{2}}{N^{2}} \left(c_{k} m_{k} + c_{w} m_{2}\right)^{2} \\ 
    &    +\left(\frac{\Delta T_2}{\Delta t}\right)^{2} \frac{\sigma_{N}^{2}}{N^{4}} \left(c_{k} m_{k} + c_{w} m_{2}\Biggr)^{2}\right)^{\frac{1}{2}}.
    \end{split}
    \label{eq:massendurchsatzFehler}
\end{equation}

%Fehler der Dichte
\noindent Die Formel der Gauß'schen Fehlerfortpflanzung für die Formel \ref{eqn:dichte}, mit der die Dichte bestimmt wird, ergibt sich mit den fehlerbehafteten Größen 
$p_a$, $T_2$ zu:
\begin{equation}
 \sigma_{\rho} =\sqrt{\frac{T_{0}^{2} \sigma_{p_{a}}^{2} \rho_{0}^{2}}{T_{a}^{2} p_{0}^{2}} + \frac{T_{0}^{2} \sigma_{T_{a}}^{2} p_{a}^{2} \rho_{0}^{2}}{T_{a}^{4} p_{0}^{2}}}.
    \label{eq:dichteFehler}
\end{equation}

%Fehler der mechanischen Kompressorleistung
\noindent Die Formel der Gauß'schen Fehlerfortpflanzung für die Formel \ref{eqn:nmech}, mit der die mechanische Leistung bestimmt wird, ergibt sich mit den fehlerbehafteten Größen 
$p_a$, $p_b$,$\rho$ und $\frac{\Delta m}{\Delta t}$ zu:
\begin{equation}
    \begin{split}
        \sigma_{N_{mech}}= {} & \Biggl(\frac{\sigma_{\frac{\Delta m}{\Delta t}}^{2} \left(- p_{a} + p_{b} \left(\frac{p_{a}}{p_{b}}\right)^{\frac{1}{\kappa}}\right)^{2}}{\rho_{0}^{2} \left(\kappa - 1\right)^{2}} 
            + \frac{\sigma_{p_{a}}^{2} \left(\frac{\Delta m}{\Delta t}\right)^{2} \left(-1 + \frac{p_{b} \left(\frac{p_{a}}{p_{b}}\right)^{\frac{1}{\kappa}}}{\kappa p_{a}}\right)^{2}}{\rho_{0}^{2} \left(\kappa - 1\right)^{2}} \\ 
                       & + \frac{\sigma_{p_{b}}^{2} \left(\frac{\Delta m}{\Delta t}\right)^{2} \left(\left(\frac{p_{a}}{p_{b}}\right)^{\frac{1}{\kappa}} - \frac{\left(\frac{p_{a}}{p_{b}}\right)^{\frac{1}{\kappa}}}{\kappa}\right)^{2}}{\rho_{0}^{2} \left(\kappa - 1\right)^{2}} \Biggr)^{\frac{1}{2}}.
    \end{split}
    \label{eq:nmechFehler}
\end{equation}