\section{Diskussion}
\label{sec:Diskussion}

\subsection{Kennlinien der Hochvakuumdiode}
Die Kennlinien sehen wie erwartet aus. Es lässt sich bei allen Messungen ein Sättigungsstrom am Ende der Kurve erkennen. 
Bei linearem Anstieg des Beschleunigungsstroms ist eine exponentielle Zunahme zwischen den Kennlinien zu erkennen. %Sagt die Theorie das vorher?

\subsection{Gültigkeitsbereich des Raumladungsgesetzes}
Es wurde ein recht kleiner Gültigkeitsbereich gewählt, da der Exponent in diesem Bereich dem zu erwartenden Wert am 
nächsten gekommen ist. 
Der ermittelte Wert besitzt einen relativen Fehler von \SI{1.63}{\percent} und weicht um \SI{18}{\percent} vom zu 
erwartenden Literaturwert ab. 
Dies ist aber auch im Plot schon zu erkennen, da die Kurve nicht ganz wie in der Theorie erwartet aussieht. 

\subsection{Anlaufstromgebiet der Diode und Bestimmung der Kathodentemperatur}
Die Messung des Anlaufstromgebiets mittels des Gegenstroms, der erzeugt wurde, hat relativ gut funktioniert. 
Die Werte sind trotz zweifachen Wechsels der Skala am Messgerät ziemlich linear in der logarithmischen Auftragung. 
Es ergibt sich eine Temperatur mit einem relativen Fehler von \SI{2.69}{\percent}. Dabei haben wir aber keinen 
Literaturwert, mit dem wir diese Temperatur vergleichen können. %oder?

\subsection{Leistungsbilanz des Heizstromkreises und Abschätzung der Kathodentemperatur}
Die Leistungsbilanz ist auch ähnlich wie erwartet. Die Temperaturen, die sich hier ergeben, liegen zumindest in der 
gleichen Größenordnung, wie die Temperatur bei der vorherigen Messung. Die Abweichung zwischen den Werten bei der 
Bestimmung der Temperatur mittels des Anlaufstromgebiets und der Bestimmung der Leistungsbilanz liegt bei 
\SI{19.6}{\percent}. Mit dem Verlust durch Innenwiderstände und anderen Faktoren, die ein gewisses Maß an Fehlern 
erzeugen, ist der Fehler in Ordnung. %Oder? Keine Ahnung.. Welche anderen Faktoren gibt es noch?

\subsection{Austrittsarbeit für Wolfram}
Der relative Fehler des Mittelwerts der Austrittsarbeit liegt bei \SI{2.11}{\percent}. Die Abweichung zum Literaturwert 
für die Austrittsarbeit eines Wolframdrahtes liegt bei \SI{4.40}{\percent}. 

\subsection{Fazit}
Somit ist die Messung insgesamt als relativ gut zu bezeichnen. Die Ergebnisse liegen alle halbwegs in dem zu 
erwartenden Gebiet und es gibt keine außergewöhnlich großen Abweichungen, die nicht durch die üblichen Fehlerquellen 
zu erklären sind. %Was sind die üblichen Fehlerquellen? -> zB. unsere Inkompetenz
