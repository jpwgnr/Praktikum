\section{Auswertung}
\label{sec:Auswertung}

Für die Auswertung wurde Python und im Speziellen Matplotlib \cite{matplotlib}, SciPy \cite{scipy},
Uncertainties \cite{uncertainties} und NumPy \cite{numpy} verwendet.

\subsection{Einstellung des Selektivverstärkers}
In Tabelle \ref{taba} sind die Frequenzen $\nu$ und die jeweils gemessenen Ausgangspannungen $U_\text{A}$ 
bei einer Eingangsspannung von $U_\text{E} = \SI{100}{\milli\volt}$ eingetragen. 
In Abb. \ref{plota} sind diese Werte gegeneinander aufgetragen.

\begin{table}\caption{Die Anzahl der Impulse, der Startwert auf der Mikrometerschraube und der Endwert auf der Mikrometerschraube.}
\label{taba}
\centering
\sisetup{round-mode = places, round-precision=2, round-integer-to-decimal=true}
\begin{tabular}{S[]S[]S[]} 
\toprule
{Anzahl} & {$d_\text{Start} / \si{\milli\meter}$} & {$d_\text{Start} / \si{\milli\meter}$}\\
\midrule
3001.0 & 6.73 & 2.0\\
3002.0 & 6.73 & 2.0\\
3000.0 & 1.82 & 6.5\\
3000.0 & 6.74 & 2.0\\
3000.0 & 1.83 & 6.5\\
3000.0 & 6.74 & 2.0\\
3001.0 & 1.84 & 6.5\\
3000.0 & 2.83 & 7.5\\
3001.0 & 7.77 & 3.0\\
3002.0 & 2.75 & 7.5\\
\bottomrule
\end{tabular}\end{table}

% Plot einfügen 
\begin{figure}
    \centering
    \includegraphics[width=15cm, height=8cm]{build/plota2.pdf} %
    \caption{Die Filterkurve des Selektivverstärkers. Das Spannungsverhältnis $\frac{U_\text{A}}{U_\text{E}}$
    ist gegen die Frequenz $\nu$ aufgetragen. Es sind die Daten und ein Fit eingetragen.}
    \label{plota}
\end{figure}

\noindent Der Fit wurde mit Gleichung \eqref{cauchyverteilung} gefittet. Die Werte, die sich daraus ergeben, sind
\begin{align*} 
  a &= \SI{1.06(3)}{\kilo\hertz} \\
  s &= \SI{0.36(2)}{\kilo\hertz} \\
  t &= \SI{35.13(1)}{\kilo\hertz}. 
\end{align*}


\noindent Aus der Tabelle lassen sich der Wert $\nu_0$ und die beiden Werte $\nu_\text{-}$ und $\nu_\text{+}$ ungefähr ablesen.
Da das Maximum bei $\SI{99}{\milli\volt}$ liegt, liegen die anderen beiden Frequenzen auf der Höhe von $\SI{70.004}{\milli\volt}$.
Die Werte ergeben sich zu 
\begin{align*} 
 \nu_0 &= \SI{35.1}{\kilo\hertz} \\
 \nu_{-} &= \SI{35.0}{\kilo\hertz} \\
 \nu_{+} &= \SI{35.3}{\kilo\hertz}. 
\end{align*}

\noindent Daraus lässt sich mit Gleichung \eqref{eqn:guete} erkennen, dass die Güte den Wert 
\begin{align*} 
    Q_\text{exp} = \num{100}
\end{align*}
hat. 


\noindent Aus der Werten des Fits und der Gleichung \eqref{cauchyverteilung} lassen sich der Wert $\nu_0$ und die beiden Werte $\nu_\text{-}$ und $\nu_\text{+}$ rechnerisch ermitteln.
Die Werte ergeben sich zu 
\begin{align*} 
 \nu_0 &= \SI{35.1}{\kilo\hertz} \\
 \nu_{-} &= \SI{34.9}{\kilo\hertz} \\
 \nu_{+} &= \SI{35.3}{\kilo\hertz}. 
\end{align*}

\noindent Daraus lässt sich mit Gleichung \eqref{eqn:guete} erkennen, dass die Güte den Wert 
\begin{align*} 
    Q_\text{Fit} = \num{88}
\end{align*}
hat. 

\noindent Der gegebene Wert für $Q$ liegt bei 
\begin{align*} 
    Q = \num{100}.
\end{align*}


\subsection{Theoretische Suszeptibilität} 
Für die verschiedenen Stoffe ergeben sich aufgrund der verschiedenen Elemente und Zusammensetzungen auch verschiedene 
Werte für die Suszeptibilität. 
Die Werte, die zur Berechnung nötig sind, sind die Temperatur $T$, die als Raumtemperatur von $\SI{20}{\celsius}$ angenommen wird, was einem Wert von \SI{293.15}{\kelvin} entspricht. 

\noindent Die Anzahl der Momente N wird mit der Formel \eqref{N} berechnet. Dafür wird die Dichte $\rho$ und die Molmasse $M$ benötigt. Um $Q_{real}$ zu bestimmen, wird auch noch die tatsächliche Masse der jeweiligen Stoffe benötigt. Alle drei Werte sind in Tab. \ref{tab2} eingetragen.
Die Dichten und Massen der unteren drei Elemente sind der Anleitung \cite{V606} sowie der Beschriftung auf den Proben entnommen.
Die Masse des ersten Elements ist ebenfalls der Probe entnommen. Als Dichte wird die Dichte von Praseodym \cite{Dichte} angenommen. In der Tabelle wird also nur der Stoff $\text{Pr}_2$ betrachtet und nicht $\text{C}_6 \text{O}_{12} \text{Pr}_2$. Somit wird auch im Anschluss immer mit dem Dichtewert von $\text{Pr}_2$ und nicht $\text{C}_6 \text{O}_{12} \text{Pr}_2$ gerechnet, da dieser nicht gegeben ist.

\begin{table}\caption{Die Spannung, die Stromstärke, die Anzahl der Impulse, die transportierte Ladungsmenge und die transporte Ladungsmenge in Einheiten der Elementarladung.}
\label{tab1}
\centering
\sisetup{round-mode = places, round-precision=2, round-integer-to-decimal=true}
\begin{tabular}{S[]S[] S[]@{${}\pm{}$}S[] S[]@{${}\pm{}$} S[] S[]@{${}\pm{}$} S[]} 
\toprule
{U / \si{\volt}} & {I / \si{\ampere}} & \multicolumn{2}{c}{N/second} &  \multicolumn{2}{c}{$\Delta Q / \si{\coulomb}$} &  \multicolumn{2}{c}{$\Delta Q \si{\elementarycharge}$}\\
\midrule
320.0 & 0.1     & 86.91 & 0.07 &  8.975  &  0.007  & 5.602   &  0.005e+19\\
400.0 & 0.2     & 90.92 & 0.07 & 17.157  &  0.014  & 1.0709  &  0.0009e+20\\
480.0 & 0.3     & 93.35 & 0.07 & 25.068  &  0.020  & 1.5646  &  0.0012e+20\\
540.0 & 0.35    & 94.62 & 0.07 & 28.851  &  0.023  & 1.8008  &  0.0014e+20\\
560.0 & 0.4     & 92.83 & 0.07 & 33.610  &  0.027  & 2.0977  &  0.0017e+20\\
600.0 & 0.45    & 95.03 & 0.07 & 36.935  &  0.029  & 2.3053  &  0.0018e+20\\
640.0 & 0.5     & 95.41 & 0.08 & 40.877  &  0.032  & 2.5514  &  0.0020e+20\\
660.0 & 0.55    & 96.21 & 0.08 & 44.591  &  0.035  & 2.7832  &  0.0022e+20\\
680.0 & 0.6     & 97.38 & 0.08 & 48.06   &  0.04   & 2.9997  &  0.0023e+20\\
\bottomrule
\end{tabular}\end{table}

\noindent Für den Stoff $\text{Nd}_2 \text{O}_3$ ergibt sich die konkrete Rechnung des Landé-Faktors $g_J$ folgendermaßen: 

\begin{align*}
    g_J &= \frac{3 J (J+1) + S (S+1) - L (L+1)}{2 J (J+1)} \\
        &= 0,73.
\end{align*} 

\noindent Der Wert für $S$ ergibt sich, da wir \num{3} Elektronen auf der $4f$ Schale haben, die sich nach der ersten Hundschen Regel erstmal alle parallel ausrichten und sich somit zu einem Spin von $S = \frac{1}{2}+ \frac{1}{2} + \frac{1}{2}$ ergeben. Nach der zweiten Regel dürfen sie nicht den gleichen Drehimpuls $l$ besitzen. So ergibt sich, dass $L = 3 + 2 + 1$ ist. 

\noindent Die anderen drei Werte lassen sich analog ermitteln. 
In Tab. \ref{tab1} befinden sich die Werte für $L$, $S$ und $J$. Daneben stehen jeweils die Werte, die sich für 
$g_J$ ergeben. 

\begin{table}\caption{Erste Messung.}
\label{tab1}
\centering
\sisetup{round-mode = places, round-precision=1, round-integer-to-decimal=true}
\begin{tabular}{S[]S[]S[]} 
\toprule
{$g / \si{\centi\meter}$} & {$b / \si{\centi\meter}$} & {$B / \si{\centi\meter}$}\\
\midrule
12.700000000000003 & 38.9 & 8.4\\
13.700000000000003 & 32.2 & 6.5\\
14.700000000000003 & 28.200000000000003 & 5.3\\
15.700000000000003 & 25.299999999999997 & 4.4\\
16.700000000000003 & 22.5 & 3.7\\
17.700000000000003 & 21.299999999999997 & 3.3\\
18.700000000000003 & 19.799999999999997 & 3.0\\
19.700000000000003 & 18.5 & 2.6\\
20.700000000000003 & 17.799999999999997 & 2.4\\
21.700000000000003 & 17.400000000000006 & 2.2\\
\bottomrule
\end{tabular}\end{table}

\noindent Die Werte für die Suszeptibilität, die mittels Gleichung \eqref{eqn:chitheo} berechnet werden können, ergeben sich für die jeweiligen Stoffe 
zu folgenden Werten 

\begin{align*}
   \chi_{\text{C}_6 \text{O}_{12} \text{Pr}_2} &= \num{1.6e-3}\\
   \chi_{\text{Gd}_2 \text{O}_3} &= \num{1.2e-2}\\
   \chi_{\text{Nd}_2 \text{O}_3} &= \num{2.8e-3}\\
   \chi_{\text{Dy}_2 \text{O}_3} &= \num{2.8e-2}.
\end{align*}

\subsection{Suszeptibilität mittels Spannungsverhältnis}
In Tab. \ref{tab3} sind die Spannungsdifferenzen vor und nach dem Einlegen der Probe und daneben die Widerstandsdifferenzen beim gleichen Vorgang für die verschiedenen Elemente aufgelistet.

\begin{table}\caption{Die Zeit des Durchschallungsverfahrens gegen die Länge des Zylinders.}
\label{tab3}
\centering
\sisetup{round-mode = places, round-precision=2, round-integer-to-decimal=true}
\begin{tabular}{S[]S[]} 
\toprule
{t/ \si{\second}} & {l/ \si{\meter}}\\
\midrule
8.95e-05 & 0.1208\\
7.8e-05 & 0.1023\\
5.93e-05 & 0.0805\\
3.08e-05 & 0.0404\\
2.47e-05 & 0.0311\\
\bottomrule
\end{tabular}\end{table}
\noindent Die Werte für die Suszeptibilität, die mittels Gleichung \eqref{eqn:chiexp1} berechnet werden können, ergeben sich für die jeweiligen Stoffe 
zu folgenden Werten 

\begin{align*} 
   \chi_{\text{C}_6 \text{O}_{12} \text{Pr}_2} &= \num{8(4)e-5}\\ %ändert sich noch
   \chi_{\text{Gd}_2 \text{O}_3} &= \num{268(2)e-5}\\
   \chi_{\text{Nd}_2 \text{O}_3} &= \num{15(4)e-5}\\
   \chi_{\text{Dy}_2 \text{O}_3} &= \num{703(9)e-5}.
\end{align*}


\subsection{Suszeptibilität mittels Widerstandsverhältnis}
Mit Gleichung \eqref{eqn:chiexp2} und den Werten aus Tab. \ref{tab3}
wird die Suszeptibilität der Proben bestimmt.
Es ergeben sich die Werte

\begin{align*} 
   \chi_{\text{C}_6 \text{O}_{12} \text{Pr}_2} &= \num{120(50)e-5}\\
   \chi_{\text{Gd}_2 \text{O}_3} &= \num{980(17)e-5}\\
   \chi_{\text{Nd}_2 \text{O}_3} &= \num{190(18)e-5}\\
   \chi_{\text{Dy}_2 \text{O}_3} &= \num{2302(20)e-5}.
\end{align*}
