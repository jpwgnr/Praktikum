\begin{table}\caption{Die Masse der Probe, die Dichte des Probenmaterials, die Molmasse und die reale Querschnittsfläche. Für den ersten Stoff wird die Dichte von Praseodym angenommen. Für die anderen Stoffe ist die Dichte in der Anleitung gegeben. Die Querschnittsfläche ist $F = \SI{86.6}{\milli\meter\squared}$.}
\label{tab2}
\centering
\sisetup{round-mode = places, round-precision=2, round-integer-to-decimal=true}
\begin{tabular}{l S S S S} 
\toprule
{Stoffe} & {$m$ / \si{\gram}} & {$\rho_\text{W}$ / \si[per-mode=fraction]{\kilo\gram\per\cubic\meter}} & {$M$ / \si[per-mode=fraction]{\gram\per\mol}} & {$Q_\text{real} / \si{\milli\meter\squared}$}\\
\midrule
$\text{Pr}_2$  & 7.87 & 978.5542204165753 &   & 7.596525096525096\\
$\text{Gd}_2 \text{O}_3$         & 14.08 & 6400.0 & 373.0             & 13.75\\
$\text{Nd}_2 \text{O}_3$         & 9.0 & 7240.0 & 362.5               & 7.7693370165745845\\
$\text{Dy}_2 \text{O}_3$         & 14.38 & 7800.0 & 336.48            & 11.522435897435898\\
\bottomrule
\end{tabular}\end{table}
