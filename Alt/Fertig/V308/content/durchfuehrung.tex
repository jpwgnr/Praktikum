\section{Durchführung}
\label{sec:Durchfuehrung}

\subsection{Magnetfelder von langen und kurzen Spulen}
Im ersten Teil des Versuchs wird eine lange Spule an ein Netzgerät angeschlossen. 
Anschließend wird die Stromstärke eingestellt.
Die Werte werden mittels einer longitudinalen Hall-Sonde innerhalb und außerhalb der
Spule gemessen. Danach wird dieselbe Messung mit einer kurzen Spule 
wiederholt. 

\subsection{Magnetfelder von Spulenpaaren} %Andere Überschrift? Die alte Überschrift war ja einfach falsch haha. 
Es wird das Magnetfeld eines in Reihe geschalteten Spulenpaares gemessen. Dabei wird
der Abstand der Spulen zunächst so gewählt, dass dieser den Radien der Spule entspricht.
Es handelt sich also um ein Helmholtz-Spulenpaar. Es wird eine Stromstärke von
$I= \SI{4}{\ampere}$ eingestellt. Das Magnetfeld wird mittels einer 
transversalen Hall-Sonde innerhalb und außerhalb des Spulenpaares an verschiedenen Positionen auf der $x$-Achse gemessen.
\newline
Anschließend wird der Abstand der Spulen auf den Durchmesser der Spulen erhöht.
Die Stromstärke wird zuerst auf $I= \SI{4}{\ampere}$ eingestellt.
Die magnetische Flussdichte wird mit der transversalen Hall-Sonde innerhalb und
außerhalb des Spulenpaares gemessen.
\newline
Die gleiche Messung wird mit einer Stromstärke von
$I = \SI{3}{\ampere}$ wiederholt.

\subsection {Hysteresekurve einer Ringspule} 
Die Ringspule hat einen Luftspalt. Mit einer transversalen Hall-Sonde wird die 
magnetische Flussdichte der Ringspule in Abhängigkeit vom Spulenstrom in dem Luftspalt gemessen.
Die Stromstärke wird von $\num{0}$ auf $\SI{10}{\ampere}$ erhöht.%nicht besser Amperé ausschreiben? Sieht irgendwie hübscher aus, find ich. 

\noindent Anschließend wird $I$ nach und nach auf $\SI{-10}{\ampere}$ verringert. Danach wird die Stromstärke
von $\num{-10}$ auf $\SI{10}{\ampere}$ erhöht.

% Aus den bestimmten 
% Daten lässt sich anschließend eine Hysteresekurve ermitteln. Aus der graphischen 
% Darstellung lassen sich verschiedene Faktoren, wie Sättigungsmagnetisierung, die 
% Remanenz und die Koerzitivkraft ablesen.

%Das gehört eher in die Auswertung.
