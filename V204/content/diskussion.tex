\section{Diskussion}
\label{sec:Diskussion}

\subsection{Statische Methode}
%Temperaturverläufe von T1, T4, T5, T8 vergleichen
In den Temperaturverläufen $T_1/T_4$ und $T_5/T_8$ sind einige Merkmale zu erkennen. 
So sieht man, dass in der Grafik $T_1/T_4$ die Kurven ...<++>
In Grafik $T_5//T_8$ ist zu erkennen, dass ...<++>
%Welcher hat die beste Wärmeleitung?
Daher ist daraus zu schließen, dass das Material <++> die beste Wärmeleitfähigkeit besitzt. 
Im Vergleich zu <++> und <++> ist zu erkennen, dass das Metall am schnellsten heiß wurde. Da die geometrischen 
Bedingungen bei allen drei Materialien dieselben waren, ist klar zu erkennnen, dass es eine Material- und keine
Geometrieeigenschaft ist, die die Wärmeleitfähigkeit bestimmt. 
%Wärmestrom Meßzeiten
Die einzelnen Wärmeströme haben einen relativen Fehler, der sich zwischen \SI{<++>}{\percent} und \SI{<++>}{\percent} befindet. 
Zu erkennen ist, dass der Wärmestrom am Anfang noch groß ist und mit der Zeit dann abnimmt. Daran lässt sich erkennen, dass die Temperatur 
in einer <++>"e-Funktion" zunimmt.  
%Graphiken T7-T8 und T2-T1 vergleichen
Aus den Differenzen $T_7$ und $T_8$ ergibt sich im Gegensatz zu der Differenz aus $T_2-T_1$, dass ...


\subsection{Dynamische Methode}
%Wärmeleitfähigkeit kappa 
Für die Wärmeleitfähigkeit des Messings ergibt sich ein relativer Fehler von \SI{<++>}{\percent}. Für Aluminium liegt der relative 
Fehler bei \SI{<++>}{\percent}. Für Edelstahl liegt der Fehler bei \SI{<++>}{\percent}. 

\noindent Der experimentelle Werte von $\kappa_\text{Messing}$ entspricht \SI{<++>}{\percent} des Literaturwertes. 
Für $\kappa_\text{Aluminium}$ ergibt sich ein prozentualer Wert von \SI{<++>}{\percent} und für $\kappa_\text{Edelstahl}$ ergibt sich ein Wert 
von \SI{<++>}{\percent}. 
Somit liegt die relative Abweichung in einem Bereich von \SI{<++>}{\percent} bis \SI{<++>}{\percent}, was als <++> einzuschätzen ist. 



