\section{Auswertung}
\label{sec:Auswertung}
Der Abstand der Thermoelemente der jeweiligen Stäbe beträgt
\begin{equation*}
    x = \SI{3}{\centi\meter}.
\end{equation*}
Die Werte für die Dichte \cite{V204}, die spezifische Wärme \cite{V204} und 
die Wärmeleitfähigkeit \cite{wiki} von Messing sind: 
\begin{align*}
    \rho &= \SI{8520}{\kilo\gram\per\cubic\meter} \\
    c &= \SI{385}{\joule\per\kilo\gram\per\kelvin} \\
    \kappa &= \SI{120}{\watt\per\meter\per\kelvin}.
\end{align*}
Die Werte für Aluminium \cite{V204}, \cite{wiki} sind:
\begin{align*}
    \rho &= \SI{2800}{\kilo\gram\per\cubic\meter} \\
    c &= \SI{830}{\joule\per\kilo\gram\per\kelvin} \\
    \kappa &= \SI{236}{\watt\per\meter\per\kelvin}.
\end{align*}
Die Werte für Edelstahl \cite{V204}, \cite{edelstahl} sind:
\begin{align*}
    \rho &= \SI{8000}{\kilo\gram\per\cubic\meter} \\
    c &= \SI{400}{\joule\per\kilo\gram\per\kelvin} \\
    \kappa &= \SI{21}{\watt\per\meter\per\kelvin}.
\end{align*}

\subsection{Statische Methode}
Die Temperaturverläufe der fernen Thermoelemente
befinden sich im Anhang. %Seiten erwähnen
\newline
Der Wärmestrom für die fernen Thermoelemente für fünf Zeiten kann mittels
Gleichung \ref{eqn:dQ} bestimmt werden. %Gleichung noch umstellen?
%Für den breiten Messingstab(T1):
%Für den schmalen Messingstab(T4):
%Für den Aluminiumstab(T5):
%Für den Edelstahlstab(T8):
\newline
Die Temperaturdifferenz der beiden Thermoelemente des Edelstahlstabs
ist graphisch dargestellt und befindet sich im Anhang. %Seite erwähnen
Für den breiten Messingstab ist die Temperaturdifferenz der beiden
Thermoelemente ebenfalls graphisch dargestellt. %Seite erwähnen

\subsection{Dynamische Methode}
\subsubsection{Breiter Messingstab}
\ref{sec:messing}
Die Temperaturverläufe für den breiten Messingstab (Thermoelemente $T1$ und $T2$)
sind graphisch dargestellt. %Seite erwähnen
\newline
Durch Abmessung der Amplituden $A1$ und $A2$ und der Phasendifferenzen $\Delta t$ 
und Mittelung dieser, ergeben sich folgende Werte:
\begin{align*}
    A_{1} &= \SI{}{} \\
    A_{2} &= \SI{}{} \\
    \Delta t &= \SI{}{}.
\end{align*}
Aus diesen Werten lässt sich mittels Gleichung \eqref{eqn:Wärme} die Wärmeleitfähigkeit
bestimmen. Diese ergibt sich zu
\begin{equation*}
    \kappa = \SI{}{}.
\end{equation*}

\subsubsection{Aluminiumstab}
Die Temperaturverläufe für die Thermoelemente des Aluminiumstabes befinden sich im Anhang. %Seite erwähnen
Auf die selbe Weise wie in \ref{sec:messing} beschrieben lassen sich die Werte
für den Aluminiumstab bestimmen:
\begin{align*}
    A_{5} &= \SI{}{} \\
    A_{6} &= \SI{}{} \\
    \Delta t &= \SI{}{}.
\end{align*}
Damit ergibt sich mit \eqref{eqn:Wärme} eine Wärmeleitfähigkeit von
\begin{equation*}
    \kappa = \SI{}{}.
\end{equation*}

\subsubsection{Edelstahlstab}
Die Temperaturverläufe für die Thermoelemente des Edelstahlstabes befinden sich im Anhang. %Seite erwähnen
Die Amplituden und die Phasendifferenz des Edelstahlstabes ergeben sich zu
\begin{align*}
    A_{7} &= \SI{}{} \\
    A_{8} &= \SI{}{} \\
    \Delta t &= \SI{}{}.
\end{align*}
Die Wärmeleitfähigkeit ist mit Gleichung \eqref{eqn_Wärme}
\begin{equation*}
    \kappa = \SI{}{}.
\end{equation*}