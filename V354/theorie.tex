\section{Theorie}
\label{sec:Theorie}
\cite{V354}

% Amplitude einer gedämpften Schwingung mit Dämpfungswiderstand 
Die Spannung lässt sich im allgemeinen als Differentialgleichung darstellen. Nach dem 2. Kirchhoffschen Gesetz gilt für die Spannung in einer Masche, also einem Schaltkreis, dass die Summe aller Spannungen gleich Null ergeben muss, also 
\begin{align*}
U_{R}(t)+U_{C}(t)+U_{L}(t) &= 0 
\end{align*}
, wobei U_{R}(t) die zeitlich veränderliche Spannung an dem Widerstand, U_{C}(t) die Spannung am Kondensator und U_{L}(t) die Spannung an der Spule ist, also die induktive Spannung. 
\begin{equation}
I(t)= A_{0}* e^{-2\pi\mue t} cos(2\pi \nue t +\eta)
\end{equation}
mit $2\pi\mue = R/2L$. 

%R(aperiodischer Grenzfall) Formel mit Amplitude, Schwingungsdauer und was sonst aus a gegeben ist 
Wenn die Spannung ohne Überschwingung an schnellsten gegen null geht (gestrichelte Linie Abb. \ref{Abb.1}), dann heißt dieser Fall aperiodischer Grenzfall. Dieser kann durch die Gleichung 
\begin{equation} 
I(t)=A e^{-\frac{R}{2L}*t}= A e^{-\frac{t}{\sqrt{LC}}}
\end{equation}
.

%Frequenz zu Kondensatorspannung Verbindung

Die Abhängigkeit zwischen Frequenz und Kondensatorspannung lässst sich mittels 

\begin{equation}
U_{C}(\omega)= \frac{U_{0}}{\sqrt{(1-LC\omega^{2})^{2} + (\omega R C)^{2}}}
\end{equation}
% Frequenz zu Erreger und Kondensatorspannung, bzw. die jeweiligen Spannung in ihrer Verbindung zu einander
Mit 
\begin{equation}
U_{C, max} = \frac{1}{\omega_{0}RC} U_{0} = \frac{1}{R} \sqrt{\frac{L}{C}} U_{0}
\end{equation}
ergibt sich eine Abhängigkeit zwischen der Erregerspannung $U_0$ und der Kondensatorspannung U_C. 
% Frequenz zu Scheinwiderstand entweder mit R(klein) oder Impedanzmeter 

Die Impedanz $Z$ kann in einem Schwingkreis über die Formel 

\begin{equation}
\abs{Z} = \sqrt{(R^2+(\omega L- \frac{1}{\omega C}))}
