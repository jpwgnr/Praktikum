\section{Diskussion}
\label{sec:Diskussion}

Im ersten Aufgabenteil wurde der Wert für den effektiven Widerstand bestimmt. Dieser liegt bei $34,2 \pm 1,5$\si{\ohm} was somit einer relativen Abweichung von \SI{28.9}{\percent} zum Literaturwert \SI{48.1(1)}{\ohm} entspricht. Die Messung ist also nicht als besonders exakt zu bewerten. Dasselbe gilt auch für die Periodendauer $T_{exp}$. Der bestimmte Wert und der Literaturwert haben eine Abweichung von \SI{40.34}{\percent}.

Auch die Messung des Widerstands, bei dem der aperiodische Grenzfall zum ersten Mal auftritt, hat eine ähnliche starke Abweichung. Der relative Fehler liegt bei \SI{20.27}{\percent}.

Der relative Fehler der Resonanzüberhöhung q, liegt bei \SI{20.97}{\percent}. Dies passt auch in die Größenordnung der vorherigen Abweichungen. Grund hierfür könnte sein, dass mit einem konstanten Wert für $U_0$ gerechnet wurde, obwohl dieser bei q niedriger geworden st. Im Gegensatz dazu ist der Wert $\omega_{res}$ im Experiment ziemlich nah an dem theoretisch berechneten werd dran. Die Abweichung beträgt hierbei lediglich \SI{0.267}{\percent}. Die Differenz der Werte $\omega_{-}$ und $omega_{+}$ hat zur Differenz der theoretisch berechneten Werte $/omega_{1}$ und $\omega_{2}$ eine Abweichung von \SI{24.66}{\percent}. Dieser Wert liegt wieder in der Größenordnung der vorherigen Fehlerwerte. 

Somit ist die Messung insgesamt als relativ exakt zu bezeichnen. Es gab lediglich einen durchgehenden Fehler in der Höhe von ca. $20$ bis $30 \si{\percent}$.  
