\section{Auswertung}
\label{sec:Auswertung}
%Tabellen:
%taba: Komma nach "Zylinder" weg
%Ist tab2.tex oder tab3.tex das Durchschallungsverf.?
%tabd: andere caption?

% Tabelle mit Länge, Spannung 1, Zeit 1, Spannung 2 und Zeit 2 
%Grafik mit diesen Werten, also Screenshots 
\subsection{Vermessung der Acrylzylinder}
Die Längen der fünf Acrylzylinder sowie die Laufzeiten und Amplituden
der ersten beiden reflektierten Pulse, %Spannungsamplituden?
die mittels A-Scan bei dem Impuls-Echo-Verfahren aufgenommen werden, %eigentlich "wurden"
befinden sich in Tab. \ref{taba}.
Die Bildschirmaufnahmen der A-Scans sind in der Reihenfolge
vom längsten bis zum kürzesten Zylinder eingefügt.

\begin{table}\caption{Die Anzahl der Impulse, der Startwert auf der Mikrometerschraube und der Endwert auf der Mikrometerschraube.}
\label{taba}
\centering
\sisetup{round-mode = places, round-precision=2, round-integer-to-decimal=true}
\begin{tabular}{S[]S[]S[]} 
\toprule
{Anzahl} & {$d_\text{Start} / \si{\milli\meter}$} & {$d_\text{Start} / \si{\milli\meter}$}\\
\midrule
3001.0 & 6.73 & 2.0\\
3002.0 & 6.73 & 2.0\\
3000.0 & 1.82 & 6.5\\
3000.0 & 6.74 & 2.0\\
3000.0 & 1.83 & 6.5\\
3000.0 & 6.74 & 2.0\\
3001.0 & 1.84 & 6.5\\
3000.0 & 2.83 & 7.5\\
3001.0 & 7.77 & 3.0\\
3002.0 & 2.75 & 7.5\\
\bottomrule
\end{tabular}\end{table}

\begin{figure}
    \centering
    \includegraphics[width=15cm, height=5cm]{build/Messung1.1.png}
    \caption{Bildschirmaufnahme des A-Scans des ersten Zylinders mit dem Impuls-Echo-Verfahren.}
    \label{m1.1}
\end{figure}

\begin{figure}
    \centering
    \includegraphics[width=15cm, height=5cm]{build/Messung1.2.png}
    \caption{Bildschirmaufnahme des A-Scans des zweiten Zylinders mit dem Impuls-Echo-Verfahren.}
    \label{m1.2}
\end{figure}

\begin{figure}
    \centering
    \includegraphics[width=15cm, height=5cm]{build/Messung1.3.png}
    \caption{Bildschirmaufnahme des A-Scans des dritten Zylinders mit dem Impuls-Echo-Verfahren.}
    \label{m1.3}
\end{figure}

\begin{figure}
    \centering
    \includegraphics[width=15cm, height=5cm]{build/Messung1.4.png}
    \caption{Bildschirmaufnahme des A-Scans des vierten Zylinders mit dem Impuls-Echo-Verfahren.}
    \label{m1.4}
\end{figure}

\begin{figure}
    \centering
    \includegraphics[width=15cm, height=5cm]{build/Messung1.5.png}
    \caption{Bildschirmaufnahme des A-Scans des fünften Zylinders mit dem Impuls-Echo-Verfahren.}
    \label{m1.5}
\end{figure}

%Ergebnisse mit Formel c = 2* Länge durch (Zeit 2 - Zeit 1), fünf Geschwindigkeiten c1 bis c5 
%Theoriewert für c angeben
\noindent Mit der Gleichung \eqref{eqn:c} lassen sich die Schallgeschwindigkeiten
für die Zylinder berechnen:
\begin{align*}
    c_1 &= \SI{2742}{\meter\per\second} \\
    c_2 &= \SI{2692}{\meter\per\second} \\
    c_3 &= \SI{2720}{\meter\per\second} \\
    c_4 &= \SI{2721}{\meter\per\second} \\
    c_5 &= \SI{2670}{\meter\per\second}.
\end{align*}
Der theoretische Wert für die Schallgeschwindigkeit in Acryl beträgt
\begin{equation*}
    c_{\text{theo}} = \SI{2730}{\meter\per\second}. %Quelle: olympus-ims
\end{equation*}

%Dämpfungskoeffizient aus denselben Daten, tab1,  Ausgleichsrechnung mit y= - ln(U2/U1) und x als Länge l. Ergebnis alpha und Plot dazu. 
%Theoriewert für alpha angeben, falls wir den finden
\subsection{Bestimmung der Dämpfung mittels Impuls-Echo-Verfahren}
Für die Bestimmung des materialspezifischen Schwächungskoeffizienten werden
die Amplituden aus Tab. \ref{taba} verwendet.
Das negative logarithmierte Verhältnis der Spannungen ist in Plot \ref{fig:plot1}
gegen die Länge aufgetragen. Die Werte dazu befinden sich in Tab. \ref{tab1}.
\begin{table}\caption{Erste Messung.}
\label{tab1}
\centering
\sisetup{round-mode = places, round-precision=1, round-integer-to-decimal=true}
\begin{tabular}{S[]S[]S[]} 
\toprule
{$g / \si{\centi\meter}$} & {$b / \si{\centi\meter}$} & {$B / \si{\centi\meter}$}\\
\midrule
12.700000000000003 & 38.9 & 8.4\\
13.700000000000003 & 32.2 & 6.5\\
14.700000000000003 & 28.200000000000003 & 5.3\\
15.700000000000003 & 25.299999999999997 & 4.4\\
16.700000000000003 & 22.5 & 3.7\\
17.700000000000003 & 21.299999999999997 & 3.3\\
18.700000000000003 & 19.799999999999997 & 3.0\\
19.700000000000003 & 18.5 & 2.6\\
20.700000000000003 & 17.799999999999997 & 2.4\\
21.700000000000003 & 17.400000000000006 & 2.2\\
\bottomrule
\end{tabular}\end{table}
\begin{figure}
 \centering
 \includegraphics[width=15cm, height=10cm]{build/plot1.pdf}
 \caption{Das negative logarithmierte Spannungsverhältnis ist gegen die Länge aufgetragen.
 Zu sehen sind die Daten und ein Fit.}
 \label{fig:plot1}
\end{figure}
\noindent Die Steigung entspricht der Dämpfung:
\begin{equation*}
    \alpha = \SI{24.87}{\per\meter}.
\end{equation*}
Der theoretische Wert für den Schwächungskoeffizienten von Acryl ist %Theoriewert?
\begin{equation*}
    \alpha_{\text{theo}} = \SI{}{\per\meter}. 
\end{equation*}
%https://www.tu-chemnitz.de/physik/FPRAK/F-Praktikum/Versuche_alt/V35_Musterprotokoll.pdf
%Polyacryl: 0.017 1/mm

%Ausgleichsrechnung c mit Durchschall- und Impuls-Echo Verfahren. Dazu Tabellen tab b, tab c, tab2, Plot 2, tab 3, Plot 3 und Ergebnisse. 
\subsection{Bestimmung der Schallgeschwindigkeit mittels Durchschallungs-Verfahren}
Die Laufzeiten, die der Schallimpuls jeweils zum Durchlaufen des Zylinders benötigt,
sind in Tab. \ref{tabc} aufgeführt. In Tab. \ref{tab3} ist die Zeit des
Durchschallungs-Verfahrens gegen die Länge des Zylinders aufgetragen.
Der Plot \ref{fig:plot3} stellt diese Auftragung dar.
\begin{table}\caption{Der magnetische Fluss $B$ des gemessenen Magnetfelds gegen den Strom $I$ des erzeugenden Magnetfelds, Neukurve.}
\label{tabc}
\centering
\sisetup{round-mode = places, round-precision=1, round-integer-to-decimal=true}
\begin{tabular}{S[]S[]} 
\toprule
{$B$/ \si{\milli\tesla}} & {$I$/ \si{\ampere}}\\
\midrule
0.0 & 0.0\\
111.19999999999999 & 1.0\\
273.5 & 2.0\\
397.8 & 3.0\\
479.9 & 4.0\\
537.9000000000001 & 5.0\\
585.0999999999999 & 6.0\\
621.8000000000001 & 7.0\\
653.1 & 8.0\\
679.9 & 9.0\\
704.3000000000001 & 10.0\\
\bottomrule
\end{tabular}\end{table}
\begin{table}\caption{Die Zeit des Durchschallungsverfahrens gegen die Länge des Zylinders.}
\label{tab3}
\centering
\sisetup{round-mode = places, round-precision=2, round-integer-to-decimal=true}
\begin{tabular}{S[]S[]} 
\toprule
{t/ \si{\second}} & {l/ \si{\meter}}\\
\midrule
8.95e-05 & 0.1208\\
7.8e-05 & 0.1023\\
5.93e-05 & 0.0805\\
3.08e-05 & 0.0404\\
2.47e-05 & 0.0311\\
\bottomrule
\end{tabular}\end{table}
\begin{figure}
    \centering
    \includegraphics[width=15cm, height=10cm]{build/plot3.pdf}
    \caption{Die Laufzeit der Durchschallung ist gegen die Länge des Zylinders
    aufgetragen. Es sind die Daten und ein Fit zu sehen.}
    \label{fig:plot3}
\end{figure}
\noindent Die Schallgeschwindigkeit ist dabei die Steigung:
\begin{equation*}
    c_{\text{Durchschallung}} = \SI{2723.15}{\meter\per\second}.
\end{equation*}

\subsection{Bestimmung der Schallgeschwindigkeit mittels Impuls-Echo-Verfahren}
Die Laufzeiten, die der Schallimpuls mittels Impuls-Echo-Verfahren benötigt hat,
befinden sich in Tab. \ref{tabb}. In Tab. \ref{tab2} ist die Zeit des Impuls-Echo-Verfahrens
gegen die Länge des Zylinders aufgetragen. Diese Werte sind in Plot \ref{fig:plot2}
gegeneinander aufgetragen.
\begin{table}\caption{Die Frequenzen der Sägezahnspannung.}
\label{tabb}
\centering
\sisetup{round-mode = places, round-precision=2, round-integer-to-decimal=true}
\begin{tabular}{S[]S[]} 
\toprule
{Index} & {$\nu_\text{Sä} / \si{\hertz}$}\\
\midrule
1.0 & 25.02\\
2.0 & 49.95\\
3.0 & 99.99\\
4.0 & 149.97\\
\bottomrule
\end{tabular}\end{table}
\begin{table}\caption{Die Spannung, die Stromstärke, die Anzahl der Impulse, die transportierte Ladungsmenge und die transporte Ladungsmenge in Einheiten der Elementarladung.}
\label{tab1}
\centering
\sisetup{round-mode = places, round-precision=2, round-integer-to-decimal=true}
\begin{tabular}{S[]S[] S[]@{${}\pm{}$}S[] S[]@{${}\pm{}$} S[] S[]@{${}\pm{}$} S[]} 
\toprule
{U / \si{\volt}} & {I / \si{\ampere}} & \multicolumn{2}{c}{N/second} &  \multicolumn{2}{c}{$\Delta Q / \si{\coulomb}$} &  \multicolumn{2}{c}{$\Delta Q \si{\elementarycharge}$}\\
\midrule
320.0 & 0.1     & 86.91 & 0.07 &  8.975  &  0.007  & 5.602   &  0.005e+19\\
400.0 & 0.2     & 90.92 & 0.07 & 17.157  &  0.014  & 1.0709  &  0.0009e+20\\
480.0 & 0.3     & 93.35 & 0.07 & 25.068  &  0.020  & 1.5646  &  0.0012e+20\\
540.0 & 0.35    & 94.62 & 0.07 & 28.851  &  0.023  & 1.8008  &  0.0014e+20\\
560.0 & 0.4     & 92.83 & 0.07 & 33.610  &  0.027  & 2.0977  &  0.0017e+20\\
600.0 & 0.45    & 95.03 & 0.07 & 36.935  &  0.029  & 2.3053  &  0.0018e+20\\
640.0 & 0.5     & 95.41 & 0.08 & 40.877  &  0.032  & 2.5514  &  0.0020e+20\\
660.0 & 0.55    & 96.21 & 0.08 & 44.591  &  0.035  & 2.7832  &  0.0022e+20\\
680.0 & 0.6     & 97.38 & 0.08 & 48.06   &  0.04   & 2.9997  &  0.0023e+20\\
\bottomrule
\end{tabular}\end{table}
\begin{figure}
    \centering
    \includegraphics[width=15cm, height=10cm]{build/plot2.pdf}
    \caption{Die Laufzeit des Impuls-Echo-Verfahrens ist gegen die Länge des Zylinders
    aufgetragen. Es sind die Daten und ein Fit zu sehen.}
    \label{fig:plot2}
\end{figure}
\noindent Die Steigung entspricht der Schallgeschwindigkeit im Acryl:
\begin{equation*}
    c_{\text{Impuls-Echo}} = \SI{2738.45}{\meter\per\second}.
\end{equation*}


%Auge, tab d, Screenshot, Ergebnis mit s= 0.5*c*t 
%Theoriewerte für Augenabstände angeben, falls diese zu finden sind, verschiedene Maßstäbe betrachten
\subsection{Biometrische Untersuchung eines Augenmodells}
Die Laufzeiten des Echos beim Impuls-Echo-Verfahren für das Auge
sind in Tab. \ref{tabd} aufgelistet. Eine Bildschirmaufnahme des
A-Scans ist in Abb. \ref{fig:auge} zu sehen.
\begin{table}\caption{Kreisfrequenz $\omega$ gegen die Phasenverschiebung $\varphi$ der Kondensatorspannung $U_C$ und der Generatorspannungi $U_0$.}
\label{tabd}
\centering
\sisetup{round-mode = places, round-precision=2, round-integer-to-decimal=true}
\begin{tabular}{S[]S[]} 
\toprule
{$\omega\cdot 10^{5}$ /\si[per-mode=fraction]{\per\second}} & {$Phase \varphi$}\\
\midrule
0.5654866776461628 & 0.12440706908215582\\
0.6911503837897545 & 0.11058406140636072\\
0.8168140899333463 & 0.13069025438933538\\
0.9424777960769379 & 0.1696460032938488\\
1.0681415022205296 & 0.16022122533307945\\
1.1938052083641213 & 0.20294688542190062\\
1.319468914507713 & 0.23750440461138836\\
1.4451326206513049 & 0.26012387171723483\\
1.5707963267948966 & 0.34557519189487723\\
1.6964600329384882 & 0.4750088092227767\\
1.8221237390820801 & 0.546637121724624\\
1.8849555921538759 & 0.6785840131753952\\
1.9477874452256716 & 0.818070726994782\\
2.0106192982974673 & 0.9650972631827843\\
2.0734511513692637 & 1.1611326447667876\\
2.1362830044410592 & 1.4099467829310992\\
2.199114857512855 & 1.6713272917097701\\
2.261946710584651 & 1.9000352368911066\\
2.324778563656447 & 2.092300707290802\\
2.3876104167282426 & 2.244353791724548\\
2.450442269800039 & 2.4014334244040376\\
2.5761059759436304 & 2.5761059759436304\\
2.701769682087222 & 2.6477342884454775\\
2.827433388230814 & 2.770884720466197\\
2.9530970943744057 & 2.894035152486917\\
3.078760800517997 & 2.8940351524869175\\
3.204424506661589 & 2.948070546128662\\
3.330088212805181 & 2.9970793915246623\\
3.4557519189487724 & 2.9719466502959446\\
3.581415625092364 & 3.008389125077586\\
3.7070793312359562 & 3.0398050516134836\\
\bottomrule
\end{tabular}\end{table}
\begin{figure}
 \centering
 \includegraphics[width=15cm, height=5cm]{build/Auge.png}
 \caption{Aufnahme des A-Scans des Auges beim Impuls-Echo-Verfahren.}
 \label{fig:auge}
\end{figure}
\noindent Mit Gleichung \eqref{eqn:s} werden die Längen im Auge bestimmt.
Die erste Strecke $s_1$ ist die zwischen der Hornhaut und der Linse.
Die zweite Strecke $s_2$ ist die Dicke der Linse. Die dritte Strecke $s_3$ ist der Abstand
der Retina zur Linse.
Die Schallgeschwindigkeiten sind dabei unterschiedlich.
Zwischen Hornhaut und Linse befindet sich (zumindest wird das angenommen)
Wasser. Zwischen Linse und Retina befindet sich Glaskörperflüssigkeit.
Die Schallgeschwindigkeiten sind:
\begin{align*}
    c_{\text{W}} &= \SI{1485}{\meter\per\second} \\ %QUELLE c_W
    c_{\text{L}} &= \SI{2500}{\meter\per\second} \\ %Quellen: Anleitung
    c_{\text{GK}} &= \SI{1410}{\meter\per\second}.
\end{align*}
Mit den gemessenen Zeiten und den verschiedenen Schallgeschwindigkeiten
ergeben sich folgende Längen:
\begin{align*}
    s_1 &= \SI{3.3}{\milli\meter} \\
    s_2 &= \SI{8.5}{\milli\meter} \\
    s_3 &= \SI{34.8}{\milli\meter}.
\end{align*}
Das Augenmodell ist in einem Maßstab von $1:3$ gebaut.
Die dreifachen theoretischen Längen eines echten Auges sind:
\begin{align*}
    s_{\text{theo,1}} &= \SI{12.4}{\milli\meter} \\
    s_{\text{theo,2}} &= \SI{10.8}{\meter} \\
    s_{\text{theo,3}} &= \SI{45.5}{\milli\meter}. %geometrische-optik.de
\end{align*}
