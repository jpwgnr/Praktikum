\section{Diskussion}
\label{sec:Diskussion}

%Paar Worte zur Grafik 
\subsection{Graphische Auswertung}
Die Aufnahmen der Ultraschallverfahren waren meist relativ exakt, aber vor allem bei der Spannungsamplitude war es schwierig einen konstanten Wert zu messen. Bei dem Auge Sprang die Intensität zum Beispiel die ganze Zeit zwischen verschiedenen Werten hin und her. Die Zeiten waren dafür aber immer recht gut abzulesen. 
%Relativen Fehler der Geschwindigkeiten, falls Fehler auf Länge l, und relative Abweichungen der Geschwindigkeiten zum Theoriewert
\subsection{Messung der Schallgeschwindigkeit}
Die Messung der Schallgeschwindigkeit innerhalb von Acryl hatte eine gewisse Abweichung, je nachdem in welchem Zylinder man diese gemessen hat. Die Abweichung der Werte zum Literaturwert variiert dabei zwischen \SI{0.33}{\percent} und \SI{2.20}{\percent}. 

%Relativen Fehler des Dämpfungskoeffizienten, falls Fehler auf Länge l oder Fehler der Ausgleichsrechnung und relative Abweichungen des Koeffizenten zum Theoriewert
\subection{Dämpfungskoeffizient}
Der Dämpfungskoeffizient besitzt einen relativen Fehler von \SI{16.85}{\percent} und hat eine Abweichung von \SI{}{\percent} zum Literaturwert. 

%Fehler der Ausgleichsrechnung von c mit Durchschall- und Impuls-Echo Verfahren, relative Abweichung zum Theoriewert 
\subsection{Durchschall- und Impuls-Echo-Verfahren} 
Die Geschwindigkeit die sich durch das Impuls-Echo-Verfahren ergibt hat einen relativen Fehler von \SI{0.58}{\percent} und eine Abweichung zum Literaturwert von \SI{0.3}{\percent}. Das Durchschallungsverfahren besitzt einen relativen Fehler von \SI{2.21}{\percent} und weicht um \SI{0.37}{\percent} vom Literaturwert ab.

%Abweichung der Augenwerte zum Theoriewerte angeben, verschiedene Maßstäbe betrachten
\subsection{Die Abstände in dem Augenmodell}
Im Vergleich mit den Abständen der Schichten in einem echten Auge, sind die Werte zumindest im Bereich der gleichen Größenordnung. 
Der Abstand zwischen Hornhaut und Linse weicht vom Literaturwert um \SI{73.4}{\percent} ab, die Breite der Linse weicht um \SI{21.3}{\percent} ab und der Abstand zwischen Linse und Retina weicht um \SI{23.5}{\percent} ab.
%Fazit welches Messverfahren am exaktesten ist in Bezug auf die Literaturwerte und welche Fehlerquellen für die Abweichungen verantwortlich sein könnten
\subsection{Fazit}
Somit scheinen die Werte, die beim Impuls-Echo-Verfahren bestimmt wurden am besten zu sein, da der Fehler und die Abweichung dabei am kleinsten waren. 
Die großen Abweichungen beim Auge können unterschiedliche Ursachen haben. Unter anderem kann das Modell nicht ganz korrekt sein. Dann war aber auch die Messung natürlich nicht ganz exakt, da sich die Werte die ganze Zeit über stark verändert haben und es dadurch schwieriger wurde, sie ordentlich abzulesen. Die andere Schallgeschwindigkeit in der Koppelflüssigkeit wird zum Beispiel völlig außer acht gelassen bei der Messung und die Flüssigkeit ist auch nicht immer gleichmäßig verteilt.

Das Problem ist nur, dass das Durchschallungsverfahren vermutlich nicht richtig funktioniert hat. Denn die Werte, die dabei rauskamen, lagen bei ganz ähnlichen Werten wie die Werte des Impuls-Echo-Verfahrens, was eigentlich keinen Sinn ergibt, da der Schall dabei ja nur die Hälfte des Weges zurücklegt und die Zeiten somit auch nur die Hälfte betragen sollten. 
