\section{Diskussion}
\label{sec:Diskussion}


\subsection{Beugung am ersten Einzelspalt}
Die Ausgleichsrechnung ergibt für den Wert der Amplitude einen Wert, dessen relative Abweichung bei \SI{6.12}{\percent} liegt. Für die Spaltbreite ergibt sich ein Wert, dessen relativer Fehler \SI{3.69}{\percent} beträgt und der um \SI{47.95}{\percent} vom Literaturwert abweicht.
Der gefittete Wert für den Dunkelstrom hat einen relativen Fehler von \SI{9.80}{\percent} und eine relative Abweichung zum gemessenen Wert von \SI{4.38}{\percent}.


\subsection{Beugung am zweiten Einzelspalt}
Die Ausgleichsrechnung ergibt für den Wert der Amplitude einen Wert, dessen relative Abweichung bei \SI{4.85}{\percent} liegt. Für die Spaltbreite ergibt sich ein Wert, dessen relativer Fehler \SI{2.66}{\percent} beträgt und der um \SI{28.32}{\percent} vom Literaturwert abweicht.
Der gefittete Wert für den Dunkelstrom hat einen relativen Fehler von \SI{4.32}{\percent} und eine relative Abweichung zum gemessenen Wert von \SI{1.25}{\percent}.

\subsection{Interferenz am Doppelspalt}
Das Doppelspalt-Experiment hat genau die Ergebnisse ergeben, die zu erwarten gewesen sind. Auf der selben Breite  wie beim Einzelspalt-Experiment, gibt es deutlich mehr Maxima und Minima, aber die Amplitude bleibt ungefähr gleich, sodass das Beugungsmuster was gefittet wurde als Einhüllende des Doppelspaltmusters gedeutet werden kann. Was lediglich überrascht, ist dass der Dunkelstrom bei dieser Ausgleichsrechnung deutlich höher ist als bei den anderen beiden Messungen und somit auch \SI{89.38}{\percent} über dem gemessenen Wert liegt. Der relative Fehler liegt bei \SI{6.60}{\percent}.

\subsection{Allgemeine Probleme bei der Auswertung}
Die Daten wurden mittels Python ausgewertet. Zur Erstellung der Fits wurde das Package scipy \cite{scipy} genutzt. Es scheint aber so, als habe die "curve_fit"-Funktion Schwierigkeiten dabei quadratische trigonometrische Funktionen auszuwerten. Aus dem Grund wurde die Wurzel aus der Funktion und den y-Werten gezogen, anschließend wurde gefittet und am Ende wurden alle Werte wieder quadriert. Das Ergebnis ist damit deutlich besser, als alle vorherigen Versuche. Trotzdem ist die Abweichung speziell beim ersten und dritten Graphen noch relativ groß, was dann auch die Abweichung zum Literaturwert um fast \SI{50}{\percent} erklärt. 
Außerdem wurden die Werte beim Hauptmaximum, also bei 0 Grad herausgenommen um zu vermeiden, dass durch null geteilt wird während des "curve_fit"-Vorgangs. 
Das einzige tatsächliche Problem sind die Werte für den Dunkelstrom. Der Faktor $d$ wurde in der Wurzel-Variante des Terms hinzugefügt. Das bedeutet wenn man die Zahlen einfach quadriert kommt durch die binomische Formel noch eine zusätzliche Komponente hinzu. Da $d$ im Verhältnis zu den Werten im vorderen Teil der Gleichung einige Größenordnungen kleiner ist, wird dieser $2 I d$ wohl keinen großen Einfluss haben, denn vor allem bei der zweiten Messung ist der Wert $d$ sehr nah am gemessenen Wert, wenn man ihn dann quadriert hat. Insofern scheint unsere Rechnung schon okay zu sein, aber nur weil wir davon ausgehen können, dass der zusätzliche Teil keine große Auswirkung auf die Formel hat. 

\subsection{Fazit} 
Insgesamt kann der Versuch und die Messung als relativ exakt betrachtet werden. Nur die Auswertung stellt sich als recht kompliziert heraus, was dazu führt, dass die Ergebnisse nicht so gut sind, wie ursprünglich erwartet.  

\newpage
