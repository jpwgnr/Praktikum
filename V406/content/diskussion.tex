\section{Diskussion}
\label{sec:Diskussion}

\subsection{Beugung am ersten Einzelspalt}
Die Ausgleichsrechnung ergibt für den Wert der Amplitude einen Wert, dessen relative Abweichung bei \SI{6.12}{\percent} liegt. Für die Spaltbreite ergibt sich ein Wert, dessen relativer Fehler \SI{3.69}{\percent} beträgt und der um \SI{47.95}{\percent} vom Literaturwert abweicht.
Der gefittete Wert für den Dunkelstrom hat einen relativen Fehler von \SI{9.80}{\percent} und eine relative Abweichung zum gemessenen Wert von \SI{4.38}{\percent}.


\subsection{Beugung am zweiten Einzelspalt}
Die Ausgleichsrechnung ergibt für den Wert der Amplitude einen Wert, dessen relative Abweichung bei \SI{4.85}{\percent} liegt. Für die Spaltbreite ergibt sich ein Wert, dessen relativer Fehler \SI{2.66}{\percent} beträgt und der um \SI{28.32}{\percent} vom Literaturwert abweicht.
Der gefittete Wert für den Dunkelstrom hat einen relativen Fehler von \SI{4.32}{\percent} und eine relative Abweichung zum gemessenen Wert von \SI{1.25}{\percent}.

\subsection{Interferenz am Doppelspalt}
Das Doppelspalt-Experiment hat genau die Ergebnisse ergeben, die zu erwarten gewesen sind. Auf der selben Breite gibt es deutlich mehr Maxima und Minima, aber die Amplitude bleibt ungefähr gleich, sodass das Beugungsmuster als Einhüllende des Doppelspaltmusters gedeutet werden kann. 

\subsection{Fazit} 
Insgesamt kann der Versuch als relativ exakt betrachtet werden. Zumindest ergibt sich der Fitparameter zu einem entsprechend genauen Wert. 

\newpage
