\section{Diskussion}
\label{sec:Diskussion}

\subsection{Beugung am ersten Einzelspalt}
Die Ausgleichsrechnung ergibt einen Wert, dessen relativer Fehler \SI{<++>}{\percent} beträgt und der um \SI{<++>}{\percent} vom Literaturwert abweicht. 


\subsection{Beugung am zweiten Einzelspalt}
Die Ausgleichsrechnung ergibt einen Wert, dessen relativer Fehler \SI{<++>}{\percent} beträgt und der um \SI{<++>}{\percent} vom Literaturwert abweicht. 

\subsection{Interferenz am Doppelspalt}

Das Doppelspalt Experiment hat genau die Ergebnisse ergeben, die zu erwarten gewesen sind. Auf der selben Breite gibt es deutlich mehr Maxima und Minima, aber die Amplitude bleibt ungefähr gleich, sodass das Beugungsmuster als Einhüllende des Doppelspaltmusters gedeutet werden kann. 

\subsection{Fazit} 

Insgesamt kann der Versuch als relativ exakt betrachtet werden. Zumindest ergibt sich der Fit Parameter zu einem entsprechend genauen Wert. 
