\begin{table}\caption{Der Winkel \varphi gegen die Stromstärke I aufgetragen.}
\label{tab1}
\centering
\sisetup{round-mode = places, round-precision=2, round-integer-to-decimal=true}
\begin{tabular}{S[]S[]} 
\toprule
{$\varphi / \si{\radian}$} & {$I / \si{\nano\ampere}$}\\
\midrule
-0.004132950412107405 & 2.3\\
-0.003953258830669266 & 2.7\\
-0.003773566993935379 & 3.1000000000000005\\
-0.0035938749135094187 & 3.6\\
-0.003414182600995156 & 4.199999999999999\\
-0.00323449006799645 & 4.800000000000001\\
-0.0030547973261172466 & 5.300000000000001\\
-0.0028751043869615716 & 5.9\\
-0.0026954112621335275 & 6.599999999999999\\
-0.002515717963237289 & 7.3\\
-0.0023360245018770993 & 8.399999999999999\\
-0.002156330889657261 & 8.399999999999999\\
-0.0019766371381821396 & 9.400000000000002\\
-0.0017969432590561516 & 10.4\\
-0.0016172492638837637 & 10.4\\
-0.001437555164269488 & 11.4\\
-0.0012578609718178768 & 12.400000000000002\\
-0.0010781666981335184 & 12.400000000000002\\
-0.0008984723548210324 & 13.4\\
-0.0007187779534850654 & 14.4\\
-0.0005390835057302868 & 14.4\\
-0.00035938902316138383 & 14.4\\
-0.000179694517383057 & 15.399999999999999\\
-1.5960079419588954e-17 & 15.399999999999999\\
\bottomrule
\end{tabular}\end{table}