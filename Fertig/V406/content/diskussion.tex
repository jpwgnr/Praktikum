\section{Diskussion}
\label{sec:Diskussion}


\subsection{Beugung am ersten Einzelspalt}
Der negative Wert der Spaltbreite ergibt sich, da die Curve-Fit Funktion für die Amplitude einen negativen Wert ermittelt hat. Dadurch wird dann auch die Spaltbreite negativ. Dieses Ergebnis ist aber als nicht physikalisch zu betrachten, insofern wird einfach im Folgenden angenommen, dass sich die beiden Vorzeichen gegenseitig aufheben. 
Die Ausgleichsrechnung ergibt für die Amplitude einen Wert, dessen relativer Fehler bei \SI{6.12}{\percent} liegt. Für die Spaltbreite ergibt sich ein Wert, dessen relativer Fehler \SI{3.69}{\percent} beträgt und der um \SI{47.95}{\percent} vom Literaturwert abweicht.
Der gefittete Wert für den Off-Strom hat einen relativen Fehler von \SI{9.80}{\percent} und eine relative Abweichung zum gemessenen Wert von \SI{4.38}{\percent}.


\subsection{Beugung am zweiten Einzelspalt}
Die Ausgleichsrechnung ergibt für die Amplitude einen Wert, dessen relativer Fehler bei \SI{4.85}{\percent} liegt. Für die Spaltbreite ergibt sich ein Wert, dessen relativer Fehler \SI{2.66}{\percent} beträgt und der um \SI{28.32}{\percent} vom Literaturwert abweicht.
Der gefittete Wert für den Off-Strom hat einen relativen Fehler von \SI{4.32}{\percent} und eine relative Abweichung zum gemessenen Wert von \SI{1.25}{\percent}.

\subsection{Interferenz am Doppelspalt}
Die Spaltbreite besitzt einen relativen Fehler von \SI{3.7}{\percent}. Die Abweichung zum Literaturwert beträgt \SI{41}{\percent}.  
Die Breite zwischen den Spalten hat einen relativen Fehler von \SI{2.16e-9}{\percent}. Der Off-Strom liegt \SI{89.38}{\percent} über dem gemessenen Wert. Der relative Fehler liegt bei \SI{6.6}{\percent}. Der Fit des Doppelspalts passt vom Bild her nicht sonderlich zu den richtigen Werten. Die ermittelten Werte passen aber, bis auf den Off-Strom, einigermaßen.
\newline
Die durch die Theoriekurve ermittelte Breite hat eine Abweichung von \SI{45}{\percent} zum Literaturwert.
Der Spaltabstand weicht um \SI{140}{\percent} ab. Die Abweichung des Off-Stroms zum gemessenen Wert beträgt \SI{0}{\percent}. 

\subsection{Allgemeine Probleme bei der Auswertung}
Die Daten wurden mittels Python ausgewertet. Zur Erstellung der Fits wurde das Package scipy \cite{scipy} genutzt. Es scheint aber so, als habe die "curve fit"-Funktion Schwierigkeiten dabei quadratische trigonometrische Funktionen auszuwerten. Aus dem Grund wurde die Wurzel aus der Funktion und den y-Werten gezogen, anschließend wurde gefittet und am Ende wurden alle Werte wieder quadriert. Das Ergebnis ist damit deutlich besser, als alle vorherigen Versuche. Trotzdem ist die Abweichung speziell beim ersten und dritten Graphen noch relativ groß, was dann auch die Abweichung zum Literaturwert um fast \SI{50}{\percent} erklärt. 
Außerdem wurden die Werte beim Hauptmaximum, also bei $\SI{0}{\degree}$, herausgenommen, um zu vermeiden, dass während des "curve fit"-Vorgangs durch null geteilt wird. 
Das einzige tatsächliche Problem sind die Werte für den Off-Strom. Der Faktor $d$ wurde in der Wurzel-Variante des Terms hinzugefügt. Das bedeutet, dass wenn die Zahlen einfach quadriert werden, durch die binomische Formel noch eine zusätzliche Komponente hinzukommt. Da $d$ im Verhältnis zu den Werten im vorderen Teil der Gleichung einige Größenordnungen kleiner ist, wird dieser $2 B d$-Term wohl keinen großen Einfluss haben, denn vor allem bei der zweiten Messung ist der Wert $d$ sehr nah am gemessenen Wert, wenn man ihn dann quadriert hat. Insofern scheint unsere Rechnung in Ordnung zu sein, aber nur weil wir davon ausgehen können, dass der zusätzliche Teil keine große Auswirkung auf die Formel hat. 

\subsection{Fazit} 
Insgesamt kann der Versuch und die Messung als relativ exakt betrachtet werden. Nur die Auswertung stellt sich als recht kompliziert heraus, was dazu führt, dass die Ergebnisse nicht so gut sind, wie ursprünglich erwartet.  

\newpage
