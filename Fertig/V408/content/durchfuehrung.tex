\section{Durchführung}
\label{sec:Durchführung}

Es wird eine optische Bank benutzt, auf der die optischen Elemente, also die Lampe, die Linsen und der Schirm, befestigt werden.
Die Lichtquelle ist eine Halogenlampe und der Gegenstand ein "Perl L", eine Art Filter, durch den ein L-förmiges Bild durchgelassen wird. 

\subsection{Brennweitenbestimmung}
Im ersten Schritt soll bei einer festen Gegenstandsweite $g$ die Position des Schirms variiert werden, bis ein scharfes Bild zu erkennen ist. Dabei sollen die Gegenstands- und Bildweite sowie die Bildgröße gemessen werden. 
Die Messung wird für zehn verschiedene Gegenstandsweiten durchgeführt.

\subsection{Methode von Bessel}
Bei dieser Methode wird die Brennweite einer Linse bestimmt, indem der Abstand zwischen Gegenstand und Bild konstant bleibt und dabei zwei Linsenpositionen eingestellt werden, bei denen das Bild scharf wird.
Es werden die Bild- und Gegenstandsweiten aufgenommen.
Die Messung wird zehn mal durchgeführt.

\noindent Anschließend soll die chromatische Abberation für blaues und rotes Licht bestimmt werden. Dazu werden ein blauer und ein roter Filter verwendet. Die vorherige Messung wird jeweils fünf mal durchgeführt. 

\subsection{Methode von Abbe}
Eine Zerstreuungslinse und eine Sammellinse werden zwischen der Lampe und dem Schirm positioniert. Die beiden Linsen werden so dicht zusammengestellt, dass sie sich berühren und beim Verschieben wird ein fester Abstand zwischen den beiden Linsen beibehalten. Ein Referenzpunkt A wird gewählt. Es wird für zehn verschiedene Gegenstandsweiten ein scharfes Bild erzeugt und die Bild- und Gegenstandsweite $b'$ und $g'$ sowie die Bildgröße $B$ werden gemessen.