\section{Diskussion}
\label{sec:Diskussion}

\subsection{Verifizierung der Linsengleichung und des Abbildungsgesetzes}
Der mit der Bild- und Gegenstandsweite ermittelte Mittelwert des Abbildungsmaßstabes weicht um $\SI{6.7}{\percent}$ von dem Wert des mit der Bild- und Gegenstandsgröße berechneten Mittelwerts des Abbildungsmaßstabes ab.
Die Werte liegen also relativ nah aneinander und das Abbildungsgesetz kann als bestätigt angesehen werden.

\noindent Die mit der Linsengleichung ermittelte Brennweite weicht von der vom Hersteller angegebenen Brennweite um $\SI{3.8}{\percent}$ ab. Die Linsengleichung kann also auch als bestätigt angesehen werden.

\subsection{Überprüfung der Messgenauigkeit}
Die Geraden in \ref{fig:brennweite} schneiden sich alle in einem Punkt, der zuvor ermittelten Brennweite.
Dort gilt $x = y$.
Das bedeutet, dass die Messgenauigkeit hoch ist.

\subsection{Brennweitenbestimmung mit der Methode von Bessel}
Die erste Brennweite für das weiße Licht weicht um $\SI{2.7}{\percent}$ von der zweiten Brennweite ab.
Die erste Brennweite weicht von der Herstellerangabe um $\SI{3.8}{\percent}$ ab, die zweite um $\SI{1.1}{\percent}$.

\noindent Die erste Brennweite für das rote Licht weicht um $\SI{2.2}{\percent}$ von der zweiten Brennweite ab.

\noindent Die erste Brennweite für das blaue Licht weicht um $\SI{2.0}{\percent}$ von der zweiten Brennweite ab.

\noindent Die zuvor in \ref{sec:brennweite1} ermittelte Brennweite bei weißem Licht weicht von der ersten Brennweite, die mit der Methode von Bessel mit weißem Licht bestimmt wurde, um $\SI{0}{\percent}$ ab.
\noindent Von der zweiten Brennweite weicht die zuvor berechnete um $\SI{2.7}{\percent}$ ab.

\noindent Die jeweils erste Berechnung führt für die Brennweite des roten Lichts zu einer Abweichung von \SI{0.8}{\percent} zur Brennweite des weißen Lichts und die Abweichung der Brennweite des blauen Lichts beträgt \SI{0.5}{\percent}. Die zweite Berechnung ergibt, dass die Abweichung des roten Filters bei \SI{0.3}{\percent} und die Abweichung für den blauen Filter bei \SI{0.2}{\percent} liegt. Somit kann von einer geringen chromatischen Abberation gesprochen werden, da die Brennweite des roten Lichts jeweils weiter von der des weißen Lichts entfernt liegt, als die des blauen Lichts. 

\subsection{Brennweitenbestimmung mit der Methode von Abbe}
Die mit Abb. \ref{fig:gstrich} bestimmte Brennweite nach der Methode von Abbe weicht von der mit Abb. \ref{fig:bstrich} bestimmten Brennweite um $\SI{2.0}{\percent}$ ab.

\subsection{Fazit}
Da die Fehler der einzelnen berechneten Größen und die relativen Abweichungen alle klein sind, lässt sich sagen, dass die Messung ziemlich exakt ist.

\noindent Fehler bei der Messung könnte sein, dass die Definition eines scharfen Bildes nicht immer dieselbe war und somit Unregelmäßigkeiten bei der Messung entstanden sind.
