\documentclass[
  bibliography=totoc,     % Literatur im Inhaltsverzeichnis
  captions=tableheading,  % Tabellenüberschriften
  titlepage=firstiscover, % Titelseite ist Deckblatt
]{scrartcl}


% Paket float verbessern
\usepackage{scrhack}

% Warnung, falls nochmal kompiliert werden muss
\usepackage[aux]{rerunfilecheck}

% unverzichtbare Mathe-Befehle
\usepackage{amsmath}
% viele Mathe-Symbole
\usepackage{amssymb}
% Erweiterungen für amsmath
\usepackage{mathtools}

% Fonteinstellungen
\usepackage{fontspec}
% Latin Modern Fonts werden automatisch geladen
% Alternativ:
%\setromanfont{Libertinus Serif}
%\setsansfont{Libertinus Sans}
%\setmonofont{Libertinus Mono}
\recalctypearea % Wenn man andere Schriftarten gesetzt hat,
% sollte man das Seiten-Layout neu berechnen lassen

% deutsche Spracheinstellungen
\usepackage{polyglossia}
\setmainlanguage{german}


\usepackage[
  math-style=ISO,    % ┐
  bold-style=ISO,    % │
  sans-style=italic, % │ ISO-Standard folgen
  nabla=upright,     % │
  partial=upright,   % ┘
  warnings-off={           % ┐
    mathtools-colon,       % │ unnötige Warnungen ausschalten
    mathtools-overbracket, % │
  },                       % ┘
]{unicode-math}

% traditionelle Fonts für Mathematik
\setmathfont{Latin Modern Math}
% Alternativ:
%\setmathfont{Libertinus Math}

\setmathfont{XITS Math}[range={scr, bfscr}]
\setmathfont{XITS Math}[range={cal, bfcal}, StylisticSet=1]

% Zahlen und Einheiten
\usepackage[
  locale=DE,                   % deutsche Einstellungen
  separate-uncertainty=true,   % immer Fehler mit \pm
  per-mode=symbol-or-fraction, % / in inline math, fraction in display math
]{siunitx}

% chemische Formeln
\usepackage[
  version=4,
  math-greek=default, % ┐ mit unicode-math zusammenarbeiten
  text-greek=default, % ┘
]{mhchem}

% richtige Anführungszeichen
\usepackage[autostyle]{csquotes}

% schöne Brüche im Text
\usepackage{xfrac}

% Standardplatzierung für Floats einstellen
\usepackage{float}
\floatplacement{figure}{H}
\floatplacement{table}{H}

% Floats innerhalb einer Section halten
\usepackage[
  section, % Floats innerhalb der Section halten
  below,   % unterhalb der Section aber auf der selben Seite ist ok
]{placeins}

%dassselbe für Subsections 
\makeatletter
\AtBeginDocument{%
  \expandafter\renewcommand\expandafter\subsection\expandafter{%
    \expandafter\@fb@secFB\subsection
  }%
}
\makeatother

% Seite drehen für breite Tabellen: landscape Umgebung
\usepackage{pdflscape}

% Captions schöner machen.
\usepackage[
  labelfont=bf,        % Tabelle x: Abbildung y: ist jetzt fett
  font=small,          % Schrift etwas kleiner als Dokument
  width=0.9\textwidth, % maximale Breite einer Caption schmaler
]{caption}
% subfigure, subtable, subref
\usepackage{subcaption}

% Grafiken können eingebunden werden
\usepackage{graphicx}
% größere Variation von Dateinamen möglich
\usepackage{grffile}

% schöne Tabellen
\usepackage{booktabs}

% Verbesserungen am Schriftbild
\usepackage{microtype}

% Literaturverzeichnis
\usepackage[
  backend=biber,
]{biblatex}
% Quellendatenbank
\addbibresource{programme.bib}

% Hyperlinks im Dokument
\usepackage[
  unicode,        % Unicode in PDF-Attributen erlauben
  pdfusetitle,    % Titel, Autoren und Datum als PDF-Attribute
  pdfcreator={},  % ┐ PDF-Attribute säubern
  pdfproducer={}, % ┘
]{hyperref}
% erweiterte Bookmarks im PDF
\usepackage{bookmark}

% Trennung von Wörtern mit Strichen
\usepackage[shortcuts]{extdash}

%selbst hinzugefügt
\usepackage{physics}

\title{V703 - Das Geiger-Müller-Zählrohr}
\date{Durchführung: 28.05.2019, Abgabe: 04.06.2019}
\author{
  Jan Herdieckerhoff
  \texorpdfstring{%
    \\%
    \href{mailto:jan.herdieckerhoff@tu-dortmund.de}{jan.herdieckerhoff@tu-dortmund.de}
  }{}%
  \texorpdfstring{\and}{, }
  Karina Overhoff
  \texorpdfstring{%
    \\%
    \href{mailto:karina.overhoff@tu-dortmund.de}{karina.overhoff@tu-dortmund.de}
  }{}%
}
\publishers{TU Dortmund – Fakultät Physik}


\begin{document}
%ZAHLEN
%\num{Zahl eingeben. Punkt als Komma}
%\num{6.022e23} 10er Potenzen
%\num{1.54 \pm 0.1} Fehler
%EINHEITEN
%\si{\meter\per\second}
%\si[per-mode=fraction]{meter\per\second}
%ZAHL MIT EINHEIT
%1.Aurgument: num, 2.Argument: si
%\SI{2.5e6}{\kilo\gram\square\meter\per\second\squared}
%WINKEL
%\ang{5;;} -> 5°
%\ang{;5;} -> 5'

%Neue Zeile nicht einrücken: \noindent
\tableofcontents
\newpage

\section{Ziel}
Das Ziel dieses Versuches ist es, anhand
der Drehschwingungen eines Drahtes seine
elastischen Konstanten zu bestimmen.
Außerdem soll durch Erweiterung um ein Magnetfeld und einen
Permanentmagneten das magnetische Moment
dieses Magneten ermittelt werden.

\section{Theorie}
%Spannung (Normal-, Tangential) \sigma \tau
Die Kraft pro Flächeneinheit ist die Spannung.
Ihre Komponente, die senkrecht zur Oberfläche  steht,
ist die Normalspannung $\sigma$.
Die Tangentialspannung (oder Schubspannung) $\tau$
ist die Komponente der Spannung, die parallel
zur Oberfläche steht.
%Isotrope Körper
%Schubmodul/ Torsionsmodul G
%Kompressionsmodul Q
%Elastizitätsmodul E
%Poissonsche Querkontrktionszahl \mu

\noindent Ein isotroper Körper ist ein Material, bei dem 
die elastischen Konstanten richtungsunabhängig
sind. Man beschreibt sein elastisches Verhalten durch
vier Konstanten: Der Schubmodul $G$ charakterisiert
die Gestaltselastizität (solche Verformung tritt auf,
wenn an Probe ausschließlich Tangentialspannungen
angreifen), der Kompressionsmodul $Q$
die Volumenelastizität. Der Elastizitätsmodul $E$
beschreibt die relative Längenänderung eines unter
dem Einfluss einer Normalspannung stehenden Körpers
in Spannungsrichtung.
Die Poissonsche Querkontraktionszahl $\mu$ beschreibt
die Längenänderung eines solchen Körpers senkrecht zur
Richtung von $\vec{\sigma}$.
Bei diesem Versuch wird nur der Schubmodul $G$
bestimmt. Für $E$ wird ein Literaturwert genommen.
$Q$ und $\mu$ können aus den anderen beiden Modulen
mittels
%$\sigma = E \frac{\Delta L}{L}$
%$P = Q \frac{\Delta V}{V}$
%$\mu = \frac{\Delta B}{B} \cdot \frac{L}{\Delta L}$
\begin{gather}
    E = 2G(\mu + 1) \\
    E = 3(1-2\mu)Q %mu wird nicht angezeigt
    \label{eqn:E}
\end{gather}
berechnet werden.
%Torsion (/Scherung)

\noindent Der Schubmodul $G$ lässt sich aus der Torsion eines Drahtes bestimmen.
Dieser wird an einer Seite fest eingespannt. An der anderen Seite greift
an zwei diametral gegenüberliegenden Punkten ein Kräftepaar an.
%Drehmoment M
Dadurch entsteht ein Drehmoment auf den Draht, welches vom Hebelarm, der
über den Probendurchmesser varriert, abhängig ist. Also wird die Probe in
Hohlzylinder infinitesimaler Dicke $dr$ zerlegt. Für jeden wird das
infinitesimale Drehmoment $dM$ angegeben und über den gesamten Probenradius
$R$ integriert:
\begin{equation*}
    dM = r dK.
\end{equation*}
Mit der Schubspannung $\tau = \frac{dK}{dF}$, die als Tangentialkraft pro
Flächeneinheit gegeben ist, ergibt sich:
\begin{equation*}
    dM = r \tau dF.
\end{equation*}
Die Schubspannung ist auch durch das Hookesche Gesetz
\begin{equation*}
    \tau = G \alpha
\end{equation*}
gegeben. Damit folgt
\begin{equation}
    dM = r G \alpha dF.
    \label{eqn:dM}
\end{equation}
Der Zusammenhang zwischen dem Scherungswinkel $\alpha$, dem Torsionswinkel
$\phi$ und der Probenlänge $L$ ist folgender:
\begin{equation}
    \alpha = \frac{r \phi}{L}.
    \label{eqn:alpha}
\end{equation}
Der Flächeninhalt eines Kreisrings ist durch
\begin{equation}
    dF = 2 \pi r dr
    \label{eqn:dF}
\end{equation}
gegeben. Setzt man \eqref{eqn:dM}, \eqref{eqn:alpha} und \eqref{eqn:dF} zusammen, erhält man:
\begin{equation*}
    dM = 2 \pi \frac{G}{L} \phi r^3 dr.
\end{equation*}
Also ist das gesamte Drehmoment
\begin{equation}
    M = \int 2 \pi \frac{G}{L} \phi r^3 dr = \frac{\pi}{2} G \frac{R^4}{L} \phi.
    \label{eqn:M}
\end{equation}

%Proportionalitätsfaktor D
\noindent Dabei ist 
\begin{equation}
    D = \frac{\pi}{2} G \frac{R^4}{L}
    \label{eqn:D}
\end{equation}
der Proportionalitätsfaktor zwischen $M$ und $\phi$.
An den am Torsionsdraht hängenden Körper greifen zwei entgegengesetzt
wirkende Drehmomente an: das durch die Torsion des Drahtes \eqref{eqn:M} und das durch die
Trägheit der rotierenden Masse. %???

%Schwingungsdauer T
\noindent Die Schwingungsdauer des Systems ist gegeben durch 
\begin{equation}
    T = 2 \pi \sqrt{\frac{\theta}{D}}.
    \label{eqn:T}
\end{equation}

%Trägheitsmoment
\noindent Das Trägheitsmoment $\theta$ setzt sich dabei aus dem Trägheitsmoment der Kugel
$\theta_K$ und dem der Halterung der Kugel $\theta_H$ zusammen.
Das Trägheitsmoment einer Kugel wird durch
\begin{equation}
    \theta_{K} = \frac{2}{5} m_{K} R_{K}^2
    \label{eqn:theta}
\end{equation}
bestimmt.
$\theta_{H}$ ist bei diesem Versuch angegeben.

%Schubmodul endgültige Gleichung
\noindent Setzt man jetzt \eqref{eqn:D} und die Trägheitsmomente \eqref{eqn:theta} und $\theta_{H}$ in \eqref{eqn:T} ein, ergibt sich
\begin{equation*}
    T = 2 \pi \sqrt{\frac{\frac{2}{5} m_{K} R_{K}^2 + \theta_{H}}{\frac{\pi G R^4}{2 L}}}.
\end{equation*}
Quadriert man diesen Ausdruck und stellt ihn nach $G$ um, erhält man für
den Schubmodul $G$ die endgültige Gleichung
\begin{equation}
    G = \frac{\frac{16}{5} \pi L m_{K} R_{K}^{2} + 8 \pi L \theta_{H}}{T^2 R^4}.
    \label{eqn:G}
\end{equation}

%Magnetisches Moment m
\noindent Das magnetische Moment ist gegeben durch
\begin{equation*}
    \vec{m} = p \cdot \vec{a}.
\end{equation*}
$p$ ist die Polstärke und $\vec{a}$ der Abstand der Pole.
In einem homogenen Magnetfeld, wie es von der Helmholtz-Spule erzeugt wird,
wirken auf einen Magneten zwei entgegengesetzt gleiche Kräfte, die an den
Polen angreifen. Es gibt also keine resultierende Kraft, sondern ein
Drehmoment $\vec{M}_{Mag}$, das den Magneten in Feldrichtung dreht:
\begin{equation*}
    \vec{M}_{Mag} = p \vec{a} \times \vec{B} = \vec{m} \times \vec{B}.
\end{equation*}
Der Betrag des Drehmoments ist 
\begin{equation*}
    M_{Mag} = mB \sin(\gamma).
\end{equation*}

%neue Periodendauer
\noindent Die Periodendauer der Torsionsschwingung ist jetzt natürlich eine andere.
Sie ist gegeben durch
\begin{equation*}
    T_{m} = 2 \pi \sqrt{\frac{\theta}{mB+D}}.
\end{equation*}
Quadriert man diesen Ausdruck und stellt ihn nach $m$ um, erhält man
für das magnetische Moment
\begin{equation}
    m = \frac{4 \pi^2 \theta}{T m^2 B} - \frac{D}{B}.
\end{equation}
\section{Durchführung}
%Kugel um Fehler zu beheben
%Beschreibung der Messapparatur 1
Die Messapparatur besteht aus einem Torsionsdraht
und einer an diesem aufgehängten Kugel.
Das System soll Torsionsschwingungen ausführen.
Die Schwingungsdauer soll dabei mit einer
elektronischen Stoppuhr gemessen werden.
Die Signale, die nötig sind, um die Stoppuhr
zu steuern, werden durch eine Lichtschranke
erzeugt. Eine Beleuchtungsvorrichtung neben
der Apparatur wirft einen Lichtstrahl auf einen
am Torsionsdraht angebrachten Spiegel. Wenn
der Spiegel vom Lichtstrahl getroffen wird,
gibt der Lichtdetektor, also die Photodiode,
ein elektrisches Signal an das Zählwerk der
Stoppuhr ab.

%Skizze
\noindent SKIZZE 1

%Versuchsteil 1
\noindent Zuerst wird der Spiegel so justiert, dass
der reflektierte Lichtstrahl ein wenig neben
die Photodiode fällt. Die Beleuchtungsvorrichtung
wird so eingestellt, dass ein scharfes Bild
des Spaltes auf der Mattscheibe entsteht.
Wenn das System durch Hin- und Herbewegen des
Justierrads zu Torsionsbewegungen angeregt
wird, wandert der Lichtstrahl zu einem Umkehrpunkt
und wieder zurück. Dabei geht er über die
Photodiode und löst einen elektrischen Impuls
aus, der die Uhr startet. Der nächste Impuls
entsteht, wenn der Lichtstrahl vom Umkehrpunkt
zurückkommt und wieder über die Photodiode
wandert. Dieser Impuls wird allerdings nicht
benötigt. Der darauf folgende, dritte Impuls, der 
entsteht, wenn der Strahl vom anderen Umkehrpunkt
zurückkehrt und die Diode überstreicht, stoppt
die Uhr. Es wurde eine volle Periode ausgeführt.
Der vierte Impuls löst einen Rückstellimpuls
für das Zählwerk aus. Die Stoppuhr wird also
wieder auf Null gesetzt. Die nächste Periode
wird danach genauso gemessen. Um das Schubmodul $G$ zu bestimmen, werden
zehn Schwingungsdauern gemessen.

%Beschreibung der Messapparatur 2
\noindent Zuletzt soll noch das magnetische Moment eines
Permanentmagneten gemessen werden. Die Versuchsapparatur
wird dafür erweitert. Es wird ein Helmholtz-Spulenpaar,
welches ein homogenes Magnetfeld erzeugt, hinzugestellt. %bessere Formulierung?
Die Kugel, die unten am Draht angehängt ist,
wird in der Halterung so gedreht, dass der darin enthaltene
Magnet parallel zum Magnetfeld steht. Die
Dipolachse soll also in Ruhelage parallel zur Feldrichtung
stehen.

%Skizze 2
\noindent SKIZZE 2

%Versuchsteil 2
\noindent Der Spiegel muss jetzt so justiert werden, dass
deutlich kleinere Auslenkungen als beim
vorherigen Versuchsteil möglich sind. Das
Justierrad wird kurz aus der Ruhelage ausgelenkt,
um eine Drehschwingung zu erzeugen.
Die Schwingungsdauer wird für jede eingestellte
Stromstärke der Helmholtz-Spulen fünf mal
gemessen. Es wird für fünf verschiedene
Stromstärken gemessen.

\section{Fehlerrechnung}
%Mittelwert
Der Mittelwert einer Stichprobe von $N$ Werten
wird durch
\begin{equation*}
    \overline{x} = \frac{1}{N} \sum_{i=1}^N x_i
\end{equation*}
bestimmt.
%Standardabweichung
Die Standardabweichung der Stichprobe wird berechnet mit:
\begin{equation*}
    \sigma_x = \sqrt{\frac{1}{N-1} \sum_{i=1}^N (x_i - \overline{x})^2}.
\end{equation*}
%Gauß'sche Fehlerfortpflanzung
Das sogenannte Gauß'sche Fehlerfortpflanzungsgesetz
ist gegeben durch
\begin{equation*}
    \sigma_f = \sqrt{\sum_{i=1}^N (\frac{\partial f}{\partial x_i} \sigma_i)^2}.
\end{equation*}
Dabei ist $f$ eine von unsicheren Werten $x_i$
abhängige Funktion mit Standardabweichungen $\sigma_i$.

%Lineare Regression
Bei der linearen Regression wird die Gerade
\begin{equation*}
    y(x) = mx + b
\end{equation*}
durch das Streudiagramm gelegt.
Dabei ist $m$ die Steigung mit
\begin{equation*}
    m = \frac{\overline{xy} - \overline{x} \cdot \overline{y}}{\overline{x^2} - \overline{x}^2}
\end{equation*}
und $b$ der $y$-Achsenabschnitt mit
\begin{equation*}
    b = \frac{\overline{y} \cdot \overline{x^2} - \overline{xy} \cdot \overline{x}}{\overline{x^2} - \overline{x}^2}.
\end{equation*}
(Wenn im Folgenden Mittelwert, Standardabweichung
oder die Standardabweichung von Funktionen unsicherer
Größen berechnet werden, werden diese Formeln
verwendet.)
Zur Auswertung und Berechnung der Fehler
wurde Python, im Speziellen uncertainties.unumpy
und numpy, benutzt.

\section{Auswertung}
Einige Eigenschaften der Messapparatur sind angegeben.
Diese sind in Tabelle 1 dargestellt.

TABELLE 1

Dabei ist $m_{K}$ die Masse der Kugel, $2R_{K}$ der Durchmesser
der Kugel, $R_{K}$ der Radius der Kugel und $\theta_{H}$ das
Trägheitsmoment der Kugelhalterung.
$N$ ist die Windungszahl der Helmholtz-Spule und $R_{H}$ ihr Radius.
\subsection{Bestimmung des Schubmoduls $G$}
Die Abmessungen des Torsionsdrahtes, die zur Berechnung des Schubmoduls $G$ benötigt werden,
sind in Tabelle 2 zu finden. $L$ ist hier die Länge des Drahtes,
$d$ sein Durchmesser und $R$ sein Radius. Die Periodendauern $T$ der
Torsionsschwingungen sind in Tabelle 3 aufgelistet.

TABELLE 2

TABELLE 3

Die gemessenen Werte werden gemittelt.
Damit ergibt sich für die Länge des Drahtes $L = \SI{0.6860 \pm 0.0008}{\metre}$.
Der gemittelte Durchmesser ist %$d = \SI{1.767 \cdot 10^{-4} \pm 0.024}{\metre}$.
Die mittlere Periodendauer ist $T = \SI{20.036 \pm 0.012}{\second}$.
Für den Schubmodul ergibt sich also mittels \eqref{eqn:G}
%\begin{equation*}
 %   G = \SI[per-mode=fraction]{9.5e10 \pm 0.5e10}{\kilo\gram\per\square\metre\per\second\squared}.
%\end{equation*}
\subsection{Bestimmung der anderen elastischen Konstanten}
Die elastischen Konstanten lassen sich aus dem Schubmodul $G$
und dem Elastizitätsmodul $E$ mittels \eqref{eqn:E} bestimmen.
Für $E$ wird ein Literaturwert genommen:
%$E = \SI[per-mode=fraction]{2.1e13 \pm 0}{\Newton\per\square\metre}$.
Damit ergibt sich für das Kompressionsmodul %$Q = \SI[per-mode=fraction]{-3.21e10 \pm 0.17e10}{\Newton\per\square\metre}$
und für die Poissonsche Querkontraktionszahö $\mu = \num{110 \pm 6}$.
\subsection{Bestimmung des magnetischen Moments}

\section{Diskussion}
Die Methode zur Bestimmung des Schub- oder Torsionsmoduls $G$ und des magnetischen Moments $m$ sind nach der Auswertung als relativ genau zu bewerten.
Die Messung des Elastizitätsmoduls $E$ konnte leider nicht durchgeführt werden, da die Messgeräte dafür nicht zur Verfügung standen. 
Der Wert des Schubmoduls $G$ liegt mit $\SI{95 \pm 5}{\giga\Pa}$ unter Berücksichtigung des Fehlers bei einer Abweichung von $\SI{12.23}{\percent}$ vom Literaturwert. Ein Grund dafür könnte die vermutlich nicht ganz exakte Messung der Breite des Drahtes sein. Auch die Messung der Länge des Drahtes stellte sich als relativ kompliziert heraus, da der Draht recht schwierig zu erreichen war. Er war bereits in der Vorrichtung mit der Helmholtz-Spule verbaut.
Auch die Reibung der Luft und eine eventuell zu starke Auslenkung der Vorrichtung können zu den Fehlern des Systems geführt haben. 
Das Gewicht und der Durchmesser der Kugel, sowie das Trägheitsmoment der Vorrichtung und die Werte des Helmholtz-Spulenpaars waren gegeben. 
Das Bauen der Schaltung wurde durch eine vorliegende Skizze unterstützt.
Beim Messen der Schwingungsdauer $T$ kam es erst zu Schwierigkeiten, da der am Draht befestigte Spiegel zu sehr gewackelt hat. Dadurch hat die Photodiode ein zu schwaches Signal erhalten und konnte dieses nicht messen. Nachdem dieser Fehler behoben wurde, verlief die restliche Messung recht positiv, was sich auch an den gemessenen Daten erkennen lässt. 
Die Werte zur Bestimmung des magnetischen Moments, also die Drehschwingung $T_{m}$ in Korrelation zum magnetischen Fluss $B$, ergeben eine lineare Funktion, wenn man das Inverse des Quadrats von $T_{m}$ gegen $B$ aufträgt. Durch den Wert der Steigung und die Trägheitsmomente der Vorrichtung sowie der Kugel ließ sich das magnetische Moment problemlos berechnen. Ob dies einen realistischen Wert hat, lässt sich leider nicht sagen, da wir zu dem Permanentmagneten im Inneren der Kugel keinen Literaturwert kennen.



\section{Literatur}

\end{document}
