\begin{table}\caption{Die Masse der Probe, die Dichte des Probenmaterials und die Molmasse. Für den ersten Stoff wurde dabei angenommen, dass die Dichte näherungsweise dieselbe ist, wie die Dichte der Probe. Die Dichte wurde hierbei mit dem Volumen und der angegebenen Probenmasse bestimmt. Für die anderen Stoffe war die Dichte in der Anleitung gegeben. Außerdem ist der Querschnitt $Q_\text{real}$ gegeben und die Querschnittsfläche beträgt bei allen Stoffen $F = \SI{86.6}{\milli\meter\squared}$.}
\label{tab2}
\centering
\sisetup{round-mode = places, round-precision=2, round-integer-to-decimal=true}
\begin{tabular}{S[]S[]S[]S[]} 
\toprule
{$m$ \si{\gram}} & {$\rho_\text{W}$ / \si{\kilo\gram\per\cubic\meter}} & {$M$ \si{\gram\per\mol}} & {$Q_\text{real} / \si{\milli\meter\squared}$}\\
\midrule
7.87 & 6475.0 & 545.87 & 7.596525096525096\\
14.08 & 6400.0 & 373.0 & 13.75\\
9.0 & 7240.0 & 362.5 & 7.7693370165745845\\
14.38 & 7800.0 & 336.48 & 11.522435897435898\\
\bottomrule
\end{tabular}\end{table}