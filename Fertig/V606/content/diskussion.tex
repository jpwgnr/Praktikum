\section{Diskussion}
\label{sec:Diskussion}

\subsection{Berechnung der Güte}
Die relative Abweichung des berechneten Wertes für die Güte zum erwarteten Gütewert beträgt \SI{0.29}{\percent}. 

\subsection{Suszeptibilität}
\subsubsection{Kohlenstoff-Sauerstoff-Praseodym-Verbindung $C_6 O_{12} Pr_2$}
%Bei $C_6 O_{12} Pr_2$ handelt es sich um eine Verbindung von Kohlenstoff, Sauerstoff und Praseodym.
%\newline
Der relative Fehler der Suszeptibilität für $C_6 O_{12} Pr_2$ bei der Berechnung mit dem Spannungsverhältnis beträgt \SI{50}{\percent}. 
Die Abweichung des gemessenen Wertes zum Theoriewert mit dem Spannungsverhältnis beträgt \SI{95.12}{\percent}.
\newline
Bei der Berechnung mit dem Widerstandsverhältnis beträgt der relative Fehler \SI{44.4}{\percent}.
Die Abweichung des Wertes zum theoretischen Wert beträgt \SI{26.8}{\percent}.
\newline
Grund für die große Abweichung könnte hier sein, dass die Dichte des Stoffs für die Berechnung nicht bekannt war und 
näherungsweise mit der Dichte der Probe gerechnet wurde. 
Dafür passen vor allem die Werte beim Widerstandsverhältnis ziemlich gut zum Theoriewert.

\subsubsection{Neodym(III)-oxid $Nd_2 O_3$}
Der relative Fehler der Suszeptibilität für $Nd_2 O_3$ bei der Berechnung mit dem Spannungsverhältnis beträgt \SI{26.67}{\percent}. 
Die Abweichung des gemessenen Wertes zum Theoriewert mit dem Spannungsverhältnis beträgt \SI{94.65}{\percent}.
\newline
Bei der Berechnung mit dem Widerstandsverhältnis beträgt der relative Fehler \SI{9.47}{\percent}.
Die Abweichung des Wertes zum theoretischen Wert beträgt \SI{32.21}{\percent}.

\subsubsection{Gadolinium(III)-oxid $Gd_2 O_3$}
Der relative Fehler der Suszeptibilität für $Gd_2 O_3$ bei der Berechnung mit dem Spannungsverhältnis beträgt \SI{0.63}{\percent}. 
Die Abweichung des gemessenen Wertes zum Theoriewert mit dem Spannungsverhältnis beträgt \SI{76.87}{\percent}.
\newline
Bei der Berechnung mit dem Widerstandsverhältnis beträgt der relative Fehler \SI{1.73}{\percent}.
Die Abweichung des Wertes zum theoretischen Wert beträgt \SI{15.44}{\percent}.

\subsubsection{Dysprosium(III)-oxid $Dy_2 O_3$}
Der relative Fehler der Suszeptibilität für $Dy_2 O_3$ bei der Berechnung mit dem Spannungsverhältnis beträgt \SI{1.28}{\percent}. 
Die Abweichung des gemessenen Wertes zum Theoriewert mit dem Spannungsverhältnis beträgt \SI{75.04}{\percent}.
\newline
Bei der Berechnung mit dem Widerstandsverhältnis beträgt der relative Fehler \SI{0.87}{\percent}.
Die Abweichung des Wertes zum theoretischen Wert beträgt \SI{18.27}{\percent}.


\subsection{Auffälligkeiten und Erklärungsversuche}
Die Ergebnisse für die Berechnung mittels Spannungsverhältnis lagen alle weit entfernt von den Theoriewerten. Dafür kann es verschiedene Gründe geben. Zum einen wäre es 
möglich, dass irgendwo ein Fehler mit den Einheiten unterlaufen ist. 
Auffällig ist nämlich, dass die Werte alle um einen ähnlichen Faktor abweichen. 
\newline
Die Ergebnisse der Widerstände waren etwas besser. Man könnte die Ergebnisse vielleicht noch verbessern, indem man den Widerstand exakter als 
nur in fünf Milliohm Schritten misst. Ansonsten war auch hier bei allen Messungen eine ähnlich große Abweichung zu erkennen, die aber 
trotzdem noch deutlich geringer war als bei der Messung der Spannungen. Gründe dafür könnten Innenwiderstände und Verlust durch Wärme sein. 
\newline
Je größer die Suszeptibilität der jeweiligen Probe war, desto besser stimmten die experimentellen Ergebnisse mit den Theoriewerten überein. 
Das lässt vermuten, dass die Messgeräte für kleine Suszeptibilitäten vielleicht nicht exakt genug gemessen haben. Tatsächlich sah man bei 
einigen Messungen kaum Abweichungen zwischen der Messung mit und der Messung ohne Probe. 

\subsection{Fazit}
Positiv ist, dass die Größenordnung der Ergebnisse zumindest mit den Theoriewerten übereinstimmt. Auch eine gewisse  Proportionalität konnte 
erkannt werden, obwohl die Werte nicht wirklich gut mit den erwarteten Ergebnissen übereinstimmten.
