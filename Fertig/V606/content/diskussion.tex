\section{Diskussion}
\label{sec:Diskussion}

\subsection{Berechnung der Güte}
Die relative Abweichung des berechneten Wertes für die Güte zum erwarteten Gütewert beträgt \SI{0.29}{\percent}. 

\subsection{Suszeptibilität}

Die relative Fehler und die relative Abweichung zum Literaturwert bei dem Spannungsverhältnis und der relative Fehler und die relative Abweichung zum Literaturwert beim Widerstandsverhältnis stehen in Tab. \ref{taberr}. 
\begin{table}\caption{Die relativen Fehler bei der Messung mit dem Spannungsverhältnis und die Abweichung zu den Literaturwerten. Außerdem das Gleiche für die Widerstandsmessung.}
\label{taberr}
\centering
\sisetup{round-mode = places, round-precision=2, round-integer-to-decimal=true}
\begin{tabular}{l S S S S} 
\toprule
{Stoffe} & {Relativer Fehler (Spannung) / $\si{\percent}$} & {Abweichung Literaturwert (Spannung) / $\si{\percent}$} & {Relativer Fehler (Widerstand) / $\si{\percent}$} & {Abweichung Literaturwert (Widerstand) / $\si{\percent}$} \\
\midrule
$\text{C}_6 \text{O}_{12} \text{Pr}_2$  & 4.0 & 1.0 & 5.0 & 1.0 \\
$\text{Gd}_2 \text{O}_3$                & 0.63 & 76.87 & 1.73 & 15.44 \\
$\text{Nd}_2 \text{O}_3$                & 26.67 & 94.65 & 9.47 & 32.21 \\
$\text{Dy}_2 \text{O}_3$                & 1.28 & 75.04 & 0.87 & 18.27 \\
\bottomrule
\end{tabular}\end{table}


\subsubsection{Kohlenstoff-Sauerstoff-Praseodym-Verbindung $\text{C}_6 \text{O}_{12} \text{Pr}_2$}
%Bei $C_6 O_{12} Pr_2$ handelt es sich um eine Verbindung von Kohlenstoff, Sauerstoff und Praseodym.
%\newline
%Der relative Fehler der Suszeptibilität für $\text{C}_6 \text{O}_{12} \text{Pr}_2$ bei der Berechnung mit dem Spannungsverhältnis beträgt \SI{50}{\percent}. 
%Die Abweichung des gemessenen Wertes zum Theoriewert mit dem Spannungsverhältnis beträgt \SI{95.12}{\percent}.
%\newline
%Bei der Berechnung mit dem Widerstandsverhältnis beträgt der relative Fehler \SI{44.4}{\percent}.
%Die Abweichung des Wertes zum theoretischen Wert beträgt \SI{26.8}{\percent}.
%\newline
Hier wurde statt des Stoffs nur $\text{Pr}_2$ betrachtet. Daher könnten die Werte ein wenig abweichen, aber die tatsächliche Dichte des Stoffes fehlte hier, sodass dies die bestmögliche Näherung war.

%\subsubsection{Neodym(III)-oxid $\text{Nd}_2 \text{O}_3$}
%Der relative Fehler der Suszeptibilität für $\text{Nd}_2 \text{O}_3$ bei der Berechnung mit dem Spannungsverhältnis beträgt \SI{26.67}{\percent}. 
%Die Abweichung des gemessenen Wertes zum Theoriewert mit dem Spannungsverhältnis beträgt \SI{94.65}{\percent}.
%\newline
%Bei der Berechnung mit dem Widerstandsverhältnis beträgt der relative Fehler \SI{9.47}{\percent}.
%Die Abweichung des Wertes zum theoretischen Wert beträgt \SI{32.21}{\percent}.

%\subsubsection{Gadolinium(III)-oxid $\text{Gd}_2 \text{O}_3$}
%Der relative Fehler der Suszeptibilität für $\text{Gd}_2 \text{O}_3$ bei der Berechnung mit dem Spannungsverhältnis beträgt \SI{0.63}{\percent}. 
%Die Abweichung des gemessenen Wertes zum Theoriewert mit dem Spannungsverhältnis beträgt \SI{76.87}{\percent}.
%\newline
%Bei der Berechnung mit dem Widerstandsverhältnis beträgt der relative Fehler \SI{1.73}{\percent}.
%Die Abweichung des Wertes zum theoretischen Wert beträgt \SI{15.44}{\percent}.

%\subsubsection{Dysprosium(III)-oxid $\text{Dy}_2 \text{O}_3$}
%Der relative Fehler der Suszeptibilität für $\text{Dy}_2 \text{O}_3$ bei der Berechnung mit dem Spannungsverhältnis beträgt \SI{1.28}{\percent}. 
%Die Abweichung des gemessenen Wertes zum Theoriewert mit dem Spannungsverhältnis beträgt \SI{75.04}{\percent}.
%\newline
%Bei der Berechnung mit dem Widerstandsverhältnis beträgt der relative Fehler \SI{0.87}{\percent}.
%Die Abweichung des Wertes zum theoretischen Wert beträgt \SI{18.27}{\percent}.


\subsection{Auffälligkeiten und Erklärungsversuche}
Die Ergebnisse für die Berechnung mittels Spannungsverhältnis lagen alle weit entfernt von den Theoriewerten.  
Auffällig ist nämlich, dass die Werte alle um einen ähnlichen Faktor abweichen. 
\newline
Die Ergebnisse der Widerstände waren etwas besser. Man könnte die Ergebnisse vielleicht noch verbessern, indem man den Widerstand exakter als 
nur in fünf Milliohm Schritten misst. Ansonsten war auch hier bei allen Messungen eine ähnlich große Abweichung zu erkennen, die aber 
trotzdem noch deutlich geringer war als bei der Messung der Spannungen. Gründe dafür könnten Innenwiderstände und Verlust durch Wärme sein. 
\newline
Je größer die Suszeptibilität der jeweiligen Probe war, desto besser stimmten die experimentellen Ergebnisse mit den Theoriewerten überein. 
Das lässt vermuten, dass die Messgeräte für kleine Suszeptibilitäten vielleicht nicht exakt genug gemessen haben. Tatsächlich sah man bei 
einigen Messungen kaum Abweichungen zwischen der Messung mit und der Messung ohne Probe. 

\subsection{Fazit}
Positiv ist, dass die Größenordnung der Ergebnisse zumindest mit den Theoriewerten übereinstimmt. Auch eine gewisse  Proportionalität konnte 
erkannt werden, obwohl die Werte nicht wirklich gut mit den erwarteten Ergebnissen übereinstimmten.
