\begin{table}\caption{Die relativen Fehler bei der Messung mit dem Spannungsverhältnis $(1)$ und die Abweichung zu den Literaturwerten. Außerdem das Gleiche für die Widerstandsmessung $(2)$.}
\label{taberr}
\centering
\sisetup{round-mode = places, round-precision=2, round-integer-to-decimal=true}
\begin{tabular}{l S S S S} 
\toprule
{Stoffe} & {$\text{Fehler}_1 / \si{\percent}$} & {$\text{Abweichung}_1 / \si{\percent}$} & {$\text{Fehler}_2 / \si{\percent}$} & {$\text{Abweichung}_2 / \si{\percent}$} \\
\midrule
$\text{C}_6 \text{O}_{12} \text{Pr}_2$  & 50  & 95.1 & 41.67 & 26.38  \\
$\text{Gd}_2 \text{O}_3$                & 0.63 & 76.87 & 1.73 & 15.44 \\
$\text{Nd}_2 \text{O}_3$                & 26.67 & 94.65 & 9.47 & 32.21 \\
$\text{Dy}_2 \text{O}_3$                & 1.28 & 75.04 & 0.87 & 18.27 \\
\bottomrule
\end{tabular}\end{table}
