\section{Diskussion}
\label{sec:Diskussion}
%irgendwas dazwischen schreiben
Im Folgenden wird die statische sowie die dynamische Methode diskutiert.
\subsection{Statische Methode}
Zunächst wird die Auswertung der statischen Methode diskutiert.
%Temperaturverläufe von T1, T4, T5, T8 vergleichen
\subsubsection{Vergleich der Temperaturverläufe der fernen Thermoelemente}
In den Temperaturverläufen $T1/T4$ und $T5/T8$ sind einige Merkmale zu erkennen. 
So ist erkennbar, dass in der Grafik $T1/T4$ die Kurve für $T1$ über der Kurve von $T4$ liegt. Dies deutet darauf hin, dass der Messingstab mit 
der größeren Querschnittsfläche eine bessere Wärmeleitfähigkeit als der Messingstab mit der kleineren Oberfläche besitzt. 
In Grafik $T5/T8$ ist zu erkennen, dass der Aluminiumstab eine bessere Wärmeleitfähigkeit als die anderen Stäbe hat. Edelstahl fällt sogar in den ersten 50 Sekunden 
noch ein wenig ab, obwohl die Temperatur des Aluminiumstabs sofort anfängt zu steigen. 
Bei allen vier Kurven ist es so, dass die Temperatur anfangs stärker steigt und dann immer langsamer steigt. Beim Edelstahl sieht es sogar 
beinah linear aus. Die anderen Kurven steigen ähnlich wie eine Wurzel- oder eine logarithmische Funktion.

\subsubsection{Wärmeleitfähigkeit}
%Welcher hat die beste Wärmeleitung?
Die Abweichung der Temperaturen bei $T5$ und $T6$ war mit \SI{49.94}{\degreeCelsius} und \SI{50.00}{\degreeCelsius} nach \SI{700}{\second} am geringsten. Daher 
ist daraus zu schließen, dass das Material Aluminium die beste Wärmeleitfähigkeit besitzt. 
Im Vergleich zu Messing und Edelstahl ist zu erkennen, dass das Aluminium  am schnellsten heiß wurde. Da die geometrischen
Bedingungen bei allen drei Materialien dieselben waren, ist klar zu erkennen, dass es eine Material- und keine
Geometrieeigenschaft ist, die die Wärmeleitfähigkeit in diesem Fall bestimmt.

\subsubsection{Wärmestrom}
%Wärmestrom Meßzeiten
Die Wärmeströme liegen in der zu erwartenden Größenordnung. Interessant ist dabei, dass der Wärmestrom in Messing größer ist als im Aluminium. 
Aluminium scheint wohl die größte Wärmeleitfähigkeit zu besitzen, aber einen geringeren Wärmestrom innerhalb des Stabs.
Das bedeutet, dass $\frac{\partial T}{\partial x}$ bei Messing deutlich größer sein muss, als bei Aluminium.
Die Temperaturdifferenz pro Strecke ist bei größerer Wärmeleitfähigkeit also geringer. Der Wärmestrom ist höher. %reicht die Erklärung?
Zu erkennen ist, dass der Wärmestrom am Anfang noch groß ist und mit der Zeit dann abnimmt. Daran lässt sich erkennen, dass die Temperatur 
in einer Wurzel oder logarithmischen Funktion zunimmt.

\subsubsection{Vergleich der Temperaturdifferenzen der Thermoelemente von Edelstahl und Messing}
%Graphiken T7-T8 und T2-T1 vergleichen
In der Grafik $T7-T8$ ist zu erkennen, dass die Differenz zwischen dem nahen und dem fernen Thermoelement anfangs stark zunimmt, aber nach ca. 
\SI{200}{\second} ein Maximum erreicht und anschließend fällt. Die Grafik $T2-T1$ lässt erkennen, dass auch hier die 
Differenz steigt, bis sie bei ca. \SI{75}{\second} ein Maximum erreicht und anschließend fällt.
Dabei fällt auf, dass das Gefälle deutlich größer ist und der Hochpunkt deutlich eher erreicht wird. Dafür ist in Grafik 
$T7-T8$ aber der Hochpunkt bei einer höheren Temperatur. Er liegt bei ca. \SI{10}{\degreeCelsius}, wohingegen der Graph in Grafik $T2-T1$ sein
Maximum bei ca. \SI{5.25}{\degreeCelsius} hat. Das bedeutet, dass die Temperaturdifferenz beim Edelstahl nicht schnell zunimmt, aber sobald sie ein 
sehr hohes Maximum erreicht hat, nur langsam abfällt. Somit ist zu erkennen, dass Edelstahl keine hohe Wärmeleitfähigkeit besitzt, da der Zeitraum, bis das Maximum erreicht wurde, größer als beim Messing ist. Anschließend ist der Zeitraum auch groß, bis die Differenz wieder abgenommen hat.
%Was bedeutet das?

\subsection{Dynamische Methode}
%Wärmeleitfähigkeit kappa 
Für die Wärmeleitfähigkeit des Messings ergibt sich ein relativer Fehler von \SI{12.64}{\percent}. Für Aluminium liegt der relative 
Fehler bei \SI{14.77}{\percent}. Für Edelstahl liegt der Fehler bei \SI{9.33}{\percent}. 

\noindent Der experimentelle Wert von $\kappa_\text{Mes}$ entspricht \SI{72.5}{\percent} des Literaturwertes. 
Für $\kappa_\text{Alu}$ ergibt sich ein prozentualer Wert von \SI{74.57}{\percent} und für $\kappa_\text{Edel}$ ergibt sich ein Wert 
von \SI{71.43}{\percent}. Interessant ist, dass somit alle drei Metalle in einem ziemlich ähnlichen Bereich liegen, was die Abweichung zum 
Literaturwert angeht. 
Die Auswertung mit dem Gerät scheint wohl einen Fehler von \num{25} bis \SI{30}{\percent} zu haben. Grund dafür wird zum Beispiel die 
nicht ideale Wärmeisolierung sein.
