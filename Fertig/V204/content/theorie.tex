\section{Ziel}
\label{sec:Ziel}

Bei diesem Versuch soll die Wärmeleitung von Aluminium, Messing und Edelstahl untersucht werden. 

\section{Theorie}
\label{sec:Theorie}

In einem Körper, der sich in einem Temperaturungleichgewicht befindet, kommt es zu einem Wärmetransport entlang 
des Temperaturgefälles. Dies kann z.B. durch Wärmeleitung geschehen. In festen Körpern erfolgt der Wärmetransport 
über Phononen und frei bewegliche Elektronen. In Metallen ist der Gitterbeitrag (Phononen) zu vernachlässigen. 

\noindent Ist das eine Ende eines Stabes wärmer als das andere, durchfließt die Wärmemenge $dQ$ die Querschnittsfläche $A$ in der Zeit $dt$. Es gilt:
\begin{equation}
\frac{dQ}{dt} = -\kappa \, A \frac{\partial T}{\partial x},
\label{eqn:dQ}
\end{equation}
wobei $\kappa$ die Wärmeleitfähigkeit ist.

\noindent Für die Wärmestromdichte $j_{\omega}$ gilt:
\begin{equation*}
    j_{\omega} = -\kappa \, \frac{\partial T}{\partial x}.
\end{equation*}
Mit Hilfe der Kontinuitätsgleichung kann hieraus die eindimensionale Wärmeleitungsgleichung aufgestellt werden:
\begin{equation*}
    \frac{\partial T}{\partial t} = \frac{\kappa}{\rho c} \frac{\partial^2 T}{\partial x^2},
\end{equation*}
wobei $\rho$ die Dichte und $c$ die spezifische Wärme ist.
Dabei ist $\frac{\kappa}{\rho c}$ die Temperaturleitfähigkeit. Sie gibt die Schnelligkeit an, mit der sich ein 
Temperaturunterschied ausgleicht. Die Lösung der Wärmeleitungsgleichung hängt von der Geometrie des Stabes und den Anfangsbedingungen ab. 

\noindent Wird ein Stab abwechselnd erwärmt und abgekühlt, breitet sich wegen der periodischen Temperaturwechsel eine räumliche 
und zeitliche Temperaturwelle aus. Diese wird folgendermaßen beschrieben:
\begin{equation*}
    T(x,t) = T_\text{max} \,  e^{\mathlarger{-\sqrt{\frac{\omega \rho c}{2 \kappa}} x}} cos\!\!\left(\omega t - \sqrt{\frac{\omega \rho c}{2 \kappa}} x\right).
\end{equation*}
Die Phasengeschwindigkeit $v$ der Welle ist
\begin{equation*}
    v= \frac{\omega}{k} = \sqrt{\frac{2\kappa \omega}{\rho c}}.
\end{equation*}
Für die Wärmeleitfähigkeit ergibt sich nach einigen kleinen Umformungen: %ohne "kleinen"?
\begin{equation}
    \kappa = \frac{\rho c (\Delta x)^2}{2 \Delta t \, ln\!\!\left(\frac{A_\text{nah}}{A_\text{fern}}\right)}.
    \label{eqn:Wärme}
\end{equation}
Dabei sind $A_\text{nah}$ und $A_\text{fern}$ die Amplituden an verschiedenen Stellen $x_\text{nah}$ und $x_\text{fern}$. Die Größe $\Delta x$ ist 
der Abstand zwischen diesen beiden Messstellen und $\Delta t$ die Phasendifferenz der Temperaturwelle zwischen beiden 
Messstellen.
