\section{Auswertung}
\label{sec:Auswertung}
Der Abstand der Thermoelemente der jeweiligen Stäbe beträgt
\begin{equation*}
    x = \SI{3}{\centi\meter}.
\end{equation*}
Die Werte für die Dichte \cite{V204}, die spezifische Wärme \cite{V204} und 
die Wärmeleitfähigkeit \cite{wiki} von Messing sind: 
\begin{align*}
    \rho &= \SI{8520}{\kilo\gram\per\cubic\meter} \\
    c &= \SI{385}{\joule\per\kilo\gram\per\kelvin} \\
    \kappa &= \SI{120}{\watt\per\meter\per\kelvin}.
\end{align*}
Die Werte für Aluminium \cite{V204}, \cite{wiki} sind:
\begin{align*}
    \rho &= \SI{2800}{\kilo\gram\per\cubic\meter} \\
    c &= \SI{830}{\joule\per\kilo\gram\per\kelvin} \\
    \kappa &= \SI{236}{\watt\per\meter\per\kelvin}.
\end{align*}
Die Werte für Edelstahl \cite{V204}, \cite{edelstahl} sind:
\begin{align*}
    \rho &= \SI{8000}{\kilo\gram\per\cubic\meter} \\
    c &= \SI{400}{\joule\per\kilo\gram\per\kelvin} \\
    \kappa &= \SI{21}{\watt\per\meter\per\kelvin}.
\end{align*}

\subsection{Statische Methode}

\begin{table}\caption{Der maximale Drehimpuls $L$, der Gesamtspin $S$ und der Gesamtdrehimpuls $J$ ergeben sich zum Landé-Faktor $g_\text{J}$ für die vier verschiedenen Elemente.}
\label{tab1}
\centering
\sisetup{round-mode = places, round-precision=2, round-integer-to-decimal=true}
\begin{tabular}{S[]S[]S[]S[]} 
\toprule
{$L$} & {$S$} & {$J$} & {$g_\text{J}$}\\
\midrule
5.0 & 1.0 & 4.0 & 0.8\\
0.0 & 3.5 & 3.5 & 2.0\\
6.0 & 1.5 & 4.5 & 0.7272727272727273\\
5.0 & 2.5 & 7.5 & 1.3333333333333333\\
\bottomrule
\end{tabular}\end{table}

Die Temperaturverläufe der fernen Thermoelemente
befinden sich im Anhang (I,II).
\newline
Der Wärmestrom für die fernen Thermoelemente für fünf Zeiten kann mittels
Gleichung \ref{eqn:dQ} bestimmt werden.
Für den breiten Messingstab (T1/T2) ergibt sich somit:
\begin{align*} %-0.84864 -0.6144  -0.53184 -0.50304 -0.49536
   \frac{\Delta Q}{\Delta t}(\SI{140}{\second}) &= \SI{-0.848}{\watt} \\ 
   \frac{\Delta Q}{\Delta t}(\SI{280}{\second}) &= \SI{-0.614}{\watt} \\ 
   \frac{\Delta Q}{\Delta t}(\SI{420}{\second}) &= \SI{-0.532}{\watt} \\ 
   \frac{\Delta Q}{\Delta t}(\SI{560}{\second}) &= \SI{-0.503}{\watt} \\ 
   \frac{\Delta Q}{\Delta t}(\SI{700}{\second}) &= \SI{-0.495}{\watt} \\ 
\end{align*}

Für den schmalen Messingstab (T3/T4) ergibt sich:
\begin{align*} %-0.35168 -0.30688 -0.27776 -0.26768 -0.26096
   \frac{\Delta Q}{\Delta t}(\SI{140}{\second}) &= \SI{-0.351}{\watt} \\ 
   \frac{\Delta Q}{\Delta t}(\SI{280}{\second}) &= \SI{-0.307}{\watt} \\ 
   \frac{\Delta Q}{\Delta t}(\SI{420}{\second}) &= \SI{-0.278}{\watt} \\ 
   \frac{\Delta Q}{\Delta t}(\SI{560}{\second}) &= \SI{-0.268}{\watt} \\ 
   \frac{\Delta Q}{\Delta t}(\SI{700}{\second}) &= \SI{-0.261}{\watt} \\ 
\end{align*}

Für den Aluminiumstab (T5/T6) ergibt sich:
\begin{align*} %-0.449344 -0.143488 -0.05664  -0.033984 -0.022656
   \frac{\Delta Q}{\Delta t}(\SI{140}{\second}) &= \SI{-0.449}{\watt} \\ 
   \frac{\Delta Q}{\Delta t}(\SI{280}{\second}) &= \SI{-0.143}{\watt} \\ 
   \frac{\Delta Q}{\Delta t}(\SI{420}{\second}) &= \SI{-0.057}{\watt} \\ 
   \frac{\Delta Q}{\Delta t}(\SI{560}{\second}) &= \SI{-0.034}{\watt} \\ 
   \frac{\Delta Q}{\Delta t}(\SI{700}{\second}) &= \SI{-0.023}{\watt} \\ 
\end{align*}

Für den Edelstahlstab (T7/T8) ergibt sich:
\begin{align*} %-0.320208 -0.33432  -0.313152 -0.29904  -0.289968
   \frac{\Delta Q}{\Delta t}(\SI{140}{\second}) &= \SI{-0.320}{\watt} \\ 
   \frac{\Delta Q}{\Delta t}(\SI{280}{\second}) &= \SI{-0.334}{\watt} \\ 
   \frac{\Delta Q}{\Delta t}(\SI{420}{\second}) &= \SI{-0.313}{\watt} \\ 
   \frac{\Delta Q}{\Delta t}(\SI{560}{\second}) &= \SI{-0.299}{\watt} \\ 
   \frac{\Delta Q}{\Delta t}(\SI{700}{\second}) &= \SI{-0.290}{\watt} \\ 
\end{align*}
\newline
Die Temperaturdifferenz der beiden Thermoelemente des Edelstahlstabs
ist graphisch dargestellt und befindet sich im Anhang (III).
Für den breiten Messingstab ist die Temperaturdifferenz der beiden
Thermoelemente ebenfalls graphisch dargestellt (IV).

\subsection{Dynamische Methode}
Die Auswertung der dynamischen Methode für den breiten Messingstab, 
den Aluminiumstab und den Edelstahlstab ist im Folgenden zu sehen.
\subsubsection{Breiter Messingstab}
\label{sec:messing}

\begin{table}\caption{Die Phasenverschiebung der T1- und T2-Funktion der Maxima und der Minima aus dem Plot von Seite V im Anhang.}
\label{tab3a}
\centering
\sisetup{round-mode = places, round-precision=2, round-integer-to-decimal=true}
\begin{tabular}{S[]S[]} 
\toprule
{$\varphi_\text{max}/ \si{\second}$} & {$\varphi_\text{min} /\si{\second}$}\\
\midrule
0.35 & 0.2\\
0.4 & 0.25\\
0.45 & 0.2\\
0.35 & 0.2\\
0.3 & 0.25\\
0.3 & 0.3\\
0.3 & 0.2\\
0.35 & 0.25\\
0.35 & 0.25\\
0.35 & 0.3\\
\bottomrule
\end{tabular}\end{table}
\begin{table}\caption{Die Amplitude der T1-Funktion und die Amplitude der T2-Funktion aus dem Plot von Seite V im Anhang.}
\label{tab3b}
\centering
\sisetup{round-mode = places, round-precision=2, round-integer-to-decimal=true}
\begin{tabular}{S[]S[]} 
\toprule
{$Amp_\text{T1}/ \si{\degreeCelsius}$} & {$Amp_\text{T2}/ \si{\degreeCelsius}$}\\
\midrule
8.7 & 3.1999999999999997\\
8.7 & 3.0\\
8.7 & 2.7\\
8.5 & 2.5999999999999996\\
8.4 & 2.7\\
8.399999999999999 & 2.5\\
8.3 & 2.6\\
8.2 & 2.5\\
8.3 & 2.4\\
\bottomrule
\end{tabular}\end{table}

Die Temperaturverläufe für den breiten Messingstab (Thermoelemente $T1$ und $T2$)
sind graphisch dargestellt (V).
\newline
Durch Abmessung der Amplituden $A1$ und $A2$ und der Phasendifferenzen $\Delta t$ 
und Mittelung dieser, ergeben sich folgende Werte:
\begin{align*}
    A_{1} &= \SI{8.47 \pm 0.18}{\kelvin} \\
    A_{2} &= \SI{2.69 \pm 0.24}{\kelvin} \\
    \Delta t &= \SI{14.8 \pm 1.5}{\second}.
\end{align*}
Aus diesen Werten lässt sich mittels Gleichung \eqref{eqn:Wärme} die Wärmeleitfähigkeit
bestimmen. Diese ergibt sich zu
\begin{equation*}
    \kappa = \SI{87 \pm 11}{\watt\per\meter\per\kelvin}.
\end{equation*}

\subsubsection{Aluminiumstab}

\begin{table}\caption{Die Phasenverschiebung der T5- und T6-Funktion der Maxima und der Minima aus dem Plot von Seite VI im Anhang.}
\label{tab4a}
\centering
\sisetup{round-mode = places, round-precision=2, round-integer-to-decimal=true}
\begin{tabular}{S[]S[]} 
\toprule
{$\varphi_\text{max} /\si{\second}$} & {$\varphi_\text{min} /\si{\second}$}\\
\midrule
12.5 & 7.5\\
10.0 & 7.5\\
10.0 & 7.5\\
10.0 & 7.5\\
10.0 & 10.0\\
10.0 & 7.5\\
10.0 & 7.5\\
15.0 & 10.0\\
10.0 & 10.0\\
\bottomrule
\end{tabular}\end{table}
\begin{table}\caption{Die Amplitude der T5-Funktion und die Amplitude der T6-Funktion aus dem Plot von Seite VI im Anhang.}
\label{tab4b}
\centering
\sisetup{round-mode = places, round-precision=2, round-integer-to-decimal=true}
\begin{tabular}{S[]S[]} 
\toprule
{$Amp_\text{T5}/ \si{\degreeCelsius}$} & {$Amp_\text{T6} /\si{\degreeCelsius}$}\\
\midrule
9.333333333333334 & 5.25\\
9.416666666666666 & 5.0\\
9.083333333333332 & 5.0\\
8.833333333333334 & 4.833333333333334\\
8.916666666666666 & 4.833333333333334\\
8.75 & 4.5\\
8.583333333333334 & 4.583333333333333\\
8.583333333333334 & 4.583333333333333\\
8.5 & 4.5\\
\bottomrule
\end{tabular}\end{table}

Die Temperaturverläufe für die Thermoelemente des Aluminiumstabes befinden sich im Anhang (VI).
Auf die selbe Weise wie in \ref{sec:messing} beschrieben lassen sich die Werte
für den Aluminiumstab bestimmen:
\begin{align*}
    A_{5} &= \SI{4.79 \pm 0.25}{\kelvin} \\
    A_{6} &= \SI{8.89 \pm 0.31}{\kelvin} \\
    \Delta t &= \SI{9.6 \pm 1.0}{\second}.
\end{align*}
Damit ergibt sich mit \eqref{eqn:Wärme} eine Wärmeleitfähigkeit von
\begin{equation*}
    \kappa = \SI{176 \pm 26}{\watt\per\meter\per\kelvin}.
\end{equation*}

\subsubsection{Edelstahlstab}

\begin{table}\caption{Die Phasenverschiebung der T7- und T8-Funktion der Maxima und der Minima aus dem Plot von Seite V im Anhang.}
\label{tab5}
\centering
\sisetup{round-mode = places, round-precision=2, round-integer-to-decimal=true}
\begin{tabular}{S[]S[]} 
\toprule
{$\varphi_\text{max} /\si{\second}$} & {$\varphi_\text{min} /\si{\second}$}\\
\midrule
0.5 & 0.3\\
0.55 & 0.3\\
0.45 & 0.35\\
0.5 & 0.35\\
0.5 & 0.35\\
0.55 & 0.35\\
0.45 & 0.35\\
0.45 & 0.35\\
0.5 & 0.35\\
0.55 & 0.45\\
\bottomrule
\end{tabular}\end{table}
\begin{table}\caption{Die Amplitude der T7-Funktion und die Amplitude der T8-Funktion aus dem Plot von Seite V im Anhang.}
\label{tab5b}
\centering
\sisetup{round-mode = places, round-precision=2, round-integer-to-decimal=true}
\begin{tabular}{S[]S[]} 
\toprule
{$Amp_\text{T7}/ \si{\degreeCelsius}$} & {$Amp_\text{T8} /\si{\degreeCelsius}$}\\
\midrule
3.2249999999999996 & 0.65\\
3.175 & 0.6\\
3.125 & 0.55\\
3.175 & 0.5\\
3.15 & 0.475\\
3.2 & 0.5\\
3.2 & 0.475\\
3.125 & 0.475\\
3.1500000000000004 & 0.475\\
3.175 & 0.47500000000000003\\
\bottomrule
\end{tabular}\end{table}

Die Temperaturverläufe für die Thermoelemente des Edelstahlstabes befinden sich im Anhang (VII).
Die Amplituden und die Phasendifferenz des Edelstahlstabes ergeben sich zu
\begin{align*}
    A_{7} &= \SI{12.19 \pm 0.12}{\kelvin} \\
    A_{8} &= \SI{1.99 \pm 0.23}{\kelvin} \\
    \Delta t &= \SI{53.1 \pm 3.4}{\second}.
\end{align*}
Die Wärmeleitfähigkeit ist mit Gleichung \eqref{eqn:Wärme}
\begin{equation*}
    \kappa = \SI{15.0 \pm 1.4}{\watt\per\meter\per\kelvin}.
\end{equation*}
