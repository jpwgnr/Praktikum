\section{Diskussion}
\label{sec:Diskussion}

Die lange Spule hat einen maximalen Wert von \SI{2.19}{\milli\tesla}. Der theoretisch berechnete Wert liegt bei %gleich viele Nachkommastellen angeben?
\SI{2.36}{\milli\tesla} und hat somit eine relative Abweichung von \SI{7.2}{\percent}. Ansonsten entsprechen die 
beobachteten Messungen dem Bild, das zu erwarten war.

\noindent Bei der kurzen Spule passen der Theoriewert im Maximum und die experimental gemessene Kurve nicht sonderlich gut zusammen. Die Abweichung im 
Hochpunkt liegt bei \SI{40}{\percent}, was recht viel ist. 

\noindent Das Spulenpaar, als Helmholtzspule, passt ebenfalls nicht zum Theoriewert. Der relative Fehler des Maximums in der Mitte 
des Paares liegt bei einem Wert von \SI{55.5}{\percent}. %Dein Einwand war berechtigt, aber der Satz war eh redundant, da wir ja keine Theoriekurven betrachten
Bei einem Strom von \SI{4}{\ampere} und einem Abstand der Spulen, der dem Durchmesser entspricht, liegt die relative Abweichung in der Mitte des Paares bei \SI{10.71}{\percent}. %"lag" Warte mal, soll ich hier im Präteritum schreiben oder nicht? Die Diskussion ist doch immer danach oder nicht? Also wenn ich jetzt schreibe die Apparatur "ist" wackelig stimmt das vielleicht, aber es geht doch darum, was bei uns falsch "war". Oder hattest du die schon korrigiert?
Bei der dritten Messung wurde ein %"hatte" 
relativer Fehler von \SI{6.84}{\percent} festgestellt. Somit liegen alle Fehler in einem Bereich von \SI{5}{\percent} 
bis \SI{60}{\percent}, was relativ viel ist. Grund dafür könnte die etwas wackelige Apparatur gewesen sein. %"gewesen"
Außerdem musste die Hall-Sonde immer exakt gleich ausgerichtet sein. Drehte man sie nur leicht, änderten sich die %"musste", "drehte", ..
gemessenen Werte bereits extrem.
\newline
Vermutlich war die Hallsonde bei der Messung der kurzen Spule und der ersten Messung %"war"
des Spulenpaares nicht ganz senkrecht, wodurch die starke Abweichung zu den Theoriewerten entstanden ist. 

\noindent Die Magnetisierung funktionierte recht gut. Die Hysteresekurve sieht genauso aus, wie sie zu erwarten war. %"funktionierte", "war"
