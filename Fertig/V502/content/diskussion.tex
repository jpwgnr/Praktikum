\section{Diskussion}
\label{sec:Diskussion}

\subsection{Elektrisches Feld}
Im Folgenden wird die Auswertung der Ablenkung eines Elektronenstrahls durch ein elektrisches Feld diskutiert. 

\subsubsection{Proportionalität zwischen Leuchtfleckverschiebung und Ablenkspannung}
Die Steigung des Fits hat einen relativen Fehler von \SI{2.14}{\percent} und weicht von der 
theoretischen Steigung um \SI{15.92}{\percent} ab. Damit sind die Ergebnisse als relativ gut zu bewerten.

\subsubsection{Bestimmung der Frequenz der Sinusspannung}
Die ermittelte Sinusfrequenz hat einen relativen Fehler von \SI{0.06}{\percent}. Auch hier ist der Fehler relativ klein, 
weshalb die Messung als ziemlich gut bewertet werden kann. 


\subsection{Magnetfeld}
In diesem Teil wird die Auswertung der Ablenkung eines Elektronenstrahls durch ein magnetisches Feld diskutiert. 

\subsubsection{Bestimmung der spezifischen Elektronenladung}
Für den Mittelwert der spezifischen Elektronenladung ergibt sich ein relativer Fehler von \SI{4.14}{\percent}. Die relative 
Abweichung zum Literaturwert beträgt \SI{17.56}{\percent}. Der relative Fehler ist zwar recht klein, aber trotzdem weicht das Ergebnis 
ziemlich stark ab. 
Grund dafür ist vermutlich, dass das Inklinatorium nicht sehr exakt war. Das lag vor allem daran, dass die Aufhängung nicht komplett reibungsfrei war. 


\subsubsection{Bestimmung der Intensität des lokal Erdmagnetfelds}
Das Erdmagnetfeld, das sich durch die Messung ergeben hat, weicht vom Literaturwert um \SI{34.6}{\percent} ab. Auch hierbei ist die Messung 
nicht sehr exakt gewesen, was wieder mit dem schlecht kalibrierten Inklinatorium in Verbindung gebracht werden kann.
Allerdings liegt das Ergebnis trotz der schlechten Messung in der gleichen Größenordnung.

\subsection{Fazit}
Abschließend kann gesagt werden, dass die Messung im ekeltrischen Feld deutlich exakter und näher an den Theoriewerten lag, als bei der 
zweiten Messung, bei der die Abweichung von den Literaturwerten doch als recht groß einzuschätzen ist. 
