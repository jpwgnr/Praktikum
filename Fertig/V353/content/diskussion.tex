\section{Diskussion}
\label{sec: dis}

Die Auswertung der Funktionen ist als recht exakt zu bewerten. 

\noindent Die gemessenen Werte ergeben nach der halblogarithmischen Auftragung einen Fit, der eine Steigung von 
$\SI[per-mode=fraction]{836.50}{\per\second}$ hat. Der relative Fehler der linearen Regression liegt bei $\SI{6.15}{\percent}$. 

\noindent Im Abschnitt \ref{sec: b} ergibt sich nach der Umformung der Funktion \eqref{eqn: A2}  auch eine lineare Gleichung.
Die Daten führen zu einer linearen Regression, deren Steigung bei \SI[per-mode=fraction]{885.68}{\per\second} liegt. Der relative
Fehler liegt bei \SI{2.197}{\percent}. Auffällig ist dabei, dass die Werte für \SI{10}{\hertz} und \SI{20}{\hertz} rausgelassen werden mussten,
um ein realistisches Ergebnis zu bekommen. Auch zwei weitere Werte weichen auffällig stark vom Fit ab. Die Ursache dafür könnte sein,
dass der Wert, der für die maximale Amplitude $U_0$ gewählt wurde, nach Anleitung demselben Wert entsprechen sollte, der als erstes gemessen wurde. Dies führt aber dazu, dass durch null geteilt wird. Also muss ein Wert gewählt werden, der etwas größer ist. Wäre dieser Wert exakt bestimmt worden, würden vermutlich auch die rausgenommenen Werte sinnvoll auf dem Fit liegen. 
Somit wird angenommen, dass der Wert bei mindestens \SI{621}{\milli\volt} liegt. 

\noindent Im Abschnitt \ref{sec: c} wird ein Wert von \SI[per-mode=fraction]{873.02}{\per\second} für die inverse Zeitkonstante $1/RC$ bei einem relativen
Fehler von \SI{3.056}{\percent} festgestellt.
Alle drei Ergebnisse überschneiden sich in einem Bereich von
\SI[per-mode=fraction]{877.09 \pm 10.92}{\per\second}. Somit ist eine systematische Abweichung nicht zu erkennen.
Die Werte für $1/RC$ im Abschnitt \ref{sec: a} und \ref{sec: b} weichen dennoch um \SI{5.55}{\percent} voneinander ab. Dieser Fehler könnte durch den nicht
betrachteten Innenwiderstand des Sinusfrequenzgenerators entstanden sein. Der Wert dieses Widerstands liegt laut Anleitung bei
\SI{600}{\ohm} \cite{versuch}. 

\noindent Die Funktion der Kondensatorspannung $U_{C}$ als Integrator der Spannung $U(t)$ scheint sich anhand der ermittelten
Schaubilder zu bestätigen. Die annähernde Korrektheit lässt sich zumindest gut an den Schaubildern erkennen,
indem die Abhängigkeit der Hoch- und Tiefpunkte von den Nullstellen der jeweils anderen Funktionen betrachtet wird. 

\noindent Die Abhängigkeit der Relativamplitude zur Phase $\phi$ lässt sich in den Polarkoordinatensystemen gut ablesen.
Dabei ist die Eigenschaft von $\phi$ als $arctan$-Funktion gut zu erkennen, da die gemessenen Werte den Werten der Theoriekurve
eindeutig ähneln, lediglich phasenverschoben sind. 
