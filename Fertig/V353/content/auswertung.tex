\section{Auswertung}
\label{sec:Auswertung}

Für die Berechnung vieler Ergebnisse wird der Wert $U_0$ benötigt. Dieser wird auf einen Wert von \SI{621}{\milli\volt} gesetzt, damit die Rechnungen funktionieren. \ref{sec: dis}

\subsection{Bestimmung von RC über den Entladevorgang des Kondensators}
\label{sec: a}
Die Werte, die für die Bestimmung der Zeitkonstanten $RC$ nötig sind, befinden sich in Tabelle \ref{tab1}.
\begin{table}\caption{Die Anzahl der Impulse, der Startwert auf der Mikrometerschraube und der Endwert auf der Mikrometerschraube.}
\label{taba}
\centering
\sisetup{round-mode = places, round-precision=2, round-integer-to-decimal=true}
\begin{tabular}{S[]S[]S[]} 
\toprule
{Anzahl} & {$d_\text{Start} / \si{\milli\meter}$} & {$d_\text{Start} / \si{\milli\meter}$}\\
\midrule
3001.0 & 6.73 & 2.0\\
3002.0 & 6.73 & 2.0\\
3000.0 & 1.82 & 6.5\\
3000.0 & 6.74 & 2.0\\
3000.0 & 1.83 & 6.5\\
3000.0 & 6.74 & 2.0\\
3001.0 & 1.84 & 6.5\\
3000.0 & 2.83 & 7.5\\
3001.0 & 7.77 & 3.0\\
3002.0 & 2.75 & 7.5\\
\bottomrule
\end{tabular}\end{table}
Die $x$ und $y$ Werte der linearen Regression \eqref{eqn: linreg1} sind in Tabelle \ref{taba} dargestellt.
Abbildung \ref{fig:plota} stellt die Gerade dieser linearen Regression dar.
\begin{table}\caption{Die Zeit $t$ gegen den negativen Logarithmus der Spannungswerte geteilt durch die maximale Spannung.}
\label{taba}
\centering
\sisetup{round-mode = places, round-precision=3, round-integer-to-decimal=true}
\begin{tabular}{S[]S[]} 
\toprule
{$t/ \si{\milli\second}$} & {$-log(\frac{U(t)}{U_{0}})$}\\
\midrule
0.0 & -0.0\\
0.4 & 0.13372577497521726\\
0.8 & 0.4281699018019215\\
1.2 & 0.705467664947986\\
1.6 & 0.9977516783331359\\
2.0 & 1.3015531326648002\\
2.4 & 1.623136756792262\\
2.8 & 1.8993901334204206\\
3.2 & 2.2178438645389553\\
3.6 & 2.505525936990737\\
4.0 & 2.793208009442516\\
4.4 & 3.19867311755068\\
4.8 & 3.604138225658845\\
5.2 & 4.29728540621879\\
5.6 & 4.990432586778735\\
\bottomrule
\end{tabular}\end{table}

\begin{figure}
  \centering
  \includegraphics[width=13cm, height=9cm]{build/plota.pdf}
  \caption{Auftragung des negativen Logarithmus der Spannung geteilt durch die maximale Spannung gegen die Zeit.
  Dargestellt werden die Daten, ein Fit und der Fehler des Fits.}
  \label{fig:plota}
\end{figure}

\noindent Die Steigung der Geraden ist das Inverse der Zeitkonstante.
Mittels Gleichung \eqref{eqn: RC} und der Gleichung für den Fehler \eqref{eqn: fehler} berechnet
sich das Inverse der Zeitkonstante zu $\frac{1}{RC} = \SI[per-mode=fraction]{836.500 \pm 51.514}{\per\second}$.

\subsection{Bestimmung von RC über die Amplitude der Kondensatorspannung}
\label{sec: b}
Die Kondensatorspannungen und die zeitlichen Phasenverschiebungen in Abhängigkeit
von der Frequenz sind in Tabelle \ref{tab2} dargstellt.
\begin{table}\caption{Die Frequenzen der Sägezahnspannung.}
\label{tabb}
\centering
\sisetup{round-mode = places, round-precision=2, round-integer-to-decimal=true}
\begin{tabular}{S[]S[]} 
\toprule
{Index} & {$\nu_\text{Sä} / \si{\hertz}$}\\
\midrule
1.0 & 25.02\\
2.0 & 49.95\\
3.0 & 99.99\\
4.0 & 149.97\\
\bottomrule
\end{tabular}\end{table}
Um die Zeitkonstante aus der Amplitude der Kondensatorspannung errechnen zu können, wird eine lineare Regression 
\eqref{eqn: linreg2} durchgeführt. Die $x$ und $y$ Werte dazu finden sich in Tabelle \ref{tabb}.
In Abbildung \ref{fig:plotb} sind diese $x$ und $y$ Werte gegeneinander aufgetragen.
\begin{table}\caption{Der Kehrwert der Kreisfrequenz $\omega$ gegen die Wurzel aus dem Bruch in dessen Nenner die maximale Spannung durch die Amplitudenwerte von $U_{C}$ zum Quadrat um eins subtrahiert werden}
\label{tabb}
\centering
\sisetup{round-mode = places, round-precision=5, round-integer-to-decimal=true}
\begin{tabular}{S[]S[]} 
\toprule
{$\frac{1}{\omega}/ \si{\second}$} & {$\sqrt{\frac{1}{(\frac{U_{0}}{A(\omega)})^{2}-1}}$}\\
\midrule
0.0024485375860291594 & 2.262660907951623\\
0.0019894367886486917 & 1.7091833258800144\\
0.0015915494309189533 & 1.2706566931710195\\
0.0006366197723675814 & 0.48280454958526764\\
0.00039788735772973834 & 0.2991883988251616\\
0.0002448537586029159 & 0.18505403427568887\\
0.00019894367886486917 & 0.14980117725462763\\
0.00015915494309189535 & 0.14980117725462763\\
6.366197723675813e-05 & 0.0483656490240811\\
3.978873577297384e-05 & 0.030287634503871775\\
2.448537586029159e-05 & 0.01884392449684891\\
1.989436788648692e-05 & 0.015621873649013022\\
1.5915494309189534e-05 & 0.01240030915098555\\
\bottomrule
\end{tabular}\end{table}

\begin{figure}
  \centering
  \includegraphics[width=13cm, height=9cm]{build/plotb.pdf}
  \caption{Auftragung der Werte aus der linearen Regression. Dargestellt werden die Daten, ein Fit und der Fehler des Fits.}
  \label{fig:plotb}
\end{figure}

\noindent Das Inverse der Zeitkonstante wird wieder durch \eqref{eqn: RC} und der Fehler durch \eqref{eqn: fehler}
berechnet. Es ergibt sich $\frac{1}{RC} = \SI[per-mode=fraction]{885.682  \pm 19.461}{\per\second}$.

\subsection{Bestimmung von RC über die Phasenverschiebung der Kondensatorspannung}
\label{sec: c}
Die Abstände der Nulldurchgänge der Spannung des Kondensators und der Spannung des Generators in Abhängigkeit
von der Frequenz sind bereits in Tabelle \ref{tab2} aufgelistet.
Um die Zeitkonstante aus dieser Phasenverschiebung zu bestimmen, wird die lineare Regression \eqref{eqn: linreg3} benötigt.
Die $x$ und $y$ Werte befinden sich in Tabelle \ref{tabc}. Abbildung \ref{fig:plotc} stellt die Gerade der
linearen Regression dar, deren Steigung das Inverse der Zeitkonstante ist.
\begin{table}\caption{Der Kehrwert der Kreisfrequenz gegen den negativen Kehrwert des Tangens der Phase, die sich durch die negative Division der zeitlichen Phasenverschiebung durch die Periodendauer multipliziert mit $\pi$ ergibt}
\label{tabc}
\centering
\sisetup{round-mode = places, round-precision=5, round-integer-to-decimal=true}
\begin{tabular}{S[]S[]} 
\toprule
{$\frac{1}{\omega}/ \si{\second}$} & {$-\frac{1}{tan(\phi(\omega))}$}\\
\midrule
0.0024485375860291594 & 3.117626187195526\\
0.0019894367886486917 & 2.723786837133246\\
0.0015915494309189533 & 2.2715973027329865\\
0.0006366197723675814 & 1.4714553158199692\\
0.00039788735772973834 & 1.2401991640705359\\
0.0002448537586029159 & 1.2010895549922853\\
0.00019894367886486917 & 1.120011792429241\\
0.00015915494309189535 & 1.0000000000000002\\
6.366197723675813e-05 & 0.8540806854634666\\
3.978873577297384e-05 & 1.0126459941540735\\
2.448537586029159e-05 & 1.0190294678742615\\
1.989436788648692e-05 & 1.064891840324792\\
1.5915494309189534e-05 & 1.064891840324792\\
\bottomrule
\end{tabular}\end{table}

\begin{figure}
  \centering
  \includegraphics[width=13cm, height=9cm]{build/plotc.pdf}
  \caption{Auftragung der Werte aus der linearen Regression. Dargestellt werden die Daten, ein Fit und der Fehler des Fits.}
  \label{fig:plotc}
\end{figure}

\noindent Mittels \eqref{eqn: RC} und \eqref{eqn: fehler} berechnet sich das Inverse der Zeitkonstante zu
$\frac{1}{RC} = \SI[per-mode=fraction]{873.015 \pm 26.684}{\per\second}$.

\subsection{Abhängigkeit der Relativamplitude von der Phasenverschiebung}
Die Werte, die zur Bestimmung der Abhängigkeit der Relativamplitude $\frac{A(\omega)}{U_{0}}$ von der
Phasenverschiebung $\phi$ nötig sind, befinden sich in Tabelle \ref{tabd}.
\begin{table}\caption{Die Phasenverschiebung gegen die Amplitude der Spannung $U_{C}$ geteilt durch die maximale Spannung $U_{0}}$
\label{tabd}
\centering
\sisetup{round-mode = places, round-precision=5, round-integer-to-decimal=true}
\begin{tabular}{S[]S[]} 
\toprule
{$\phi/ \si{\radian}$} & {$\frac{A(\omega)}{U_{0}}$}\\
\midrule
-0.31038935417467156 & 0.9146537842190016\\
-0.3518583772020568 & 0.8631239935587762\\
-0.4146902302738527 & 0.7858293075684379\\
-0.5969026041820606 & 0.4347826086956522\\
-0.6785840131753953 & 0.286634460547504\\
-0.6942919764433444 & 0.1819645732689211\\
-0.728849495632832 & 0.14814814814814814\\
-0.7853981633974483 & 0.14814814814814814\\
-0.8639379797371932 & 0.04830917874396135\\
-0.7791149780902686 & 0.03027375201288245\\
-0.7759733854366789 & 0.01884057971014493\\
-0.7539822368615503 & 0.015619967793880838\\
-0.7539822368615503 & 0.012399355877616747\\
\bottomrule
\end{tabular}\end{table}
Um die verschiedenen errechneten Werte für die Zeitkonstante $RC$ vergleichen
zu können, wird für die in \ref{sec: a}, \ref{sec: b} und \ref{sec: c} berechneten Werte in den folgenden Abbildungen
\ref{fig: plotd1}, \ref{fig: plotd2} und \ref{fig: plotd3} die Relativamplitude $\frac{A(\omega)}{U_{0}}$ als Radius
und die Phase $\phi$ als Winkel dargestellt. Die rote Linie entspricht den berechneten Werten 
$\frac{A(\omega)}{U_{0}}$, die durch Einsetzen von $\phi$ in die Formel \eqref{eqn: A1} bestimmt werden. Die grüne Linie entspricht der 
Theoriekurve, welche durch die Formel \eqref{eqn: phi} mit den Werten für $RC$ bestimmt wird.
\begin{figure}
  \centering
  \includegraphics[width=12cm, height=7cm]{build/plotd1.pdf}
  \caption{Abhängigkeit der Relativamplitude in Abhängigkeit von der Phase für die in \ref{sec: a} 
  berechnete Zeitkonstante in einem Polarkoordinatensystem dargestellt. Es sind Daten, ein Fit, der Fehler
  des Fits und die Theoriekurve eingezeichnet.}
  \label{fig: plotd1}
\end{figure}

\begin{figure}
  \centering
  \includegraphics[width=12cm, height=7cm]{build/plotd2.pdf}
  \caption{Abhängigkeit der Relativamplitude in Abhängigkeit von der Phase für die in \ref{sec: b} 
  berechnete Zeitkonstante in einem Polarkoordinatensystem dargestellt. Es sind Daten, ein Fit, der Fehler
  des Fits und die Theoriekurve eingezeichnet.}
  \label{fig: plotd2}
\end{figure}

\begin{figure}
  \centering
  \includegraphics[width=12cm, height=7cm]{build/plotd3.pdf}
  \caption{Abhängigkeit der Relativamplitude in Abhängigkeit von der Phase für die in \ref{sec: c} 
  berechnete Zeitkonstante in einem Polarkoordinatensystem dargestellt. Es sind Daten, ein Fit, der Fehler
  des Fits und die Theoriekurve eingezeichnet.}
  \label{fig: plotd3}
\end{figure}

\subsection{Spannung als Integrator}
%Integrator Bilder
Die folgenden Abbildungen zeigen die integrierte, sowie die zu integrierende Spannung.
Dabei wird eine Sinusspannung angelegt. Die Kondensatorspannung wird dadurch zu einem Cosinus. Die Dreiecksspannung wird zu einer Funktion integriert, die einem Sinus ähnelt. Die Rechteckspannung integriert sich zur Dreiecksspannung. \newline \newline \newline %Noch was?
\begin{figure}
  \centering
  \includegraphics[width=9cm, height=6cm]{build/integrator1.pdf}
  \caption{Aufnahme des Bildschirms des Oszillokops bei eingestellter Sinusspannung.}
  \label{fig: sinus}
\end{figure}

\begin{figure}
  \centering
  \includegraphics[width=9cm, height=6cm]{build/integrator2.pdf}
  \caption{Aufnahme des Bildschirms des Oszillokops bei eingestellter Dreiecksspannung.}
  \label{fig: dreieck}
\end{figure}

\begin{figure}
  \centering
  \includegraphics[width=9cm, height=6cm]{build/integrator3.pdf}
  \caption{Aufnahme des Bildschirms des Oszillokops bei eingestellter Rechteckspannung.}
  \label{fig: rechteck}
\end{figure}



