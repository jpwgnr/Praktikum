\section{Fehlerrechnung}

Der Mittelwert einer Stichprobe von $N$ Werten wird durch
\begin{equation}
    \overline{x} = \frac{1}{N} \sum_{i=1}^N x_i
    \label{eqn:mittelwert}
\end{equation}
bestimmt.
\newline
Die Standardabweichung der Stichprobe wird berechnet mit:
\begin{equation}
    \sigma_x = \sqrt{\frac{1}{N-1} \sum_{i=1}^N (x_i - \overline{x})^2}.
    \label{eqn:standard}
\end{equation}
\newline
Der realtive Fehler zwischen zwei Werten kann durch
\begin{equation*}
    \frac{a-b}{a}
\end{equation*}
bestimmt werden.
\newline
Bei der linearen Regression wird die Gerade
\begin{equation*}
    y(x) = mx + b
\end{equation*}
durch das Streudiagramm gelegt.
Dabei ist $m$ die Steigung mit
\begin{equation*}
    m = \frac{\overline{xy} - \overline{x} \cdot \overline{y}}{\overline{x^2} - \overline{x}^2}
\end{equation*}
und $b$ der $y$-Achsenabschnitt mit
\begin{equation*}
    b = \frac{\overline{y} \cdot \overline{x^2} - \overline{xy} \cdot \overline{x}}{\overline{x^2} - \overline{x}^2}.
\end{equation*}
\\

\noindent Der Fehler der Gleichung \ref{eqn:eta}, die zur Bestimmung der Viskosität $\eta$ verwendet wird, ergibt sich zu 

\begin{equation}
    \sigma_{\eta} = \sqrt{K^{2} \left(\rho_{K}-\rho_{Fl}\right)^{2} \sigma_{t}^{2}}.
    \label{eqn:erreta}
\end{equation}
\\
\noindent Der Fehler der Gleichung \ref{eqn:K}, die zur Ermittlung des Proportionalitätskonstante $K$ verwendet wird, ergibt sich zu 

\begin{equation}
    \sigma_{K} = \sqrt{\frac{\eta^{2} \sigma_{t}^{2}}{\left(\rho_{K}-\rho_{Fl}\right)^{2} t^{4}} + \frac{\sigma_{\eta}^{2}}{\left(\rho_{K}-\rho_{Fl}\right)^{2}t^{2}}}.
    \label{eqn:errK}
\end{equation}
\\
\noindent Der Fehler der Gleichung \ref{eqn:Re}, die zur Berechnung der Reynoldszahlen $Re$ benutzt wird, ergibt sich zu 

\begin{equation}
    \sigma_{Re} = \sqrt{\frac{d^{2} \rho^{2} x^{2} \sigma_{t}^{2} }{\eta^{2} t^{4}} + \frac{d^{2} \rho^{2} x^{2} \sigma_{\eta}^{2}}{\eta^{4} t^{2}}}.
    \label{eqn:errRe}
\end{equation}
