\section{Auswertung}
\label{sec:Auswertung}
Für die Auswertung werden Matplotlib \cite{matplotlib}, NumPy \cite{numpy}, SciPy \cite{scipy} 
und Uncertainties \cite{uncertainties} benutzt.

\subsection{Bestimmung der Apparaturkonstante für die große Kugel}
Die Fallstrecke beträgt 
\begin{equation*}
    x = \SI{10}{\centi\meter}.
\end{equation*}
Die Masse, der Durchmesser und die Apparaturkonstante der kleinen Kugel sind:
\begin{align*}
    m_\text{klein} &= \SI{3.71}{\gram} \\
    d_\text{klein} &= \SI{0.0156}{\meter} \\
    K_\text{klein} &= \SI{76.4}{\nano\pascal\cubic\meter\per\kilo\gram}.
\end{align*}
Die Masse und der Durchmesser der großen Kugel sind:
\begin{align*}
    m_\text{groß} &= \SI{4.21}{\gram} \\
    d_\text{groß} &= \SI{0.0158}{\meter}. \\
\end{align*}
Die Dichte des destillierten Wassers beträgt
\begin{equation*}
    \rho_\text{Fl} = \SI{998.2067}{\kilo\gram\per\cubic\meter}.
\end{equation*}
Aus der Masse und dem Volumen der Kugeln ergibt sich jeweils mittels Gleichung \eqref{eqn:dichte}
die Dichte der Kugel:
\begin{align*}
    \rho_\text{klein} &= \SI{1866.39}{\kilo\gram\per\cubic\meter} \\
    \rho_\text{groß} &= \SI{2038.51}{\kilo\gram\per\cubic\meter}.
\end{align*}
Die Falldauern der kleinen und der großen Kugel im Kugelfall-Viskosimeter
bei Raumtemperatur befinden sich in Tabelle \ref{tab1}.
%Daraus besser zwei Tabellen machen?: Nö, denke nicht. Oder wieso meinst du denn?
\begin{table}\caption{Erste Messung.}
\label{tab1}
\centering
\sisetup{round-mode = places, round-precision=1, round-integer-to-decimal=true}
\begin{tabular}{S[]S[]S[]} 
\toprule
{$g / \si{\centi\meter}$} & {$b / \si{\centi\meter}$} & {$B / \si{\centi\meter}$}\\
\midrule
12.700000000000003 & 38.9 & 8.4\\
13.700000000000003 & 32.2 & 6.5\\
14.700000000000003 & 28.200000000000003 & 5.3\\
15.700000000000003 & 25.299999999999997 & 4.4\\
16.700000000000003 & 22.5 & 3.7\\
17.700000000000003 & 21.299999999999997 & 3.3\\
18.700000000000003 & 19.799999999999997 & 3.0\\
19.700000000000003 & 18.5 & 2.6\\
20.700000000000003 & 17.799999999999997 & 2.4\\
21.700000000000003 & 17.400000000000006 & 2.2\\
\bottomrule
\end{tabular}\end{table}
\noindent Für die mittleren Falldauern ergibt sich mit den Gleichungen \eqref{eqn:mittelwert} 
und \eqref{eqn:standard}
\begin{align*}
    t_\text{klein} &= \SI{11.7 \pm 0.1}{\second} \\
    t_\text{groß} &= \SI{65.2 \pm 0.3}{\second}.
\end{align*}
Mit Gleichung \eqref{eqn:eta} berechnet sich die Viskosität mit $K_\text{klein}$,
$\rho_\text{klein}$ und $t_\text{klein}$ zu
\begin{equation*}
    \eta = \SI{773 \pm 7}{\micro\pascal\second}.
\end{equation*}
Mit diesem Wert folgt für die Apparaturkonstante $K_\text{groß}$
mittels Gleichung \eqref{eqn:K}
\begin{equation*}
    K_\text{groß} = \SI{11.40(11)}{\nano\pascal\cubic\meter\per\kilo\gram}.
\end{equation*}

\subsection{Bestimmung der Temperaturabhängigkeit der Viskosität von destilliertem Wasser}
Die Falldauern der großen Kugel für verschiedene Temperaturen sind in den Tabellen
\ref{tab2} und \ref{tab3} dargestellt.
%Daraus besser eine Tabelle machen?:
\begin{table}\caption{Die Spannung, die Stromstärke, die Anzahl der Impulse, die transportierte Ladungsmenge und die transporte Ladungsmenge in Einheiten der Elementarladung.}
\label{tab1}
\centering
\sisetup{round-mode = places, round-precision=2, round-integer-to-decimal=true}
\begin{tabular}{S[]S[] S[]@{${}\pm{}$}S[] S[]@{${}\pm{}$} S[] S[]@{${}\pm{}$} S[]} 
\toprule
{U / \si{\volt}} & {I / \si{\ampere}} & \multicolumn{2}{c}{N/second} &  \multicolumn{2}{c}{$\Delta Q / \si{\coulomb}$} &  \multicolumn{2}{c}{$\Delta Q \si{\elementarycharge}$}\\
\midrule
320.0 & 0.1     & 86.91 & 0.07 &  8.975  &  0.007  & 5.602   &  0.005e+19\\
400.0 & 0.2     & 90.92 & 0.07 & 17.157  &  0.014  & 1.0709  &  0.0009e+20\\
480.0 & 0.3     & 93.35 & 0.07 & 25.068  &  0.020  & 1.5646  &  0.0012e+20\\
540.0 & 0.35    & 94.62 & 0.07 & 28.851  &  0.023  & 1.8008  &  0.0014e+20\\
560.0 & 0.4     & 92.83 & 0.07 & 33.610  &  0.027  & 2.0977  &  0.0017e+20\\
600.0 & 0.45    & 95.03 & 0.07 & 36.935  &  0.029  & 2.3053  &  0.0018e+20\\
640.0 & 0.5     & 95.41 & 0.08 & 40.877  &  0.032  & 2.5514  &  0.0020e+20\\
660.0 & 0.55    & 96.21 & 0.08 & 44.591  &  0.035  & 2.7832  &  0.0022e+20\\
680.0 & 0.6     & 97.38 & 0.08 & 48.06   &  0.04   & 2.9997  &  0.0023e+20\\
\bottomrule
\end{tabular}\end{table}
\begin{table}\caption{Die Zeit des Durchschallungsverfahrens gegen die Länge des Zylinders.}
\label{tab3}
\centering
\sisetup{round-mode = places, round-precision=2, round-integer-to-decimal=true}
\begin{tabular}{S[]S[]} 
\toprule
{t/ \si{\second}} & {l/ \si{\meter}}\\
\midrule
8.95e-05 & 0.1208\\
7.8e-05 & 0.1023\\
5.93e-05 & 0.0805\\
3.08e-05 & 0.0404\\
2.47e-05 & 0.0311\\
\bottomrule
\end{tabular}\end{table}
\noindent Die Viskositäten für die Falldauern der ersten und der zweiten Messung
sind in Tabelle \ref{tab4} zu finden.
\begin{table}\caption{Der Abstand $r$ zwischen Leucht- und Photodiode aufgetragen gegen die Spannung U_{Out}. Dazu jeweils den Wert für die Verstärkung des Tiefpasses und des Detektors.}
\label{tab4}
\centering
\sisetup{round-mode = places, round-precision=1, round-integer-to-decimal=true}
\begin{tabular}{S[]S[]S[]S[]} 
\toprule
{$r / \si{\centi\meter}$} & {$U_{Out} / \si{\volt}$} & {Gain Tiefpass} & {Gain Detektor}\\
\midrule
10.0 & 4.0 & 20.0 & 100.0\\
15.0 & 4.1 & 50.0 & 100.0\\
20.0 & 4.2 & 100.0 & 100.0\\
25.0 & 5.3 & 200.0 & 100.0\\
30.0 & 8.7 & 500.0 & 100.0\\
35.0 & 6.5 & 500.0 & 100.0\\
40.0 & 4.9 & 500.0 & 100.0\\
45.0 & 7.6 & 1000.0 & 100.0\\
50.0 & 6.0 & 1000.0 & 100.0\\
55.00000000000001 & 5.0 & 1000.0 & 100.0\\
60.0 & 4.2 & 1000.0 & 100.0\\
65.0 & 7.1 & 1000.0 & 200.0\\
70.0 & 6.2 & 1000.0 & 200.0\\
75.0 & 5.4 & 1000.0 & 200.0\\
80.0 & 4.8 & 1000.0 & 200.0\\
85.0 & 4.2 & 1000.0 & 200.0\\
90.0 & 3.7 & 1000.0 & 200.0\\
95.0 & 9.0 & 1000.0 & 500.0\\
100.0 & 8.0 & 1000.0 & 500.0\\
\bottomrule
\end{tabular}\end{table}
\noindent Das Inverse der Zeit gegen die logarithmierte Viskosität für die erste
Messung ist in Tabelle \ref{tab5} und für die zweite Messung in Tabelle
\ref{tab6} dargestellt.
Diese Werte sind für die erste Messung in Abb. \ref{fig:plot1} und für
die zweite Messung in Abb. \ref{fig:plot2} gegeneinander aufgetragen.
\begin{table}\caption{Die invertierte Temperatur gegen die logarithmierte Viskosität für die erste Messung.}
\label{tab5}
\centering
\sisetup{round-mode = places, round-precision=1, round-integer-to-decimal=true}
\begin{tabular}{S[]S[]} 
\toprule
{$\frac{10^{3}}{T_1} /\si[per-mode=fraction]{\per\kelvin}$} & {$\eta_1 /\si{\pascal\second}$}\\
\midrule
3.0660738923808064 & -7.497305275002141\\
3.0473868657626086 & -7.5327861977670985\\
3.028926245645919 & -7.555420217220707\\
3.0197795560924057 & -7.591653556586626\\
3.0016509079994 & -7.621470911639022\\
2.9837386244964943 & -7.656207745688693\\
2.966038855109002 & -7.708174169692305\\
2.948547840188707 & -7.769145431988264\\
2.9312619082515026 & -7.860478126926399\\
2.914177473408131 & -8.001593998980967\\
\bottomrule
\end{tabular}\end{table}
\begin{table}\caption{Die invertierte Temperatur gegen die logarithmierte Viskosität für die zweite Messung.}
\label{tab5}
\centering


\begin{tabular}{S[table-format=1.3]  
        @{$ \:\:\:\: $}
        S[table-format=3.3]
        @{${} \pm{}$}
        S[table-format=1.3]} 
\toprule
 \multicolumn{2}{c}{$\frac{10^{3}}{T_1} /\si[per-mode=fraction]{\per\kelvin}$} & {$ln(\eta_1) /\si{\pascal\second}$}\\
\midrule
3.066 & -7.504 & 0.010\\
3.047 & -7.547 & 0.010\\
3.029 & -7.586 & 0.010\\
3.020 & -7.599 & 0.010\\
3.002 & -7.638 & 0.010\\
2.984 & -7.692 & 0.010\\
2.966 & -7.741 & 0.010\\
2.949 & -7.820 & 0.010\\
2.931 & -7.908 & 0.010\\
2.914 & -7.998 & 0.010\\
\bottomrule
\end{tabular}\end{table}

\begin{figure}
    \centering
    \includegraphics[width=14cm, height=10cm]{build/plot1.pdf}
    \caption{Die logarithmierte Viskosität ist gegen das Inverse
    der Falldauer der ersten Messung aufgetragen. Die Parameter
    aus Gleichung \eqref{eqn:temp} sind $A_1=\SI{0.566(515)}{\nano\pascal\second}$
    und $B_1=\SI{3012.12 \pm 304.10}{\kelvin}$.}
    \label{fig:plot1}
\end{figure}

\begin{figure}
    \centering
    \includegraphics[width=14cm, height=10cm]{build/plot2.pdf}
    \caption{Die logarithmierte Viskosität ist gegen das Inverse
    der Falldauer der zweiten Messung aufgetragen. Die Parameter
    aus Gleichung \eqref{eqn:temp} sind $A_2=\SI{0.376(251)}{\nano\pascal\second}$
    und $B_2=\SI{3140.87 \pm 223.67}{\kelvin}$.}
    \label{fig:plot2}
\end{figure}

\subsection{Bestimmung der Reynoldszahlen}
Die Geschwindigkeit der kleinen Kugel ergibt sich mit Gleichung \eqref{eqn:v}
und der Falldauer $t_\text{klein}$ zu
\begin{equation*}
    v_\text{klein} = \SI{8.58(08)}{\milli\meter\per\second}.
\end{equation*}
Mit Gleichung \eqref{eqn:Re} folgt für die Reynoldszahl für die kleine Kugel
\begin{equation*}
    Re_\text{klein} = \num{172.7 \pm 3.1}.
\end{equation*}
Die Geschwindigkeit der großen Kugel ist
\begin{equation*}
    v_\text{groß} = \SI{1.534(007)}{\milli\meter\per\second}.
\end{equation*}
Die Reynoldszahl für die große Kugel ergibt sich auf die gleiche Weise zu
\begin{equation*}
    Re_\text{groß} = \num{31.27 \pm 0.31}.
\end{equation*}

\noindent Die Reynoldszahlen für die große Kugel bei verschiedenen Temperaturen ergeben sich auf die gleiche Weise und sind in folgender Tabelle aufgetragen. 
\begin{table}\caption{Die Temperatur und die Reynoldszahlen der erste und zweite Messung.}
\label{tab7}
\centering
\sisetup{round-mode = places, round-precision=2, round-integer-to-decimal=true}
\begin{tabular}{S[]S[]S[]} 
\toprule
{$T /\si{\kelvin}$} & {$Re_1$} & {$Re_2$}\\
\midrule
326.15 & 60.8192940846345 & 61.58075867442081\\
328.15 & 65.29196107212914 & 67.15376369132561\\
330.15 & 68.31551907276062 & 72.56044280868363\\
331.15 & 73.44990716651733 & 74.5672655167899\\
333.15 & 77.9633113577189 & 80.516146521398\\
335.15 & 83.57229092824743 & 89.76242107453724\\
337.15 & 92.72562511274035 & 98.9476442151419\\
339.15 & 104.75113452403667 & 115.98891142199052\\
341.15 & 125.74458914652912 & 138.2887489038005\\
343.15 & 166.74813886555594 & 165.69035598392597\\
\bottomrule
\end{tabular}\end{table}

