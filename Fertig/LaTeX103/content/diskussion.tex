\section{Diskussion}
\label{sec:Diskussion}

% Tabellen und Plot zu E1, erste Stange
In der ersten Messung bei dem goldenen Stab mit einer quadratischen Querschnittsfläche und einer Länge von $\SI{55}{\centi\meter}$, ergibt sich für den Elastizitätsmodul ein Wert, der weit von den zu erwartenden Werten abweicht. Die Messdaten liegen alle, bis auf den unteren und oberen Rand, ziemlich genau auf einer Ausgleichsgeraden, deren Steigung durch die Auswertung auf einen deutlich zu hohen Wert hindeutet. Ursachen für die Abweichung könnten verschiedene Fehler, eventuell sogar systematische, die nicht erkannt wurden, sein. Zum einen sprang die Messuhr teilweise zwischen Stücken $ \Delta x$, in denen gar keine Steigung mit dem bloßen Auge zu erkennen war und bei denen beim erneuten Überfahren des Bereichs keine Veränderung stattfand. Dies deutet darauf hin, dass die gemessenen Ergebnisse der Uhr nicht fehlerfrei sind, wie in der Auswertung angenommen. 
Zum anderen ist, obwohl der Körper $K$ mit seiner Masse $m$ bereits $\SI{17.6}{\percent}$ über der Masse der Stange lag, die maximale Auslenkung($\SI{1.64}{\milli\meter}$), die stattgefunden hat, zu niedrig gewesen. Diese sollte mindestens $\SI{3}{\milli\meter}$ betragen. Da während des Messens die Differenzen der Auslenkungen noch nicht überprüft wurden, ließ sich diese Fehlerquelle nicht früh genug erkennen. 
Eine weitere Ursache für die Fehler an den Rändern der Messung ist, dass die Stäbe dort kaum ausgelenkt waren und daher schon leichte Deformationen der Stäbe zu Unregelmäßigkeiten in der Messung geführt haben. Je nachdem wie stark die Stange eingespannt war, änderte sich die Ausrichtung ein wenig und die Stäbe waren leicht nach oben gerichtet und nicht im $90^\circ$ Winkel zur Erdoberfläche.  
Auch mit der Dichte, bestimmt aus dem Volumen und dem Gewicht der Stange, lässt sich die Stoffanalyse der Stange nicht genau durchführen. Mit einer Dichte von $\SI{13,076}{\kilo\gram\per\cubic\deci\meter}$ lässt sich kein exakter Stoff kombinieren. Mögliche Stoffe wären Quecksilber ($\SI{13,55}{\kilo\gram\per\cubic\deci\meter}$) oder Blei($\SI{11,34}{\kilo\gram\per\cubic\deci\meter}$), wobei Quecksilber bei Raumtemperatur flüssig ist und somit als Element der Stange keinen Sinn ergibt. Wäre der zu messende Stoff aus Blei gewesen, müsste der Elastizitätsmodul bei einem Wert von $E =\SI{19}{\giga\pascal}$ liegen. Der gemessene Wert liegt bei $E =\SI{459}{\giga\pascal}$. Somit ist der relative Fehler $\SI{2315}{\percent}$, falls es sich tatsächlich um Blei handelt. 


% Tabellen und Plot zu E2, zweite Stange
\noindent Bei der Messung des silbernen Stabes mit der runden Grundfläche und einer Länge von $\SI{55}{\centi\meter}$ wurde der Elastizitätsmodul  $E = \SI{160,12}{\giga\pascal}$ gemessen. Dabei ist diese Messung deutlich exakter gewesen, denn die Ausgleichsgerade hat keine Werte, die wirklich abweichen. Außerdem lag die maximale Auslenkung bei $\SI{6,23}{\milli\meter}$, also zwischen $3$ und $\SI{7}{\milli\meter}$. Somit scheinen systematische Fehler bei dieser Messung ausgeschlossen zu sein. Trotzdem lässt sich der Wert keinem Metall exakt zuordnen. Die Dichte, die bei $\SI{3.98}{\kilo\gram\per\cubic\deci\meter}$ liegt, lässt auf Aluminium($\SI{2,7}{\kilo\gram\per\cubic\deci\meter}$) schließen. Der Elastizitätsmodul liegt nur bei $E =\SI{70}{\giga\pascal}$, was einen Fehler von $\SI{128,7}{\percent}$ bedeutet, falls es sich tatsächlich um Aluminium handelt.

\noindent Bei der dritten Messung wurde das Gewicht in die Mitte gehängt. Einer der groben Fehler dabei könnte sein, dass zwei verschiedene Messuhren benutzt wurden. Die berechneten Werte hätten sich den Erwartungen nach symmetrisch zur Mitte der Stange äquivalent zueinander verhalten sollen. Tatsächlich unterscheiden sich die maximalen Auslenkungen der links und rechts des Gewichts gemessenen Werte um $\SI{45,95}{\percent}$ ($\SI{0,4}{\milli\meter}$ und $\SI{0,74}{\milli\meter}$). 
Die maximale Auslenkung liegt somit auch wieder weit unter dem für eine exakte Messung geforderten Wert von mindestens $\SI{3}{\milli\meter}$. 
Der hier gemessene Stab hat eine Länge von $\SI{60}{\centi\meter}$ und eine kreisförmigen Grundfläche, was mit seinem Gewicht zu einer Dichte von $\SI{13,073}{\kilo\gram\per\cubic\deci\meter}$ führt. Diese liegt in der Nähe der Dichte des ersten Stabes. Wie beim ersten Stab ist der Wert für Blei hier der passenste, wobei der Elastizitätsmodul ebenfalls stark abweicht. 
Der von rechts gemessene Modul liegt bei einem Wert von $E =\SI{226,349}{\giga\pascal}$, was einer Abweichung von $\SI{1091}{\percent}$ entspricht. Der von links gemessene Modul liegt bei $E =\SI{166,05}{\giga\pascal}$, was eine Abweichung von $\SI{773}{\percent}$ bedeutet. 
\noindent Insgesamt ist das Verfahren in der durchgeführten Art und Weise für den Zweck der Stoffanalyse als ziemlich ungenau zu bewerten. 


