\section{Diskussion}
\label{sec:Diskussion}

Die Messung des Widerstands bei der Wheatstone Brückenschaltung liefert bei dem Widerstand Nummer \num{13} 
einen Wert von \SI{319.8}{\ohm} mit einem relativen Fehler von \SI{0.38}{\percent} und beim Widerstand Nummer 
\num{14} einen Wert von \SI{899}{\ohm} mit einem relative Fehler von \SI{1.00}{\percent}. 

\noindent Für den Widerstand des Bauteils Nummer \num{8} ergibt sich ein Wert von 
\SI{510}{\ohm} und ein relativer Fehler von \SI{7.84}{\percent}. Für die Kapazität ergibt sich ein Wert von \SI{293}{\nano\farad} mit einem relativen Fehler von \SI{8.87}{\percent}.

\noindent Die Induktivität der Spule Nummer \num{19} wurde auf einen Wert von \SI{26.9}{\milli\henry} bestimmt. 
Der relative Fehler liegt dabei bei \SI{2.97}{\percent}. Der Widerstand hatte einen Wert von \SI{100}{\ohm} 
und sein relativer Fehler wurde auf \SI{4}{\percent} bestimmt. 
Die Spule wurde mit der Maxwell-Brücke erneut gemessen. Dabei kam für die Spule ein Wert von \SI{26.1}{\milli\henry} 
und für den relativen Fehler ein Wert von \SI{1.91}{\percent} heraus. Die Werte des Widerstands lagen bei 
\SI{98.9}{\ohm} und \SI{1.82}{\percent}. Die relative Abweichung der Induktivität der Spulen beträgt \SI{2.97}{\percent}. Die relative Abweichung liegt für die Widerstände bei \SI{1.1}{\percent}.

\noindent Die Messung der Frequenzabhängigkeit der Brückenspannung der Wien-Robinson-Brücke lässt einen Wert für 
$\omega_0$ bei \SI[per-mode=fraction]{2513.27}{\per\second} erkennen. Der theoretische Wert liegt bei \SI[per-mode=fraction]{2380.95}{\per\second}. Die Abweichung dieser 
beiden Werte liegt bei \SI{5.26}{\percent}. Der Klirrfaktor liegt bei \num{0.148}, was relativ hoch ist.

\noindent Die Werte liegen alle ziemlich genau in dem zu erwartenden Bereich und auch die Abweichung ist nicht besonders groß. Somit ist die Messung als recht exakt zu bewerten. 
