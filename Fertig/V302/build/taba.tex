\begin{table}\caption{ Die verschiedenen Werte der bekannten Widerstände der Wheatstoneschen Brücke. In Zeile 1-3 werden die Werte für den Widerstand 13 angegeben. In Zeile 4-6 für Widerstand 14.}
\label{taba}
\centering
\sisetup{round-mode = places, round-precision=1, round-integer-to-decimal=true}
\begin{tabular}{S[]S[]S[]} 
\toprule
{$R_3/\si{\ohm}$} & {$R_4/\si{\ohm}$} & {$R2/\si{\ohm}$}\\
\midrule
491.0 & 509.0 & 332.0\\
324.0 & 676.0 & 664.0\\
243.0 & 757.0 & 1000.0\\
728.0 & 272.0 & 332.0\\
578.0 & 422.0 & 664.0\\
473.0 & 527.0 & 1000.0\\
\bottomrule
\end{tabular}\end{table}