\section{Auswertung}
\label{sec:Auswertung}


\subsection{Zählrohr-Charakteristik}

Im Folgenden wird die Zählrohr-Charakteristik eines Halogenzählrohrs bestimmt. Die Werte für die angelegte Spannung $U$ und die Anzahl der gemessenen Impulse sind in Tab. \ref{taba} dargestellt und die normierten und fehlerbehafteten Werte für die Anzahl der Impulse befinden sich daneben. 

\begin{table}\caption{Die Anzahl der Impulse, der Startwert auf der Mikrometerschraube und der Endwert auf der Mikrometerschraube.}
\label{taba}
\centering
\sisetup{round-mode = places, round-precision=2, round-integer-to-decimal=true}
\begin{tabular}{S[]S[]S[]} 
\toprule
{Anzahl} & {$d_\text{Start} / \si{\milli\meter}$} & {$d_\text{Start} / \si{\milli\meter}$}\\
\midrule
3001.0 & 6.73 & 2.0\\
3002.0 & 6.73 & 2.0\\
3000.0 & 1.82 & 6.5\\
3000.0 & 6.74 & 2.0\\
3000.0 & 1.83 & 6.5\\
3000.0 & 6.74 & 2.0\\
3001.0 & 1.84 & 6.5\\
3000.0 & 2.83 & 7.5\\
3001.0 & 7.77 & 3.0\\
3002.0 & 2.75 & 7.5\\
\bottomrule
\end{tabular}\end{table}

\noindent Die Anzahl der Impulse wird in einer Zeit von 
\begin{equation*}
    \Delta t = \SI{130}{\second}
\end{equation*}
gemessen. 

\noindent In Abb. \ref{fig1} sind diese Werte gegeneinander aufgetragen und es wird eine Ausgleichsgerade im Bereich von $\num{360}$ bis $\SI{620}{\volt}$ mit der allgemeinen Gleichung \eqref{linReg} für die lineare Regressionen bestimmt. 

\begin{figure}
    \centering
    \includegraphics[width=15cm, height=9cm]{build/plot1.pdf}
    \caption{Die Zählrate $N$ ist gegen die angelegte Spannung $U$ aufgetragen. Es sind die Daten mit Fehlern sowie eine Ausgleichsgerade im Bereich von $\num{360}$ bis $\SI{620}{\volt}$ eingetragen.}
    \label{fig1}
\end{figure}

\noindent Die lineare Regression ergibt als Parameter

\begin{align*} 
   m &= \SI{0.012(4)}{\per\volt\per\second} \\
   n &= \SI{86.9(20)}{\per\second}.
\end{align*}

\noindent Mit diesen Parametern lässt sich aus den Werten für die Anzahl bei \num{450} und \SI{550}{\volt} die Steigung des Plateaus mit Gleichung \eqref{steigung} bestimmen.
Die Steigung des Plateaus beträgt 

\begin{align*} 
    m_\text{Plateau} = \num{1.3(4)} \frac{\si{\percent}}{\SI{100}{\volt}}.
\end{align*}

\subsection{Totzeitbestimmung}

Bei einer Spannung von $\SI{500}{\volt}$ ergibt sich durch das Ablesen vom Oszilloskop (Abb. \ref{foto}) eine Totzeit von 
\begin{equation*}
    T_\text{Tot,1} = \SI{54(10)}{\micro\second}.
\end{equation*}

\begin{figure}
    \centering
    \includegraphics[width=12cm, height=8cm]{build/foto.jpg}
    \caption{Das Foto des Bildschrims des Oszilloskops. Die $x$-Achse entspricht der Zeit (\SI{20}{\micro\second} pro Kästchen) und die $y$-Achse (\SI{1}{\volt} pro Kästchen) der Spannung. Die beiden rechten Kurven entsprechen den Nachentladungen.}
    \label{foto}
\end{figure}

\noindent Die Totzeit lässt sich auch durch die Zwei-Quellen-Methode berechnen. 
Bei der Messung haben sich für die erste und zweite Quelle, sowie für die Kombination aus beiden folgende Werte ergeben
\begin{align*} 
   N_1 &= 9730 \frac{1}{\SI{60}{\second}},\\
   N_2 &= 11918 \frac{1}{\SI{60}{\second}}, \\
   N_{1+2} &= 21187 \frac{1}{\SI{60}{\second}}.
\end{align*}

\noindent Diese Werte werden durch die Messdauer von
\begin{equation*}
    \Delta t = \SI{60}{\second}
\end{equation*}
geteilt und ein Fehler entsteht ebenfalls dadurch %?

\begin{align*} 
   N_1 &= \SI{162.17(10)}{\per\second},\\
   N_2 &= \SI{198.63(11)}{\per\second}, \\
   N_{1+2} &=\SI{353.12(14)}{\per\second}.
\end{align*}

\noindent Daraus ergibt sich für die Totzeit mit Gleichung \eqref{totzeit}
ein Wert von 
\begin{equation*} 
    T_\text{Tot,2} = \SI{119.3(31)}{\micro\second}.
\end{equation*} 

\subsection{Transportierte Ladungsmenge}
In Tab. \ref{tabb} stehen die Anzahl der gemessenen Impulse und die Stromstärke $I$ bei verschiedenen angelegten Spannungen $U$.
In Abb. \ref{fig2} sind die Spannung und die Stromstärke gegeneinander aufgetragen. 

\begin{table}\caption{Die Frequenzen der Sägezahnspannung.}
\label{tabb}
\centering
\sisetup{round-mode = places, round-precision=2, round-integer-to-decimal=true}
\begin{tabular}{S[]S[]} 
\toprule
{Index} & {$\nu_\text{Sä} / \si{\hertz}$}\\
\midrule
1.0 & 25.02\\
2.0 & 49.95\\
3.0 & 99.99\\
4.0 & 149.97\\
\bottomrule
\end{tabular}\end{table}

\begin{figure}
    \centering
    \includegraphics[width=15cm, height=9cm]{build/plot2.pdf}
    \caption{Die Stromstärke $I$ ist gegen die angelegte Spannung $U$ aufgetragen. Es sind nur die Daten eingetragen.}
    \label{fig2}
\end{figure}

\noindent Aus den Werten für die Stromstärke und der Anzahl der Impulse ergibt sich mit Gleichung \eqref{ladung} die transportierte Ladungsmenge $\Delta Q$ und daraus wiederum die transportierte Ladungsmenge in Abhängigkeit von der Elementarladung $e$. Die Ergebnisse sind in Tab. \ref{tab2} eingetragen. 
%Muss N hier nicht die Anzahl der Impulse in 130s sein?

\begin{table}\caption{Die Spannung, die Stromstärke, die Anzahl der Impulse, die transportierte Ladungsmenge und die transporte Ladungsmenge in Einheiten der Elementarladung.}
\label{tab1}
\centering
\sisetup{round-mode = places, round-precision=2, round-integer-to-decimal=true}
\begin{tabular}{S[]S[] S[]@{${}\pm{}$}S[] S[]@{${}\pm{}$} S[] S[]@{${}\pm{}$} S[]} 
\toprule
{U / \si{\volt}} & {I / \si{\ampere}} & \multicolumn{2}{c}{N/second} &  \multicolumn{2}{c}{$\Delta Q / \si{\coulomb}$} &  \multicolumn{2}{c}{$\Delta Q \si{\elementarycharge}$}\\
\midrule
320.0 & 0.1     & 86.91 & 0.07 &  8.975  &  0.007  & 5.602   &  0.005e+19\\
400.0 & 0.2     & 90.92 & 0.07 & 17.157  &  0.014  & 1.0709  &  0.0009e+20\\
480.0 & 0.3     & 93.35 & 0.07 & 25.068  &  0.020  & 1.5646  &  0.0012e+20\\
540.0 & 0.35    & 94.62 & 0.07 & 28.851  &  0.023  & 1.8008  &  0.0014e+20\\
560.0 & 0.4     & 92.83 & 0.07 & 33.610  &  0.027  & 2.0977  &  0.0017e+20\\
600.0 & 0.45    & 95.03 & 0.07 & 36.935  &  0.029  & 2.3053  &  0.0018e+20\\
640.0 & 0.5     & 95.41 & 0.08 & 40.877  &  0.032  & 2.5514  &  0.0020e+20\\
660.0 & 0.55    & 96.21 & 0.08 & 44.591  &  0.035  & 2.7832  &  0.0022e+20\\
680.0 & 0.6     & 97.38 & 0.08 & 48.06   &  0.04   & 2.9997  &  0.0023e+20\\
\bottomrule
\end{tabular}\end{table}


