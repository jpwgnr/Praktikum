\begin{table}\caption{Die angelegte Spannung des elektrischen Feldes innerhalb des Geiger-Müller-Zählrohrs und die Anzahl der jeweils gemessenen Impulse.}
\label{taba}
\centering
\sisetup{round-mode = places, round-precision=1, round-integer-to-decimal=true}
\begin{tabular}{c c r @{${}\pm{}$} S[]} 
\toprule
{$U / \si{\volt}$} & {$N / \frac{1}{130\,\si{\second}}$} & \multicolumn{2}{c}{$N / \frac{1}{\si{\second}}$}\\
\midrule
320 & 11298 & 86,9 & 0.8\\
340 & 11674 & 89,8 & 0.8\\
360 & 11921 & 91,7 & 0.8\\
380 & 11839 & 91,1 & 0.8\\
400 & 11820 & 90,9 & 0.8\\
420 & 12087 & 93,0 & 0.8\\
440 & 11940 & 91,8 & 0.8\\
460 & 12259 & 94,3 & 0.9\\
480 & 12135 & 93,3 & 0.8\\
500 & 12013 & 92,4 & 0.8\\
520 & 12099 & 93,1 & 0.8\\
540 & 12301 & 94,6 & 0.9\\
560 & 12068 & 92,8 & 0.8\\
580 & 12097 & 93,1 & 0.8\\
600 & 12354 & 95,0 & 0.9\\
620 & 12289 & 94,5 & 0.9\\
640 & 12403 & 95,4 & 0.9\\
660 & 12507 & 96,2 & 0.9\\
680 & 12659 & 97,4 & 0.9\\
\bottomrule
\end{tabular}\end{table}