\section{Auswertung}
\label{sec:Auswertung}

Die aufgenommenen Werte der Spannungen $U_{C}$ wurden mit dem Schalter am Tastkopf
auf $X10$ gemessen. Deshalb werden die Werte mit einem Faktor $10$ multipliziert.
%a)
\subsection{Effektiver Widerstand und Abklingdauer} 
Die Zeit und die Amplitude der Spannung sind in Tabelle
\ref{tab1} dargestellt. Um den effektiven Dämpfungswiderstand
zu berechnen, werden die Werte linearisiert. Diese Werte sind in
Tabelle \ref{taba} zu finden. In Abb. \ref{fig:plota}
sind die linearisierten Werte dargestellt. Es wird der
Logarithmus der Spannung gegen die Zeit aufgetragen.
\begin{table}\caption{Erste Messung.}
\label{tab1}
\centering
\sisetup{round-mode = places, round-precision=1, round-integer-to-decimal=true}
\begin{tabular}{S[]S[]S[]} 
\toprule
{$g / \si{\centi\meter}$} & {$b / \si{\centi\meter}$} & {$B / \si{\centi\meter}$}\\
\midrule
12.700000000000003 & 38.9 & 8.4\\
13.700000000000003 & 32.2 & 6.5\\
14.700000000000003 & 28.200000000000003 & 5.3\\
15.700000000000003 & 25.299999999999997 & 4.4\\
16.700000000000003 & 22.5 & 3.7\\
17.700000000000003 & 21.299999999999997 & 3.3\\
18.700000000000003 & 19.799999999999997 & 3.0\\
19.700000000000003 & 18.5 & 2.6\\
20.700000000000003 & 17.799999999999997 & 2.4\\
21.700000000000003 & 17.400000000000006 & 2.2\\
\bottomrule
\end{tabular}\end{table}
\begin{table}\caption{Die Anzahl der Impulse, der Startwert auf der Mikrometerschraube und der Endwert auf der Mikrometerschraube.}
\label{taba}
\centering
\sisetup{round-mode = places, round-precision=2, round-integer-to-decimal=true}
\begin{tabular}{S[]S[]S[]} 
\toprule
{Anzahl} & {$d_\text{Start} / \si{\milli\meter}$} & {$d_\text{Start} / \si{\milli\meter}$}\\
\midrule
3001.0 & 6.73 & 2.0\\
3002.0 & 6.73 & 2.0\\
3000.0 & 1.82 & 6.5\\
3000.0 & 6.74 & 2.0\\
3000.0 & 1.83 & 6.5\\
3000.0 & 6.74 & 2.0\\
3001.0 & 1.84 & 6.5\\
3000.0 & 2.83 & 7.5\\
3001.0 & 7.77 & 3.0\\
3002.0 & 2.75 & 7.5\\
\bottomrule
\end{tabular}\end{table}
\begin{figure}
  \centering
  \includegraphics{build/plota.pdf}
  \caption{Die logarithmierte Spannung ist gegen die Zeit aufgetragen.} %Erklären, was die Steigung ist
  \label{fig:plota}
\end{figure}
\noindent Daraus ergibt sich aus der Steigung und durch die Gleichung \eqref{eqn:reff}
der effektive Widerstand zu $R_{eff} = \SI{34.2 \pm 1.5}{\ohm}$.
Die Abklingdauer \ref{eqn:t_ex} ist somit $T_{ex} = \SI{0.000590 \pm 0.000026}{\second}$.
%Hier wurde jetzt dieselbe Gleichung für den experimentalen
%und den theoretischen Wert benutzt.
\newline
Der theoretische effektive Widerstand \eqref{eqn:reff} ist $R_{eff,theo} = \SI{48.1 \pm 0.1}{\ohm}$ %Gleichung angeben
und die Abklingdauer \eqref{eqn:t_ex} $T_{ex,theo} = \SI{0.0004204 \pm 0.0000015}{\second}$.

\subsection{Dämpfungswiderstand beim aperiodischen Grenzfall}
%b)
Der Wert für den Dämpfungswiderstand beim aperiodischen Grenzfall
lässt sich ablesen. Er ist $R_{ap} = \SI{3500.0}{\ohm}$.
\newline
Der theoretisch berechnete Wert, der mittels \eqref{eqn:r_ap}
bestimmt wird, ist $R_{ap,theo} = (4390 \pm 9) \si{\ohm}$.

\subsection{Resonanzüberhöhung und Breite der Resonanzkurve}
%c)
Tabelle \ref{tab2} beinhaltet die Spannung der Kondensators
sowie die Abstände der Nulldurchgänge der Kondensatorspannung
und der Erregerspannung zu verschiedenen Frequenzen.
Die Werte zur Bestimmung der Güte und der Breite der Resonanzkurve
sind in Tabelle \ref{tabc} aufgelistet.
Die Abbildung \ref{fig:plotc} stellt diese Werte dar.
\begin{table}\caption{Die Spannung, die Stromstärke, die Anzahl der Impulse, die transportierte Ladungsmenge und die transporte Ladungsmenge in Einheiten der Elementarladung.}
\label{tab1}
\centering
\sisetup{round-mode = places, round-precision=2, round-integer-to-decimal=true}
\begin{tabular}{S[]S[] S[]@{${}\pm{}$}S[] S[]@{${}\pm{}$} S[] S[]@{${}\pm{}$} S[]} 
\toprule
{U / \si{\volt}} & {I / \si{\ampere}} & \multicolumn{2}{c}{N/second} &  \multicolumn{2}{c}{$\Delta Q / \si{\coulomb}$} &  \multicolumn{2}{c}{$\Delta Q \si{\elementarycharge}$}\\
\midrule
320.0 & 0.1     & 86.91 & 0.07 &  8.975  &  0.007  & 5.602   &  0.005e+19\\
400.0 & 0.2     & 90.92 & 0.07 & 17.157  &  0.014  & 1.0709  &  0.0009e+20\\
480.0 & 0.3     & 93.35 & 0.07 & 25.068  &  0.020  & 1.5646  &  0.0012e+20\\
540.0 & 0.35    & 94.62 & 0.07 & 28.851  &  0.023  & 1.8008  &  0.0014e+20\\
560.0 & 0.4     & 92.83 & 0.07 & 33.610  &  0.027  & 2.0977  &  0.0017e+20\\
600.0 & 0.45    & 95.03 & 0.07 & 36.935  &  0.029  & 2.3053  &  0.0018e+20\\
640.0 & 0.5     & 95.41 & 0.08 & 40.877  &  0.032  & 2.5514  &  0.0020e+20\\
660.0 & 0.55    & 96.21 & 0.08 & 44.591  &  0.035  & 2.7832  &  0.0022e+20\\
680.0 & 0.6     & 97.38 & 0.08 & 48.06   &  0.04   & 2.9997  &  0.0023e+20\\
\bottomrule
\end{tabular}\end{table}
\begin{table}\caption{Der magnetische Fluss $B$ des gemessenen Magnetfelds gegen den Strom $I$ des erzeugenden Magnetfelds, Neukurve.}
\label{tabc}
\centering
\sisetup{round-mode = places, round-precision=1, round-integer-to-decimal=true}
\begin{tabular}{S[]S[]} 
\toprule
{$B$/ \si{\milli\tesla}} & {$I$/ \si{\ampere}}\\
\midrule
0.0 & 0.0\\
111.19999999999999 & 1.0\\
273.5 & 2.0\\
397.8 & 3.0\\
479.9 & 4.0\\
537.9000000000001 & 5.0\\
585.0999999999999 & 6.0\\
621.8000000000001 & 7.0\\
653.1 & 8.0\\
679.9 & 9.0\\
704.3000000000001 & 10.0\\
\bottomrule
\end{tabular}\end{table}
\begin{figure}
  \centering
  \includegraphics{build/plotc.pdf}
  \caption{Die Amplitude der Kondensatorspannung geteilt durch die Generatorspannung
  ist gegen die Kreisfrequenz aufgetragen. Das Maximum ist die Güte. Es liegt bei der
  Resonanzfrequenz.}
  \label{fig:plotc}
\end{figure}
\noindent Daraus lässt sich der Wert für die Resonanzüberhöhung, bzw. die Güte $q$ entnehmen.
Sie ist nach Gleichung \eqref{eqn:ucmax} das Maximum.
%Die Resonanzfrequenz ist der zugehörige $\omega$-Wert.
Also ergibt sich für die Güte $q = \num{3.40}$.
%und für die Resonanzfrequenz $\omega_{res} = \SI[per-mode=fraction]{213628.30}{\per\second}$.
Die Breite der Resonanzkurve ist mit Gleichung \eqref{eqn:breite}
$\omega_{+} - \omega_{-} = \SI[per-mode=fraction]{62831.85}{\per\second}$.
\newline
Der theoretisch errechnete Wert für die Güte, der mit Gleichung
\eqref{eqn:q} bestimmt werden kann, ist $q_{theo} = \num{4.309 \pm.010}$.
Der mit Gleichung \eqref{eqn:diff} berechnete Wert für die Breite der
Resonanzkurve ist
$\omega_{+,theo} - \omega_{-,theo} = \SI[per-mode=fraction]{5.040(016)e4}{\per\second}$.

\subsection{Resonanzfrequenz} %und omega_1, omega_2?
%d)
Die Abstände der Nulldurchgänge der Kondensatorspannung und
der Erregerspannung sind bereits in Tabelle \ref{tab2}
dargestellt. In Tabelle \ref{tabd} befinden sich die
zur Berechnung von $\omega_{res}$, $\omega_{1}$ und $\omega_{2}$
nötigen Werte, welche auch in Abb. \ref{fig:plotd} gegeneinander
aufgetragen sind. Die Phase wird dabei durch Gleichung \eqref{eqn:phi}
bestimmt.
\begin{table}\caption{Kreisfrequenz $\omega$ gegen die Phasenverschiebung $\varphi$ der Kondensatorspannung $U_C$ und der Generatorspannungi $U_0$.}
\label{tabd}
\centering
\sisetup{round-mode = places, round-precision=2, round-integer-to-decimal=true}
\begin{tabular}{S[]S[]} 
\toprule
{$\omega\cdot 10^{5}$ /\si[per-mode=fraction]{\per\second}} & {$Phase \varphi$}\\
\midrule
0.5654866776461628 & 0.12440706908215582\\
0.6911503837897545 & 0.11058406140636072\\
0.8168140899333463 & 0.13069025438933538\\
0.9424777960769379 & 0.1696460032938488\\
1.0681415022205296 & 0.16022122533307945\\
1.1938052083641213 & 0.20294688542190062\\
1.319468914507713 & 0.23750440461138836\\
1.4451326206513049 & 0.26012387171723483\\
1.5707963267948966 & 0.34557519189487723\\
1.6964600329384882 & 0.4750088092227767\\
1.8221237390820801 & 0.546637121724624\\
1.8849555921538759 & 0.6785840131753952\\
1.9477874452256716 & 0.818070726994782\\
2.0106192982974673 & 0.9650972631827843\\
2.0734511513692637 & 1.1611326447667876\\
2.1362830044410592 & 1.4099467829310992\\
2.199114857512855 & 1.6713272917097701\\
2.261946710584651 & 1.9000352368911066\\
2.324778563656447 & 2.092300707290802\\
2.3876104167282426 & 2.244353791724548\\
2.450442269800039 & 2.4014334244040376\\
2.5761059759436304 & 2.5761059759436304\\
2.701769682087222 & 2.6477342884454775\\
2.827433388230814 & 2.770884720466197\\
2.9530970943744057 & 2.894035152486917\\
3.078760800517997 & 2.8940351524869175\\
3.204424506661589 & 2.948070546128662\\
3.330088212805181 & 2.9970793915246623\\
3.4557519189487724 & 2.9719466502959446\\
3.581415625092364 & 3.008389125077586\\
3.7070793312359562 & 3.0398050516134836\\
\bottomrule
\end{tabular}\end{table}
\begin{figure}
  \centering
  \includegraphics{build/plotd.pdf}
  \caption{Die Phasenverschiebung der Kondensatorspannung und der
  Generatorspannung ist gegen die Kreisfrequenz aufgetragen.}
  \label{fig:plotd}
\end{figure}
\noindent Aus Abbildung \ref{fig:plotd} kann man die gesuchten Werte in Verbindung mit Tabelle
\ref{tabc} entnehmen.
Die Resonanzfrequenz liegt an der Stelle des Maximums und ist somit
 $\omega_{res} = \SI[per-mode=fraction]{213628.30}{\per\second}$.
Die Werte $\omega_{1}$ und $\omega_{2}$ sind näherungsweise $\omega_{-}$ und $\omega_{+}$.
$\omega_{1}$ und $\omega_{2}$ haben die Werte
$\omega_{1} = \SI[per-mode=fraction]{182212.37}{\per\second}$ und
$\omega_{2} = \SI[per-mode=fraction]{245044.23}{\per\second}$.
\newline
Die Resonanzfrequenz lässt sich in der Theorie durch 
Gleichung \eqref{eqn:omega_res} bestimmen.
Der Wert für diese ist $\omega_{res,theo} = \SI[per-mode=fraction]{2.142(004)e5}{\per\second}$.
Für die Frequenzen $\omega_{1,theo}$ und $\omega_{2,theo}$ ergibt sich
mittels Gleichung \eqref{eqn:omega_12}:
$\omega_{1,theo} = \SI[per-mode=fraction]{1.934(004)e5}{\per\second}$ und
$\omega_{2,theo} = \SI[per-mode=fraction]{2.438(005)e5}{\per\second}$.
