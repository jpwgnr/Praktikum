\documentclass[
  bibliography=totoc,     % Literatur im Inhaltsverzeichnis
  captions=tableheading,  % Tabellenüberschriften
  titlepage=firstiscover, % Titelseite ist Deckblatt
]{scrartcl}


% Paket float verbessern
\usepackage{scrhack}

% Warnung, falls nochmal kompiliert werden muss
\usepackage[aux]{rerunfilecheck}

% unverzichtbare Mathe-Befehle
\usepackage{amsmath}
% viele Mathe-Symbole
\usepackage{amssymb}
% Erweiterungen für amsmath
\usepackage{mathtools}

% Fonteinstellungen
\usepackage{fontspec}
% Latin Modern Fonts werden automatisch geladen
% Alternativ:
%\setromanfont{Libertinus Serif}
%\setsansfont{Libertinus Sans}
%\setmonofont{Libertinus Mono}
\recalctypearea % Wenn man andere Schriftarten gesetzt hat,
% sollte man das Seiten-Layout neu berechnen lassen

% deutsche Spracheinstellungen
\usepackage{polyglossia}
\setmainlanguage{german}


\usepackage[
  math-style=ISO,    % ┐
  bold-style=ISO,    % │
  sans-style=italic, % │ ISO-Standard folgen
  nabla=upright,     % │
  partial=upright,   % ┘
  warnings-off={           % ┐
    mathtools-colon,       % │ unnötige Warnungen ausschalten
    mathtools-overbracket, % │
  },                       % ┘
]{unicode-math}

% traditionelle Fonts für Mathematik
\setmathfont{Latin Modern Math}
% Alternativ:
%\setmathfont{Libertinus Math}

\setmathfont{XITS Math}[range={scr, bfscr}]
\setmathfont{XITS Math}[range={cal, bfcal}, StylisticSet=1]

% Zahlen und Einheiten
\usepackage[
  locale=DE,                   % deutsche Einstellungen
  separate-uncertainty=true,   % immer Fehler mit \pm
  per-mode=symbol-or-fraction, % / in inline math, fraction in display math
]{siunitx}

% chemische Formeln
\usepackage[
  version=4,
  math-greek=default, % ┐ mit unicode-math zusammenarbeiten
  text-greek=default, % ┘
]{mhchem}

% richtige Anführungszeichen
\usepackage[autostyle]{csquotes}

% schöne Brüche im Text
\usepackage{xfrac}

% Standardplatzierung für Floats einstellen
\usepackage{float}
\floatplacement{figure}{H}
\floatplacement{table}{H}

% Floats innerhalb einer Section halten
\usepackage[
  section, % Floats innerhalb der Section halten
  below,   % unterhalb der Section aber auf der selben Seite ist ok
]{placeins}

%dassselbe für Subsections 
\makeatletter
\AtBeginDocument{%
  \expandafter\renewcommand\expandafter\subsection\expandafter{%
    \expandafter\@fb@secFB\subsection
  }%
}
\makeatother

% Seite drehen für breite Tabellen: landscape Umgebung
\usepackage{pdflscape}

% Captions schöner machen.
\usepackage[
  labelfont=bf,        % Tabelle x: Abbildung y: ist jetzt fett
  font=small,          % Schrift etwas kleiner als Dokument
  width=0.9\textwidth, % maximale Breite einer Caption schmaler
]{caption}
% subfigure, subtable, subref
\usepackage{subcaption}

% Grafiken können eingebunden werden
\usepackage{graphicx}
% größere Variation von Dateinamen möglich
\usepackage{grffile}

% schöne Tabellen
\usepackage{booktabs}

% Verbesserungen am Schriftbild
\usepackage{microtype}

% Literaturverzeichnis
\usepackage[
  backend=biber,
]{biblatex}
% Quellendatenbank
\addbibresource{programme.bib}

% Hyperlinks im Dokument
\usepackage[
  unicode,        % Unicode in PDF-Attributen erlauben
  pdfusetitle,    % Titel, Autoren und Datum als PDF-Attribute
  pdfcreator={},  % ┐ PDF-Attribute säubern
  pdfproducer={}, % ┘
]{hyperref}
% erweiterte Bookmarks im PDF
\usepackage{bookmark}

% Trennung von Wörtern mit Strichen
\usepackage[shortcuts]{extdash}

%selbst hinzugefügt
\usepackage{physics}

\title{V703 - Das Geiger-Müller-Zählrohr}
\date{Durchführung: 28.05.2019, Abgabe: 04.06.2019}
\author{
  Jan Herdieckerhoff
  \texorpdfstring{%
    \\%
    \href{mailto:jan.herdieckerhoff@tu-dortmund.de}{jan.herdieckerhoff@tu-dortmund.de}
  }{}%
  \texorpdfstring{\and}{, }
  Karina Overhoff
  \texorpdfstring{%
    \\%
    \href{mailto:karina.overhoff@tu-dortmund.de}{karina.overhoff@tu-dortmund.de}
  }{}%
}
\publishers{TU Dortmund – Fakultät Physik}

\begin{document}
\section{Ziel}
Ziel dieses Versuches ist es, die Elastizitätsmodule
verschiedener Stäbe durch Messung ihrer Biegung
zu bestimmen.
\section{Theorie}
%Spannung
Die Spannung ist die Kraft auf einen Körper pro Flächeneinheit.
Die Komponente, die senkrecht zur Oberfläche steht,
ist die Normalspannung $\sigma$. Ihre oberflächenparallele
Komponente heißt Tangentialspannung.
%Hookesches Gesetz
Das Hookesche Gesetz stellt den Zusammenhang zwischen
der Spannung $\sigma$, die am Körper angreift, und der
Deformation des Körpers dar:
\begin{equation}
\sigma = E \frac{\Delta L}{L}
\label{eqn:Hooke}
\end{equation}
%Elastizitätsmodul
$E$ ist dabei das Elastizitätsmodul.
Das Elastizitätsmodul ist eine Materialkonstante, die
anhand der Deformation eines Körpers bestimmt werden kann.
%Biegung
Eine Art der Deformation ist die Biegung. Sie entsteht, wenn
eine Kraft, wie in Abbildung 1 und in Abbildung 2 gezeigt, auf einen Körper wirkt. %Abbildung 1 erwähnen
%Berechnung der Durchbiegung, einseitig
Zunächst wird die Berechnung der Biegung eines Stabes bei einseitiger
Einspannung beschrieben. %Abbildung 1
Die Durchbiegung $D(x)$ bezeichnet die Verschiebung eines Oberflächenpunktes
an der Stelle x zwischen dem belasteten und unbelasteten Zustand des Stabes.
Es wird eine Drehmomentgleichung aufgestellt, um $D(x)$ zu bestimmen.
Die Zug- und Druckspannungen, die an der Querschnittsfläche $Q$ angreifen,
sind entgegesetzt gleich und bewirken deshalb ein Drehmoment $M_{\sigma}$:
\begin{equation}
M_{\sigma} = \int_{Q} y \sigma(y) dq
\end{equation}
$y$ ist der Abstand des Flächenelementes $dq$ von der neutralen
Faser. Die neutrale Faser ist die Fläche, in der keine Spannungen
auftreten. Ihre Länge ändert sich bei der Biegung folglich nicht.
Ein weiteres Drehmoment $M_{F}$ entsteht durch die Kraft auf einen senkrecht
zur Stabachse stehenden Querschnitt. Es verdreht den Querschnitt aus 
seiner ursprünglichen vertikalen Lage.
Die Deformation des Körpers stellt sich so ein, dass die Drehmomente an
jeder Stelle $x$ übereinstimmen:
\begin{equation}
M_{F} = M_{\sigma}.
\end{equation}
Dabei ist
\begin{equation}
M_{F} = F (L-x),
\end{equation}
da die Kraft $F$ über den Hebelarm $L-x$ an $Q$ angreift.
Damit ist das Gleichgewicht der Drehmomente durch
\begin{equation}
\int_{Q} y \sigma(y) dq = F(L-x)
\label{eqn:Momente}
\end{equation}
gegeben.
Mit dem Hookeschen Gesetz \eqref{eqn:Hooke} wird die Normalspannung
$\sigma(y)$ mittels
\begin{equation}
\sigma(y) = E \frac{\delta x}{\Delta x}
\end{equation}
berechnet. Hier ist $\Delta x$ die Länge eines kurzen Stabstücks
und $\delta x$ die Längenänderung der Faser.
Es gilt außerdem
\begin{equation}
\delta x = y \Delta \phi = y \frac{\Delta x}{R},
\end{equation}
wobei $R$ der Krümmungsradius der Faser bei $x$ ist.
Damit ist
\begin{equation}
\sigma(y) = E \frac{y}{R} = E y \frac{d^2D}{dx^2},
\end{equation}
da für geringe Kurvenkrümmungen
\begin{equation}
\frac{1}{R} = \frac{d^2D}{dx^2} %nicht gleich sondern ungefähr!
\end{equation}
gilt, falls
\begin{equation}
(\frac{dD}{dx})^2 << 1
\end{equation}
ist. Für \eqref{eqn:Momente} ergibt sich damit:
\begin{equation}
E \frac{d^2D}{dx^2} \int_{Q} y^2 dq = F(L-x).
\label{eqn:Momente2}
\end{equation}
Dabei ist
\begin{equation}
I = \int_{Q} y^2 dq(y)
\end{equation}
das Flächenträgheitsmoment.
Integriert man \eqref{eqn:Momente2} und stellt die Gleichung
nach $D(x)$ um, erhält man für die Biegung bei einseitiger Einspannung
\begin{equation}
D(x) = \frac{F}{2EI} (Lx^2- \frac{x^3}{3}).
\label{eqn:D1}
\end{equation}
Diese Gleichung ist für $0 \leq x \leq L$ definiert.
Die Integrationskonstanten verschwinden, weil $D(0) = 0$ und $\frac{dD}{dx} = 0$ sein müssen.
%Berechnung der Durchbiegung, zweiseitig
Liegen beide Stabenden auf und lässt man in der Mitte des Stabes
eine Kraft angreifen, greift an der Querschnittsfläche die Kraft
$\frac{F}{2}$ mit dem Hebelarm $x$ an. Für die erste Stabhälfte
gilt für das Drehmoment
\begin{equation}
M_{F} = - \frac{F}{2} x, 0 \leq x \leq \frac{L}{2}.
\end{equation}
Für die zweite Hälfte gilt
\begin{equation}
M_{F} = - \frac{F}{2} (L-x), \frac{L}{2} \leq x \leq L.
\end{equation}
Damit ergibt sich hier für \eqref{eqn:Momente2}
\begin{equation}
\frac{d^2D}{dx^2} = - \frac{F}{EI} \frac{x}{2}, 0 \leq x \leq \frac{L}{2}
\label{eqn:links}
\end{equation}
und
\begin{equation}
\frac{d^2D}{dx^2} = -\frac{1}{2} \frac{F}{EI} (L-x), \frac{L}{2} \leq x \leq L.
\label{eqn:rechts}
\end{equation}
Integriert man beide Gleichungen, ergibt sich 
\begin{equation}
\frac{dD}{dx} = - \frac{F}{EI} \frac{x^2}{4} + C , 0 \leq x \leq \frac{L}{2}
\end{equation}
und
\begin{equation}
\frac{dD}{dx} = - \frac{1}{2} \frac{F}{EI} (Lx-\frac{x^2}{2}) + C' , \frac{L}{2} \leq x \leq L.
\end{equation}
Da die Biegekurve in der Mitte des Stabes eine horizontale Tangente
haben muss, muss für die Konstanten gelten:
\begin{equation}
C = \frac{F}{EI} \frac{L^2}{16}
\end{equation}
und
\begin{equation}
C' = \frac{3}{16} \frac{F}{EI} L^2.
\end{equation}.
Setzt man die Konstanten in \eqref{eqn:links} und \eqref{eqn:rechts}
ein und integriert die Ausdrücke, erhält man die Gleichungen
für die Biegung bei zweiseitiger Auflage des Stabes:
\begin{equation}
D(x) = \frac{F}{48EI} (3L^2x - 4x^3), 0 \leq x \leq \frac{L}{2}
\label{eqn:D2links}
\end{equation}
und
\begin{equation}
D(x) = \frac{F}{48EI} (4x^3 - 12Lx^2 + 9L^2x - L^3) , \frac{L}{2} \leq x \leq L.
\label{eqn:D2rechts}
\end{equation}
Die Integrationskonstanten verschwinden hier, weil $D(0) = 0$ und $D(L) = 0$ sein müssen.
Die Biegung eines elastischen Stabes kann also durch
\eqref{eqn:D1}, \eqref{eqn:D2links} und \eqref{D2rechts} bestimmen.
%D(x)=D_M(x)-D_0(x)
Weil die Stäbe nicht als exakt gerade angenommen werden können,
muss die Biegung ohne angehängtes Gewicht gemessen werden.
Die Biegung des Stabes durch die Last ist dann 
\begin{equation}
D(x) = D_{M}(x) - D_{0}(x).
\end{equation}
%Diskussion: Wert hat sich bei Erschütterung geändert
\end{document}