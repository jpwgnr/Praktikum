\section{Diskussion}
\label{sec:Diskussion}

% Tabellen und Plot zu E1, erste Stange
In der ersten Messung bei dem goldenen Stab, mit einer quadratischen Querschnittsfläche und einer Länge von $\SI{55}{\centi\meter}$, ergibt sich für den Elastizitätsmodul ein Wert, der weit von den zu erwartenden Werten abweicht. Die Messdaten liegen alle, bis auf den unteren und oberen Rand, ziemlich genau auf einer Ausgleichsgeraden, deren Steigung durch die Auswertung auf einen deutlich zu hohen Wert hindeutet. Ursachen für die Abweichung könnten verschiedene Fehler sein, eventuell sogar systematische Fehler, die nicht erkannt wurden. Zum einen sprang die Messuhr teilweise zwischen Stücken $ \Delta x$, in denen gar keine Steigung mit dem bloßen Auge zu erkennen war und bei denen beim erneuten Überfahren des Bereichs auch keine Veränderung stattfand. Dies deutet darauf hin, dass die gemessenen Ergebnisse der Uhr nicht so fehlerfrei sind, wie nun für die Auswertung angenommen werden sollte. 
Zum anderen ist, obwohl der Körper $K$ mit seiner Masse $m$ bereits $\SI{17.6}{\percent}$ über der Masse der Stange lag, die maximale Auslenkung, die stattgefunden hat, zu niedrig gewesen ($\SI{1.64}{\milli\meter}$). Diese sollte mindestens $\SI{3}{\milli\meter}$ betragen, aber da während des Messens die Differenzen der Auslenkung noch nicht überprüft wurden, ließ sich diese Fehlerquelle nicht früh genug erkennen. 
Aber auch mit der Dichte, bestimmt aus dem Volumen der Stange und dem Gewicht derselbigen, lässt sich die Stoffanalyse der Stange nicht genau erkennen. Mit einer Dichte von $\SI{13,076}{\kilo\gram\per\cubic\deci\meter}$ lässt sich kein exakter Stoff kombinieren. Mögliche Stoffe wären Quecksilber ($\SI{13,55}{\kilo\gram\per\cubic\deci\meter}$) oder Blei($\SI{11,34}{\kilo\gram\per\cubic\deci\meter}$), wobei Quecksilber bei Raumtemperatur im flüssigen Zustand ist. Wäre der zu messende Stoff aus Blei gewesen, müsste der Elastizitätsmodul bei einem Wert von $\SI{19}{\giga\pascal}$ liegen. Der gemessene Wert liegt bei $\SI{459}{\giga\pascal}$. Somit liegt der relative Fehler, falls es sich tatsächlich um Blei gehandelt haben sollte bei $\SI{2315}{\percent}$ Abweichung. 
\noindent

% Tabellen und Plot zu E2, zweite Stange
Bei der Messung des silbernen Stabs, mit der runden Grundfläche und einer Länge von $\SI{55}{\centi\meter}$ wurde der Elastizitätsmodul $\SI{160,12}{\giga\pascal}$ gemessen. Dabei ist diese Messung deutlich exakter gewesen, denn sowohl die Ausgleichsgerade hat keine Werte die wirklich abweichen, aber auch die maximale Auslenkung lag bei $\SI{6,23}{\milli\meter}$ also zwischen $3$ und $\SI{7}{\milli\meter}$. Somit scheinen systematische Fehler bei dieser Messung ausgeschlossen zu sein. Trotzdem lässt sich der Wert keinem Metall exakt zuordnen. Der Dichte nach, die bei $\SI{3.98}{\kilo\gram\per\cubic\deci\meter}$ liegt, lässt auf Aluminium($\SI{2,7}{\kilo\gram\per\cubic\deci\meter}$) schließen. Der Modul liegt dabei nur bei $\SI{70}{\giga\pascal}$, was auf einen Fehler von $\SI{128,7}{\percent}$ hindeutet, vorausgesetzt es handelt sich tatsächlich um Alluminium.
\noindent
Bei der dritten Messung wurde das Gewicht in die Mitte gehängt. Einer der groben Fehler könnte dabei sein, dass zwei verschiedene Messuhren benutzt wurden. Die berechneten Werte hätten sich den Erwartungen nach symmetrischen zur Mitte der Stange äquivalent zu einander verhalten sollen, aber tatsächlich unterscheiden sich die maximalen Auslenkungen der von links und rechts des Gewichts gemessenen Werte um $\SI{45,95}{\percent}$ ($\SI{0,4}{\milli\meter}$ und $\SI{0,74}{\milli\meter}$). 
Die maximale Auslenkung liegt somit auch wieder weit unter den, für eine exakte Messung geforderten Werte von mindestens $\SI{3}{\milli\meter}$ Auslenkung. 
Der hier gemessene Stab hat eine Länge von $\SI{60}{\centi\meter}$ und hat eine kreisförmigen Grundfläche, was mit seinem Gewicht zu einer Dichte von $\SI{13,073}{\kilo\gram\per\cubic\deci\meter}$ führt. Dieser liegt sehr in der Nähe des in der ersten Messung verwendeten Stabes. Wie auch dort ist der Wert für Blei der passenste, wobei auch dort der Elastizitätsmodul stark abweicht. 
Der von rechts gemessene Modul liegt bei einem Wert von $\SI{226,349}{\giga\pascal}$, was einer Abweichung von $\SI{1091}{\percent}$ entspricht. Der von links gemessene Modul liegt bei $\SI{166,05}{\giga\pascal}$, was eine Abweichung von $\SI{773}{\percent}$ bedeutet. 
\noindent

Insgesamt ist das Verfahren in der durchgeführten Art und Weise für den Zweck der Stoffanalyse als ziemlich ungenau zu bewerten. 


