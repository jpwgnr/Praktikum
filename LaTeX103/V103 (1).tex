%new header 
% This example is meant to be compiled with lualatex or xelatex
% The theme itself also supports pdflatex
\PassOptionsToPackage{unicode}{hyperref}
\documentclass[aspectratio=1610, 9pt]{beamer}

% Load packages you need here
\usepackage{polyglossia}
\usefonttheme{professionalfonts}
%\setmainlanguage{english}

\usepackage{csquotes}
    

\usepackage{amsmath}
\usepackage{amssymb}
\usepackage{mathtools}

\usepackage{hyperref}
\usepackage{bookmark}

% Paket float verbessern
\usepackage{scrhack}

% Warnung, falls nochmal kompiliert werden muss

% unverzichtbare Mathe-Befehle
\usepackage{amsmath}
% viele Mathe-Symbole
\usepackage{amssymb}
% Erweiterungen für amsmath
\usepackage{mathtools}

% Fonteinstellungen
\usepackage{fontspec}
% Latin Modern Fonts werden automatisch geladen
% Alternativ:
%\setromanfont{Libertinus Serif}
%\setsansfont{Libertinus Sans}
%\setmonofont{Libertinus Mono}
% sollte man das Seiten-Layout neu berechnen lassen

% deutsche Spracheinstellungen
%\usepackage{polyglossia}
\setmainlanguage{english}

\usepackage[
  math-style=ISO,    % ┐
  bold-style=ISO,    % │
  sans-style=italic, % │ ISO-Standard folgen
  nabla=upright,     % │
  partial=upright,   % ┘
  warnings-off={           % ┐
    mathtools-colon,       % │ unnötige Warnungen ausschalten
    mathtools-overbracket, % │
  },                       % ┘
]{unicode-math}

% Zahlen und Einheiten
\usepackage[
  locale=UK,                   % deutsche Einstellungen
  separate-uncertainty=true,   % immer Fehler mit \pm
  per-mode=symbol-or-fraction, % / in inline math, fraction in display math
]{siunitx}

% schöne Brüche im Text
\usepackage{xfrac}

% Standardplatzierung für Floats einstellen
\usepackage{float}
\floatplacement{figure}{H}
\floatplacement{table}{H}

% Floats innerhalb einer Section halten
\usepackage[
  section, % Floats innerhalb der Section halten
  below,   % unterhalb der Section aber auf der selben Seite ist ok
]{placeins}

%dassselbe für Subsections 
\makeatletter
\AtBeginDocument{%
  \expandafter\renewcommand\expandafter\subsection\expandafter{%
    \expandafter\@fb@secFB\subsection
  }%
}
\makeatother

% Seite drehen für breite Tabellen: landscape Umgebung
\usepackage{pdflscape}

% Captions schöner machen.
\usepackage[
  labelfont=bf,        % Tabelle x: Abbildung y: ist jetzt fett
  font=small,          % Schrift etwas kleiner als Dokument
  width=0.7\textwidth, % maximale Breite einer Caption schmaler
]{caption}
% subfigure, subtable, subref
\usepackage{subcaption}

% Grafiken können eingebunden werden
\usepackage{graphicx}
% größere Variation von Dateinamen möglich
%\usepackage{grffile}

% schöne Tabellen
\usepackage{booktabs}
\usepackage{physics}
% Verbesserungen am Schriftbild
\usepackage{microtype}

% Literaturverzeichnis
\usepackage[
  sorting=none,
  style=authortitle,
  autolang=hyphen,
  backend=biber,
]{biblatex}
% Quellendatenbank
\addbibresource{lit.bib}

% Hyperlinks im Dokument

% Trennung von Wörtern mit Strichen
\usepackage[shortcuts]{extdash}

%selbst hinzugefügt
\usepackage{physics}

% load the theme after all packages

\usetheme[
  showtotalframes, % show total number of frames in the footline
]{tu}

% Put settings here, like
\unimathsetup{
  math-style=ISO,
  bold-style=ISO,
  nabla=upright,
  partial=upright,
  mathrm=sym,
}


% traditionelle Fonts für Mathematik
\setmathfont{Latin Modern Math}
% Alternativ:
%\setmathfont{Libertinus Math}
\setmathfont{XITS Math}[range={scr, bfscr}]
\setmathfont{XITS Math}[range={cal, bfcal}, StylisticSet=1]

% title etc. 
\title{Heavy Quarks (ARGUS at DORIS and CLEO at CESR)}
\date{25.06.2020}
\author{Jan Herdieckerhoff}
%\institute[E5]{Experimentelle Physik V\\ Fakultät Physik}




\begin{document}
\maketitle
\tableofcontents
\newpage
\section{Ziel}
Ziel dieses Versuches ist es, die Elastizitätsmodule
verschiedener Stäbe durch Messung ihrer Biegung
zu bestimmen.
\section{Theorie}
%Spannung
Die Spannung ist die Kraft auf einen Körper pro Flächeneinheit.
Die Komponente, die senkrecht zur Oberfläche steht,
ist die Normalspannung $\sigma$. Ihre oberflächenparallele
Komponente heißt Tangentialspannung.
%Hookesches Gesetz
Das Hookesche Gesetz stellt den Zusammenhang zwischen
der Spannung $\sigma$, die am Körper angreift, und der
Deformation des Körpers dar:
\begin{equation}
\sigma = E \frac{\Delta L}{L}
\label{eqn:Hooke}
\end{equation}
%Elastizitätsmodul
$E$ ist dabei das Elastizitätsmodul.
Das Elastizitätsmodul ist eine Materialkonstante, die
anhand der Deformation eines Körpers bestimmt werden kann.
%Biegung
Eine Art der Deformation ist die Biegung. Sie entsteht, wenn
eine Kraft, wie in Abbildung 1 und in Abbildung 2 gezeigt, auf einen Körper wirkt. %Abbildung 1 erwähnen
%Berechnung der Durchbiegung, einseitig
Zunächst wird die Berechnung der Biegung eines Stabes bei einseitiger
Einspannung beschrieben. %Abbildung 1
\includegraphics{plot/abb1.jpg}
Die Durchbiegung $D(x)$ bezeichnet die Verschiebung eines Oberflächenpunktes
an der Stelle x zwischen dem belasteten und unbelasteten Zustand des Stabes.
Es wird eine Drehmomentgleichung aufgestellt, um $D(x)$ zu bestimmen.
Die Zug- und Druckspannungen, die an der Querschnittsfläche $Q$ angreifen,
sind entgegesetzt gleich und bewirken deshalb ein Drehmoment $M_{\sigma}$:
\begin{equation*}
M_{\sigma} = \int_{Q} y \sigma(y) dq
\end{equation*}
$y$ ist der Abstand des Flächenelementes $dq$ von der neutralen
Faser. Die neutrale Faser ist die Fläche, in der keine Spannungen
auftreten. Ihre Länge ändert sich bei der Biegung folglich nicht.
Ein weiteres Drehmoment $M_{F}$ entsteht durch die Kraft auf einen senkrecht
zur Stabachse stehenden Querschnitt. Es verdreht den Querschnitt aus 
seiner ursprünglichen vertikalen Lage.
Die Deformation des Körpers stellt sich so ein, dass die Drehmomente an
jeder Stelle $x$ übereinstimmen:
\begin{equation*}
M_{F} = M_{\sigma}.
\end{equation*}
Dabei ist
\begin{equation*}
M_{F} = F (L-x),
\end{equation*}
da die Kraft $F$ über den Hebelarm $L-x$ an $Q$ angreift.
Damit ist das Gleichgewicht der Drehmomente durch
\begin{equation}
\int_{Q} y \sigma(y) dq = F(L-x)
\label{eqn:Momente}
\end{equation}
gegeben.
Mit dem Hookeschen Gesetz \eqref{eqn:Hooke} wird die Normalspannung
$\sigma(y)$ mittels
\begin{equation*}
\sigma(y) = E \frac{\delta x}{\Delta x}
\end{equation*}
berechnet. Hier ist $\Delta x$ die Länge eines kurzen Stabstücks
und $\delta x$ die Längenänderung der Faser.
Es gilt außerdem
\begin{equation*}
\delta x = y \Delta \phi = y \frac{\Delta x}{R},
\end{equation*}
wobei $R$ der Krümmungsradius der Faser bei $x$ ist.
Damit ist
\begin{equation*}
\sigma(y) = E \frac{y}{R} = E y \frac{d^2D}{dx^2},
\end{equation*}
da für geringe Kurvenkrümmungen
\begin{equation*}
\frac{1}{R} \approx \frac{d^2D}{dx^2}
\end{equation*}
gilt, falls
\begin{equation*}
(\frac{dD}{dx})^2 << 1
\end{equation*}
ist. Für \eqref{eqn:Momente} ergibt sich damit:
\begin{equation}
E \frac{d^2D}{dx^2} \int_{Q} y^2 dq = F(L-x).
\label{eqn:Momente2}
\end{equation}
Dabei ist
\begin{equation*}
I = \int_{Q} y^2 dq(y)
\end{equation*}
das Flächenträgheitsmoment.
Integriert man \eqref{eqn:Momente2} und stellt die Gleichung
nach $D(x)$ um, erhält man für die Biegung bei einseitiger Einspannung
\begin{equation}
D(x) = \frac{F}{2EI} (Lx^2- \frac{x^3}{3}).
\label{eqn:D1}
\end{equation}
Diese Gleichung ist für $0 \leq x \leq L$ definiert.
Die Integrationskonstanten verschwinden, weil $D(0) = 0$ und $\frac{dD}{dx} = 0$ sein müssen.

%Berechnung der Durchbiegung, zweiseitig
\noindent Liegen beide Stabenden auf und lässt man in der Mitte des Stabes
eine Kraft angreifen, greift an der Querschnittsfläche die Kraft
$\frac{F}{2}$ mit dem Hebelarm $x$ an. Für die erste Stabhälfte $0 \leq x \leq \frac{L}{2}$
gilt für das Drehmoment
\begin{equation*}
M_{F} = - \frac{F}{2} x.
\end{equation*}
Für die zweite Hälfte $\frac{L}{2} \leq x \leq L$ gilt
\begin{equation*}
M_{F} = - \frac{F}{2} (L-x).
\end{equation*}
Damit ergibt sich hier für \eqref{eqn:Momente2}
\begin{equation}
\frac{d^2D}{dx^2} = - \frac{F}{EI} \frac{x}{2} \qq{für $0 \leq x \leq \frac{L}{2}$}
\label{eqn:links}
\end{equation}
und
\begin{equation}
\frac{d^2D}{dx^2} = -\frac{1}{2} \frac{F}{EI} (L-x) \qq{für $\frac{L}{2} \leq x \leq L$}.
\label{eqn:rechts}
\end{equation}
Integriert man beide Gleichungen, ergibt sich 
\begin{equation*}
\frac{dD}{dx} = - \frac{F}{EI} \frac{x^2}{4} + C \qq{für $0 \leq x \leq \frac{L}{2}$}
\end{equation*}
und
\begin{equation*}
\frac{dD}{dx} = - \frac{1}{2} \frac{F}{EI} (Lx-\frac{x^2}{2}) + C' \qq{für $\frac{L}{2} \leq x \leq L$}.
\end{equation*}
Da die Biegekurve in der Mitte des Stabes eine horizontale Tangente
haben muss, muss für die Konstanten gelten:
\begin{equation*}
C = \frac{F}{EI} \frac{L^2}{16}
\end{equation*}
und
\begin{equation*}
C' = \frac{3}{16} \frac{F}{EI} L^2.
\end{equation*}
Setzt man die Konstanten in \eqref{eqn:links} und \eqref{eqn:rechts}
ein und integriert die Ausdrücke, erhält man die Gleichungen
für die Biegung bei zweiseitiger Auflage des Stabes:
\begin{equation}
D(x) = \frac{F}{48EI} (3L^2x - 4x^3) \qq{für $0 \leq x \leq \frac{L}{2}$}
\label{eqn:D2links}
\end{equation}
und
\begin{equation}
D(x) = \frac{F}{48EI} (4x^3 - 12Lx^2 + 9L^2x - L^3) \qq{für $\frac{L}{2} \leq x \leq L$}.
\label{eqn:D2rechts}
\end{equation}
Die Integrationskonstanten verschwinden hier, weil $D(0) = 0$ und $D(L) = 0$ sein müssen.
Die Biegung eines elastischen Stabes kann also durch
\eqref{eqn:D1}, \eqref{eqn:D2links} und \eqref{eqn:D2rechts} bestimmt werden.

%D(x)=D_M(x)-D_0(x)
\noindent Weil die Stäbe nicht als exakt gerade angenommen werden können,
muss die Biegung ohne angehängtes Gewicht gemessen werden.
Die Biegung des Stabes durch die Last ist dann 
\begin{equation}
D(x) = D_{M}(x) - D_{0}(x).
\label{eqn:D(x)}
\end{equation}

%Lineare Regression, Berechnung von E
\noindent Der Elastizitätsmodul lässt sich mittels einer
linearen Regression bestimmen. Die Werte für die Biegung \eqref{eqn:D(x)}
werden gegen eine liearisierte Form des horizontalen Abstands aufgetragen.
Diese können aus den Gleichungen \eqref{eqn:D1}, \eqref{eqn:D2links} und \eqref{eqn:D2rechts} entnommen werden.
Die Steigung berechnet sich dabei wie folgt:
\begin{equation}
m = \frac{\overline{xy} - \overline{x} \cdot \overline{y}}{\overline{x^2} - \overline{x}^2}.
\label{eqn:m}
\end{equation} %MW etc auch erwähnen?
Bei dem ersten und zweiten Stab entspricht die Steigung
$\frac{F}{2EI}$.
Stellt man den Ausdruck nach $E$ um, erhält man
eine Gleichung für den Elastizitätsmodul:
\begin{equation}
E = \frac{F}{2mI}.
\label{eqn:E12}
\end{equation}
Für den dritten Stab ergibt sich für den Elastizitätsmodul auf die
selbe Weise die Gleichung:
\begin{equation}
E = \frac{F}{48mI}.
\label{eqn:E3}
\end{equation}

\section{Durchführung}
Die Apparatur ist in Abbildung xy zu sehen. %Abbildung Apparatur erwähnen
Die Stäbe werden entweder einseitig eingeklemmt oder zweiseitig
auf den Punkten $A$ und $B$ gelagert. Die Stäbe werden belastet, indem
ein Gewicht entweder am Stabende oder in der Stabmitte angehängt wird.
Die Biegung wird mit zwei Messuhren, die sich auf einer Längen-Skala befinden
und verschiebbar sind, sodass die Biegung an verschiedenen Stellen $x$ bestimmt
werden kann, gemessen. Bei Messuhren wird die Verschiebung eines Objektes mittels
eines federnden Taststiftes gemessen. %andere Formulierung
%Hier Abbildung einfügen
\includegraphics{plot/abb2.jpg}
Zunächst wird jeweils für zwei einseitig eingespannte Stäbe die Biegung ohne
angehängtes Gewicht gemessen. Danach wird an das Stabende ein Gewicht angehängt.
Die Biegung wird mit einer der Messuhren gemessen. %Abbildung einseitiger Stab
Ein dritter Stab wird
zweiseitig aufgelegt. Es wird wieder zunächst die Biegung ohne Gewicht und dann
mit Gewicht gemessen. Hier wird für die erste Hälfte des Stabes $\frac{L}{2} \leq x \leq L$
die linke Messuhr und für die zweite Hälfte $0 \leq x \leq \frac{L}{2}$ die rechte Messuhr
verwendet. %Abbildung zweiseitiger Stab
Zuletzt werden die Längen, Breiten, beziehungsweise Durchmesser und Massen der Stäbe,
sowie die Masse des angehängten Gewichts bestimmt.

\section{Auswertung}
\label{sec:Auswertung}

Für die Auswertung wird Python und im Speziellen Matplotlib \cite{matplotlib}, SciPy \cite{scipy}, Uncertainties \cite{uncertainties} und NumPy \cite{numpy} verwendet.

\subsection{Bestimmung des Energieverlustes von Alphastrahlung in Luft}

\subsubsection{Erster Abstand}
Die gemessenen Pulse und Positionen der Energiemaxima bei den verschiedenen Drücken sind für den Abstand $d_1 = \SI{2.7}{\centi\meter}$ in Tab. \ref{taba} zu sehen. Die Drücke, die Anzahl der Pulse und die Position des jeweiligen Maximums befinden sich in Tab. \ref{taba}. 

\begin{table}\caption{Die Anzahl der Impulse, der Startwert auf der Mikrometerschraube und der Endwert auf der Mikrometerschraube.}
\label{taba}
\centering
\sisetup{round-mode = places, round-precision=2, round-integer-to-decimal=true}
\begin{tabular}{S[]S[]S[]} 
\toprule
{Anzahl} & {$d_\text{Start} / \si{\milli\meter}$} & {$d_\text{Start} / \si{\milli\meter}$}\\
\midrule
3001.0 & 6.73 & 2.0\\
3002.0 & 6.73 & 2.0\\
3000.0 & 1.82 & 6.5\\
3000.0 & 6.74 & 2.0\\
3000.0 & 1.83 & 6.5\\
3000.0 & 6.74 & 2.0\\
3001.0 & 1.84 & 6.5\\
3000.0 & 2.83 & 7.5\\
3001.0 & 7.77 & 3.0\\
3002.0 & 2.75 & 7.5\\
\bottomrule
\end{tabular}\end{table}

%Zählrate als Funktion der effektiven Länge für den ersten Abstand
Die mit Gleichung \eqref{eqn:x} ermittelten Abstände, die Anzahl der Pulse (Zählrate) und die mit Gleichung \eqref{eqn:energie} ermittelten Energien befinden sich in Tab. \ref{tab1}. 

\noindent Die Zählrate ist in Abb. \ref{zaehlrate1} gegen die mit Gleichung \eqref{eqn:abstand} bestimmte effektive Länge aufgetragen.
\begin{figure}
    \centering
    \includegraphics[width=12cm, height=9cm]{build/plota.pdf}
    \caption{}
    \label{fig:zaehlrate1}
\end{figure}

\noindent Die Fitparameter der linearen Regression ergeben sich dadurch zu 
\begin{align*}
    m &= \num{-11300952.96(324920)} \frac{N}{\SI{120}{\second} \si{\meter}} \\
    n &= \num{289663.5} \frac{N}{\SI{120}{\second}} .
\end{align*}


\noindent Mit dem Umformen dieser linearen Gleichung ergibt sich bei $y = \frac{1}{2} max$ die mittlere Reichweite der $\alpha$-Teilchen zu dem Wert %wie genau?
\begin{equation*}
    R_\text{m,1} = \SI{22.7}{\milli\meter}
\end{equation*}
bestimmen.

\noindent Das entspricht  nach Ablesen in Tab. \ref{tab1} einer Energie von %wie genau?
\begin{equation*}
    E_1 = \SI{12.7}{\mega\electronvolt}.
\end{equation*}

%Energie als Funktion der effektiven Länge für den ersten Abstand
Die Energie ist in Abb. \ref{fig:energie1} gegen die effektive Länge aufgetragen.
\begin{figure}
    \centering
    \includegraphics[width=12cm, height=9cm]{build/plotb.pdf}
    \caption{}
    \label{fig:energie1}
\end{figure}

\noindent Die Fitparameter der linearen Regression ergeben sich dadurch zu 
\begin{align*}
    m &= \SI{-114512929.24(426762263)}{\mega\electronvolt\per\meter} \\
    n &= \SI{4132682.60}{\electronvolt} .
\end{align*}

\noindent Daraus lässt sich anhand der Steigung der Energieverlust der Strahlung bestimmen %wie genau?
\begin{equation*}
    - \left( \frac{dE}{dx} \right)_1 = - \SI{11.45}{\mega\electronvolt}.
\end{equation*}


\subsubsection{Zweiter Abstand}
Die gemessenen Pulse und Positionen der Energiemaxima bei den verschiedenen Drücken sind für den Abstand $d_2 = \SI{2}{\centi\meter}$ in Tab. \ref{tabb} zu sehen. Die Drücke, die Anzahl der Pulse und die Position des jeweiligen Maximums befinden sich in Tab. \ref{tabb}. 

\begin{table}\caption{Die Frequenzen der Sägezahnspannung.}
\label{tabb}
\centering
\sisetup{round-mode = places, round-precision=2, round-integer-to-decimal=true}
\begin{tabular}{S[]S[]} 
\toprule
{Index} & {$\nu_\text{Sä} / \si{\hertz}$}\\
\midrule
1.0 & 25.02\\
2.0 & 49.95\\
3.0 & 99.99\\
4.0 & 149.97\\
\bottomrule
\end{tabular}\end{table}

%Zählrate als Funktion der effektiven Länge für den ersten Abstand
Die mit Gleichung \eqref{eqn:x} ermittelten Abstände, die Anzahl der Pulse (Zählrate) und die mit Gleichung \eqref{eqn:energie} ermittelten Energien befinden sich in Tab. \ref{tab2}. 

\noindent Die Zählrate ist in Abb. \ref{fig:zaehlrate2} gegen die mit Gleichung \eqref{eqn:abstand} bestimmte effektive Länge aufgetragen.
\begin{figure}
    \centering
    \includegraphics[width=12cm, height=9cm]{build/plotc.pdf}
    \caption{}
    \label{fig:zaehlrate2}
\end{figure}

\noindent Die Fitparameter der linearen Regression ergeben sich dadurch zu 
\begin{align*}
    m &= \num{-11300952.96(324920)} \frac{N}{\SI{120}{\second} \si{\meter}} \\
    n &= \num{289663.5} \frac{N}{\SI{120}{\second}} .
\end{align*}


\noindent Mit dem Umformen dieser linearen Gleichung ergibt sich bei $y = \frac{1}{2} max$ die mittlere Reichweite der $\alpha$-Teilchen zu dem Wert %wie genau?
\begin{equation*}
    R_\text{m,2} = \SI{<++>}{\milli\meter}
\end{equation*}
bestimmen.

\noindent Das entspricht  nach Ablesen in Tab. \ref{tab2} einer Energie von %wie genau?
\begin{equation*}
    E_2 = \SI{<++>}{\mega\electronvolt}.
\end{equation*}

%Energie als Funktion der effektiven Länge für den ersten Abstand
Die Energie ist in Abb. \ref{fig:energie2} gegen die effektive Länge aufgetragen.
\begin{figure}
    \centering
    \includegraphics[width=12cm, height=9cm]{build/plotd.pdf}
    \caption{}
    \label{fig:energie2}
\end{figure}

\noindent Die Fitparameter der linearen Regression ergeben sich dadurch zu 
\begin{align*}
    m &= \SI{<++>}{\<++>} \\
    n &= \SI{<++>}{\<++>} .
\end{align*}

\noindent Daraus lässt sich anhand der Steigung der Energieverlust der Strahlung bestimmen %wie genau?
\begin{equation*}
    - \left( \frac{dE}{dx} \right)_1 = \SI{}{}.
\end{equation*}


\subsection{Untersuchung der Statistik des radioaktiven Zerfalls}

Die Anzahl der Pulse, die jeweils in $\SI{10}{\second}$ gemessen wurde, sind in Tab. \ref{tabc} eingetragen.

\begin{table}\caption{Der magnetische Fluss $B$ des gemessenen Magnetfelds gegen den Strom $I$ des erzeugenden Magnetfelds, Neukurve.}
\label{tabc}
\centering
\sisetup{round-mode = places, round-precision=1, round-integer-to-decimal=true}
\begin{tabular}{S[]S[]} 
\toprule
{$B$/ \si{\milli\tesla}} & {$I$/ \si{\ampere}}\\
\midrule
0.0 & 0.0\\
111.19999999999999 & 1.0\\
273.5 & 2.0\\
397.8 & 3.0\\
479.9 & 4.0\\
537.9000000000001 & 5.0\\
585.0999999999999 & 6.0\\
621.8000000000001 & 7.0\\
653.1 & 8.0\\
679.9 & 9.0\\
704.3000000000001 & 10.0\\
\bottomrule
\end{tabular}\end{table}

%Zerfallsraten in Histogramm
Die Zerfallsraten sind in Abb. \ref{fig:histogramm} in einem Histogramm aufgetragen. Außerdem ist eine Gauß- und eine Poissonverteilung eingetragen. Bei der Erzeugung der beiden Verteilungen wurde ein Seed von 42 benutzt, um die Auswertung deterministisch zu machen. 
\begin{figure}
    \centering
    \includegraphics[width=12cm, height=9cm]{build/plot.pdf}
    \caption{}
    \label{fig:histogramm}
\end{figure}

%Mittelwert und Varianz
\noindent Aus den gemessenen Zählraten lassen sich der  Mittelwert und die Varianz bestimmen: %Gleichungen erwähnen, Varianz ergänzen?
\begin{align*}
    \bar{x} &= \num{4541.6} \\
    var &= \num{25065}.
\end{align*}

\newpage
\section{Diskussion}
\label{sec:Diskussion}

Die lange Spule hat einen maximalen Wert von \SI{2.19}{\milli\tesla}. Der theoretisch berechnete Wert liegt bei %gleich viele Nachkommastellen angeben?
\SI{2.36}{\milli\tesla} und hat somit eine relative Abweichung von \SI{7.2}{\percent}. Auch der gefittete Wert für die Magnetfeldstärke innerhalb der langen Spule hat einen Wert von \SI{<++>}{\<++>}<++> mit einem relativen Fehler von \SI{<++>}{\percent}. Dieser Wert hat eine Abweichung von \SI{<++>}{\percent} zum Theoriewert. Ansonsten entsprechen die 
beobachteten Messungen dem Bild, das zu erwarten war.

\noindent Bei der kurzen Spule passen der Theoriewert im Maximum und die experimental gemessene Kurve nicht sonderlich gut zusammen. Die Abweichung im 
Hochpunkt liegt bei \SI{40}{\percent}, was recht viel ist. Die Fitparameter haben einen relativ großen Fehler. Die Mittelwerte liegen aber ziemlich gut in dem Bereich der tatsächlichen Werte. Mit \SI{25}{\percent} Abweichung in der Stromstärke $I$ und einer Abweichung von \SI{43,1}{\percent} bei der Spulenanzahl liegen diese Werte noch in einem ähnlichen Rahmen, wie die Abweichung zur Theoriekurve.   

\noindent Das Spulenpaar, als Helmholtzspule, passt ebenfalls nicht zum Theoriewert. Der relative Fehler des Maximums in der Mitte 
des Paares liegt bei einem Wert von \SI{55.5}{\percent}. %Dein Einwand war berechtigt, aber der Satz war eh redundant, da wir ja keine Theoriekurven betrachten
Bei einem Strom von \SI{4}{\ampere} und einem Abstand der Spulen, der dem Durchmesser entspricht, liegt die relative Abweichung in der Mitte des Paares bei \SI{10.71}{\percent}. %"lag" Warte mal, soll ich hier im Präteritum schreiben oder nicht? Die Diskussion ist doch immer danach oder nicht? Also wenn ich jetzt schreibe die Apparatur "ist" wackelig stimmt das vielleicht, aber es geht doch darum, was bei uns falsch "war". Oder hattest du die schon korrigiert?
Bei der dritten Messung wurde ein %"hatte" 
relativer Fehler von \SI{6.84}{\percent} festgestellt. Somit liegen alle Fehler in einem Bereich von \SI{5}{\percent} 
bis \SI{60}{\percent}, was relativ viel ist. Grund dafür könnte die etwas wackelige Apparatur gewesen sein. %"gewesen"
Außerdem musste die Hall-Sonde immer exakt gleich ausgerichtet sein. Drehte man sie nur leicht, änderten sich die %"musste", "drehte", ..
gemessenen Werte bereits extrem.
\newline
Vermutlich war die Hallsonde bei der Messung der kurzen Spule und der ersten Messung %"war"
des Spulenpaares nicht ganz senkrecht, wodurch die starke Abweichung zu den Theoriewerten entstanden ist. 
Insgesamt sind die Mittelwerte aus dem Fit ähnlich weit abgewichen wie die Abweichungen zwischen Theorie- und Experimentalkurve. Dies könnte vor allem an den Rändern liegen, da diese Verhältnismäßig ungenau sind. Dadurch kommen auch entsprechend große Standardabweichungen zustande. Wenn man die Fehler ab mit betrachtet, liegen auch die gefitteten Werte zum Großteil sowohl in dem Bereich der Theorie- als auch im Bereich der Experimentalkurven.

\noindent Die Magnetisierung funktionierte recht gut. Die Hysteresekurve sieht genauso aus, wie sie zu erwarten war. %"funktionierte", "war"


%Diskussion: Wert hat sich bei Erschütterung geändert

\nocite{*}
\newpage
\printbibliography
\end{document}
