\section{Auswertung}
\label{sec:Auswertung}
%Sollen wir U_Out oder U_out nehmen?

Das sinusförmige Signal, das auf den Verstärker gegeben wird, hat eine Frequenz
von $\SI{1}{\kilo\hertz}$ und eine Spannung von $\SI{10}{\milli\volt}$. %gehört das zu allen Teilen? Sonst lieber in die Subsection schreiben

\subsection{Überprüfung der Funktion eines phasenempfindlichen Gleichrichters}
Die Aufnahmen der Ausgangssignale für fünf verschiedene Phasen sind im Folgenden abgebildet.
\begin{figure}
    \centering
    \includegraphics[width=8cm, height=8cm]{build/1.BMP}
    \caption{Ausgangsspannung bei einer Phase von $\varphi = \SI{0}{\degree}$.}
    \label{fig:bild1}
\end{figure}

\begin{figure}
    \centering
    \includegraphics[width=8cm, height=8cm]{build/2.BMP}
    \caption{Ausgangsspannung bei einer Phase von $\varphi = \SI{15}{\degree}$.}
    \label{fig:bild2}
\end{figure}

\begin{figure}
    \centering
    \includegraphics[width=8cm, height=8cm]{build/3.BMP}
    \caption{Ausgangsspannung bei einer Phase von $\varphi = \SI{30}{\degree}$.}
    \label{fig:bild3}
\end{figure}

\begin{figure}
    \centering
    \includegraphics[width=8cm, height=8cm]{build/4.BMP}
    \caption{Ausgangsspannung bei einer Phase von $\varphi = \SI{45}{\degree}$.}
    \label{fig:bild4}
\end{figure}

\begin{figure}
    \centering
    \includegraphics[width=8cm, height=8cm]{build/5.BMP}
    \caption{Ausgangsspannung bei einer Phase von $\varphi = \SI{60}{\degree}$.}
    \label{fig:bild5}
\end{figure}

\noindent Die Werte der Amplitude der Ausgangsspannung $U_{out}$ in Abhängigkeit von der Phasenverschiebung $\varphi$
zwischen der Signalspannung $U_{Sig}$ und der Referenzspannung $U_{Ref}$
sind in Tabelle \ref{tab2} dargestellt. Die Ausgangsspannung $U_{out}$ wird in Abb. \ref{fig:plot2} gegen die Phasenverschiebung $\varphi$
aufgetragen.
\begin{table}\caption{Die Spannung, die Stromstärke, die Anzahl der Impulse, die transportierte Ladungsmenge und die transporte Ladungsmenge in Einheiten der Elementarladung.}
\label{tab1}
\centering
\sisetup{round-mode = places, round-precision=2, round-integer-to-decimal=true}
\begin{tabular}{S[]S[] S[]@{${}\pm{}$}S[] S[]@{${}\pm{}$} S[] S[]@{${}\pm{}$} S[]} 
\toprule
{U / \si{\volt}} & {I / \si{\ampere}} & \multicolumn{2}{c}{N/second} &  \multicolumn{2}{c}{$\Delta Q / \si{\coulomb}$} &  \multicolumn{2}{c}{$\Delta Q \si{\elementarycharge}$}\\
\midrule
320.0 & 0.1     & 86.91 & 0.07 &  8.975  &  0.007  & 5.602   &  0.005e+19\\
400.0 & 0.2     & 90.92 & 0.07 & 17.157  &  0.014  & 1.0709  &  0.0009e+20\\
480.0 & 0.3     & 93.35 & 0.07 & 25.068  &  0.020  & 1.5646  &  0.0012e+20\\
540.0 & 0.35    & 94.62 & 0.07 & 28.851  &  0.023  & 1.8008  &  0.0014e+20\\
560.0 & 0.4     & 92.83 & 0.07 & 33.610  &  0.027  & 2.0977  &  0.0017e+20\\
600.0 & 0.45    & 95.03 & 0.07 & 36.935  &  0.029  & 2.3053  &  0.0018e+20\\
640.0 & 0.5     & 95.41 & 0.08 & 40.877  &  0.032  & 2.5514  &  0.0020e+20\\
660.0 & 0.55    & 96.21 & 0.08 & 44.591  &  0.035  & 2.7832  &  0.0022e+20\\
680.0 & 0.6     & 97.38 & 0.08 & 48.06   &  0.04   & 2.9997  &  0.0023e+20\\
\bottomrule
\end{tabular}\end{table}

\begin{figure}
    \centering
    \includegraphics[width=8cm, height=8cm]{build/plot2.pdf}
    \caption{Plot2}
    \label{fig:plot2}
\end{figure}

\noindent Der experimentell ermittelte Wert für die Signalspannung ist
\begin{equation*}
    U_{0,exp} = \SI{3.75 \pm 0.17}{\volt}.
\end{equation*}
Der mit Gleichung \eqref{eqn:u_out} berechnete theoretische Wert ergibt sich zu
\begin{equation*}
    U_{0,theo} = \SI{123456789}{\volt}. %richtigen Wert einfügen
\end{equation*}

\subsection{Überprüfung der Funktion eines Lock-In-Verstärkers}
In Tabelle \ref{tab3} befinden sich die Werte der Amplitude der Ausgangsspannung $U_{out}$ in
Abhängigkeit von der Phasenverschiebung $\varphi$ zwischen der Signalspannung $U_{Sig}$ und
der Referenzspannung $U_{Ref}$. Die Werte der Ausgangsspannung $U_{out}$ wird gegen die Phasenverschiebung $\varphi$
in Abb. \ref{fig:plot3} aufgetragen.
\begin{table}\caption{Die Zeit des Durchschallungsverfahrens gegen die Länge des Zylinders.}
\label{tab3}
\centering
\sisetup{round-mode = places, round-precision=2, round-integer-to-decimal=true}
\begin{tabular}{S[]S[]} 
\toprule
{t/ \si{\second}} & {l/ \si{\meter}}\\
\midrule
8.95e-05 & 0.1208\\
7.8e-05 & 0.1023\\
5.93e-05 & 0.0805\\
3.08e-05 & 0.0404\\
2.47e-05 & 0.0311\\
\bottomrule
\end{tabular}\end{table}

\begin{figure}
    \centering
    \includegraphics[width=8cm, height=8cm]{build/plot3.pdf}
    \caption{Plot3}
    \label{fig:plot3}
\end{figure}

\noindent Experimentell ergibt sich für die Signalspannung ein Wert von
\begin{equation*}
    U_{0} = \SI{3.86 \pm 0.09}{\volt}.
\end{equation*}
Mit Gleichung \eqref{eqn:u_out} wird der theoretische Wert zu
\begin{equation*}
    U_{0} = \SI{123456789}{\volt} %richtigen Wert einfügen
\end{equation*}
berechnet.

\subsection{Überprüfung der Rauschunterdrückung des Lock-In-Verstärkers}
Die Werte der Ausgangsspannung $U_{out}$ mit Verstärkung des Tiefpasses und des Detektors in Abhängigkeit vom Abstand $r$ der LED
zur Photodiode sind in Tabelle \ref{tab4} zu finden.
Die tatsächliche Ausgangsspannung $U_{out}$ in Abhängigkeit vom Abstand $r$ der LED
zur Photodiode befindet sich in Tabelle \ref{tab5}.
Die Werte aus Tabelle \ref{tab5} sind in Abbildung \ref{fig:plot4}
gegeneinander aufgetragen. Es ist also die Ausgangsspannung in Abhängigkeit
vom Abstand dargestellt.
\begin{table}\caption{Der Abstand $r$ zwischen Leucht- und Photodiode aufgetragen gegen die Spannung U_{Out}. Dazu jeweils den Wert für die Verstärkung des Tiefpasses und des Detektors.}
\label{tab4}
\centering
\sisetup{round-mode = places, round-precision=1, round-integer-to-decimal=true}
\begin{tabular}{S[]S[]S[]S[]} 
\toprule
{$r / \si{\centi\meter}$} & {$U_{Out} / \si{\volt}$} & {Gain Tiefpass} & {Gain Detektor}\\
\midrule
10.0 & 4.0 & 20.0 & 100.0\\
15.0 & 4.1 & 50.0 & 100.0\\
20.0 & 4.2 & 100.0 & 100.0\\
25.0 & 5.3 & 200.0 & 100.0\\
30.0 & 8.7 & 500.0 & 100.0\\
35.0 & 6.5 & 500.0 & 100.0\\
40.0 & 4.9 & 500.0 & 100.0\\
45.0 & 7.6 & 1000.0 & 100.0\\
50.0 & 6.0 & 1000.0 & 100.0\\
55.00000000000001 & 5.0 & 1000.0 & 100.0\\
60.0 & 4.2 & 1000.0 & 100.0\\
65.0 & 7.1 & 1000.0 & 200.0\\
70.0 & 6.2 & 1000.0 & 200.0\\
75.0 & 5.4 & 1000.0 & 200.0\\
80.0 & 4.8 & 1000.0 & 200.0\\
85.0 & 4.2 & 1000.0 & 200.0\\
90.0 & 3.7 & 1000.0 & 200.0\\
95.0 & 9.0 & 1000.0 & 500.0\\
100.0 & 8.0 & 1000.0 & 500.0\\
\bottomrule
\end{tabular}\end{table}
\begin{table}\caption{Die invertierte Temperatur gegen die logarithmierte Viskosität für die erste Messung.}
\label{tab5}
\centering
\sisetup{round-mode = places, round-precision=1, round-integer-to-decimal=true}
\begin{tabular}{S[]S[]} 
\toprule
{$\frac{10^{3}}{T_1} /\si[per-mode=fraction]{\per\kelvin}$} & {$\eta_1 /\si{\pascal\second}$}\\
\midrule
3.0660738923808064 & -7.497305275002141\\
3.0473868657626086 & -7.5327861977670985\\
3.028926245645919 & -7.555420217220707\\
3.0197795560924057 & -7.591653556586626\\
3.0016509079994 & -7.621470911639022\\
2.9837386244964943 & -7.656207745688693\\
2.966038855109002 & -7.708174169692305\\
2.948547840188707 & -7.769145431988264\\
2.9312619082515026 & -7.860478126926399\\
2.914177473408131 & -8.001593998980967\\
\bottomrule
\end{tabular}\end{table}

\begin{figure}
    \centering
    \includegraphics[width=8cm, height=8cm]{build/plot4.pdf}
    \caption{Plot4}
    \label{fig:plot4}
\end{figure}

\noindent Der maximale Abstand $r_{max}$, bei dem das Licht der LED %"LED"? In der Anleitung steht "Photodiode" aber das ergibt doch keinen Sinn oder?
noch nachgewiesen werden kann, ist nicht eindeutig bestimmbar. 