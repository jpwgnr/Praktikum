\section{Diskussion}
\label{sec:Diskussion}

%2&3) Wie verändern sich die Signale?
Die Amplitude der Referenzspannung ließ sich verändern am Funktionsgenerator. Die Oszillatorspannung war konstant 
mit einem Wert von \SI{3.26}{\V}. %nicht 3.28?

\noindent Der Noise Generator wurde im ersten Versuchsteil überbrückt. Die Bilder, die gemacht wurden, ergeben, dass 
die Spannung $U_{Out}$ ohne den Tiefpass wie eine Sinusfunktion aussieht, die nach etwas mehr als einer viertel 
Periode abgeschnitten wird und dieses Bild anschließend periodisch fortgesetzt wird. 
Bei einer Phasenverschiebung verschiebt sich der Sinus in in positive x-Richtung, wie sich an den Abbildungen 
erkennen lässt. Hier wurde ein Phasenverschiebung von insgesamt \SI{75}{\degree} betrachtet. %60 Grad oder?
Dabei wurde der Sinus 
bereits um eine knappe viertel Phase verschoben. Also entspricht die Phasenverschiebung auch ungefähr der 
Verschiebung des Bildes, das gesehen wird. 

\noindent Wenn das Signal mittels des Tiefpasses integriert wird, erhält man den Wert zwischen dem Graphen und der 
$x$-Achse. Für die Messung ohne Noise Generator ergibt sich damit ein Wert von \SI{3.75}{\volt} und einer relativen 
Abweichung von \SI{4.53}{\percent}. Der direkt an der Spannung  abgelesene Wert beträgt \SI{3.26}{\volt}, was einer%3.28? 
relativen Abweichung von  \SI{15.03}{\percent} zwischen dem gemessenen Wert und dem aus dem Fit ermittelten Wert 
entspricht. 

\noindent Mit dem Noise Generator ergab sich für die Spannung ein Wert von \SI{3.86}{\volt} und einem relativen 
Fehler von \SI{2.33}{\percent}. Somit liegt dieser noch weiter von dem exakten Wert entfernt. Dies war auch zu 
erwarten, da der Noise Generator das Signal ungenauer macht. Insgesamt ist die Abweichung aber trotzdem nicht so %besser ohne "so"
groß. Die Abweichung zur Messung ohne Noise Generator beträgt \SI{2.85}{\percent}. Zum exakten Wert beträgt die 
Abweichung \SI{18.4}{\percent}. Das Bild war bei der Spannung mit dem Noise Generator natürlich deutlich ungenauer, 
der Lock-In-Verstärker hat das Signal aber trotzdem ziemlich genau ermittelt. 

\noindent Die Messungen mit der LED und der Photodiode haben auch das zu erwartende Ergebnis geliefert. Es ist zu 
erkennen, dass die Werte auf einer beidseitig logarithmierten Skala linear abnehmen. Der Nullpunkt war nicht exakt 
zu bestimmen. Bei einer Entfernung von einem Meter gab es noch immer einen Ausschlag des Messgeräts. Aus der Grafik 
lässt sich dieser Wert auch nicht ablesen. Aufgrund der logarithmischen Skala wird der absolute Nullpunkt nie 
erreicht.  
