\section{Diskussion}
\label{sec:Diskussion}

%2&3) Wie verändern sich die Signale?
Die Amplitude der Signalspannung ließ sich am Funktionsgenerator verändern. Die verstärkte Spannung ergab einen konstanten 
Wert von \SI{3.28}{\V}. Dieser Wert wurde mit dem Oszilloskop als Vergleichswert gemessen.

\noindent Der Noise Generator wurde im ersten Versuchsteil überbrückt. Die Bilder, die gemacht wurden, ergeben, dass 
die Spannung $U_{Out}$ ohne den Tiefpass wie eine Sinusfunktion aussieht, die nach etwas mehr als einer viertel 
Periode abgeschnitten wird und dieses Bild anschließend periodisch fortgesetzt wird. 
Bei einer Phasenverschiebung verschiebt sich der Sinus in positive $t$-Richtung, wie sich an den Abbildungen 
erkennen lässt. Hier wurde eine Phasenverschiebung von insgesamt \SI{60}{\degree} betrachtet.
Dabei wurde der Sinus 
um etwas weniger als eine viertel Phase verschoben. Also entspricht die Phasenverschiebung auch ungefähr der 
Verschiebung des Bildes, das gesehen wird. 

\noindent Wenn das Signal mittels des Tiefpasses integriert wird, erhält man den Wert zwischen dem Graphen und der 
$t$-Achse. Für die Messung ohne Noise Generator ergibt sich damit ein Wert von \SI{3.75}{\volt} und ein relativer 
Fehler von \SI{4.53}{\percent}. Die Phasenverschiebung $\delta$ beträgt bei der Messung ohne Noise \num{0.142} mit einem relativen Fehler von \SI{19.01}{\percent}. 
Bei der Messung mit Noise wurde eine Phasenverschiebung von \num{0.049} mit einem Fehler von \SI{45.32}{\percent}. 
Die Phasenverschiebung ist bei der Messung ohne Noise deutlich größer gewesen. Bei der Messung mit Noise war aber dafür der Fehler deutlich größer. Durch das Rauschen ist die Phasenverschiebung zwar kleiner, weil der Wert im Mittel dann näher an den richtigen Daten liegt. Der Fehler ist aber aus demselben Grund relativ dazu deutlich größer.   
 


\noindent Mit dem Noise Generator ergab sich für die Spannung ein Wert von \SI{3.86}{\volt} und ein relativer 
Fehler von \SI{2.33}{\percent}. Somit liegt dieser etwas weiter von dem exakten Wert entfernt. Dies war auch zu 
erwarten, da der Noise Generator das Signal ungenauer macht. Insgesamt ist die Abweichung aber trotzdem nicht
groß. Die Abweichung zur Messung ohne Noise Generator beträgt \SI{2.85}{\percent}. Das Bild war bei der Spannung mit dem Noise Generator natürlich deutlich ungenauer, 
der Lock-In-Verstärker hat das Signal aber trotzdem ähnlich genau ermittelt. 

\noindent Die Messungen mit der LED und der Photodiode haben auch das zu erwartende Ergebnis geliefert. Es ist zu 
erkennen, dass die Werte auf einer beidseitig logarithmierten Skala linear abnehmen. Der Nullpunkt war nicht exakt 
zu bestimmen. Bei einer Entfernung von einem Meter gab es noch immer einen Ausschlag des Messgeräts. Aus der Grafik 
lässt sich dieser Wert auch nicht ablesen. Aufgrund der logarithmischen Skala wird der absolute Nullpunkt sogar nie 
erreicht.  
