\section{Ziel}
\label{sec:Ziel}
%NICHT VERGESSEN

Die Funktionsweise eines Lock-In-Verstärkers soll kennengelernt und verstanden werden. 

\section{Theorie}
\label{sec:Theorie}

Der Lock-In-Verstärker wird genutzt, um bestimmte Eigenschaften einer Signalspannung 
$U_{Sig}$ zu bestimmen. 
Hierzu wird das das Signal mit einer Referenzfrequenz $\omega_0$ moduliert.
Dieses unter Umständen verrauschte Signal wird durch einen Bandpass von Rauschanteilen 
deutlich höher oder niedriger als die Referenzfrequenz befreit. %
In einem sogenannten Mischer wird die Signalspannung dann mit einem Referenzsignal 
$U_{Ref}$, das die Referenzfrequenz $\omega_0$ hat, multipliziert. 

\noindent Dieses neue Signal wird in einen Tiefpass eingespeist. Dieser hat die 
Eigenschaft, dass er das entstandene Mischsignal über mehrere Perioden der 
Modulationsfrequenz integriert.
Die nicht zur Frequenz synchronisierten Rauschbeiträge werden sich dadurch 
herausmitteln, sodass eine zur Eingangsspannung $U_{Sig}$ proportionale Gleichspannung 
$U_{out}$ gemessen werden kann. 

\noindent Der Tiefpass entscheidet dabei über die Bandbreite des Restrauschens. 
Je größer die Zeitkonstante des Passes gewählt wird, desto kleiner wird die Bandbreite 
des Rauschens sein. Mit einem Lock-In-Verstärker kann man somit Güten erreichen, 
die weit über der Güte eines normalen Bandpasses liegt. %

Sind die Signal- und Referenzspannungen nicht in Phase, sondern haben eine 
Phasendifferenz $\varphi$, erhält man eine folgende Ausgangsspannung: 

\begin{equation}
    U_{out} \propto U_0 cos(\varphi).
\end{equation}
Sie erreicht somit ihr Maximum bei einer Phase von $\varphi = 0$. 
