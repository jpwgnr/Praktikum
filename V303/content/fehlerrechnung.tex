\section{Fehlerrechnung}

Der Mittelwert einer Stichprobe von $N$ Werten wird durch
\begin{equation*}
    \overline{x} = \frac{1}{N} \sum_{i=1}^N x_i
    \label{eqn:mittelwert}
\end{equation*}
bestimmt.
\newline
Die Standardabweichung der Stichprobe wird berechnet mit:
\begin{equation*}
    \sigma_x = \sqrt{\frac{1}{N-1} \sum_{i=1}^N (x_i - \overline{x})^2}.
    \label{eqn:standard}
\end{equation*}
\newline
Der realtive Fehler zwischen zwei Werten kann durch
\begin{equation*}
    \frac{a-b}{a}
\end{equation*}
bestimmt werden.
\newline
Für die Auswertung werden matplotlib \cite{matplotlib}, NumPy \cite{numpy} und SciPy \cite{scipy} benutzt.