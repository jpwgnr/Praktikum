\section{Vorbereitung}
Das kommt nicht ins Protokoll und ist nur fürs Kolloq.

1.
Unter welchen Bedingungen bildet sich in einem Hohlzylinder eine stehende Welle aus? 

Wie lautet die Helmholtzgleichung eines Kugelresonator? 

Wie lautet die Schrödingergleichung eines Elektrons im Wasserstoffatom? 

Vergleichen Sie die beiden Gleichungen. Welche Gemeinsamkeiten und welche Unterschiede weisen sie auf?

2.
Was wird im Kugelresonator-Experiment gemessen?

3.
Wie sehen die Energieeigenwerte eines Wasserstoffatoms und die Eigenfrequenzen eines sphärischen Hohlraumresonators aus?

Welche Energiezustände (Quantenzahlen) kann ein Wasserstoffatom mit n=4 einnehmen?

Welche Quantenzahl wird beeinflusst, wenn zwischen den Halbkugeln des Hohlraumresonators ein Ring eingesetzt wird? 

Wie lauten die ersten neun Legendrepolynome?

4.
Erklären Sie anhand des akustischen Modells die Wellenfunktion eines Wasserstoffmoleküls. 

Wie werden die parallelen und antiparallelen Zustände im Experiment realisiert? 

Welche Quantenzahlen hat der $2 sigma_u$-Zustand des Wasserstoffmoleküls?

5.
Wie sieht das Bändermodell eines Festkörpers aus? 

Was sind erlaubte und verbotene Bänder im Festkörper? 

Wie sehen die Elektronenenergien in Abhängigkeit von der Wellenzahl k aus?

6.
Was ist Dispersion? 

Erklären Sie die Dispersion einer Schallwelle und einer elektromagnetischen (optischen) Welle.

Welches sind die Unterschiede und die Gemeinsamkeiten? 

Wie lautet die spektrale Funktion eines Überganges?

7.
Welche Defekte können in einem Kristallgitter auftreten?

Welche dieser Defekte können mit dem akustischem Modell realisiert werden?