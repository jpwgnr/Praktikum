\section{Diskussion}
\label{sec:Diskussion}

Die Schallgeschwindigkeit ergibt sich zu einem Wert von \SI{3}{\meter\per\second}. Der Literaturwert liegt bei \SI{3}{\meter\per\second}, was einer Abweichung von \SI{3}{\percent} entspricht. 

Die Durchführung des Versuchs mit dem Oszilloskop war aus zeitlichen Gründen nicht möglich. 
Insgesamt wurden die Peaks gesehen, so wie sie von der Theorie erwartet wurden und die Formel \eqref{eq:lambda} konnte so bestätigt werden. 

Bei der Analogie zum Festkörper wurde entsprechend der theoretischen Erwartung das erwartete Verhalten eines 1D-Festkörpers gesehen. Bei einer Erhöhung der Elemente lässt sich erkennen, dass \dots
Beim ersetzen der Blende ergibt sich \dots
Beim ersetzen eines Zylinders durch verschiedene Größen ergibt sich, dass \dots
Die Variation der Länge der Resonatoren ergibt analog zur Festkörperphysik die Variation der Massepunkte.
Das Variieren der Blenden entspricht dem Analogon verschiedener Potentiale zwischen den Elektronen. 

Das Wasserstoff Analogon mithilfe des Kugelresonators ergibt für vier verschiedene Resonanzen Veränderungen der Druckamplitude bei Variation des Winkels. Dies war zu sehen in den Abb. \ref{fig:winkelamp}. 
Bei einer Variation der Ringdicke ergaben sich die erwarteten $m=1$ Resonanzen. 
Bei einer Zwischenringdicke von \SI{9}{\milli\meter} wurde die Amplitude erneut in Abhängigkeit von der Winkelabhängigkeit gemessen. 

Um die Frequenz \SI{2.3}{\kilo\hertz} wurde ein doppelter Kugelresonator als Analogon zum Wasserstoffmolekül untersucht. 
Dabei wurden nicht alle vier Peaks beobachtet, da \dots

