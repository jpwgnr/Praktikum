\section{Diskussion}
\label{sec:Diskussion}

Die Schallgeschwindigkeit ergibt sich zu einem Wert von \SI{342.6(7)}{\meter\per\second}. Der Literaturwert liegt bei \SI{343.38}{\meter\per\second}, was einer Abweichung von \SI{0.2}{\percent} entspricht. 

Die Durchführung des Versuchs mit dem Oszilloskop war aus zeitlichen Gründen nicht möglich. 
Insgesamt wurden die Peaks gesehen, so wie sie von der Theorie erwartet wurden und die Formel \eqref{eq:speed} konnte so bestätigt werden. 
Bei der Analogie zum Festkörper wurde entsprechend der theoretischen Erwartung das erwartete Verhalten eines 1D-Festkörpers gesehen. Bei einer Erhöhung der Elemente lässt sich erkennen, dass wie erwartet die Bänder mit den verbotenen Lücken entstanden sind. Je kleiner die Blenden waren, also je kleiner die Kopplung war, desto enger waren die Bänder. 
Durch das Einfügen einer Defektmasse ergaben sich Lücken innerhalb der Bänder. 
Bei alternierenden Massen wurden Resonanzen von beiden Massen beobachtet.
Das Variieren der Blenden entspricht dem Analogon verschiedener Potentiale zwischen den Elektronen. 

Das Wasserstoff Analogon mithilfe des Kugelresonators ergibt für vier verschiedene Resonanzen Veränderungen der Druckamplitude bei Variation des Winkels. Die Werte für Winkel unter \SI{45}{\degree} waren meist schlecht. Dies könnte auf Messfehler hindeuten oder besonders große Unsicherheiten bei niedrigen Winkeln. Die Ursache dafür ist nicht klar. 
Die Quantenzahlen $l=1,2,3,5$ bei $m=0$ konnten gemessen werden. Dank der Symmetriebrechung durch die Ringe wurde dort auch $l = 1$ bei $m=0$ bzw. $m=1$ gemessen. Die Fits an die Legendrepolynome haben gut gepasst und das Modell passt gut als Analogie. 
Zwischen der Aufspaltung der Peaks und der Ringdicke wurde eine lineare Relation analog zum Zeeman-Effekt gefunden. 

Beim Wasserstoffmolekül konnte beobachtet werden, dass die 1. und 2. Resonanz  unabhängig vom Blendendurchmesser konstant bei einer Frequenz blieben. 
Für die dritte Resonanz, bei \SI{2.31}{\kilo\hertz}, wurde eine Linearität zum Blendendurchmesser festgestellt.
Die Ergebnisse zur Winkelabhängigkeit haben erneut analog zu der Symmetriebrechung im vorherigen Abschnitt die Quantenzahlen $m = 0$ bei den Resonanzen \SI{2.3}{\kilo\hertz} und \SI{2.41}{\kilo\hertz} und $m = 1$ bei \SI{2307}{\hertz} und bei \SI{2311}{\hertz} ergeben. Dies deutet auf den bindenden $2 \sigma$-Zustand bei der \SI{2.3}{\kilo\hertz} Frequenz, auf die beiden $1 \pi$-Zustände bei \SI{2307}{\hertz} bzw. \SI{2311}{\hertz} und auf den anti-bindenden $2 \sigma$-Zustand bei \SI{2.41}{\kilo\hertz} hin. 
Auch hier funktioniert die Analogie gut.