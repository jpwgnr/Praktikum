\section{Theorie}
\label{sec:Theorie}

In diesem Versuch werden akustische Resonatoren als Analogien zu quantenmechanischen Systemen genutzt. 

\subsection{Resonanzsystem}

Das Resonanzsystem, das mittels eines Lautsprechers und eines Mikrofons, das Verhalten eines Wasserstoffatoms in Kapitel \ref{sec:Wasserstoff}, eines Wasserstoffmoleküls in Kapitel \ref{sec:Wasserstoffmolekül} und eines 1D-Festkörpersystems in Kapitel \ref{sec:Festkörper} simuliert, erzeugt nicht konstante Frequenzen auch für den Fall, dass kein Resonator vorliegt.
Dieses Verhalten kann über die sogenannte Transferfunktion erklärt werden, die durch die Eigenschaften des Lautsprechers und Mikrofons beeinflusst wird. 


\subsection{Wasserstoffatom}
\label{sec:Wasserstoff}
Das Wasserstoff wird mittels eines Kugelresonators simuliert. Der Druck lässt sich durch die Helmholtzgleichung beschreiben. 
Diese ist gegeben mit 
\begin{equation*}
    \frac{\partial^2}{\partial t^2} P(\vec r,t) = \frac{1}{\rho\kappa} \Delta P(\vec r,t),
\end{equation*}
wobei $\rho$ die Dichte ist und $\kappa$ die Kompressibilität. 

Diese Differentialgleichung lässt sich mit dem Ansatz $p(r, \theta, \phi) = Y^m_l(\theta, \phi) \cdot f(r)$ lösen.
Die $Y^m_l(\theta, \phi)$ heißen Kugelflächenfunktionen. und beschreiben wie die zeitunabhängige Druckverteilung auf der Oberfläche der Kugel aussieht. 

Die Kugelflächenfunktionen ergeben sich zu 
\begin{equation*}
    Y^m_l (\theta, \phi) = \sqrt{\frac{(2l+1)(l-m)!}{4\pi(l+m)!}} \cdot P^m_l(\cos \theta) \cdot e^{i m \phi},
\end{equation*}
wobei $P^m_l(\cos \theta)$ die Legendrepolynome 
\begin{equation*}
P^m_l(cos \theta) = \frac{(-1)^m}{2^l l!} (1- \cos^2 \theta)^{m/2} \frac{d^{l+m}}{d(\cos \theta)^{l+m}} (\cos^2 (\theta -1))^l
\end{equation*}
sind.
Bei Kugelsymmetrie ist die Energie entartet bezüglich $m$. 
Erst durch aufheben der Symmetrie wird die Entartung aufgehoben.  
Das bedeutet, dass sie in dem Spektrum dann nicht mehr bei der gleichen Frequenz zu sehen sind. Da sich $\pm m$ aber nur in der Phase unterscheiden, bleiben die beiden entartet und $l+1$ peaks sind zu sehen. 
Ein imaginärer Druck ist physikalisch nicht sinnvoll und kann im Experiment nicht gemessen werden. Die gemessenen Ergebnisse sind für also für alle $m$ proportional zu $\cos(m\phi)$. 

\subsection{Wasserstoffmolekül}
\label{sec:Wasserstoffmolekül}

Mit einem zweiten Kugelresonator wird ein Wasserstoffmolekül simuliert. 
Die beiden Kugeln können über verschiedene Blenden zusammengesetzt werden um verschieden starke Kopplungen zwischen den beiden Atomen zu simulieren. 
Die Orbitale von Molekülen werden mit griechischen Buchstaben abgekürzt. Der $1\sigma$ Zustand setzt sich aus zwei $1s$ Zuständen mit positivem Vorzeichen zusammen. 

Durch Verwenden zweier gekoppelter Kugelresonatoren kann der Grundzustand $1\sigma$ zwar nicht sichtbar gemacht werden, da dafür die Resonanzfrequenz \SI{0}{\hertz} gemessen werden müsste, aber der anti-bindende Zustand bei verschiedenen Vorzeichen, kann gemessen werden in dem Resonator. 

\subsection{Festkörper}
\label{sec:Festkörper}
Der Festkörper wird durch einen Rohrresonator simuliert. In dem Resonator werden stehende Wellen für die Bedingung 
\begin{equation*}
    L = n \frac{\lambda}{2} = n \frac{c}{2f}
\end{equation*}
erzeugt. 

Diese Bedingung folgt aus der Definition der Druckverteilung in dem Rohr. Diese kann mathematisch mithilfe der Helmholtz-Gleichung beschrieben werden als 
\begin{equation*}
    \frac{\partial^2}{\partial t^2} P(x,t) = \frac{1}{\rho\kappa}\frac{\partial^2}{\partial x^2} P(x,t)
\end{equation*}
mit der Dichte $\rho$ und der Kompressibilität $\kappa$. 

Analog beschreibt diese Resonanz ein Teilchen, das in einem Kasten eingesperrt ist. 
Die Wellenfunktion dieses Teilchens ist die Schrödingergleichung mit 
\begin{equation*}
    i \hbar \frac{\partial}{\partial t} \Psi(x,t) = \left(- \frac{\hbar^2}{2m} \frac{\partial^2}{\partial x^2} + V(x)\right) \Psi(x,t).
\end{equation*}
Dabei ist $m$ die Teilchenmasse, $V(x)$ ein Potential. 

In einem Potentialtopf ($V(0 < x < L) = 0, V(x \leq 0, x \geq L) \to \infty$) der Länge $L$ wird die zeitunabhängige Schrödingergleichung zu 
\begin{equation*}
    E \psi(x) = - \frac{\hbar^2}{2m} \frac{\partial^2}{\partial x^2} \psi(x).
\end{equation*}

Diese Gleichung kann durch eine ebene Welle gelöst werden. Damit ergibt sich dann aus der Randbedingung erneut
\begin{equation*}
    L = n \frac{\lambda}{2} = n \frac{c}{2f}.
\end{equation*}

Mit Blenden entspricht das System dem eines gekoppelten Pendels.
Wie beim gekoppelten Pendel bekommt man für jedes zusätzliche Rohrsegment eine zusätzliche resonante Frequenz. 
Die eine Resonanz ist dann an der gleichen Stelle, bei den Pendeln wäre dies der Fall des gleich Schwingens, die andere Resonanz ist dann der gegenschwingende Fall. 
Bei einem dritten Pendel gibt es dann noch zusätzlich den Fall, dass zwei gleich Schwingen und eines der Pendel entgegengesetzt.
Für unendlich viele Rohrsegmente spricht man dann von einem Band. Der Nachteil mit dem Rohr ist, dass Oberschwingungen nicht abgebildet werden können und das Gleich-Schwingen nicht beobachtet werden kann. 

Wenn die Gesamtlänge des Rohrresonators $L$ ist, ist die Länge eines Segments $L_S = \frac{1}{2} L$. 
Bei 

Die Analogie besteht darin, dass die Kopplung der Segmente mittels der Blenden einem Potential entspricht und somit auch Eigenschaften wie die Bandstrukturen beschrieben werden können. 




