\section{Theorie}
\label{sec:Theorie}

In diesem Versuch werden akustische Resonatoren als Analogien zu quantenmechanischen Systemen genutzt. 

\section{Resonanzsystem}

Das Resonanzsystem, das mittels eines Lautsprechers und eines Mikrofons, das Verhalten eines Wasserstoffatoms in Kapitel \ref{sec:Wasserstoff}, eines Wasserstoffmoleküls in Kapitel \ref{sec:Wasserstofmolekül} und eines 1D-Festkörpersystems in Kapitel \ref{sec:Festkörper} simuliert, erzeugt nicht konstante Frequenzen auch für den Fall, dass kein Resonator vorliegt.
Dieses Verhalten kann über die sogenannte Transferfunktion erklärt werden, die durch die Eigenschaften des Lautsprechers und Mikrofons beeinflusst wird. 


\section{Wasserstoffatom}


\section{Festkörper}

Der Festkörper wird durch einen Rohrresonator simuliert. In dem Resonator werden stehende Wellen für die Bedingung 
\begin{equation*}
    L = n \frac{\lambda}{2} = n \frac{c}{2f}
\end{equation*}
erzeugt. 

Diese Bedingung folgt aus der Definition der Druckverteilung in dem Rohr. Diese kann mathematisch mithilfe der Helmholtz-Gleichung beschrieben werden als 
\begin{equation*}
    \frac{\partial^2}{\partial t^2} P(x,t) = \frac{1}{\rho\kappa}\frac{\partial^2}{\partial x^2} P(x,t).
\end{equation*}

