\section{Durchführung}
\label{sec:Durchführung}

In dem Versuch werden verschiedene akustische Resonatoren genutzt, um als quantenmechanische Analogien zu dienen.
Dabei wird eine Software benutzt, die Frequenzen in bestimmten Bereichen auf einen Lautsprecher gibt, die anschließend die Daten wieder mittels eines Mikrofons misst. 

\subsection{1D-Festkörper}

Im ersten Versuchsteil werden \SI{50}{\milli\meter} lange Zylinder als Resonatoren genutzt. Anschließend wird die Anzahl der Zylinder nach und nach erhöht und die zusätzlich auftretenden Resonanzen werden gemessen. 
Dafür werden bis zu \num{12} Zylinder genutzt.
Das Frequenzspektrum, in dem gemessen wird, geht dabei von \SI{0.1}{\kilo\hertz} bis \SI{12}{\kilo\hertz}. 
Außerdem wird das Spektrum eines Zylinders aufgenommen, der die Länge \SI{75}{\milli\metre} hat.

Das Ganze wird noch einmal mit Blenden zwischen den Zylindern wiederholt für zwei bis zehn Zylinder mit jeweils ein bis neun Blenden. Die Messung wird mit Blenden von drei verschiedenen Durchmessern durchgeführt.

Anschließend wird einer der Rohrresonatoren durch Zylinder mit drei verschiedenen Längen (\SI{37.5}{\milli\meter}, \SI{62.5}{\milli\metre}, \SI{75}{\milli\metre}) ersetzt und die Veränderung analysiert. 

Als nächstes werden abwechselnd \SI{50}{\milli\metre} und \SI{75}{\milli\metre} Zylinder aneinander gereiht mit \SI{16}{\milli\meter} Blenden. 
Danach werden bei gleichbleibenden Zylindern der Länge \SI{50}{\milli\metre} die Blenden abgewechselt mit \SI{13}{\milli\metre} und \SI{16}{\milli\meter} Blenden. 

\subsection{Wasserstoffatom}

Ein Kugelresonator wird genutzt. Der Lautsprecher und das Mikrofon stehen sich am Anfang direkt gegenüber mit einem Winkel von $\alpha = \SI{180}{\degree}$. 
Es wird im Frequenzbereich von \SI{0.1}{\kilo\hertz} bis \SI{12}{\kilo\hertz} gemessen. 

Für vier Resonanzen wird die Druckamplitude als Funktion des Drehwinkels gemessen. Der Winkel wird dabei von \SI{0}{\degree} bis \SI{180}{\degree} in \SI{5}{\degree} Schritten variiert. Die interessanten Peaks sind bei \SI{2.3}{\kilo\hertz}, \SI{3.7}{\kilo\hertz}, \SI{5.0}{\kilo\hertz} und \SI{7.4}{\kilo\hertz}.

Die Symmetrie wird im nächsten Teil mittels eines Zwischenrings gebrochen. Es werden verschiedene Zwischenringe verwendet. Die Messung wird bei \SI{180}{\degree} durchgeführt und es wird nur der Bereich zwischen \SI{1.8}{\kilo\hertz} und \SI{2.6}{\kilo\hertz} gemessen. 

Für einen Zwischenring der Höhe \SI{9}{\milli\metre} wird die Winkelabhängigkeit der Amplitude wie zuvor gemessen. 

\subsection{Wasserstoffmolekül}

In diesem Teil des Versuchs wird ein zusätzlicher Kugelresonator eingesetzt mit einer Lücke zwischen beiden Kugeln. 
Im ersten Teil werden die Blenden zwischen beiden Kugeln variiert. Es wird eine \SI{10}{\milli\metre}, \SI{13}{\milli\metre} und eine \SI{16}{\milli\metre} Blende genutzt. Es wird im Bereich von \SI{2.2}{\kilo\hertz} bis \SI{2.5}{\kilo\hertz} gemessen. 

Als letztes wird bei einer Blende mit konstantem Durchmesser von \SI{16}{\milli\metre} auf dem gleichen Bereich des Spektrums die Winkelabhängigkeit gemessen. 
Bei \SI{180}{\degree} werden dann die Spektren der oberen als auch der unteren Kugel aufgenommen. 

