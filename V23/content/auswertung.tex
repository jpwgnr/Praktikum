\section{Auswertung}
\label{sec:Auswertung}

Die Versuche wurden aus Zeitgründen nur mit dem Computer durchgeführt. Analysiert wurden die Ergebnisse mittels matplotlib, numpy, scipy und uncertainties.

\subsection{1D-Festkörper}

Die Peaks der Resonanzen sind für die Messungen für ein bis zwölf Zylinder in Tabelle \ref{tab:rohr} aufgetragen. 
Daraus lässt sich mit Formel \ref{eq:speed} die Schallgeschwindigkeit zu \SI{320}{\meter\per\second} bestimmen. 

Das Spektrum für \num{12} Zylinder ist in Abb. \ref{fig:spec12} aufgetragen.

\begin{figure}
    \centering
    %\includegraphics[width=0.8\textwidth]{plots/spec12.pdf}
    \caption{Das Frequenzspektrum für 12 Zylinder mit jeweils einer Länge von \SI{55}{\milli\metre}.}
    \label{fig:spec12}
\end{figure}

Für den größeren Zylinder ergibt sich analog eine Schallgeschwindigkeit von \SI{320}{\meter\per\second}.

Für jedes Rohr mit einer Blende kommt ein zusätzlicher Peak hinzu. 
Das Spektrum für \num{10} Zylinder ist in Abb. \ref{fig:spec10} zu sehen. 

\begin{figure}
    \centering
    %\includegraphics[width=0.8\textwidth]{plots/spec10.pdf}
    \caption{Das Frequenzspektrum für 10 Zylinder mit jeweils einer Länge von \SI{55}{\milli\metre} mit Blendendurchmesser von \SI{16}{\milli\metre}.}
    \label{fig:spec10}
\end{figure}

Das ganze wird erneut durchgeführt mit einer Blende mit \SI{13}{\milli\meter}. Ein Beispiel ist in Abb. \ref{fig:spec10_13} zu sehen. 

\begin{figure}
    \centering
    %\includegraphics[width=0.8\textwidth]{plots/spec10_13.pdf}
    \caption{Das Frequenzspektrum für 10 Zylinder mit jeweils einer Länge von \SI{55}{\milli\metre} mit Blendendurchmesser von \SI{13}{\milli\metre}.}
    \label{fig:spec10_13}
\end{figure}

Die Messung wurde erneut variiert in dem einer der Zylinder durch einen Zylinder anderer Länge ersetzt wurde. Die neuen Längen sind \SI{37.5}{\milli\meter}, \SI{62.5}{\milli\meter} und \SI{75}{\milli\meter}.
Die Veränderung ist in Abb. \ref{fig:var3} zu sehen.

Eine Kette aus 10 Zylindern wurde abwechselnd aus \SI{50}{\milli\meter} und \SI{75}{\milli\meter} Zylindern zusammengesetzt mit \SI{16}{\milli\meter} Blenden.

Das Ergebnis ist in Abb. \ref{fig:var4} zu sehen. 

Die letzte Variation des Rohrresonators sind die \SI{13}{\milli\meter} und \SI{16}{\milli\meter} Blenden. 

Die Ergebnisse sind in Abb. \ref{fig:var5} zu sehen.

\subsection{Wasserstoff Atom}

In Abb. \ref{fig:Wasserstoff1} ist das Frequenzspektrum eines Kugelresonators zu sehen bei einem Winkel von $\alpha = \SI{180}{\degree}$. 
Dabei werden Daten in \SI{5}{\hertz} Schritten bei \SI{60}{\milli\second} pro Schritt aufgenommen.

Die Winkelabhängigkeit der Amplitude von den Resonanzen bei \SI{2.3}{\kilo\hertz}, \SI{3.7}{\kilo\hertz}, \SI{7.4}{\kilo\hertz} und bei \SI{1}{\kilo\hertz} wird im folgenden untersucht. 

In Tab. \ref{tab:winkelamp} sind die Amplitudenwerte für alle drei Peaks bei Winkeln von \SI{0}{\degree} bis \SI{180}{\degree} aufgetragen und in Abb. \ref{fig:Polar} als Polarplots abgebildet.

Nur der Peak bei ca. \SI{2.3}{\kilo\hertz} wird mit verschiedenen Zwischenringen überprüft. Die Aufspaltung ist in Tab. \ref{tab:aufspaltung} gegen die Dicke der Ringe aufgetragen.

Mit einer Dicke des Rings von \SI{9}{\milli\meter} wird die Winkelabhängigkeit erneut überprüft.

In Abb. \ref{fig:polar2} ist dieser als Polarplot aufgetragen. Die dazugehörigen Quantenzahlen sind $l = 0, 1,2,3,4,5$ und $ m = 0, 1, 2,3,4,5$. 

\subsection{Wasserstoff Molekül}

Das Frequenzspektrum der Resonanz ist in Abb. \ref{fig:molek1} für drei verschiedene Blenden aufgetragen. 
Es ist zu beobachten, dass \dots

Das Frequenzspektrum wurde für verschiedene Winkel aufgenommen. Es wurden die Quantenzustände x,y vermessen. 
Die Phasenverschiebung zwischen der oberen und unteren Kugel wurden bei \SI{180}{\degree} bestimmt. 
Nicht alle vier Peaks sind zu erkennen. Der xy Peak fehlt. 