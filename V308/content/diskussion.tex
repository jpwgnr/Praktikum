\section{Diskussion}
\label{sec:Diskussion}

Die lange Spule hat einen maximalen Wert von \SI{2.19}{\milli\tesla}. Der theoretisch berechnete Wert liegt bei \SI{2.356}{\milli\tesla} und hat somit eine relative Abweichung von \SI{7.05}{\percent}. Ansonsten entsprechen die beobachteten Messungen dem Bild, dass zu erwarten war. 

Bei der kurzen Spule passen Theoriekurve und experimental gemessene Kurve auch gut. Die Abweichung im Hochpunkt liegt bei SI{40}{percent}. 

Das Spulenpaar, als Helmholtzspule, passt ebenfalls gut zur Theoriekurve. Der relative Fehler des Werts in der Mitte des Paares liegt bei einem Wert von \SI{55.5}{percent}.
Bei einem größeren Abstand ließ sich der Wert, ließ sich die Kurve gut als Addition zweier Spulen beschreiben. Bei einem Strom von \SI{4}{\ampere} lag die relative Abweichung in der Mitte des Paares bei \SI{10.71}{\percent}. 
Bei der dritten Messung, die erneut den größeren Abstand hatte, aber nur einen Strom von \SI{3}{\ampere}, wurde ein relativer Fehler von \SI{6.84}{percent} festgestellt. Somit liegen alle Fehler in einem Bereich von \SI{5}{\percent} bis \SI{60}{\percent}, was doch relativ viel ist. Grund dafür könnte die etwas wackelige Apperatur gewesen sein. Außerdem musste die Hall-Sonde immer exakt gleich ausgerichtet sein. Drehte man sie nur leicht, änderten sich die gemessenen Werte bereits extrem. Vermutlich war die Hallsonde bei der Messung der kurzen Spule und der ersten Messung des Spulenpaares nicht ganz senkrecht, wodurch die starke Abweichung zu dem Theoriewerten entstanden ist. 

Die Magnetisierung funktionierte recht gut. Die Hysteresekurve sieht genauso aus, wie sie zu erwarten war. 
