\section{Durchführung}
\label{sec:Durchführung}

\subsection{Eine lange und eine kurzen Spule im Magnetfeld}
Im ersten Teil des Versuchs wird eine lange Spule an ein Netzgerät angeschlossen. 
Anschließend wird der Strom und die Spannung eingeschaltet, sodass mit der Messung 
begonnen werden kann. 
Die Werte werden mittels einer longitudinalen Hall-Sonde innerhalb und außerhalb der
Spule gemessen. Danach wird die ganze Messung noch einmal mit einer kurzen Spule 
wiederholt. 

\subsection{Spulenpaare im Magnetfeld}

Es wird das Magnetfeld eines in Reihe geschalteten Spulenpaares gemessen. Dabei werden verschiedene Spulenabstände eingestellt und das Magnetfeld wird mittels einer transversalen Hall-Sonde innerhalb und außerhalb des Spulenpaares gemessen. 

\subsection {Hysteresekurve einer Ringspule} 

Die Ringspule hat einen Luftspalt. Mit einer transversalen Hall-Sonde wird das Magnetfeld der Ringspule in Abhängigkeit zum Spulenstrom gemessen. Aus den bestimmten Daten lässt sich anschließend eine Hysteresekurve ermitteln. Aus der graphischen Darstellung lassen sich verschiedene Faktoren, wie Sättigungsmagnetisierung, die Remanenz und die Koerzitivkraft ablesen. ablesen.  
