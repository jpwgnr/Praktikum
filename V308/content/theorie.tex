\section{Theorie}
\label{sec:Theorie}

\cite{V308}

\subsection{Allgemeine Grundlagen zu Magnetfeldern}

Die magnetische Flußdichte $\symbf{B}$ wird durch die Permeabilität $\mu$ und die magnetische Feldstärke $\symbf{H}$ dargestellt. Es gilt die Beziehung 
\begin{align} 
\symbf{B} &= \mu /cdot \symbf{H}
\end{align}.

$\mu$ besteht dabei aus der Vakuum-Permeabilität $\mu_{0}$ und der relativen Permeabilität $\mu_{r}$, die wiederum von der jeweiligen Materie abhängt. 
Es gilt 
\begin{align} 
\mu &= \mu_{0} \cdot \mu_{r}
\end{align}.

Die magnetische Flußdichte im Mittelpunkt eines Ringes wird mit der Formel 

\begin{equation}
\symbf{B}(x)= n \cdot \frac{\mu_{0}}{2} \frac{R^2}{(R^2 +x^2)^{3/2}}\cdot \symbf{\hat{x}}
\end{equation}
beschrieben. 

Dabei ist R der Radius des Rings und $n$ die Anzahl der Windungen. 


Bei einer langen Spule (Solenoid) ist die magnetische Feldstärke in der Mitte der Spule konstant. Außerhalb der Spule ist der magnetische Fluß inhomogen. 
Das innere Feld ist proportional zur Länge der Spule $l$, zur Anzahl der Windungen $n$ und zum Strom $I$, der durch die Spule fließt. 
Es gilt die Beziehung
\begin{equation}
B = \mu_{r} \mu_{0} \frac{n}{l} I 
\end{equation}.

Wird dieser Solenoid zu einem Ring gebogen verschwinden die Randeffekte und das Feld außerhalb des Rings wird null. 
Es gilt

\begin{equation}
B = \mu_{r} \mu_{0} \frac{n}{2\pi r_{R}} I 
\end{equation}.

Ein Helmholtz-Spulenpaar hat ein homogenes Magnetfeld im Innern der zwei Ringe. (s. Abb) %Abbildungsref einfügen 

Es ergibt sich zu 
\begin{equation}
B(0)= B_{1}(x) + B_{1}(-x) = \frac{\mu_{0} I R^2}{(R^2 + x^2)^5/2}

Ferromagnetische Materialien besitzen ohne äußere Magnetfelder ein permanentes magnetisches Moment. Sie richten sich in einzelnen Bereichen parallel zu einander aus. Man nennt diese Bereiche Weiß'sche Bezirke. Im unmagnetischen Zustand ist die Ausrichtung der Bereiche statistisch verteilt. Ein äußeres Magnetfeld sorgt für eine Änderung der Richtung der magnetischen Momente und vergrößert somit die Weiß'schen Bereiche. 

Die relative Permeabilität \mu_{r} ist in ferromagnetischen Materialien sehr hoch und ist nicht mehr linear proportional zur magnetischen Flußdichte. 
Die andere Abhängigkeit kann durch eine Hysteresekurve dargestellt werden. 
In der Kurve lassen sich verschiedene markante Punkte erkennen. 
Ohne äußeres Magnetfeld gilt $B = H = 0$. Wird ein Magnetfeld angelegt, steigt die Magnetisierung an, bis ein Sättigungswert $B_{S}$ erreicht wird bei $H_{S}$. Der Verlauf dorthin wird Neukurve genannt. 
Wird das äußere Magnetfeld verringert bilden sich Bereiche mit entgegengesetzer Magnetisierung. So bleibt, obwohl das äußere Magnetfeld abgeschaltet wurde eine Remanenz bestehen. 
Ein Gegendfeld $H_{C}$ kann diese Ausrichtungen wieder aufheben. Wenn das Feld noch weiter erhöht wird, wird die Magnetisierung in dem Stoff negativ und erreicht auch wieder den Sättigungswert $B_{S}$. Wird das äußere Magnetfeld wieder umgekehrt entsteht eine punktsymmetrische Kurve zum Ursprung, die sich je nach Materialeigenschaften in der Schärfe unterscheidet. 

Die differentielle Permeabilität $\mu_{diff}$ deffiniert sich als 

\begin{equation}
\mu_{diff} = \frac{1}{\mu_{0}} \frac{dB}{dH}
\end{equation}.



 
