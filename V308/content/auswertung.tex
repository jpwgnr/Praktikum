\section{Auswertung}
\label{sec:Auswertung}

% \begin{figure}
%  \centering
%  \includegraphics{<++>}
%  \caption{<++>}
%  \label{fig:<++>}
% \end{figure}
Für die Auswertung wird Python, im Speziellen Matplotlib \cite{matplotlib}, NumPy \cite{numpy},
SciPy \cite{scipy} und Uncertainties \cite{uncertainties} verwendet.

\subsection{Magnetische Flussdichte einer langen und einer kurzen Spule}
%Tabelle lange Spule
Die verwendete lange Spule hat $\num{300}$ Windungen und eine Länge
von $\SI{24}{\centi\meter}$. %richtige Werte einfügen
Die Werte der magnetischen Flussdichte an den Stellen $x$ inner- und außerhalb
der langen Spule befinden sich in Tabelle \ref{taba1}.
Dieselben Werte sind in Abbildung \ref{plota1} gegeneinander aufgetragen.
\begin{table}\caption{Der magnetische Fluss $B$ an verschiedenen Stellen $x$ vor der langen Spule.}
\label{taba1}
\centering
\sisetup{round-mode = places, round-precision=3, round-integer-to-decimal=true}
\begin{tabular}{S[]S[]} 
\toprule
{$B$/ \si{\milli\tesla}} & {$x$/ \si{\centi\meter}}\\
\midrule
0.382 & 0.0\\
0.505 & 0.5\\
0.744 & 1.0\\
0.959 & 1.5\\
1.35 & 2.0\\
1.544 & 2.5\\
1.7619999999999998 & 3.0\\
1.928 & 3.5000000000000004\\
2.024 & 4.0\\
2.093 & 4.5\\
2.1380000000000003 & 5.0\\
\bottomrule
\end{tabular}\end{table}
\begin{table}\caption{Der magnetische Fluss $B$ an verschiedenen Stellen $x$ in der langen Spule.}
\label{taba12}
\centering
\sisetup{round-mode = places, round-precision=3, round-integer-to-decimal=true}
\begin{tabular}{S[]S[]} 
\toprule
{$B$/ \si{\milli\tesla}} & {$x$/ \si{\centi\meter}}\\
\midrule
2.2030000000000003 & 6.0\\
2.2239999999999998 & 6.5\\
2.238 & 7.000000000000001\\
2.248 & 7.5\\
2.2560000000000002 & 8.0\\
2.262 & 8.5\\
2.265 & 9.0\\
2.2659999999999996 & 9.5\\
2.264 & 10.0\\
2.2560000000000002 & 10.5\\
2.2520000000000002 & 11.0\\
2.247 & 11.5\\
2.241 & 12.0\\
2.231 & 12.5\\
2.219 & 13.0\\
2.221 & 13.5\\
2.222 & 14.000000000000002\\
2.22 & 14.499999999999998\\
2.216 & 15.0\\
2.212 & 15.5\\
2.238 & 16.0\\
2.245 & 16.5\\
2.2230000000000003 & 17.0\\
2.209 & 17.5\\
2.1870000000000003 & 18.0\\
2.1519999999999997 & 18.5\\
2.088 & 19.0\\
2.0 & 19.5\\
1.874 & 20.0\\
1.673 & 20.5\\
\bottomrule
\end{tabular}\end{table}
\begin{table}\caption{Der magnetische Fluss $B$ an verschiedenen Stellen $x$ nach der langen Spule.}
\label{taba1}
\centering
\sisetup{round-mode = places, round-precision=3, round-integer-to-decimal=true}
\begin{tabular}{S[]S[]} 
\toprule
{$B$/ \si{\milli\tesla}} & {$x$/ \si{\centi\meter}}\\
\midrule
1.1509999999999998 & 21.5\\
0.82 & 22.0\\
0.575 & 22.5\\
0.416 & 23.0\\
0.303 & 23.5\\
\bottomrule
\end{tabular}\end{table}

%Grafik lange Spule (xB-Diagramm)
\begin{figure}
    \centering
    \includegraphics{build/plota1.pdf}
    \caption{Die magnetische Flussdichte $B$ ist gegen die Position $x$ inner- 
    und außerhalb der langen Spule aufgetragen.}
    \label{plota1}
\end{figure}

%Vergleich der Werte mit Theorie
\noindent 

%Tabelle kurze Spule
\noindent Die kurze Spule hat $\num{100}$ Windungen und eine Länge von
$\SI{12}{\centi\meter}$. %richtige Werte einfügen
In Tabelle \ref{taba2} sind die Werte des magnetischen Flusses $B$
an verschiedenen Stellen $x$ aufgelistet.
Die Werte aus Tabelle \ref{taba2} sind in Abbildung \ref{plota2}
gegeneinadner aufgetragen.
\begin{table}\caption{Der magnetische Fluss $B$ an verschiedenen Stellen $x$ in der kurzen Spule.}
\label{taba2}
\centering
\sisetup{round-mode = places, round-precision=3, round-integer-to-decimal=true}
\begin{tabular}{S[]S[]} 
\toprule
{$B$/ \si{\milli\tesla}} & {$x$/ \si{\centi\meter}}\\
\midrule
0.231 & 0.0\\
0.325 & 0.5\\
0.475 & 1.0\\
0.67 & 1.5\\
0.94 & 2.0\\
1.2 & 2.5\\
1.4549999999999998 & 3.0\\
1.677 & 3.5000000000000004\\
1.821 & 4.0\\
1.9009999999999998 & 4.5\\
1.928 & 5.0\\
1.9020000000000001 & 5.5\\
1.8439999999999999 & 6.0\\
1.7249999999999999 & 6.5\\
1.5299999999999998 & 7.000000000000001\\
1.181 & 7.5\\
0.8630000000000001 & 8.0\\
0.486 & 8.5\\
0.35 & 9.0\\
0.255 & 9.5\\
0.19799999999999998 & 10.0\\
\bottomrule
\end{tabular}\end{table}

%Grafik kurze Spule (xB-Diagramm)
\begin{figure}
    \centering
    \includegraphics{build/plota2.pdf}
    \caption{Die magnetische Flussdichte $B$ ist gegen die Position $x$ inner-
    und außerhalb der kurzen Spule aufgetragen.}
    \label{plota2}
\end{figure}

%Vergleich der Werte mit Theorie
\noindent 

\subsection{Magnetische Flussdichte eines Spulenpaares}
%ABSTAND=RADIUS
%Tabelle mit I=4A
Zunächst wird der Abstand der beiden Spulen so festgelegt, dass
er den Radien der Spulen entspricht. Es handelt sich um ein
Helmholtz-Spulenpaar.
Es wird eine Stromstärke von $I = \SI{4}{\ampere}$ für die
Messung eingestellt.
Die Werte des magnetischen Flusses an verschiedenen Stellen
$x$ sind in Tabelle \ref{tabb1} aufgelistet. Dabei wird inner-
und außerhalb der Spulen gemessen.
Dieselben Werte sind in Abbildung \ref{plotb1} gegeneinander
aufgetragen.
\begin{table}\caption{Der magnetische Fluss $B$ an verschiedenen Stellen $x$ inner- und außerhalb des Spulenpaares bei einem Abstand von \SI{6.25}{\centi\meter} und einem Strom $I$ von \SI{4}{\ampere}.}
\label{tabb1}
\centering
\sisetup{round-mode = places, round-precision=3, round-integer-to-decimal=true}
\begin{tabular}{S[]S[]} 
\toprule
{$B$/ \si{\milli\tesla}} & {$x$/ \si{\centi\meter}}\\
\midrule
5.646 & 0.7000000000000001\\
5.635000000000001 & 0.8\\
5.632 & 0.8999999999999999\\
5.630999999999999 & 1.0\\
5.6290000000000004 & 1.0999999999999999\\
5.627 & 1.2\\
5.621 & 1.3\\
3.776 & 6.6000000000000005\\
3.75 & 6.7\\
3.7 & 6.800000000000001\\
3.662 & 6.9\\
3.5829999999999997 & 7.000000000000001\\
3.5239999999999996 & 7.1\\
3.464 & 7.199999999999999\\
3.3960000000000004 & 7.3\\
\bottomrule
\end{tabular}\end{table}

%Grafik (xB-Diagramm)
\begin{figure}
    \centering
    \includegraphics{build/plotb1.pdf}
    \caption{Die magnetische Flussdichte $B$ ist gegen die Position $x$ inner-
    und außerhalb des Helmholtz-Spulenpaares für eine Stromstärke von
    $\SI{4}{\ampere}$ aufgetragen.}
    \label{plotb1}
\end{figure}

%Vergleich mit Theorie
\noindent

%ABSTAND=DURCHMESSER
%Tabelle mit I=4A
\newline
\noindent Der Abstand der Spulen wird im nächsten Messabschnitt auf den
Durchmesser der Spulen erhöht.
Die eingestellte Stromstärke ist, wie im ersten Teil,
$I = \SI{4}{\ampere}$.
Die magnetische Flussdichte an verschiedenen Stellen $x$ ist
in Tabelle \ref{tabb2} dargestellt.
Diese Werte sind in Abbildung \ref{plotb2} gegeneinadner
aufgetragen.
\begin{table}\caption{Der magnetische Fluss $B$ an verschiedenen Stellen $x$ in- und außerhalb des Spulenpaares bei einem Abstand von \SI{6.25}{\centi\meter} und einem Strom $I$ von \SI{4}{\ampere}}
\label{tabb2}
\centering
\sisetup{round-mode = places, round-precision=3, round-integer-to-decimal=true}
\begin{tabular}{S[]S[]} 
\toprule
{$B$/ \si{\milli\tesla}} & {$x$/ \si{\centi\meter}}\\
\midrule
2.92 & 1.0\\
2.787 & 1.5\\
2.6740000000000004 & 2.0\\
2.5869999999999997 & 2.5\\
2.522 & 3.0\\
2.48 & 3.5000000000000004\\
2.467 & 4.0\\
2.4810000000000003 & 4.5\\
2.525 & 5.0\\
2.59 & 5.5\\
2.669 & 6.0\\
2.784 & 6.5\\
2.943 & 7.000000000000001\\
3.035 & 7.5\\
2.454 & 13.0\\
2.111 & 13.5\\
1.8940000000000001 & 14.000000000000002\\
1.534 & 14.499999999999998\\
1.3860000000000001 & 15.0\\
\bottomrule
\end{tabular}\end{table}

%Grafik (xB-Diagramm)
\begin{figure}
    \centering
    \includegraphics{build/plotb2.pdf}
    \caption{Die magnetische Flussdichte $B$ ist gegen die Position $x$ inner-
    und außerhalb des Spulenpaares mit einem Abstand, der gleich dem Durchmesser
    der Spulen ist, und einer Stromstärke von $\SI{4}{\ampere}$ aufgetragen.}
    \label{plotb2}
\end{figure}

%Vergleich mit Theorie
\noindent

%Tabelle mit I=3A
\noindent Zuletzt wird die angelegte Stromstärke verändert. Der Abstand
der Spulen bleibt unverändert der Durchmesser der Spulen.
Die verwendete Stromstärke ist $I = \SI{3}{\ampere}$.
Der magnetische Fluss an den Positionen $x$ inner- und außerhalb
der Spulen findet sich in Tabelle \ref{tabb3}.
Diese Werte sind in Abbildung \ref{plotb3} gegeneinadner
aufgetragen.
\begin{table}\caption{Die Brückenspannungen vor und nach dem Einlegen der Probe und die Widerstände vor- und nachher.}
\label{tabb1}
\centering
\sisetup{round-mode = places, round-precision=2, round-integer-to-decimal=true}
\begin{tabular}{S[]S[]S[]S[]} 
\toprule
{U_\text{Br,1} / \si{\milli\volt}} & {U_\text{Br,2} / \si{\milli\volt}} & {R_\text{1} / \si{\milli\ohm}} & {R_\text{2} / \si{\milli\ohm}}\\
\midrule
0.014 & 0.017 & 627.0 & 610.0\\
0.0145 & 0.017 & 629.0 & 610.0\\
0.016 & 0.014 & 625.0 & 610.0\\
\bottomrule
\end{tabular}\end{table}

%Grafik (xB-Diagramm)
\begin{figure}
    \centering
    \includegraphics{build/plotb3.pdf}
    \caption{Die magnetische Flussdichte $B$ ist gegen die Position $x$ inner-
    und außerhalb des Spulenpaares mit einem Abstand, der gleich dem Durchmesser
    der Spulen ist, und einer Stromstärke von $\SI{3}{\ampere}$ aufgetragen.}
    \label{plotb3}
\end{figure}

%Vergleich mit Theorie
\noindent

\subsection{Hysteresekurve einer Ringspule mit Luftspalt}
%Tabelle
Der magnetische Fluss $B$ der Ringspule wird im Luftspalt
des Rings gemessen.
In Tabelle \ref{tabc} befinden sich die Werte des magnetischen
Flusses für verschiedene eingestellte Stromstärken.
Diese Werte für den magnetischen Fluss sind in Abbildung
\ref{plotc} gegen die Stromstärke aufgetragen.
\begin{table}\caption{Der magnetische Fluss $B$ des gemessenen Magnetfelds gegen den Strom $I$ des erzeugenden Magnetfelds, Neukurve.}
\label{tabc}
\centering
\sisetup{round-mode = places, round-precision=1, round-integer-to-decimal=true}
\begin{tabular}{S[]S[]} 
\toprule
{$B$/ \si{\milli\tesla}} & {$I$/ \si{\ampere}}\\
\midrule
0.0 & 0.0\\
111.19999999999999 & 1.0\\
273.5 & 2.0\\
397.8 & 3.0\\
479.9 & 4.0\\
537.9000000000001 & 5.0\\
585.0999999999999 & 6.0\\
621.8000000000001 & 7.0\\
653.1 & 8.0\\
679.9 & 9.0\\
704.3000000000001 & 10.0\\
\bottomrule
\end{tabular}\end{table}
\begin{table}\caption{Die Indexwerte entsprechen der Höhe bei dem jeweiligen Strom und der Beschleunigungsspannung $U_\text{B} = \SI{360}{\volt}$.}
\label{tabc2}
\centering
\sisetup{round-mode = places, round-precision=3, round-integer-to-decimal=true}
\begin{tabular}{S[]S[]} 
\toprule
{Index} & {$I / \si{\ampere}$}\\
\midrule
1.0 & 0.0\\
2.0 & 0.325\\
3.0 & 0.75\\
4.0 & 1.175\\
5.0 & 1.55\\
6.0 & 1.95\\
7.0 & 2.375\\
8.0 & 2.8\\
9.0 & 3.225\\
\bottomrule
\end{tabular}\end{table}
\begin{table}\caption{Der magnetische Fluss $B$ des gemessenen Magnetfelds gegen das erzeugende H-Feld, Neukurve.}
\label{tabc3}
\centering
\sisetup{round-mode = places, round-precision=1, round-integer-to-decimal=true}
\begin{tabular}{S[]S[]} 
\toprule
{$B$/ \si{\milli\tesla}} & {$H$/ \si{\ampere\per\meter}}\\
\midrule
0.0 & 0.0\\
111.19999999999999 & 728.4399318436748\\
273.5 & 1456.8798636873496\\
397.8 & 2185.3197955310243\\
479.9 & 2913.759727374699\\
537.9000000000001 & 3642.199659218374\\
585.0999999999999 & 4370.639591062049\\
621.8000000000001 & 5099.079522905724\\
653.1 & 5827.519454749398\\
679.9 & 6555.959386593073\\
704.3000000000001 & 7284.399318436748\\
\bottomrule
\end{tabular}\end{table}
\begin{table}\caption{Der magnetische Fluss $B$ des gemessenen Magnetfelds gegen das erzeugende H-Feld.}
\label{tabc4}
\centering
\sisetup{round-mode = places, round-precision=1, round-integer-to-decimal=true}
\begin{tabular}{S[]S[]} 
\toprule
{$B$/ \si{\milli\tesla}} & {$H$/ \si{\ampere\per\meter}}\\
\midrule
689.1 & 6555.959386593073\\
671.6999999999999 & 5827.519454749398\\
651.9000000000001 & 5099.079522905724\\
627.7 & 4370.639591062049\\
600.7 & 3642.199659218374\\
565.1 & 2913.759727374699\\
519.2 & 2185.3197955310243\\
455.40000000000003 & 1456.8798636873496\\
326.2 & 728.4399318436748\\
122.80000000000001 & 0.0\\
-82.10000000000001 & -728.4399318436748\\
-256.29999999999995 & -1456.8798636873496\\
-391.40000000000003 & -2185.3197955310243\\
-479.6 & -2913.759727374699\\
-536.9000000000001 & -3642.199659218374\\
-582.8 & -4370.639591062049\\
-619.5 & -5099.079522905724\\
-649.5999999999999 & -5827.519454749398\\
-678.1 & -6555.959386593073\\
-702.7 & -7284.399318436748\\
-687.8 & -6555.959386593073\\
-669.6 & -5827.519454749398\\
-650.1 & -5099.079522905724\\
-627.4 & -4370.639591062049\\
-599.7 & -3642.199659218374\\
-564.2 & -2913.759727374699\\
-519.5999999999999 & -2185.3197955310243\\
-452.29999999999995 & -1456.8798636873496\\
-320.6 & -728.4399318436748\\
-123.30000000000001 & 0.0\\
78.39999999999999 & 728.4399318436748\\
258.90000000000003 & 1456.8798636873496\\
391.90000000000003 & 2185.3197955310243\\
479.0 & 2913.759727374699\\
537.6 & 3642.199659218374\\
585.0 & 4370.639591062049\\
620.1 & 5099.079522905724\\
650.1 & 5827.519454749398\\
678.6999999999999 & 6555.959386593073\\
\bottomrule
\end{tabular}\end{table}

%Grafik: Hysteresekurve
\begin{figure}
    \centering
    \includegraphics{build/plotc.pdf}
    \caption{Die magnetische Flussdichte $B$ ist gegen die Stromstärke $I$
    aufgetragen. Bei keinem angelegten Strom ist der magnetische Fluss
    gleich Null. Bei Erhöhung der Stromstärke entsteht die Neukurve, bis
    ein Sättigungswert $B_{S}$ (Maximum) erreicht wird. Beim Abschalten des Stroms
    bleibt eine Restmagnetisierung, die Remanenz $B_{r}$ (y-Achsenabschnitt),
    zurück. Ein Gegenfeld $H_{c}$, Koerzitivfeld, hebt diese Magnetisierung wieder
    auf (Schnittpunkt mit der x-Achse). Durch Erhöhung der Stromstärke entsteht
    eine Hysteresekurve.}
    \label{plotc}
\end{figure}

%Werte ablesen:
\noindent Aus der grafischen Darstellung lassen sich verschiedene Faktoren
wie die Sättigungsmagnetisierung $B_{S}$, die Remanenz $B_{r}$ und die
Koerzitivkraft $H_{c}$ ablesen.
Die Sättigungsmagnetisierung ist das Maximum der gemessenen Werte für
die magnetische Flussdichte: $B_{S} = \SI{}{\milli\tesla}$. %Wert
Die negative Sättigungsmagnetisierung ist das Minimum:
$-B_{S} = \SI{}{\milli\tesla}$. %Wert
Die Remanenz ist der obere Schnittpunkt mit der $y$-Achse.
Bestimmt man diesen aus der Grafik, ist die Remanenz also
$B_{r} = \SI{}{\milli\tesla}$. %Wert
Die Koerzitivktaft ist der linke, d.h. kleinere Schnittpunkt mit der Achse.
Hier ist die Koerzitivkraft $H_{c} = \SI{}{\milli\tesla}$. %Wert
