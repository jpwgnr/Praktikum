\section{Auswertung}
\label{sec:Auswertung}

% \begin{figure}
%  \centering
%  \includegraphics{<++>}
%  \caption{<++>}
%  \label{fig:<++>}
% \end{figure}
Für die Auswertung wird Python, im Speziellen Matplotlib \cite{matplotlib}, NumPy \cite{numpy},
SciPy \cite{scipy} und Uncertainties \cite{uncertainties} verwendet.

\subsection{Magnetische Flussdichte einer langen und einer kurzen Spule}
%Tabelle lange Spule

%Grafik lange Spule (xB-Diagramm)
\begin{figure}
    \centering
    \includegraphics{build/Hysterese.png} %richtigen Plot einfügen
    \caption{Plot}
    %\label{}
\end{figure}

%Vergleich der Werte mit Theorie


%Tabelle kurze Spule

%Grafik kurze Spule (xB-Diagramm)
\begin{figure}
    \centering
    \includegraphics{build/Hysterese.png} %richtigen Plot einfügen
    \caption{Plot}
    %\label{}
\end{figure}

%Vergleich der Werte mit Theorie

\subsection{Magnetische Flussdichte eines Spulenpaares}
%ABSTAND=RADIUS
%Tabelle mit I=4A

%Grafik (xB-Diagramm)
\begin{figure}
    \centering
    \includegraphics{build/Hysterese.png} %richtigen Plot einfügen
    \caption{Plot}
    %\label{}
\end{figure}

%Vergleich mit Theorie


%ABSTAND=DURCHMESSER
%Tabelle mit I=4A

%Grafik (xB-Diagramm)
\begin{figure}
    \centering
    \includegraphics{build/Hysterese.png} %richtigen Plot einfügen
    \caption{Plot}
    %\label{}
\end{figure}

%Vergleich mit Theorie

%Tabelle mit I=3A

%Grafik (xB-Diagramm)
\begin{figure}
    \centering
    \includegraphics{build/Hysterese.png} %richtigen Plot einfügen
    \caption{Plot}
    %\label{}
\end{figure}

%Vergleich mit Theorie

\subsection{Hysteresekurve einer Ringspule mit Luftspalt}
%Tabelle

%Grafik: Hysteresekurve
\begin{figure}
    \centering
    \includegraphics{build/Hysterese.png} %richtigen Plot einfügen
    \caption{Plot}
    %\label{}
\end{figure}

%Werte ablesen:
Aus der grafischen Darstellung lassen sich verschiedene Faktoren
wie die Sättigungsmagnetisierung $B_{S}$, die Remanenz $B_{r}$ und die
Koerzitivkraft $H_{c}$ ablesen.
Die Sättigungsmagnetisierung ist das Maximum der gemessenen Werte für
die magnetische Flussdichte: $B_{S} = \SI{}{\milli\tesla}$. %Wert
Die negative Sättigungsmagnetisierung ist das Minimum:
$-B_{S} = \SI{}{\milli\tesla}$. %Wert
Die Remanenz ist der obere Schnittpunkt mit der $y$-Achse.
Bestimmt man diesen aus der Grafik, ist die Remanenz also
$B_{r} = \SI{}{\milli\tesla}$. %Wert
Die Koerzitivktaft ist der linke, d.h. kleinere Schnittpunkt mit der Achse.
Hier ist die Koerzitivkraft $H_{c} = \SI{}{\milli\tesla}$. %Wert