\begin{table}\caption{Das Verhältnis des magnetischen Feldes durch die Wurzel der Beschleunigungsspannung aufgetragen gegen die Verschiebung $D$ durch die Summe des Wirkungsbereichs $L$ zum Quadrat und der Verschiebung $D$ zum Quadrat.}
\label{tab2}
\centering
\sisetup{round-mode = places, round-precision=2, round-integer-to-decimal=true}
\begin{tabular}{S[]S[]S[]} 
\toprule
{$\frac{B_1}{\sqrt{8 U_\text{B,1}}} / \si{\henry\volt\tothe{-\frac{1}{2}}}$} & {$\frac{B_2}{\sqrt{8 U_\text{B,2}}} / \si{\henry\volt\tothe{-\frac{1}{2}}}$} & {$\frac{D}{(L^2 + D^2)} / \si[per-mode=fraction]{\per\meter}$}\\
\midrule
0.0 & 0.0 & 0.0\\
3.5649278338607584e-07 & 3.862005153349155e-07 & 0.29289724188430566\\
8.912319584651897e-07 & 8.912319584651897e-07 & 0.5827222842713544\\
1.4259711335443034e-06 & 1.396263401595464e-06 & 0.8665094112549946\\
1.9250610302848096e-06 & 1.8418793808280586e-06 & 1.1414982164090373\\
2.3885016486867084e-06 & 2.3172030920094934e-06 & 1.4052180429996723\\
2.923240823765822e-06 & 2.822234535139767e-06 & 1.6555530006898145\\
3.4223307205063282e-06 & 3.3272659782700412e-06 & 1.8907846756403912\\
\bottomrule
\end{tabular}\end{table}