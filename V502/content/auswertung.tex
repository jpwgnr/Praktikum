\section{Auswertung}
\label{sec:Auswertung}

Für die Auswertung wird Python, im Speziellen matplotlib \cite{matplotlib}, 
SciPy \cite{scipy}, Uncertainties \cite{uncertainties}
und NumPy \cite{numpy} benutzt.

\subsection{Elektisches Feld}
\subsubsection{Proportionalität zwischen Leuchtfleckverschiebung und Ablenkspannung}
\label{sec:prop}
%Tabellen
Die gemessenen Werte für die Ablenkspannung $U_\text{d}$
auf jeder Linie des Koordinatennetzes sind für die fünf
unterschiedlichen Beschleunigungsspannungen $U_\text{B}$
in den folgenden Tabellen \ref{taba1} bis \ref{taba5} aufgelistet.
\begin{table}\caption{Der magnetische Fluss $B$ an verschiedenen Stellen $x$ vor der langen Spule.}
\label{taba1}
\centering
\sisetup{round-mode = places, round-precision=3, round-integer-to-decimal=true}
\begin{tabular}{S[]S[]} 
\toprule
{$B$/ \si{\milli\tesla}} & {$x$/ \si{\centi\meter}}\\
\midrule
0.382 & 0.0\\
0.505 & 0.5\\
0.744 & 1.0\\
0.959 & 1.5\\
1.35 & 2.0\\
1.544 & 2.5\\
1.7619999999999998 & 3.0\\
1.928 & 3.5000000000000004\\
2.024 & 4.0\\
2.093 & 4.5\\
2.1380000000000003 & 5.0\\
\bottomrule
\end{tabular}\end{table}
\begin{table}\caption{Der magnetische Fluss $B$ an verschiedenen Stellen $x$ in der kurzen Spule.}
\label{taba2}
\centering
\sisetup{round-mode = places, round-precision=3, round-integer-to-decimal=true}
\begin{tabular}{S[]S[]} 
\toprule
{$B$/ \si{\milli\tesla}} & {$x$/ \si{\centi\meter}}\\
\midrule
0.231 & 0.0\\
0.325 & 0.5\\
0.475 & 1.0\\
0.67 & 1.5\\
0.94 & 2.0\\
1.2 & 2.5\\
1.4549999999999998 & 3.0\\
1.677 & 3.5000000000000004\\
1.821 & 4.0\\
1.9009999999999998 & 4.5\\
1.928 & 5.0\\
1.9020000000000001 & 5.5\\
1.8439999999999999 & 6.0\\
1.7249999999999999 & 6.5\\
1.5299999999999998 & 7.000000000000001\\
1.181 & 7.5\\
0.8630000000000001 & 8.0\\
0.486 & 8.5\\
0.35 & 9.0\\
0.255 & 9.5\\
0.19799999999999998 & 10.0\\
\bottomrule
\end{tabular}\end{table}
\begin{table}\caption{Die Index Werte entsprechen der Höhe, die bei der jeweiligen Ablenkspannung $U_\text{d}$ und der Beschleunigungsspannung $U_\text{B} = \SI{280}{\volt}$ gemessen wurden. Der Indexwert $1$ entspricht einer Höhe von $\SI{0.6}{\centi\meter}$.}
\label{taba3}
\centering
\sisetup{round-mode = places, round-precision=1, round-integer-to-decimal=true}
\begin{tabular}{S[]S[]} 
\toprule
{Index} & {$U_\text{d} / \si{\volt}$}\\
\midrule
1.0 & -30.7\\
2.0 & -26.0\\
3.0 & -20.8\\
4.0 & -15.9\\
5.0 & -10.9\\
6.0 & -5.8\\
7.0 & -0.9\\
8.0 & 4.6\\
9.0 & 10.0\\
\bottomrule
\end{tabular}\end{table}
\begin{table}\caption{Die Index Werte entsprechen der Höhe, die bei der jeweiligen Ablenkspannung $U_\text{d}$ und der Beschleunigungsspannung $U_\text{B} = \SI{330}{\volt}$ gemessen wurden. Der Indexwert $1$ entspricht einer Höhe von $\SI{0.6}{\centi\meter}$.}
\label{taba4}
\centering
\sisetup{round-mode = places, round-precision=1, round-integer-to-decimal=true}
\begin{tabular}{S[]S[]} 
\toprule
{Index} & {$U_\text{d} / \si{\volt}$}\\
\midrule
1.0 & -36.1\\
2.0 & -30.1\\
3.0 & -24.4\\
4.0 & -18.3\\
5.0 & -12.6\\
6.0 & -6.8\\
7.0 & -0.6\\
8.0 & 5.7\\
9.0 & 12.1\\
\bottomrule
\end{tabular}\end{table}
\begin{table}\caption{Die Index Werte entsprechen der Höhe die bei der jeweiligen Ablenkspannung $U_d$ und der Beschleunigungsspannung $U_\text{B} = \SI{380}{\volt}$.}
\label{taba5}
\centering
\sisetup{round-mode = places, round-precision=1, round-integer-to-decimal=true}
\begin{tabular}{S[]S[]} 
\toprule
{Index} & {$U_\text{d} / \si{\volt}$}\\
\midrule
1.0 & nan\\
2.0 & -34.6\\
3.0 & -27.7\\
4.0 & -21.0\\
5.0 & -14.3\\
6.0 & -7.5\\
7.0 & -0.4\\
8.0 & 6.8\\
9.0 & 13.6\\
\bottomrule
\end{tabular}\end{table}

%D gegen U_d (plot1) -> D/U_d für verschiedene U_B
\noindent Die Leuchtfleckverschiebung $D$ ist in Abb. \ref{fig:plot1}
für die fünf Beschleunigungsspannungen gegen die 
Ablenkspannung $U_\text{d}$ aufgetragen.
\begin{figure}
    \centering
    \includegraphics[width=12cm, height=7cm]{build/plot1.pdf}
    \caption{Die Leuchtfleckverschiebung $D$ ist gegen die
    Ablenkspannung $U_\text{d}$ aufgetragen. Es sind die Daten
    für die fünf verschiedenen Beschleunigungsspannungen und
    jeweils ein Fit eingezeichnet.}
    \label{fig:plot1}
\end{figure}

\noindent Aus der Abb. \ref{fig:plot1} lassen sich die Werte 
$\frac{D}{U_\text{d}}$ bestimmen:
\begin{align*} 
    U_\text{B} = \SI{180}{\volt} &: \frac{D}{U_\text{d}} &= \SI{1}{\centi\meter\per\volt} \\
    U_\text{B} = \SI{230}{\volt} &: \frac{D}{U_\text{d}} &= \SI{2}{\centi\meter\per\volt} \\
    U_\text{B} = \SI{280}{\volt} &: \frac{D}{U_\text{d}} &= \SI{3}{\centi\meter\per\volt} \\
    U_\text{B} = \SI{330}{\volt} &: \frac{D}{U_\text{d}} &= \SI{4}{\centi\meter\per\volt} \\
    U_\text{B} = \SI{380}{\volt} &: \frac{D}{U_\text{d}} &= \SI{5}{\centi\meter\per\volt}.
\end{align*} %das sieht hässlich aus, weiß nicht wie man das machen könnte
Der mit Gleichung %Gleichung Mittelwert
errechnete Mittelwert für die Steigung ist
\begin{equation*}
    \frac{D}{U_\text{D}} = \SI{6}{\centi\meter\per\volt}.
\end{equation*}

%D/U_d gegen 1/U_B -> Steigung a
%noch ein Plot?

%Werte p, d, L angeben
\noindent Die angegebenen Werte für die Plattenlänge $p$,
den Plattenabstand $d$ und den Strahlweg $L$ sind
\begin{align*}
    p &= \SI{1.03}{\centi\meter} \\
    d &= \SI{0.38}{\centi\meter} \\
    L &= \SI{14.3}{\centi\meter}.
\end{align*}

%pL/2d mit a vergleichen
Der aus diesen Werten theoretisch berechnete Wert für die
Steigung ist
\begin{equation*}
    \frac{p \, L}{2 \, d} = \SI{7}{\centi\meter}.
\end{equation*}

\subsubsection{Bestimmung der Frequenz der Sinusspannung}
Die gemessenen Synchronisationsfrequenzen sind in Tabelle
\ref{tabb} zu finden. 
\begin{table}\caption{Die Frequenzen der Sägezahnspannung.}
\label{tabb}
\centering
\sisetup{round-mode = places, round-precision=2, round-integer-to-decimal=true}
\begin{tabular}{S[]S[]} 
\toprule
{Index} & {$\nu_\text{Sä} / \si{\hertz}$}\\
\midrule
1.0 & 25.02\\
2.0 & 49.95\\
3.0 & 99.99\\
4.0 & 149.97\\
\bottomrule
\end{tabular}\end{table}

%v_Sinus
\noindent Durch das Synchronisationsverhältnis
%Gleichung Verhältnis
kann jeweils die Frequenz der Sinusspannung bestimmt werden:
\begin{align*}
    \nu_1 &= \SI{1}{\hertz} \\
    \nu_2 &= \SI{2}{\hertz} \\
    \nu_3 &= \SI{3}{\hertz} \\
    \nu_4 &= \SI{4}{\hertz}.
\end{align*}
Der daraus mit Gleichung
%Gleichung Mittelwert
berechnete Mittelwert ergibt sich zu
\begin{equation*}
    \nu_\text{mittel} = \SI{5}{\hertz}.
\end{equation*}


\subsection{Magnetfeld}
\subsubsection{Bestimmung der spezifischen Elektronenladung}
Die Strahlverschiebung $D$ in Abhängigkeit von der Flussdichte $B$
für eine Beschleunigungsspannung von $U_\text{B} = \SI{250}{\volt}$
ist in Tabelle \ref{tabc1} aufgeführt. Die Strahlverschiebung in
Abhängigkeit von der Flussdichte für eine Beschleunigungsspannung
von $U_\text{B} = \SI{360}{\volt}$ befindet sich in Tabelle \ref{tabc2}.
\begin{table}\caption{Die Indexwerte entsprechen der Höhe bei dem jeweiligen Strom und der Beschleunigungsspannung $U_\text{B} = \SI{250}{\volt}$.}
\label{tabc1}
\centering
\sisetup{round-mode = places, round-precision=3, round-integer-to-decimal=true}
\begin{tabular}{S[]S[]} 
\toprule
{Index} & {$I / \si{\ampere}$}\\
\midrule
1.0 & 0.0\\
2.0 & 0.25\\
3.0 & 0.625\\
4.0 & 1.0\\
5.0 & 1.35\\
6.0 & 1.675\\
7.0 & 2.05\\
8.0 & 2.4\\
9.0 & 2.75\\
\bottomrule
\end{tabular}\end{table}
\begin{table}\caption{Die Indexwerte entsprechen der Höhe bei dem jeweiligen Strom und der Beschleunigungsspannung $U_\text{B} = \SI{360}{\volt}$.}
\label{tabc2}
\centering
\sisetup{round-mode = places, round-precision=3, round-integer-to-decimal=true}
\begin{tabular}{S[]S[]} 
\toprule
{Index} & {$I / \si{\ampere}$}\\
\midrule
1.0 & 0.0\\
2.0 & 0.325\\
3.0 & 0.75\\
4.0 & 1.175\\
5.0 & 1.55\\
6.0 & 1.95\\
7.0 & 2.375\\
8.0 & 2.8\\
9.0 & 3.225\\
\bottomrule
\end{tabular}\end{table}

%D/(L^2+D^2) gegen B (plot 2) für beide U_B -> Faktor a -> e_0/m_0
%Weg L angeben
\noindent Der Term $\frac{D}{L^2 + D^2}$ ist für beide Beschleunigungsspannungen
in Abb. \ref{fig:plot2} aufgetragen. Dabei ist $L$ der Strahlweg und
hat einen Wert von
\begin{equation*}
    L = \SI{14.3}{\centi\meter},
\end{equation*}
wie bereits in Abschnitt \ref{sec:prop} erwähnt.

\begin{figure}
    \centering
    \includegraphics[width=12cm, height=7cm]{build/plot2.pdf}
    \caption{$\frac{D}{L^2 + D^2}$ gegen $B$ aufgetragen.
    Es sind die Daten und die Fits für beide Beschleunigungsspannungen
    eingetragen.}
    \label{fig:plot2}
\end{figure}

\noindent Durch die Steigung
\begin{equation*}
    a = \SI{1000000000000000}{\per\meter\per\tesla} %den Wert auch angeben?
\end{equation*}
lässt sich die spezifische Elektronenladung bestimmen.
Diese ergibt sich zu
\begin{equation*}
    \frac{e_0}{m_0} = \SI{1}{\coulomb\per\kilogram}.
\end{equation*}

\subsubsection{Bestimmung der Intensität des lokal Erdmagnetfelds}
%Werte I_hor und phi angeben -> B_total
Der Spulenstrom, der nötig ist, um das Erdmagnetfeld zu
kompensieren, beträgt
\begin{equation*}
    I_\text{hor} = \SI{1}{\ampere}.
\end{equation*}
Der Winkel zwischen Horizontalebene und Richtung des Erdmagnetfelds
beträgt
\begin{equation*}
    \varphi = \SI{1}{\degree}.
\end{equation*}
Aus diesen beiden Werten lässt sich mittels Gleichung %Gleichung B_total
die Totalintensität des lokalen Erdmagnetfelds bestimmen.
Diese ergibt sich zu
\begin{equation*}
    B_\text{total} = \SI{1.2e-5}{\tesla}.
\end{equation*}