\section{Durchführung}
\label{sec:Durchführung}

\subsection{Kathodenstrahlröhre}
Im ersten Teil wird die Leuchtkraftverschiebung und die Ablenkspannung $U_\text{d}$ für fünf verschiedene Beschleunigungsspannungen $U_\text{B}$ zwischen \num{180} und \SI{500}{\volt} gemessen. 

Es wird ein einfacher Kathodenstrahl-Oszillograph aufgebaut und es wird versucht durch Variation der Sägezahnfrequenz stehende Bilder der Sinusspannung auf dem Leuchtschirm zu erhalten. Das ist immer dann der Fall, wenn Sägezahn- und Sinusfrequenz ein rationales Verhältnis zu bilden. Es sollen einige Fälle realisiert werden und dabei die Frequenzen abgelesen werden. Außerdem soll die durch die Sinusspannung erzeugte, maximale Strahlauslenkung in $y$-Richtung bei $U_\text{B} = const$ gemessen werden. 

\subsection{Elektronenstrahl im Magnetfeld}
Die spezifische Ladung des Elektrons wird bestimmt und die Intensität des lokalen Erdmagnetfeldes. 

Dafür erzeugt man mittels eines großen Helmholtz-Spulenpaars ein nahezu homogenes Magnetfeld, das senkrecht zum Elektronenstrahl einer Kathodenstrahlröhre ausgerichtet ist. Nach der korrekten Ausrichtung misst man bei konstanter Beschleungigungsspannung $U_B = \SI{250}{\volt}$ und $\SI{500}{\volt}$ die Strahlverschiebung $D$ in Abhängigkeit von den beiden Magnetfeldstärken. 

Die Veränderung des angezeigte Leuchtfleck im XY-Koordinatensystem wird beobachtet während die Ausrichtung von der Nord-Süd-Richtung zur Ost-West-Richtung geändert wird. Die Elektronen werden nun in Y-Richtung abgelenkt. Das Helmholtz-Spulenpaar wird eingeschaltet und die Auslenkung wird kompentisiert. Somit ergibt sich der Wert des Erdmagnetfelds. 
