\section{Durchführung}
\label{sec:Durchführung}

\subsection{Messung im elektrischen Feld}
\subsubsection{Proportionalität zwischen Leuchtfleckverschiebung und Ablenkspannung}
Im ersten Teil werden die Leuchtfleckverschiebung und die 
Ablenkspannung $U_\text{d}$ für fünf verschiedene 
Beschleunigungsspannungen $U_\text{B}$ zwischen \num{180} und 
\SI{380}{\volt} in $\SI{50}{\volt}$ Schritten gemessen,
um die Proportionalität zwischen diesen beiden Größen zu
bestimmen. Dazu wird für jede Beschleungigungsspannung der
Leuchtfleck nacheinander auf die $\num{9}$ Linien des
Koordinatennetzes geregelt. Es wird jeweils die Ablenkspannung
abgelesen.

\subsubsection{Kathodenstrahl-Oszillograph}
Es wird ein einfacher Kathodenstrahl-Oszillograph (Abb. \ref{fig:oszillograph})
aufgebaut und versucht durch Variation der Sägezahnfrequenz 
stehende Bilder der Sinusspannung auf dem Leuchtschirm zu 
erhalten. Das ist immer dann der Fall, wenn Sägezahn- und 
Sinusfrequenz ein rationales Verhältnis bilden. Es sollen 
die Fälle $n = \num{\frac{1}{2}, 1, 2, 3}$ realisiert 
und dabei die Sägezahnfrequenzen abgelesen werden. 
\begin{figure}
    \centering
    \includegraphics[width=10cm height=10cm]{build/V501_b.png}
    \caption{}
    \label{fig:oszillograph}
\end{figure}

\subsection{Messung im Magnetfeld}
In diesem Versuchsteil sollen die spezifische Elektronenladung
und die Intensität des lokalen Erdmagnetfeldes bestimmt werden.

\subsubsection{Messung der spezifischen Elektronenladung}
Um die spezifische Elektronenladung zu bestimmen, wird mittels 
eines großen Helmholtz-Spulenpaars ein nahezu homogenes 
Magnetfeld, das senkrecht zum Elektronenstrahl einer 
Kathodenstrahlröhre ausgerichtet ist, erzeugt.
Nach der korrekten Ausrichtung mithilfe eines speziellen
Kompasses wird bei konstanten
Beschleungigungsspannungen $U_B = \SI{250}{\volt}$ und 
$\SI{360}{\volt}$ die Strahlverschiebung $D$ in Abhängigkeit 
von den beiden Magnetfeldstärken gemessen. Dazu wird jeweils die
Stromstärke, die eingestellt werden muss, um den Leuchtfleck
auf die $\num{9}$ Linien des Koordinatennetzes zu bewegen,
abgelesen.

\subsubsectin{Messung der Intensität des lokalen Erdmagnetfeldes}
Die Achse der Kathodenstrahlröhre wird in Nord-Süd-Richtung



Die Veränderung des angezeigte Leuchtfleck im 
XY-Koordinatensystem wird beobachtet während die Ausrichtung 
von der Nord-Süd-Richtung zur Ost-West-Richtung geändert wird. 
Die Elektronen werden nun in Y-Richtung abgelenkt. Das 
Helmholtz-Spulenpaar wird eingeschaltet und die Auslenkung 
wird kompentisiert. Somit ergibt sich der Wert des 
Erdmagnetfelds. 
