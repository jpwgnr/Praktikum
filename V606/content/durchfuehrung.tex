\section{Durchführung}
\label{sec:Durchführung}

\subsection{Unterdrückung von Störspannung mittels Selektivverstärker}
Um die Brückenspannung zu messen muss die Störspannung herausgefiltert werden. 
Dazu wird ein Selektivverstärker verwendet, weil
die Brückenspannung eine monofrequente Spannung ist. Die Druchlassfrequenz $\nu_0$ des 
Selektivverstärkers wird auf die Signalfrequenz gestellt. 
Frequenzen, die nah an der Frequenz $\nu_0$ liegen, werden nicht herausgefiltert.
Eine Filterkurve eines Selektivverstärkers ist in Abb. \ref{abb:filterkurve} zu sehen.


\subsection{Untersuchung der Filterkurve des Selektivverstärkers}
Bei einer Güte von $G = \num{100}$ wird die Filterkurve des Selektivverstärkers
untersucht. Bei einer konstanten Eingangsspannung $U_\text{E} = \SI{100}{\milli\volt}$
wird die Ausgangsspannung $U_\text{A}$ bei variierender Frequenz gemessen.
Die Frequenz wird auf $\nu = \SI{20}{\kilo\hertz}$ gestellt und in $\SI{1}{\kilo\hertz}$
Schritten auf $\SI{40}{\kilo\hertz}$ erhöht. In einem Bereich von plus und minus 
$\SI{1}{\kilo\hertz}$ vom Spannungsmaximum wird noch einmal genauer in $\SI{0.1}{\kilo\hertz}$
Schritten gemessen.

\noindent Eine theoretische Filterkurve eines Selektivverstärkers ist in Abb. 
\ref{abb:filterkurve} dargestellt.

\begin{figure}
    \centering
    \includegraphics[width=10cm, height=8cm]{build/filterkurve.png}
    \caption{Filterkurve eines Selektivverstärkers. Die Frequenz ist gegen das
    Verhältnis von Ausggangs- und Eingangsspannung aufgetragen. Es sind die
    Druchlassfrequenz $\nu_0$ sowie die Frequenzen, bei denen das Verhältnis der
    Spannungen $\frac{1}{\sqrt{2}}$ beträgt, eingetragen. \cite{V606}}
    \label{abb:filterkurve}
\end{figure}


\subsection{Suszeptibilität mittels Spannungsverhältnis bzw. Widerstandsverhältnis}
Es wird die Schaltung aus Abb. \ref{abb:schaltbild} nachgebaut.

\begin{figure}
    \centering
    \includegraphics[width=12cm, height=6cm]{build/schaltbild.png}
    \caption{Schaltung zur Bestimmung der Suszeptibilität. \cite{V606}}
    \label{abb:schaltbild}
\end{figure}

\noindent Die Brückenspannung wird mittels der Widerstände auf ihr Minimum abgeglichen.
Die Probe wird in die Brückenschaltung geschoben. Die jetzt angezeigte Brückenspannung
wird aufgenommen. Anschließend wird die Brückenspannung mit dem Widerstand wieder auf
ihr Minimum abgeglichen. Der Wert des Widerstandes wird ebenfalls aufgenommen.
Das Ganze wird für alle vier Proben drei mal durchgeführt.

\noindent Aus der Änderung der Brückenspannung $\Delta U_\text{Br}$ bzw. aus der 
Änderung des Widerstandes $\Delta R$ kann die Suszeptibilität $\chi$ von Oxiden 
Seltener-Erd-Verbindungen bestimmt werden.