\section{Fehlerrechnung}

Der Mittelwert einer Stichprobe von $N$ Werten wird durch
\begin{equation}
    \overline{x} = \frac{1}{N} \sum_{i=1}^N x_i
    \label{eqn:mittelwert}
\end{equation}
bestimmt.
\newline
Die Standardabweichung der Stichprobe wird berechnet mit
\begin{equation*}
    \sigma_x = \sqrt{\frac{1}{N-1} \sum_{i=1}^N (x_i - \overline{x})^2}.
    \label{eqn:standard}
\end{equation*}
\newline
Die realtive Abweichung zwischen zwei Werten kann durch
\begin{equation*}
    f = \frac{x_\text{a} - x_\text{r}}{x_\text{r}}
\end{equation*}
bestimmt werden.