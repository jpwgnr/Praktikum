\section{Auswertung}
\label{sec:Auswertung}

\subsection{Einstellung des Selektivverstärkers}

In Tabelle Tab. \ref{taba} sind die Frequenzen $f$ gegen die gemessene Spannung $U_E$ aufgetragen. Der Plot befindet sich in Abb. \ref{plota}. 

\begin{table}\caption{Die Länge der Zylinder und die Spannung mit den jeweiligen Zeitenpunkten der Ausschläge.}
\label{taba}
\centering
\sisetup{round-mode = places, round-precision=2, round-integer-to-decimal=true}
\begin{tabular}{S[]S[]S[]S[]S[]} 
\toprule
{$l/ \si{\milli\meter}$} & {$U_1/ \si{\volt}$} & {$t_1/ \si{\micro\second}$} & {$U_2/ \si{\volt}$} & {$t_2/ \si{\micro\second}$}\\
\midrule
120.8 & 1.29 & 0.6 & 0.17 & 88.7\\
102.3 & 1.27 & 0.5 & 0.2 & 76.5\\
80.5 & 1.33 & 0.6 & 0.76 & 59.8\\
40.4 & 1.33 & 0.5 & 1.34 & 30.2\\
31.1 & 1.29 & 0.5 & 1.37 & 23.8\\
\bottomrule
\end{tabular}\end{table}

% Plot einfügen 

Aus der Tabelle lassen sich der Wert $\nu_0$ und die beiden Werte $\nu_{-}$ und $\nu_{+}$ ablesen. 

Diese ergeben sich zu 

\begin{align*} 
 \nu_0 &= \SI{<++>}{\<++>} \\
 \nu_{-} &= \SI{<++>}{\<++>} \\
 \nu_{+} &= \SI{<++>}{\<++>}. 
\end{align*}

Daraus lässt sich mit Gl. \eqref{güte} erkennen, dass die Güte $Q$ den Wert 

\begin{align*} 
    Q = \num{<++>}
\end{align*}
beträgt. 

Der gegebene Wert für Q liegt bei 
\begin{align*} 
    Q = \num{100}.
\end{align*}


\subsection{Theoriewerte für Suszeptibilität} 

Für die verschiedenen Stoffe haben sich aufgrund der verschiedenen Elemente und Zusammensetzungen auch verschiedene Werte für die Suszeptibilität ergeben. 
Die Werte die zur Berechnung nötig waren sind ...
Für den Stoff $<++>$ ergibt sich die Rechnung folgendermaßen: 

\begin{equation*}
    <++>.
\end{equation*} 

Die anderen drei Werte lassen sich analog ermitteln. 
In Tab. \ref{tab1} sind die Werte für L, S und J gegeneinander aufgetragen. Daneben stehen dann jeweils die Werte die sich für Suszeptibilität $\chi$ ergeben haben. 

\begin{table}\caption{Der maximale Drehimpuls $L$, der Gesamtspin $S$ und der Gesamtdrehimpuls $J$ ergeben sich zum Landé-Faktor $g_\text{J}$ für die vier verschiedenen Elemente.}
\label{tab1}
\centering
\sisetup{round-mode = places, round-precision=2, round-integer-to-decimal=true}
\begin{tabular}{S[]S[]S[]S[]} 
\toprule
{$L$} & {$S$} & {$J$} & {$g_\text{J}$}\\
\midrule
5.0 & 1.0 & 4.0 & 0.8\\
0.0 & 3.5 & 3.5 & 2.0\\
6.0 & 1.5 & 4.5 & 0.7272727272727273\\
5.0 & 2.5 & 7.5 & 1.3333333333333333\\
\bottomrule
\end{tabular}\end{table}

\subsection{Suszeptibilität mittels Spannungsverhältnis}

Die Werte für die Suszeptibilität ergeben sich für die jeweiligen Stoffe zu folgenden Werten 

\begin{align*} 
   \chi_\text{<++>} &= \SI{<++>}{\<++>}\\
   \chi_\text{<++>} &= \SI{<++>}{\<++>}\\
   \chi_\text{<++>} &= \SI{<++>}{\<++>}\\
   \chi_\text{<++>} &= \SI{<++>}{\<++>}.
\end{align*}

\subsection{Suszeptibilität mittels Widerstandsverhältnis}

Die Werte für die Suszeptibilität ergeben sich für die jeweiligen Stoffe zu folgenden Werten 

\begin{align*} 
   \chi_\text{<++>} &= \SI{<++>}{\<++>}\\
   \chi_\text{<++>} &= \SI{<++>}{\<++>}\\
   \chi_\text{<++>} &= \SI{<++>}{\<++>}\\
   \chi_\text{<++>} &= \SI{<++>}{\<++>}.
\end{align*}

