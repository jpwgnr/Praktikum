\section{Diskussion}
\label{sec:Diskussion}

\subsection{Güte Berechnung}

Die relative Abweichung zur erwarteten Gütewert beträgt \SI{<++>}{\percent}. 

\subsection{Suszeptibilität}

Der relative Fehler beträgt für $<++>$ bei dem Spannungsverhältnis \SI{<++>}{\percent} und bei dem Widerstandsverhältnis \SI{<++>}{\percent}. Für den Stoff $<++>$ hat sein ein relativer Fehler von \SI{<++>}{\percent} bei dem Spannungsverältnis und ein relativer Fehler von \SI{<++>}{\percent} bei dem Widerstandsverhältnis ergeben.  
Der relative Fehler beträgt für $<++>$ bei dem Spannungsverhältnis \SI{<++>}{\percent} und bei dem Widerstandsverhältnis \SI{<++>}{\percent}. Für den Stoff $<++>$ hat sein ein relativer Fehler von \SI{<++>}{\percent} bei dem Spannungsverältnis und ein relativer Fehler von \SI{<++>}{\percent} bei dem Widerstandsverhältnis ergeben.  

Die Abweichung beim $<++>$ vom gemessenen Wert beim Spannungsverhältnis zum Theoriewert beträgt \SI{<++>}{\percent}. Beim Widerstandsverhältnis beträgt sie \SI{<++>}{\percent}. 
Die Abweichung beim $<++>$ vom gemessenen Wert beim Spannungsverhältnis zum Theoriewert beträgt \SI{<++>}{\percent}. Beim Widerstandsverhältnis beträgt sie \SI{<++>}{\percent}. 
Die Abweichung beim $<++>$ vom gemessenen Wert beim Spannungsverhältnis zum Theoriewert beträgt \SI{<++>}{\percent}. Beim Widerstandsverhältnis beträgt sie \SI{<++>}{\percent}. 
Die Abweichung beim $<++>$ vom gemessenen Wert beim Spannungsverhältnis zum Theoriewert beträgt \SI{<++>}{\percent}. Beim Widerstandsverhältnis beträgt sie \SI{<++>}{\percent}. 

\subsection{Fehlerquellen}

Grund für die großen Fehler könnte das große Rauschen sein, das durch den Selektivverstärker versucht wurde zu beheben, aber tropsdem könnte das Rauschen nach wie vor eine Fehlerquelle sein. 

\subsection{Fazit}
Die Messung war ausgezeichnet super.
