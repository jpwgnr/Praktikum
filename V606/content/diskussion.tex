\section{Diskussion}
\label{sec:Diskussion}

\subsection{Berechnung der Güte}
Die relative Abweichung des berechneten Wertes für die Güte zum erwarteten Gütewert beträgt \SI{<++>}{\percent}. 

\subsection{Suszeptibilität}
\subsubsection{$C_6 O_{12} Pr_2$}
Bei $C_6 O_{12} Pr_2$ handelt es sich um eine Verbindung von Kohlenstoff, Sauerstoff und Praseodym.
\newline
Der relative Fehler der Suszeptibilität für $C_6 O_{12} Pr_2$ bei der Berechnung mit dem Spannungsverhältnis beträgt \SI{<++>}{\percent}. 
Die Abweichung des gemessenen Wertes zum Theoriewert mit dem Spannungsverhältnis beträgt \SI{<++>}{\percent}.
\newline
Bei der Berechnung mit dem Widerstandsverhältnis beträgt der relative Fehler \SI{<++>}{\percent}.
Die Abweichung des Wertes zum theoretischen Wert beträgt \SI{<++>}{\percent}. 

\subsubsection{Neodym(III)-oxid $Nd_2 O_3$}
Der relative Fehler der Suszeptibilität für $Nd_2 O_3$ bei der Berechnung mit dem Spannungsverhältnis beträgt \SI{<++>}{\percent}. 
Die Abweichung des gemessenen Wertes zum Theoriewert mit dem Spannungsverhältnis beträgt \SI{<++>}{\percent}.
\newline
Bei der Berechnung mit dem Widerstandsverhältnis beträgt der relative Fehler \SI{<++>}{\percent}.
Die Abweichung des Wertes zum theoretischen Wert beträgt \SI{<++>}{\percent}.

\subsubsection{Gadolinium(III)-oxid $Gd_2 O_3$}
Der relative Fehler der Suszeptibilität für $Gd_2 O_3$ bei der Berechnung mit dem Spannungsverhältnis beträgt \SI{<++>}{\percent}. 
Die Abweichung des gemessenen Wertes zum Theoriewert mit dem Spannungsverhältnis beträgt \SI{<++>}{\percent}.
\newline
Bei der Berechnung mit dem Widerstandsverhältnis beträgt der relative Fehler \SI{<++>}{\percent}.
Die Abweichung des Wertes zum theoretischen Wert beträgt \SI{<++>}{\percent}.

\subsubsection{Dysprosium(III)-oxid $Dy_2 O_3$}
Der relative Fehler der Suszeptibilität für $Dy_2 O_3$ bei der Berechnung mit dem Spannungsverhältnis beträgt \SI{<++>}{\percent}. 
Die Abweichung des gemessenen Wertes zum Theoriewert mit dem Spannungsverhältnis beträgt \SI{<++>}{\percent}.
\newline
Bei der Berechnung mit dem Widerstandsverhältnis beträgt der relative Fehler \SI{<++>}{\percent}.
Die Abweichung des Wertes zum theoretischen Wert beträgt \SI{<++>}{\percent}.


\subsection{Fehlerquellen}
Grund für die Fehler könnte das große Rauschen sein, das durch den Selektivverstärker versucht wurde zu beheben, aber tropsdem könnte das Rauschen nach wie vor eine Fehlerquelle sein. 

\subsection{Fazit}
Die Messung war ausgezeichnet super.