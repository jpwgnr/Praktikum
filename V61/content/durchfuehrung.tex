\section{Durchführung}
\label{sec:Durchführung}

Das Ziel des Versuchs ist es die Eigenschaften eines Helium-Neon-Lasers (He-Ne-Lasers) zu vermessen. 

Anfangs werden die Blenden aufgestellt und mithilfe eines Justierlasers entsprechend ausgerichtet. Der Justierlaser kann anschließend wieder ausgeschaltet werden. 
Anschließend werden die einzelnen Komponenten des He-Ne-Lasers, das Plasmarohr und die Resonatorspiegel, auf der optischen Schiene drapiert. 
Der Strom der Hochspannung wird auf $ I = \SI{6.5}{\milli\ampere}$ hochgedreht, wodurch das Laserrohr bereits anfängt rot zu leuchten.
Um mit der Laser Tätigkeit zu beginnen müssen erst die Justierschrauben der Resonatorspiegel nachjustiert werden. 
Sobald der Laser läuft, muss er mithilfe einer Photodiode auf maximale Leistung eingestellt werden.
Der maximale Abstand der Resonatoren soll eingestellt werden. Dafür werden bei laufendem Laser die Abstände der Spiegel vergrößert und die Leistung des Lasers nachjustiert. 
Diese Messung wird für einen weiteren Resonator erneut durchgeführt. 
Mit einem Draht aus Wolfram, der zwischen Resonatorspiegel und Laserrohr gebracht wird, werden verschiedene Moden auf einem optischen Schirm sichtbar gemacht. Mit einer Streulinse wird der Strahldurchmesser dafür vergrößert. 
Dasselbe wird erneut durchgeführt, nur der Schrim wird durch ein Messgerät ersetzt, eine Photodiode. 
Die Polarisation des Lasers wird mit einem Polarisator und einer Photodiode, die die Intensität für verschiedene Polarisationen bestimmt, gemessen. 
Mit einer schnellen Photodiode (Spektrumanalysator), die eine Bandbreite von $x \si{\giga\hertz}$ hat, werden die Fourierspektren für verschiedene Resonatorlängen gemessen. 
Als letzten Schritt wird aus den Beugungsmaxima und -minima eines Spaltes und eines Gitters die Wellenlänge des He-Ne-Lasers berechnet.


