\section{Diskussion}
\label{sec:Diskussion}

Im Fall der Stabilitätsbedingung ergab sich bei beiden Messungen wie erwartet ein Tiefpunkt der Leistung bei den zuvor bestimmten Stabilitätsbedingungen. 
Bei den konkaven Spiegeln fällt die Leistung in der Nähe des theoretisch bestimmten Wertes rapide ab um fast \SI{60}{\percent}.
Bei der Kombination des konkaven mit dem flachen Spiegel ließ sich ab der Stabilitätsbedingung gar keine Leistung mehr feststellen. 

Die Messung der TEM-Moden funktionierte dank der guten Justierung gut und sogar eine TEM$_{20}$-Mode ließ sich bestimmen. 
Je höher die Moden werden, desto weniger gut wird der Fit an die aus der Theorie kommenden Modelle.

Wie erwartet wird eine $2\pi$-periodische Polarisation festgestellt. Der Modell passt hier sehr gut zu der gegebenen Funktion. 
Es ist lediglich eine Verschiebung \SI{63.36}{\degree} mit einer relativen Abweichung von \SI{0.46}{\percent} festgestellt worden. 

Die theoretische Wellenlänge des He-Ne-Lasers liegt bei \SI{633}{\nano\meter}. Die gemessenen Werte weichen je Gitter zwischen \SI{1.87}{\percent} und \SI{5.17}{\percent} von dem erwarteten Wert ab. Dies ist trotz der recht groben Messung mithilfe eines Maßbandes ein genaues Ergebnis. 