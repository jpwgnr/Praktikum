\section{Diskussion}
\label{sec:Diskussion}

Im Fall der Stabilitätsbedingung ergab sich bei beiden Messungen wie erwartet ein Tiefpunkt der Leistung bei den zuvor bestimmten Stabilitätsbedingungen. 
Bei dem konkaven Spiegel fiel in der Nähe des theoretisch bestimmten Wertes die Leistung rapide ab um fast \SI{60}{\percent}.
Bei der Kombination des konkaven mit dem flachen Spiegel ließ sich ab der Stabilitätsbedingung gar keine Leistung mehr feststellen. 

Die Messung der TEM-Moden funktionierte dank der guten Justierung gut und sogar eine TEM$_{20}$-Mode ließ sich bestimmen. 
Je höher die Moden wurden, desto weniger gut wurde der Fit an die aus der Theorie kommenden Modelle.

Wie erwartet wurde eine $2\pi$-periodische Polarisation festgestellt. Der Modell passte hier sehr gut zu der gegebenen Funktion. 
Es war lediglich eine Verschiebung \SI{63.36}{\degree} mit einer relativen Abweichung von \SI{0.46}{\percent} festgestellt worden. 

Die theoretische Wellenlänge des He-Ne-Lasers liegt bei \SI{633}{\nano\meter}. Die gemessenen Werte wichen je Gitter zwischen \SI{1.87}{\percent} und \SI{5.17}{\percent} von dem erwarteten Wert ab. Dies ist trotz der recht groben Messung mithilfe eines Maßbandes ein genaues Ergebnis. 