\section{Ziel}
Das Ziel des Versuchs ist es die Eigenschaften eines Helium-Neon-Lasers (He-Ne-Lasers) zu vermessen. 

\section{Theorie}
\label{sec:Theorie}

\subsection{Aufbau eines Lasers}
%3 Komponenten
Ein Laser (light amplification by stimulated emission of radiation) besteht aus drei Komponenten: einer Energiepumpe, einem verstärkenden Medium und einem Resonator.
Durch die Energiepumpe wird dem verstärkenden Medium Energie zugeführt, was zu induzierter Emission durch die aktivierten Atome des Mediums führt. Der Resonator schickt das emittierte Licht durch optische Rückkopplung wieder durch das verstärkende Medium. Ein Laser ist also ein optischer Oszillator.

%Wellenlänge
Die Wellenlänge eines Lasers wird durch den Übergang zwischen zwei Energieniveaus des aktiven Mediums definiert. Die Atome gehen dabei unter Abgabe eines Photons der Frequenz
\begin{equation}
    \nu = \frac{E_\text{k} - E_\text{i}}{h}
    \label{eq:nu}
\end{equation}
vom angeregten Zustand $E_\text{k}$ in den tieferen Zustand $E_\text{i}$ über.

\subsection{Aktives Medium}
%Prozesse
Im aktiven Medium treten folgende Prozesse auf:
\begin{itemize}
    \item induzierte Absorption: ein Photon wird absorbiert und regt dabei das Atom vom Zustand $E_1$ in den energetisch höheren Zustand $E_2$ an. Dabei gilt die Bedingung \eqref{eq:nu}.
    \item induzierte Emission: ein Atom im angeregten Zustand $E_2$ wird durch ein äußeres Strahlungsfeld (einfallendes Photon) dazu veranlasst, unter Emission eines zweiten Photons mit Frequenz \eqref{eq:nu} in den tieferen Zustand $E_1$ überzugehen. Das zweite Photon besitzt die gleichen optischen Eigenschaften wie das einfallende.
    \item spontane Emission: ein angeregtes Atom gibt ohne den Einfluss eines äußeren Feldes seine Anregungsenergie unter Emission eines Photons ab.
\end{itemize}

%Inversion
Die Verstärkung eines Strahls durch induzierte Emission kann nur stattfinden, wenn Besetzungsinversion zwischen dem Grundzustand und dem Laserniveau herrscht. Das bedeutet, dass die Besetzungsdichte des Laserniveaus höher wird als die Besetzungsdichte des energetisch tiefer liegenden Niveaus (Zustand, in dem sich mehr Atome in einem höheren Energieniveau befinden als im tieferen). Dadurch ist die induzierte Emissionsrate größer als die Absorptionsrate und Licht kann beim Durchgang durch das aktive Medium verstärkt werden.

%beim He-Ne-Laser
Das aktive Medium beim He-Ne-Laser ist ein Helium-Neon-Gasgemisch. Diesem wird in einer elektrischen Entladung Energie zugeführt. Die durch die Ionisation gelösten Elektronen stoßen mit den He-Atomen, die dadurch in einen angeregten Zustand versetzt werden. Die Energie dieser angeregten He-Atome wird durch sogenannte Stöße zweiter Art auf die Ne-Atome übertragen. Die He-Atome wirken also als Energiepumpe für die Ne-Atome. Die nun angeregten Ne-Atome gehen unter Emission eines Photons in ein tiefer liegendes Niveau über. Dieser Übergang definiert, wie zuvor beschrieben, die Wellenlänge das Lasers.

\subsection{Pumpschemen}
Es gibt Drei-Niveau- und Vier-Niveau-Laser.

Bei Vier-Niveau-Lasern ist $E_3$ das obere Laserniveau, welches langlebig ist. Das kurzlebige Niveau $E_4$ steht für alle darüberliegenden Niveaus. Das untere Laserniveau $E_2$ ist ebenfalls kurzlebig. Das Grundniveau wird durch $E_1$ beschrieben.

Beim Drei-Niveau-Laser ist das Grundniveau $E_1$ auch das untere Laserniveau. Dadurch sind diese Laser ineffizienter als Vier-Niveau-Laser, da ... keine Ahnung.

Zwei-Niveau-Laser sind nicht möglich, da ... ?

Beim He-Ne-Laser handelt es sich um einen Vier-Niveau-Laser. (Richtig?)
Der Übergang von $^2s_2$ nach $^2p_4$ ist für die rote Linie verantwortlich.

Wie wird der Besetzungsinversion erreicht?

\subsection{Optischer Resonator}
%Stabilitätsparameter
Bei einem optischen Resonator verteilt sich das Strahlungsfeld nicht auf alle Moden, sondern bleibt auf wenige konzentriert. Für diese Moden hat der Resonator kleine Verlustfaktoren, für den Rest große. Die Wahrscheinlichkeit der induzierten Emission wird in den verlustarmen Moden größer, wodurch die Pumpenergie bevorzugt in Strahlungsenergie dieser Moden umgesetzt wird.

In einem stabilen Resonator reproduziert sich die Feldverteilung nach jedem Umlauf der Welle.
Die Stabilitätsbedingung lautet
\begin{equation}
    0 < g_1 g_2 < 1 \text{oder} g_1 = g_2 = 0,
    label{eq:stabilität}
\end{equation}
wobei $g_\text{i}$ die Stabilitätsparameter sind.

Stabilitätsparameter für zwei Resonatoren:

Maximaler Resonatorabstand:


\subsection{Moden im Resonator}
%
$TEM_\text{mn}$ Moden sind stationäre Feldverteilungen.
Moden mit $m=n=0$ heißen Fundamental- bzw. axiale Moden.
Die Fundamentalmoden haben ein radial-symmetrisches Gauß’sches Intensitätsprofil. Senkrecht zur Resonatorachse sinkt die Intensität auf $1/e^2$ ihres Wertes $I_0$ auf der Achse. Man nennt w den Radius der $TEM_{00}$-Mode. Der minimale Radius $w_0$ liegt bei $z = 0$, d. h. in der Mitte des konfokalen Resonators.
$w_0$ heißt die Strahltaille der Resonatormode.

%Modenblende:
Zur Selektion einzelner Moden eines Multimodelasers kann man in den Resonator optische Elemente einbauen, welche die Oszillation der unerwünschten Moden (höhere Moden) unterdrücken.

%Unterschied longitudinale und transversale Moden:
Transversale Moden (TEM-Moden)sind elektromagnetische Eigenschwingungszustände eines Systems in transversaler Richtung, also senkrecht zur Ausbreitungsrichtung.

Longitudinal = längs der Ausbreitungsrichtung. 
Die Anzahl der longitudinalen Moden gibt Aufschluss über die Anzahl der Wellen, die im Resonator schwingen können.
Für deren Auswahl eignet sich ein Fabry-Pérot-Interferometer.



\subsection{Multimode- und Singlemode-Laserbetrieb}
Singlemode = schmalbandig.
Man unterscheidet Single-Mode-Laser, die nahezu auf nur einer Frequenz schwingen, und Multimode-Laser, welche mehrere Farblinien emittieren. 

%Doppler-Effekt
Eine Form der inhomogenen Linienverbreiterung ist die Dopplerverbreiterung in Gasen. Die Frequenz eines Photons ist abhängig von der Geschwindigkeit des emittierenden Atoms relativ zur Beobachtungsrichtung. Die Linienform ergibt sich als Überlagerung mehrerer Beiträge, also durch die unterschiedlichen Ausbreitungsrichtungen der Gasmoleküle.
(etwas wenig)


%Verbreitung für den Neon-Übergang

%Modenspektrum für Laser mit L=1,5m

%Fabry-Pérot-Etalon
Modenselektion mithilfe des Fabry-Pérot-Etalon:
Das Fabry-Pérot-Etalon ist ein optischer Resonator, der aus zwei teildurchlässigen Spiegeln in unveränderlichem Abstand besteht. Eintreffendes Licht wird nur transmittiert, wenn die Wellenlängen die Resonanzbedingung des Resonators erfüllen.
Andere Bereiche werden durch destruktive Interferenz der Teilstrahlen fast vollständig ausgelöscht. 


\subsection{Polarisation eines Lasers}
%Brewster-Fenster
Durch Einfügen eines Brewster-Fensters in den Resonator kann eine Polarisationsrichtung selektiert werden.
Die Fensterflächen stehen  zur optischen Achse im Brewster-Winkel. Licht, das unter dem Brewster-Winkel auftrifft und parallel zur Einfallsebene polarisiert ist, wird an der Oberfläche nicht reflektiert, sondern tritt ohne Verlust durch das Brewster-Fenster. Die senkrecht zur Einfallsebene polarisierte Komponente des Lichts wird allerdings reflektiert und somit im Laser unterdrückt.

%resultierende Polarisation

\subsection{Bestimmung der Wellenlänge}
%Formel

