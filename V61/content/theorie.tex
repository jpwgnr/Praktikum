\section{Ziel}

\section{Theorie}
\label{sec:Theorie}

\subsection{Aufbau eines Lasers}
%3 Komponenten
Ein Laser (light amplification by stimulated emission of radiation) besteht aus drei Komponenten: einer Energiepumpe, einem verstärkenden Medium und einem Resonator.
Durch die Energiepumpe wird dem verstärkenden Medium Energie zugeführt, was zu induzierter Emission durch die aktivierten Atome des Mediums führt. Der Resonator schickt das emittierte Licht durch optische Rückkopplung wieder durch das verstärkende Medium. Ein Laser ist also ein optischer Oszillator.

%Wellenlänge
Die Wellenlänge eines Lasers wird durch den Übergang zwischen zwei Energieniveaus des aktiven Mediums definiert. Die Atome gehen dabei unter Abgabe eines Photons der Frequenz
\begin{equation}
    \nu = \frac{E_\text{k} - E_\text{i}}{h}
    \label{eq:nu}
\end{equation}
vom angeregten Zustand $E_\text{k}$ in den tieferen Zustand $E_\text{i}$ über.

\subsection{Aktives Medium}
%Prozesse
Im aktiven Medium treten folgende Prozesse auf:
\begin{itemize}
    \item induzierte Absorption: ein Photon wird absorbiert und regt dabei das Atom vom Zustand $E_1$ in den energetisch höheren Zustand $E_2$ an. Dabei gilt die Bedingung \eqref{eq:nu}.
    \item induzierte Emission: ein Atom im angeregten Zustand $E_2$ wird durch ein äußeres Strahlungsfeld (einfallendes Photon) dazu veranlasst, unter Emission eines zweiten Photons mit Frequenz \eqref{eq:nu} in den tieferen Zustand $E_1$ überzugehen. Das zweite Photon besitzt die gleichen optischen Eigenschaften wie das einfallende.
    \item spontane Emission: ein angeregtes Atom gibt ohne den Einfluss eines äußeren Feldes seine Anregungsenergie unter Emission eines Photons ab.
\end{itemize}

%Inversion
Die Verstärkung eines Strahls durch induzierte Emission kann nur stattfinden, wenn Besetzungsinversion zwischen dem Grundzustand und dem Laserniveau herrscht. Das bedeutet, dass die Besetzungsdichte des Laserniveaus höher wird als die Besetzungsdichte des energetisch tiefer liegenden Niveaus (Zustand, in dem sich mehr Atome in einem höheren Energieniveau befinden als im tieferen). Dadurch ist die induzierte Emissionsrate größer als die Absorptionsrate und Licht kann beim Durchgang durch das aktive Medium verstärkt werden.

%beim He-Ne-Laser
Das aktive Medium beim He-Ne-Laser ist ein Helium-Neon-Gasgemisch. Diesem wird in einer elektrischen Entladung Energie zugeführt. Die durch die Ionisation gelösten Elektronen stoßen mit den He-Atomen, die dadurch in einen angeregten Zustand versetzt werden. Die Energie dieser angeregten He-Atome wird durch sogenannte Stöße zweiter Art auf die Ne-Atome übertragen. Die He-Atome wirken also als Energiepumpe für die Ne-Atome. Die nun angeregten Ne-Atome gehen unter Emission eines Photons in ein tiefer liegendes Niveau über. Dieser Übergang definiert, wie zuvor beschrieben, die Wellenlänge das Lasers.

\subsection{Pumpschemen}
Es gibt Drei-Niveau- und Vier-Niveau-Laser.

Bei Vier-Niveau-Lasern ist $E_3$ das obere Laserniveau, welches langlebig ist. Das kurzlebige Niveau $E_4$ steht für alle darüberliegenden Niveaus. Das untere Laserniveau $E_2$ ist ebenfalls kurzlebig. Das Grundniveau wird durch $E_1$ beschrieben.

Beim Drei-Niveau-Laser ist das Grundniveau $E_1$ auch das untere Laserniveau. Dadurch sind diese Laser ineffizienter als Vier-Niveau-Laser, da ... keine Ahnung.

Zwei-Niveau-Laser sind nicht möglich, da ... ?

Beim He-Ne-Laser handelt es sich um einen Vier-Niveau-Laser. (Richtig?)
Der Übergang von $2s_2$ nach $2p_4$ ist für die rote Linie verantwortlich.

Wie wird der Besetzungsinversion erreicht?

\subsection{Optischer Resonator}
%Stabilitätsparameter

\subsection{Moden im Resonator}
%

\subsection{Multimode- und Singlemode-Laserbetrieb}
%Doppler-Effekt

%Fabry-Perot-Etalon

\subsection{Polarisation eines Lasers}
%Brewster-Fenster

%resultierende Polarisation

\subsection{Bestimmung der Wellenlänge}
%Formel

