\section{Diskussion}
\label{sec:Diskussion}

Die Auswertung der Funktionen ist recht exakt zu bewerten. 

Die gemessenen Werte ergeben nach der halblogarithmischen Auftragung einen Fit, der eine Steigung von xy hat. Der relative Fehler der linearen Regression liegt bei xy. 

Im Aufgabenteil b) ergibt der nach Umformung der Funktion xy auch eine lineare Gleichung. Die Werte führen zu einer linearen Regression, dessen Wert bei xy liegt und der relative Fehler bei xy. Auffällig ist dabei, dass die Werte für 10 Hz und 20 Hz rausgelassen wurden, da der Wert sonst deutlich abweicht. Auch ein weiterer Wert sticht besonders hervor. Das Problem dabei könnte sein, dass, wenn man den maximalen Wert von 620 Hz als U0 wählt, man durch Null teilt, weshalb man mindestens 621 mV (keine ahnung, was für einheiten), als U0 wählen muss, damit dies nicht geschieht. Würde man den Wert bei 10Hz wählen, wäre das Ergebnis imaginär, was auch keiner realistischen physikalischen Lösung entspricht. Somit wird angenommen, dass der Wert bei mind. 621 mV liegt. Die tatsächliche maximal Amplitude liegt aber vermutlich bei einem noch höheren Wert. Für diesen würden die Werte bei 10 und 20 Hz vermutlich auch passen. 

Im Teil c) wird ein Wert xy für die Zeitkonstante 1/RC festgestellt. Bei einem relativen Fehler von xy überschneiden sich alle drei Ergebnisse in einem Bereich von xy +- xy. Somit ist eine Systematische Abweichung nicht zwangsläufig zu erkennen. Die Werte für a) und b) weichen dennoch um xy Prozent von einander ab. Dieser Fehler könnte an dem nicht betrachteten Innenwiderstand des Sinusfrequenzgenerators entstanden sein. Der Wert dieses liegt laut Anleitung bei 600 Ohm \cite{V353}. 

Die Funktion der Kondensatorspannung U_C als Integrator der Spannung U(t) scheint anhand der ermittelten Schaubilder bestätigt zu sein. Die annähernde Korrektheit lässt sich zumindest gut anhand der Schaubilder erkennen, wenn die Abhängigkeit der Hoch und Tiefpunkte von den Nullstellen der anderen Funktionen betrachtet werden. 

Die Abhängigkeit der Relativamplitude zur Phase phi lässt sich in den Polarkoordinatensystemen gut ablesen. Dabei ist die Eigenschaft der phi Werte als arctan gut zu erkennen, da die gemessenen Werte der Theoriekurve eindeutig ähneln, lediglich Phasenverschoben sind. 
