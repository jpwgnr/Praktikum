\section{Auswertung}
\label{sec:Auswertung}


\subsection{Bestimmung der Zeitkonstante}
Die Werte, die für die Bestimmung der Zeitkonstante $RC$ nötig sind, befinden sich in Tabelle \ref{table: }. %Tabelle
%Hier Tabelle einfügen
\begin{table}\caption{Die Anzahl der Impulse, der Startwert auf der Mikrometerschraube und der Endwert auf der Mikrometerschraube.}
\label{taba}
\centering
\sisetup{round-mode = places, round-precision=2, round-integer-to-decimal=true}
\begin{tabular}{S[]S[]S[]} 
\toprule
{Anzahl} & {$d_\text{Start} / \si{\milli\meter}$} & {$d_\text{Start} / \si{\milli\meter}$}\\
\midrule
3001.0 & 6.73 & 2.0\\
3002.0 & 6.73 & 2.0\\
3000.0 & 1.82 & 6.5\\
3000.0 & 6.74 & 2.0\\
3000.0 & 1.83 & 6.5\\
3000.0 & 6.74 & 2.0\\
3001.0 & 1.84 & 6.5\\
3000.0 & 2.83 & 7.5\\
3001.0 & 7.77 & 3.0\\
3002.0 & 2.75 & 7.5\\
\bottomrule
\end{tabular}\end{table}
\begin{table}\caption{Die Zeit $t$ gegen den negativen Logarithmus der Spannungswerte geteilt durch die maximale Spannung.}
\label{taba}
\centering
\sisetup{round-mode = places, round-precision=3, round-integer-to-decimal=true}
\begin{tabular}{S[]S[]} 
\toprule
{$t/ \si{\milli\second}$} & {$-log(\frac{U(t)}{U_{0}})$}\\
\midrule
0.0 & -0.0\\
0.4 & 0.13372577497521726\\
0.8 & 0.4281699018019215\\
1.2 & 0.705467664947986\\
1.6 & 0.9977516783331359\\
2.0 & 1.3015531326648002\\
2.4 & 1.623136756792262\\
2.8 & 1.8993901334204206\\
3.2 & 2.2178438645389553\\
3.6 & 2.505525936990737\\
4.0 & 2.793208009442516\\
4.4 & 3.19867311755068\\
4.8 & 3.604138225658845\\
5.2 & 4.29728540621879\\
5.6 & 4.990432586778735\\
\bottomrule
\end{tabular}\end{table}
Die Zeitkonstante berechnet sich mittels Gleichung \eqref{eqn: RC} zu $RC = $. %Wert für RC 

\begin{figure}
  \centering
  \includegraphics{build/plota.pdf}
  \caption{}
  \label{fig:plota}
\end{figure}

\subsection{4b}
Die Kondensatorspannungen in Abhängigkeit von der Frequenz sind in Tabelle \ref{table: } dargstellt. %Tabelle
%Hier Tabelle einfügen
\begin{table}\caption{Die Frequenzen der Sägezahnspannung.}
\label{tabb}
\centering
\sisetup{round-mode = places, round-precision=2, round-integer-to-decimal=true}
\begin{tabular}{S[]S[]} 
\toprule
{Index} & {$\nu_\text{Sä} / \si{\hertz}$}\\
\midrule
1.0 & 25.02\\
2.0 & 49.95\\
3.0 & 99.99\\
4.0 & 149.97\\
\bottomrule
\end{tabular}\end{table}
\begin{table}\caption{Der Kehrwert der Kreisfrequenz $\omega$ gegen die Wurzel aus dem Bruch in dessen Nenner die maximale Spannung durch die Amplitudenwerte von $U_{C}$ zum Quadrat um eins subtrahiert werden}
\label{tabb}
\centering
\sisetup{round-mode = places, round-precision=5, round-integer-to-decimal=true}
\begin{tabular}{S[]S[]} 
\toprule
{$\frac{1}{\omega}/ \si{\second}$} & {$\sqrt{\frac{1}{(\frac{U_{0}}{A(\omega)})^{2}-1}}$}\\
\midrule
0.0024485375860291594 & 2.262660907951623\\
0.0019894367886486917 & 1.7091833258800144\\
0.0015915494309189533 & 1.2706566931710195\\
0.0006366197723675814 & 0.48280454958526764\\
0.00039788735772973834 & 0.2991883988251616\\
0.0002448537586029159 & 0.18505403427568887\\
0.00019894367886486917 & 0.14980117725462763\\
0.00015915494309189535 & 0.14980117725462763\\
6.366197723675813e-05 & 0.0483656490240811\\
3.978873577297384e-05 & 0.030287634503871775\\
2.448537586029159e-05 & 0.01884392449684891\\
1.989436788648692e-05 & 0.015621873649013022\\
1.5915494309189534e-05 & 0.01240030915098555\\
\bottomrule
\end{tabular}\end{table}

Die Zeitkonstante wird wieder durch \eqref{eqn: RC} berechnet. Es ergibt sich $RC = $. %Wert für RC

\begin{figure}
  \centering
  \includegraphics{build/plotb.pdf}
  \caption{}
  \label{fig:plotb}
\end{figure}

\subsection{4c}
In Tabelle \ref{table: } sind die Abstände der Nulldurchgänge der Spannung des Kondensators und der Spannung %Tabelle
des Generators in Abhängigkeit von der Frequenz dargestellt.
%Hier Tabelle einfügen
\begin{table}\caption{Der magnetische Fluss $B$ des gemessenen Magnetfelds gegen den Strom $I$ des erzeugenden Magnetfelds, Neukurve.}
\label{tabc}
\centering
\sisetup{round-mode = places, round-precision=1, round-integer-to-decimal=true}
\begin{tabular}{S[]S[]} 
\toprule
{$B$/ \si{\milli\tesla}} & {$I$/ \si{\ampere}}\\
\midrule
0.0 & 0.0\\
111.19999999999999 & 1.0\\
273.5 & 2.0\\
397.8 & 3.0\\
479.9 & 4.0\\
537.9000000000001 & 5.0\\
585.0999999999999 & 6.0\\
621.8000000000001 & 7.0\\
653.1 & 8.0\\
679.9 & 9.0\\
704.3000000000001 & 10.0\\
\bottomrule
\end{tabular}\end{table}
\begin{table}\caption{Der Kehrwert der Kreisfrequenz gegen den negativen Kehrwert des Tangens der Phase, die sich durch die negative Division der zeitlichen Phasenverschiebung durch die Periodendauer multipliziert mit $\pi$ ergibt}
\label{tabc}
\centering
\sisetup{round-mode = places, round-precision=5, round-integer-to-decimal=true}
\begin{tabular}{S[]S[]} 
\toprule
{$\frac{1}{\omega}/ \si{\second}$} & {$-\frac{1}{tan(\phi(\omega))}$}\\
\midrule
0.0024485375860291594 & 3.117626187195526\\
0.0019894367886486917 & 2.723786837133246\\
0.0015915494309189533 & 2.2715973027329865\\
0.0006366197723675814 & 1.4714553158199692\\
0.00039788735772973834 & 1.2401991640705359\\
0.0002448537586029159 & 1.2010895549922853\\
0.00019894367886486917 & 1.120011792429241\\
0.00015915494309189535 & 1.0000000000000002\\
6.366197723675813e-05 & 0.8540806854634666\\
3.978873577297384e-05 & 1.0126459941540735\\
2.448537586029159e-05 & 1.0190294678742615\\
1.989436788648692e-05 & 1.064891840324792\\
1.5915494309189534e-05 & 1.064891840324792\\
\bottomrule
\end{tabular}\end{table}
Mittels \eqref{eqn: RC} berechnet sich die Zeitkonstante zu $RC = $. %Wert für RC

\begin{figure}
  \centering
  \includegraphics{build/plotc.pdf}
  \caption{}
  \label{fig:plotc}
\end{figure}

\subsection{4d}

%Tabellen
\begin{table}\caption{Die Phasenverschiebung gegen die Amplitude der Spannung $U_{C}$ geteilt durch die maximale Spannung $U_{0}}$
\label{tabd}
\centering
\sisetup{round-mode = places, round-precision=5, round-integer-to-decimal=true}
\begin{tabular}{S[]S[]} 
\toprule
{$\phi/ \si{\radian}$} & {$\frac{A(\omega)}{U_{0}}$}\\
\midrule
-0.31038935417467156 & 0.9146537842190016\\
-0.3518583772020568 & 0.8631239935587762\\
-0.4146902302738527 & 0.7858293075684379\\
-0.5969026041820606 & 0.4347826086956522\\
-0.6785840131753953 & 0.286634460547504\\
-0.6942919764433444 & 0.1819645732689211\\
-0.728849495632832 & 0.14814814814814814\\
-0.7853981633974483 & 0.14814814814814814\\
-0.8639379797371932 & 0.04830917874396135\\
-0.7791149780902686 & 0.03027375201288245\\
-0.7759733854366789 & 0.01884057971014493\\
-0.7539822368615503 & 0.015619967793880838\\
-0.7539822368615503 & 0.012399355877616747\\
\bottomrule
\end{tabular}\end{table}

%Plots einfügen
\begin{figure}
  \centering
  \includegraphics{build/plotd1.pdf}
  \caption{Plot}
  \label{fig:plotd1}
\end{figure}

\begin{figure}
  \centering
  \includegraphics{build/plotsd2.pdf}
  \caption{Plot}
  \label{fig:plotd2}
\end{figure}
\begin{figure}
  \centering
  \includegraphics{build/plotd3.pdf}
  \caption{Plot}
  \label{fig:plotd3}
\end{figure}

%Integrator Bilder
\begin{figure}
  \centering
  \includegraphics{build/integrator1.pdf}
  \caption{Plot}
  \label{fig:plot}
\end{figure}
\begin{figure}
  \centering
  \includegraphics{build/integrator2.pdf}
  \caption{Plot}
  \label{fig:plot}
\end{figure}
\begin{figure}
  \centering
  \includegraphics{build/integrator3.pdf}
  \label{fig:plot}
\end{figure}
%jeweils einzelne Formeln für RC in der Theorie angeben und hier erwähnen!
