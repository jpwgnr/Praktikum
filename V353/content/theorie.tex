
\section{Ziel}
In diesem Versuch soll das Relaxationsverhalten eines RC-Kreises untesucht und ausgewertet werden. 

\section{Theorie }
%\cite{V353}
\label{sec:Theorie}
\subsection{Auf- und Entladevorgang}

Der Aufladevorgang eines Kondensators mit Kapazität $C$, der über einen Widerstand $R$ mit der Spannung $U_{0}$ verbunden ist, wird durch die Gleichung
\begin{align*}
     Q(t)&= CU_{0} (1-\mathbf{exp}(-\frac{t}{RC}))
\end{align*}
beschrieben. Der Vorgang wird durch die Ladung $Q$ zum Zeitpunkt $t$ dargestellt. 

Auf dieselbe Art und Weise wird der Entladevorgang durch
\begin{align*}
     Q(t)&= Q(0) \mathbf{exp}(-\frac{t}{RC})
\end{align*}
beschrieben. 

\subsection{Spannung messen}

Eine Wechselspannung $U(t)$ wird durch die Formel 

\begin{align*} 
    U(t)&= U_{0} \mathbf{cos}\omega t 
\end{align*}
dargestellt. Dabei ist $U_{0}$ die maximale Spannung/Auslenkung und der zweite Teil der Funktion beschreibt die Oszillation um den Nullpunkt in Abhängigkeit der Frequenz $\omega$ und der Zeit $t$. 

Bei einer Phasenverschiebung $\phi$ verschiebt sich diese Oszillation um einen gewissen Wert. Die neue Formel lautet dann 

\begin{align*} 
    U(t)&= U_{0} \mathbf{cos}(\omega t + \phi)
\end{align*}

Ein $RC$-System setzt sich nach der zweiten Kirchhoffschen Regel aus der Spannung $U_{R}$ des Widerstands und der Spannung $U_{C}$ des Kondensators zusammen.
Es gilt 
\begin{align*} 
    U(t) &= U_{R}(t) + U_{C}(t).
\end{align*}

Werden die oberen Zusammenhänge und das Ohmsche Gesetz in diesen Zusammenhang eingesetzt ergibt sich 

\begin{align*} 
U_{0} \mathbf{cos}\omega t &= -A(\omega) RC \mathbf{sin}(\omega t + \phi) + A(\omega) \mathbf{cos}(\omega t + \phi),  
\end{align*}

wobei hierbei für die Phasenverschiebung $\phi$ gilt 

\begin{align*} 
\phi (\omega)&= \mathbf{arctan}(-\omega RC) 
\end{align*}

und für die Amplitude $A(\omega)$ 

\begin{align*} 
    A(\omega) &= - frac{\mathbf{sin}\phi}{\omega RC} U_{0}.
\end{align*}
    
Durch einige Umformungen ergibt sich dann 

\begin{equation}
A(\omega) = \frac{U_{0}}{sqrt{1 + \omega^2 R^2 C^2}}
\end{equation}

Dabei wird die Amplitude $A$ der Kondensatorspannung durch die Frequenz $\omega$ der Erregerspannung beeinflusst. 

\subsection{RC-Schwingkreis als Integrator der Spannung U(t)}

Ein RC-SChwingkreis kann dazu genutzt werden eine zeitlich veränderliche Spannung $U(t)$ unter bestimmten Bedingungen zu integrieren. 
Es wird ein proportionaler Zusammenhang zwischen der Spannung des Kondensators $U_{C}$ und dem Integral $int{U(t) dt}$ festgestellt. Dieser ergibt sich durch 

\begin{equation}
    U_{C} = \frac{1}{RC} \int{0}{t}{U(t') dt'} 
\end{equation} 



