
\section{Ziel}
In diesem Versuch soll das Relaxationsverhalten eines RC-Kreises untesucht und ausgewertet werden. 

\section{Theorie}
%\cite{V353}
\label{sec:Theorie}
\subsection{Bestimmung der Zeitkonstante}
Die Zeitkonstante $RC$ kann durch die Messung des Auflade- bzw. Entladevorgangs eines Kondensators bestimmt werden.
Der Aufladevorgang eines Kondensators mit Kapazität $C$, der über einen Widerstand $R$ mit der Spannung $U_{0}$ verbunden ist, wird durch die Gleichung
\begin{align*}
     U(t)&= U_{0} (1-\mathbf{exp}(-\frac{t}{RC}))
\end{align*}
beschrieben. Der Vorgang wird durch die Spannung $U$ zum Zeitpunkt $t$ dargestellt. 
Auf dieselbe Art und Weise wird der Entladevorgang durch
\begin{align*}
     U(t)&= U_{0} \mathbf{exp}(-\frac{t}{RC})
\end{align*}
beschrieben.
%Lineare Regression: mehr erklären?
Die Zeitkonstante wird anhand der Steigung einer linearen Regression bestimmt.
Die negative logarithmierte Kondensatorspannung wird gegen die Zeit aufgetragen.
Die Steigung bei der linearen Regression ist
\begin{equation}
    m = \frac{\overline{xy} - \overline{x}\overline{y}}{\overline{x^2} - \overline{x}^2}.
    \label{eqn: m}
\end{equation}
%x und y angeben?
Da die Steigung
\begin{equation*}
    m = \frac{1}{RC}
\end{equation*}
ist, wird $RC$ bestimmt durch
\begin{equation}
    RC= \frac{1}{m} = \frac{\overline{x^2} - \overline{x}^2}{\overline{xy} - \overline{x} \overline{y}}.
    \label{eqn: RC}
\end{equation}

\subsection{Bestimmung der Kondensatorspannung} %Andere Überschrift?
Eine Wechselspannung $U(t)$ wird durch die Formel 
\begin{align*} 
    U(t)&= U_{0} \mathbf{cos}\omega t 
\end{align*}
dargestellt. Dabei ist $U_{0}$ die maximale Spannung. $\mathbf{cos}\omega t$ beschreibt die Oszillation um den Nullpunkt in Abhängigkeit von der Frequenz $\omega$ und der Zeit $t$. 
Mit einer Phasenverschiebung $\phi$ verschiebt sich die Oszillation der Kondensatorspannung um einen gewissen Wert. Die neue Formel lautet dann 
\begin{align*} 
    U_{C}(t)&= U_{0} \mathbf{cos}(\omega t + \phi).
\end{align*}
Ein $RC$-System setzt sich nach der zweiten Kirchhoffschen Regel aus der Spannung $U_{R}$ des Widerstands und der Spannung $U_{C}$ des Kondensators zusammen.
Es gilt 
\begin{align*} 
    U(t) &= U_{R}(t) + U_{C}(t).
\end{align*}
Mit den oberen Gleichungen für $U(t)$, $U_{C}(t)$ und dem Ohmsche Gesetz, ergibt sich 
\begin{align*} 
U_{0} \mathbf{cos}\omega t &= -A(\omega) RC \mathbf{sin}(\omega t + \phi) + A(\omega) \mathbf{cos}(\omega t + \phi),  
\end{align*}
wobei für die Phasenverschiebung 
\begin{equation} 
\phi (\omega) = \mathbf{arctan}(-\omega RC) 
\label{eqn: phi}
\end{equation}
gilt. Die Amplitude $A(\omega)$ ist
\begin{equation} 
    A(\omega) = - \frac{\mathbf{sin}\phi}{\omega RC} U_{0}.
    \label{eqn: A1}
\end{equation}
Durch einige Umformungen ergibt sich dann 
\begin{equation}
A(\omega) = \frac{U_{0}}{\sqrt{1 + \omega^2 R^2 C^2}}.
\label{eqn: A2}
\end{equation}
Dabei wird die Amplitude $A(\omega)$ der Kondensatorspannung durch die Frequenz $\omega$ der Erregerspannung beeinflusst.
%Lineare Regression:
Um die Zeitkonstante $RC$ zu bestimmen, wird wieder eine lineare Regression $y=mx$ durchgeführt. %Formulierung ändern
Die Gleichung \eqref{eqn: A2} wird so umgeformt, dass $\frac{1}{RC}$ die Steigung $m$ ist.
Für x und y ergeben sich: $x=$ und $y=$. %x und y einfügen
Durch die Steigung der Geraden kann dann $RC$ mittels \eqref{eqn: RC} bestimmt werden.

\subsection{Bestimmung der Phasenverschiebung} %Andere Überschrift?
Die Phasenverschiebung $\phi(\omega)$ lässt sich mit
\begin{equation}
    \phi = \frac{a}{T}2\pi
\end{equation}
bestimmen. Dabei ist $a$ der Abstand der Nulldurchgänge der Spannung des Kondensators und der Spannung des Generators.
$T$ ist die Schwingungsdauer, die durch
\begin{equation}
    T = \frac{1}{\omega}
\end{equation}
gegeben ist. $\omega$ ist die Kreisfrequenz.
%Gibt der Generator die Frequenz oder Kreisfrequez an?
%Lineare Regression:
Um die Zeitkonstante zu berechnen, wird eine lineare Regression benötigt.
Die Gleichung \eqref{eqn: phi} wird so umgestellt, dass die Steigung $m = \frac{1}{RC}$ ist.
Dann ist $x= $ und $y= $. %x und y einfügen
$RC$ kann somit wieder durch \eqref{eqn: RC} bestimmt werden.

\subsection{RC-Schwingkreis als Integrator der Spannung U(t)}
Ein RC-Schwingkreis kann dazu genutzt werden eine zeitlich veränderliche Spannung $U(t)$ unter bestimmten Bedingungen zu integrieren. 
Es wird ein proportionaler Zusammenhang zwischen der Spannung des Kondensators $U_{C}$ und dem Integral $\int U(t) dt$ festgestellt. Dieser ergibt sich durch 
\begin{equation}
    U_{C}(t) = \frac{1}{RC} \int_{0}^{t} U(t') dt'.
\end{equation}
%Formel A1 benutzen?



