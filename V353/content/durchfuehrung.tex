\section{Durchführung}
\label{sec:Durchführung}

\subsection{Bestimmung der Zeitkonstante}
Die Zeitkonstante $RC$ wird durch die Messung des Entladevorgangs des Kondensators bestimmt. 
Mit der in Abb. \ref{fig:Abb1} dargestellten Schaltung wird die am Kondensator gemessene Spannung $U_{C}(t)$ auf einem
Oszilloskop in Abhängigkeit von der Zeit $t$ angezeigt. Dabei muss darauf geachtet werden, dass sich die Spannung 
$U_{C}(t)$ innerhalb des Aufzeichnungszeitraums um den Faktor 5 bis 10 ändert. Sobald eine geeignete Kurve auf dem 
Bildschirm zu erkennen ist, werden mindestens $10$ Messwertpaare aufgenommen.

\subsection{Messung der Kondensatorspannung}
Mittels der Schaltung, die in Abb. \ref{fig:Abb2} dargestellt wird, wird die Amplitude der Kondensatorspannung in 
Abängigkeit von der Frequenz gemessen.

\subsection{Messung der Phasenverschiebung}
Zur Ermittlung der Phasenverschiebung werden, wie in Abb. \ref{fig:Abb3} dargestellt, die Kondensatorspannung $U_{C}(t)$ 
und die Generatorspannung $U_{G}(t)$ an ein Zweistrahl-Oszilloskop angeschlossen. Dabei wird der Abstand $a$ der 
Nulldurchgänge der beiden Kurven gemessen. Die Periodendauer $T$ ergibt sich aus der eingestellten Frequenz.%geteilt um auf den Winkel $\phi$ der Phasenverschiebung zu kommen. 
%Formel phi= a/b * 2pi in Theorie ergänzen S.7 unten Anleitung 

\subsection{Nachweis der Integrator-Eigenschaft eines RC-Kreises}
Es wird erneut die Schaltung aus Abb.\ref{fig:Abb3} benutzt. Am Sinusgenerator werden nacheinander eine Rechteck-, 
Sinus- und Dreiecksspannung auf den RC-Kreis gegeben. Dabei werden auf dem Zweikanal-Oszilloskop sowohl die 
zu integrierende und die integrierte Spannung angezeigt. Von den angezeigten Spannungen werden jeweils Aufnahmen
des Bildschirms gespeichert. 
