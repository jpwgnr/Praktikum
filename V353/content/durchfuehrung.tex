\section{Durchführung}
\label{sec:Durchführung}

\subsection{Bestimmung der Zeitkonstante}

Die Zeitkonstante wird durch die Messung und Beobachtung des Ent- und Aufladevorgangs des Kondensators bestimmt. 
Mit der in Abb.\ref{fig:Abb1} dargestellten Schaltung wird die am Kondensator gemessene Spannung $U_{C}$ auf einem Oszilloskop in Abhängigkeit der Zeit angezeigt. 
Dabei muss darauf geachtet werden, dass sich die Spannung $U_{C}(t)$ innerhalb des Aufzeichnungszeitraums um den Faktor 5 bis 10 ändert. 
Sobald eine geeignete Kurve auf dem Bildschirm zu erkennen ist, wird das Signal $U_{C}(t)$ auf ein Speicheroszilloskop übertragen und es wird ein Thermodruck erstellt. 

\subsection{Messung der Spannung $U_{C}$}

Mittels der Schaltung, die in Abb. \ref{fig:Abb2} dargestellt wird, wird die Amplitude der Kondensatorspannung in Abängigkeit von der Frequenz gemessen. Dafür wird die Frequenz des Sinusgenerators mittels eines Frequenzmessers aufgenommen und gegen die Werte des \si{\milli\Volt}-Meters aufgetragen, das die Amplitude $A(\omega)$ misst.

\subsection{Phasenverschiebung}

Zur Ermittlung der Phasenverschiebung wird, wie in Abb. \ref{fig:Abb3} dargestellt, die Kondensatorspannung $U_{C}$ und die Generatorspannung $U_{G}$ an ein Zweistrahl-Oszilloskop angeschlossen. 
Dabei wird der Abstand $a$ der beiden Nulldurchgänge gemessen und durch die Periodendauer $\lambda$ geteilt um auf den Winkel $\phi$ der Phasenverschiebung zu kommen. %Formel phi= a/b * 2pi in Theorie ergänzen S.7 unten Anleitung 

\subsection{Nachweis der Integrator-Eigenschaft eines RC-Kreises}

Es wird erneut die Schaltung aus Abb.\ref{fig:Abb3} benutzt. Am Sinusgenerator werden nacheinander eine Rechteck-, Sinus- und Dreiecksspannung auf den RC-Kreis gegeben. Dabei werden auf dem Zweikanal-Speicherozilloskop sowohl die zu integrierende und die integrierte Spannung angezeigt und anschließend als Thermodruck ausgegeben.   
