\begin{table}\caption{Der Kehrwert der Kreisfrequenz $\omega$ gegen die Wurzel aus dem Bruch in dessen Nenner die maximale Spannung durch die Amplitudenwerte von $U_{C} zum Quadrat um eins subtrahiert werden}
\label{tabb}
\centering
\sisetup{round-mode = places, round-precision=5, round-integer-to-decimal=true}
\begin{tabular}{S[]S[]} 
\toprule
{$\frac{1}{\omega}/ \si{\second}$} & {$\sqrt{\frac{1}{(\frac{U_{0}}{A(\omega)})^{2}-1}}$}\\
\midrule
0.0024485375860291594 & 2.262660907951623\\
0.0019894367886486917 & 1.7091833258800144\\
0.0015915494309189533 & 1.2706566931710195\\
0.0006366197723675814 & 0.48280454958526764\\
0.00039788735772973834 & 0.2991883988251616\\
0.0002448537586029159 & 0.18505403427568887\\
0.00019894367886486917 & 0.14980117725462763\\
0.00015915494309189535 & 0.14980117725462763\\
6.366197723675813e-05 & 0.0483656490240811\\
3.978873577297384e-05 & 0.030287634503871775\\
2.448537586029159e-05 & 0.01884392449684891\\
1.989436788648692e-05 & 0.015621873649013022\\
1.5915494309189534e-05 & 0.01240030915098555\\
\bottomrule
\end{tabular}\end{table}