\begin{table}\caption{Verschiedene Frequenzen und die dazu entstehende Amplitude der Spannung des Kondensatorsi, $U_{C}$, und die zeitliche Phasenverschiebung zur Spannung $U(t)$}
\label{tab2}
\centering
\sisetup{round-mode = places, round-precision=5, round-integer-to-decimal=true}
\begin{tabular}{S[]S[]S[]} 
\toprule
{$f/ \si{\Hertz}$} & {$A(\omega)/ \si{V}$} & {$a / \si{\second}$}\\
\midrule
65.0 & 0.568 & 0.00152\\
80.0 & 0.536 & 0.0014\\
100.0 & 0.488 & 0.00132\\
250.0 & 0.27 & 0.00076\\
400.0 & 0.178 & 0.00054\\
650.0 & 0.113 & 0.00034\\
800.0 & 0.092 & 0.00029\\
1000.0 & 0.092 & 0.00025\\
2500.0 & 0.03 & 0.00011\\
4000.0 & 0.0188 & 6.2e-05\\
6500.0 & 0.0117 & 3.8e-05\\
8000.0 & 0.0097 & 3e-05\\
10000.0 & 0.0077 & 2.4e-05\\
\bottomrule
\end{tabular}\end{table}