\begin{table}\caption{Die Phasenverschiebung gegen die Amplitude der Spannung $U_{C}$ geteilt durch die maximale Spannung $U_{0}}$
\label{tabd}
\centering
\sisetup{round-mode = places, round-precision=5, round-integer-to-decimal=true}
\begin{tabular}{S[]S[]} 
\toprule
{$\phi/ \si{\radian}$} & {$\frac{A(\omega)}{U_{0}}$}\\
\midrule
-0.31038935417467156 & 0.9146537842190016\\
-0.3518583772020568 & 0.8631239935587762\\
-0.4146902302738527 & 0.7858293075684379\\
-0.5969026041820606 & 0.4347826086956522\\
-0.6785840131753953 & 0.286634460547504\\
-0.6942919764433444 & 0.1819645732689211\\
-0.728849495632832 & 0.14814814814814814\\
-0.7853981633974483 & 0.14814814814814814\\
-0.8639379797371932 & 0.04830917874396135\\
-0.7791149780902686 & 0.03027375201288245\\
-0.7759733854366789 & 0.01884057971014493\\
-0.7539822368615503 & 0.015619967793880838\\
-0.7539822368615503 & 0.012399355877616747\\
\bottomrule
\end{tabular}\end{table}