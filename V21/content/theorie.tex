\section{Theorie}
\label{sec:Theorie}

In diesem Versuch soll mittels optischen Pumpens die Kernspins $I$ der Isotope von Rubidium, $^{85}$Rb und $^{87}$Rb, bestimmt werden.

\subsection{Landé-Faktoren}

Ein Teilchen, das sich wie ein Elektron in einem Atom bewegt, besitzt einen Drehimpuls $\vec L$ bei der Bewegung um den Kern. 
Der Spin $\vec S$, eine Größe, die aus der relativistischen Mechanik resultiert, beschreibt das Teilchen neben seiner Masse und Ladung. 

Beide Größen, $\vec L$ und $\vec S$ besitzen ein magnetisches Moment, die sich als 
\begin{align*}
    \mu_\text{L} &= \mu_\text{B} g_\text{L} \sqrt{L(L+1)} \\ 
    \mu_\text{S} &= \mu_\text{B} g_\text{S} \sqrt{S(S+1)}
\end{align*}
darstellen lassen mit dem Bohrschen Magneton und den Landé-Faktoren $g_\text{L,S}$. 

Da die Bewegung des Elektrons ein magnetisches Feld erzeugt und der Spin, der eine magnetische Größe ist, insofern mit diesem Magnetfeld interagiert, Koppeln der Spin und der Drehimpuls miteinander. 
Die resultierende Größe ist der Gesamtdrehimpuls $\vec J = \vec L + \vec S$. 

Auch hier ergibt sich ein magnetisches Moment $\mu_\text{J}$, das um die Richtung des Gesamtdrehimpulses präzediert, wobei der senkrechte Anteil im Mittel herausfällt und nur der parallele Anteil relevant ist. 

Für $\mu_\text{J}$ ergibt sich 
\begin{equation*}
    \mu_\text{J} = \mu_\text{B} g_\text{J} \sqrt{J(J+1)}
\end{equation*}
und für $g_\text{J}$ gilt 
\begin{equation*}
    g_\text{J} = 1 + \frac{J(J+1) + S(S+1) - L(L+1)}{2J(J+1)}.
\end{equation*}

\subsection{Energieaufspaltung}

Durch die Spin-Bahn-Kopplung werden die Energie Niveaus zum ersten mal gespalten, je nachdem in welche Richtung der Drehimpuls und der Spin gerichtet sind (parallel, orthogonal oder antiparallel).
Diese Aufspaltung nennt sich Feinstruktur. 

Die Energieverschiebung ergibt sich dann mit 
\begin{equation*}
    \Delta E = g_\text{J} \mu_\text{B} B m_\text{J}
\end{equation*}
wobei $m_\text{J}$ die Magnetquantenzahl ist.

Die Hyperfeinstruktur spaltet die Energieniveaus ein weiteres mal. 
Dabei kommt der Kernspin $I$ des Atoms in Spiel.
Die Kopplung $\vec F = \vec J + \vec I$ nimmt Werte von $| I - J|$ bis $J+I$ an. 
Mit einem angelegten Magnetfeld können diese Energieniveaus weiter gespalten werden, da das Magnetfeld auf die Hülle des Atoms wirkt, so dass sich $2F+1$ Unterniveaus ergeben. 
Dabei ist die Energieverschiebung 
\begin{equation*}
    \Delta E_\text{Zeeman} = g_\text{F} \mu_\text{B} B
\end{equation*}

Der neue Landé-Faktor $g_\text{F}$ ist 
\begin{equation*}
    g_\text{F} = g_\text{J} \frac{F(F+1) + J(J+1) - I(I+1)}{2F(F+1)}.
\end{equation*}

\subsection{Optisches Pumpen}

Um die einzelnen Energieniveaus zu messen, wird die Methode des optischen Pumpens genutzt. Dabei werden die Elektronen aus einem der zwei niedrigen Niveaus in ein höheres Niveau angeregt. 
Anschließend wird dieses Niveau wieder geleert mittels (spontaner und induzierter) Emission und in nur in einen der beiden niedrigen Niveaus gefüllt. 

Es gibt $\sigma^+$- (rechtszirkulares Licht), $\sigma^-$- (linkszirkulares Licht) und $\pi$-Übergänge (linear polarisiertes Licht), die jeweils bestimmte Magnetquantenzahlen $\Delta m_\text{J}$ mit sich bringen (+1, -1, 0).


Dadurch erhöht sich die Transparenz. 

Für bestimmte Resonanzen des Magnetfelds, also genau dann, wenn 
\begin{equation*}
    \Delta E_\text{Zeeman} = h \nu = g_\text{F} \mu_\text{B} B \Delta m_\text{J}
\end{equation*}
ist, sinkt die Transparenz wieder.

Induzierte Transparenz ist ein Prozess, bei dem ein Photon eingestrahlt wird, welches genau die Energie, der Differenz zwischen den beiden Niveaus besitzt. Dadurch treten dann zwei Photonen aus, die sich in Energie, Ausbreitungsrichtung und Polarisation nicht unterscheiden. 

Die induzierte Emission ist hier der dominante Prozess aufgrund der niedrigen Frequenz der Zeeman-Aufspaltung. 









