%\section{Ziel}

\section{Theorie}
\label{sec:Theorie}

\subsection{Horizontalfeldkomponente, Landé-Faktoren und Kernspins}

\subsection{Quadratischer Zeeman-Effekt}

\subsection{Vorbereitungsfragen}
Das wird später noch geändert:

%1
%Zur welchen Klasse der Materialien zählt das Rubidium?
Rubidium zählt zu den Alkalimetallen. Es befindet sich in der $\num{1}$. Hauptgruppe. 

%Welchen Wert haben die Quantenzahlen: Bahndrehimpuls (L), Spin (S), Gesamtdrehimpuls der Elektronenschale (J), und Kernspin (I) eines 85Rb-Atoms im Grundzustand?
Die Quantenzahlen eines $^{85}$Rb-Atoms in Grundzustand sind
\begin{align*}
    Bahndrehimpuls \, L = \num{0}, \\
    Spin \, S = \frac{1}{2}, \\
    Gesamtdrehimpuls \, J = \frac{1}{2}, \\
    Kernspin \, I = \frac{5}{2}. %stimmen die Zahlen?
\end{align*}

%Welche Werte ergeben sich für Gesamtdrehimpuls des Atoms (Quantenzahl F) im Grund- und in dem ersten angeregten Zustand (Kapitel 2 in Ref. [1])?
Der Gesamtdrehimpuls des Atoms im Grundzustand ($^{85}$Rb) ist
\begin{equation*}
    F = \num{3},
\end{equation*}
im erstem angeregten Zustand ($^{87}$Rb) beträgt der Gesamtdrehimpuls des Atoms %ist 87Rb der erste angeregte Zustand?
\begin{equation*}
    F = \num{2}.
\end{equation*} %stimmen die Zahlen?

%2
%Welche Aufspaltungen entstehen durch das äußere Magnetfeld bei den 2S1/2 und 2P1/2 Niveaus?


%Wie sieht ein Niveauschema aus?

%Wovon hängt die Zeeman-Aufspaltung ( ΔEZ ) zwischen einzelnen Niveaus ab (Kapitel 2.5.7 in Ref. [1])?
Die Zeeman-Aufspaltung $\Delta$E ist abhängig vom Landé-Faktor g, der Magnetfeldstärke B und xy:
\begin{equation}
    \Delta E = \mu_\text{B} \cdot g \cdot B.
\end{equation}


%3
%Welche Auswahlregeln gibt es für einen elektrischen Dipolübergang zwischen den beiden 2S1/2 und 2P1/2 Niveaus?

%Was bewirkt die Anregung dieser Zustände mit einem zirkular polarisierten Licht bei einem angelegten Magnetfeld?

%Woran liegt der Unterschied zwischen stimulierten und spontanen Emission?

%Wie hängt die spontane Emission von der Frequenz ab (Kapitel 1.7.3 in Ref. [1])?


%4
%Welcher Zustand stellt sich bei der kontinuierlichen Beleuchtung mit zirkular polarisierten Licht ein? 

%Was bedeutet Optisches Pumpen?

%Wie sieht dabei der zeitliche Verlauf der Intensität des transmittierten Lichtes [2,3]?


%5
%Welche Auswahlregeln gibt es für magnetische Dipolübergänge?

%Was passiert in dem optisch gepumpten System, wenn durch die RF-Feld-Frequenz (Hochfrequenz) f gegebene Energie ( ERF=hf ) der Zeeman Energie ( ΔEZ ) entspricht? 

%Was geschieht dabei mit dem Absorptionsverhalten? 

%Überlegen Sie wie eine Absorptionspeak-vs.-Magnetfeld Abhängigkeit aussehen soll.

%Wie ermittelt man dadurch den Lande-Faktor gF?

%Was passiert mit der transmittierten Lichtintensität bei null Magnetfeld?


%6
%Was passiert wenn man die ERF gleich der ΔEZ wählt und dabei die RF- Feld-Amplitude variiert? 

%Welche Beziehung erwarten Sie dort zwischen der Frequenz der Rabi-Oszillationen und der Stärke des angelegten RF-Feldes [2,3]?