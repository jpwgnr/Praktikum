\section{Auswertung}
\label{sec:Auswertung}

\subsection{Horizontalfeldkomponente, Landé-Faktoren und Kernspins}

Nachdem der Tisch so ausgerichtet wurde, dass nur die Horizontalkomponente das Magnetfelds\dots 

Somit ergibt sich für das Erdmagnetfeld ein Wert von 
\begin{equation*}
    B_\text{Erdmagnetfeld} = \SI{34.94}{\micro\tesla}.
\end{equation*}

\subsection{Landé-Faktoren und Kernspins der Isotope} 

Die gemessenen Werte sind in Tabelle \ref{tab1}
aufgelistet. 
Diese Werte werden mit Formel \eqref{eq:xy} in die jeweilige Magnetfeldstärke umgerechnet. 
In Abb. \ref{fig:g} werden die Magnetfeldstärken gegen die Frequenzen aufgetragen und eine Ausgleichsgerade wird durch die Messwerte gelegt. 




\subsection{Isotopenverhältnis}
%aus den beobachteten Amplituden bei 100kHz

%Wert in der Natur: