\section{Diskussion}
\label{sec:Diskussion}

Die gemessene Horizontalkompente des Erdmagnetfelds weicht um \SI{74.7}{\percent} vom Literaturwert des Erdmagnetfelds in Europa ab.
Der Wert liegt zwar in der selben Größenordnung, aber das Feld wurde somit anscheinend überkompensiert. 
Dies führt zu einem Systematischen Fehler in der Messung. Eine andere Quelle für systematische Fehler ist die Decke, die genutzt wurde zur Abdeckung der Apparatur.

Die theoretischen Werte des Kernspins von $^{87}$Rb sind \num{1.5} und für $^{85}$Rb \num{2.5}.
Der experimentelle Wert für den ersten Peak ist \num{1.553(23)} und für den zweiten Peak \num{2.552(26)}. Somit lässt die geringe Abweichung vermuten, dass der erste Peak zu $^{87}$Rb und der zweite Peak zu $^{85}$Rb gehört. 
Die Abweichungen sind können durch die oben zuvor genannten systematischen Fehler begründet werden. 
Das Verhältnis zwischen den beiden Isotopen wurde anhand der Peaks in dem Foto bestimmt. 
Es beträgt \num{2.037} während das Verhältnis in der Natur \num{2.593} beträgt. 


