\section{Diskussion}
\label{sec:Diskussion}

Die Messung des Widerstands bei der Wheatstone Brückenschaltung liefert bei dem Widerstand Nummer \num{<++>}<++> 
einen Wert von \SI{<++>}{\<++>}<++> mit einem relativen Fehler von \SI{<++>}{\<++>}<++> und bei dem Widerstand Nummer 
\num{<++>}<++> liegt der Wert bei \SI{<++>}{\<++>}<++> und der relative Fehler bei \SI{<++>}{\<++>}<++>. 

\noindent Für den Widerstand und die Kapazität des Bauteils Nummer \num{<++>}<++> ergibt sich ein Wert von 
\SI{<++>}{\<++>}<++> und ein Wert von \SI{<++>}{\<++>}<++>. Der relative Fehler des Widerstands liegt somit bei einem 
Wert von \SI{<++>}{\<++>}<++>. Der relative Fehler des Kondensators bei \SI{<++>}{\<++>}<++>. 

\noindent Die Induktivität der Spule Nummer \num{<++>}<++> wurde auf einen Wert von \SI{<++>}{\<++>}<++> bestimmt. 
Der relative Fehler liegt dabei bei \SI{<++>}{\<++>}<++>. Der Widerstand hatte einen Wert von \SI{<++>}{\<++>}<++> 
und sein relativer Fehler wurde als \SI{<++>}{\<++>}<++> bestimmt. 
Die Spule wurde mit der Maxwell-Brücke erneut gemessen. Dabei kam für die Spule ein Wert von \SI{<++>}{\<++>}<++> 
und für den relativen Fehler ein Wert von \SI{<++>}{\<++>}<++> heraus. Die Werte des Widerstands lagen bei 
\SI{<++>}{\<++>}<++> und \SI{<++>}{\<++>}<++>. Die relative Abweichung der beiden Messungen liegen für den Widerstand 
bei \SI{<++>}{\<++>}<++>. Bei der Induktivität der Spulen beträgt die Abweichung \SI{<++>}{\<++>}<++>. 

\noindent Die Messung der Frequenzabhängigkeit der Brückenspannung der Wien-Robinson-Brücke lässt einen Wert für 
$\omega_0$ bei \SI{<++>}{\<++>}<++> erkennen. Der theoretische Wert liegt bei \SI{<++>}{\<++>}. Die Abweichung dieser 
beiden Werte liegt bei \SI{<++>}{\<++>}<++>. Der Klirrfaktor liegt bei \SI{<++>}{\<++>}<++>. 
