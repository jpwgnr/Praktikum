\section{Durchführung}
\label{sec:Durchführung}

Die Messungen werden alle mehrfach ausgeführt, um ein Maß für die Zufallsfehler zu bekommen. Die Toleranz der Referenzbauteile liegt bei $\pm \SI{0.2}{\percent}$.

Mit der Wheatoneschen Brückenschaltung werden im ersten Aufgabenteil zwei unbekannte Widerstände ausgemessen. 

Als nächstes soll anhand einer Kapazitätsmessbrücke die Kapazität von zwei verschiedenen Kondensatoren gemessen werden. 

Die Induktivität und der Verlustwiderstand einer unbekannten Spule werden bei einer Induktivitätsmessbrücke gemessen. 

Anschließend wird die Spule ein zweites Mal mit Hilfe der Maxwell-Brücke durchgemessen. 

Die Frequenzabhängigkeit der Brückenspannung in einer Wien-Robinson-Brücke und einer TT-Brücke werden miteinander und mit dem Theoriewert verglichen. 

Als letztes wird eine Klirrfaktor-Messung durchgeführt. Dabei wird das Minimum der Spannung bestimmt und der Quotient aus der vorhandenen Oberwellen und der Grundwelle bestimmt. 
