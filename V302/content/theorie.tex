\section{Theorie}
\label{sec:Theorie}

Die Spannung $U$, also die Potentialdiffferenz zwischen zwei Punkten, wird bei Brückenschaltungen in Abhängigkeit von ihren Widerstandsverhältnissen untersucht.
Bei einer allgemeinen Brückenschaltung (s. Abb 1), bezeichnet man die Spannung, die zwischen Punkt A und B auftritt als Brückenspannung. $U_S$ wird dabei als Speisespannung bezeichnet. 
Wichtig sind dabei die Kirchhoffschen Gesetze. Das erste lautet, dass die Summe der zufließenden Ströme gleich der Summe der abfließenden Ströme sein muss. 
Das zweite Gesetzt heißt Maschenregel und besagt, dass innerhalb einer Masche, also einem in sich geschlossenen Stromkreis die Summe der elektromotorischen Kräfte, also zum Beispiel die Spannung der Stromquelle, gleich der Summe aus der Stromstärke und den Widerständen der Bauteile sein muss. 

Aus diesen Gesetzmäßigkeiten lässt sich folgern, dass die Brückenspannung $U$ und die Speisespannung $U_S$ in Abb.1 in folgendem Verhältnis zu einander stehen:

\begin{equation}
    U = \frac{R_2R_3 - R_1R_4}{(R_3 + R_4)(R_1 + R_2)}U_S 
\end{equation}.

Wenn gilt, $R_1R_4 = R_2R_3$ nennt man diesen Fall abgeglichene Brücke. Da die Abgleichbedingung nur vom Verhältnis der Widerstände abhängt, ist mit der Brückenschaltung eine Widerstandsmessung durchführbar. 

Kennt man Beispielsweise $R_1$ nicht, so variiert man einen der drei anderen Widerstände solang bis die Brückenspannung verschwindet und kann anschließend über die Abgleichbedingung den Widerstand bestimmen. 

Mit Kapazitiv- und Induktivwiderständen ist es sinnvoll komplexe Widerstandsoperatoren zu benutzen. 
Eine Brückenschaltung mit vier komplexen Widerständen hat die gleiche Abgleichbedingung wie eine reele Brückenschaltung. Der einzige Unterschied ist, dass sich daraus zwei Bedingungen ergeben. Der Realteil $X$ und der Imaginärteil $Y$ müssen dabei von einander separiert betrachtet werden. 

\begin{equation}
    X_1X_4 - Y_1Y_4 = X_2X_3 - Y_2Y_3 
\end{equation}

ist die erste Bedingung und die zweite lautet: 
\begin{equation}
    X_1Y_4 + X_4Y_1 = X_2Y_3 + X_3Y_2 
\end{equation}.

\subsection{Wheatstonesche Brücke}

Diese Brücke enthält ohmsche Widerstände (s. Abb xy). Sie kann, wie oben beschrieben, zur Bestimmung eines unbekannten Widerstands $R_X$ benutzt werden:
\begin{equation}
    R_X = R_2 \frac{R_3}{R_4}
\end{equation}.

\subsection{Kapazitätsmessbrücke}

Für die Berechnung muss die Eigenschaft eines realen Kondensators berücksichtigt werden, dass dieser zum Teil Energie in Wärme umwandelt. Dafür wird ein Ersatzschaltbild betrachtet, bei dem man einen fiktiven ohmschen Widerstand mit dem Kondensator in Reihe schaltet. Für die Messung der Kapazität eines unbekannten Kondensators $C_X$ gilt somit unter Berücksichtigung der Abgleichbedingungen, für den ohmschen Widerstand 
\begin{equation}
    R_X = R_2 \frac{R_3}{R_4}
\end{equation}
und für den Kondensator 
\begin{equation}
    C_X = C_2 \frac{R_4}{R_3}
\end{equation}. 

\subsection{Induktivitätamessbrücke}

Analog zur Kapazitätsmessbrücke verliert auch ein induktives Bauteil Energie indem sie irreversibel in Wärme umgewandelt wird. Diese Verluste und die Phasenverschiebung werden erneut durch einen fiktiven Widerstand kompensiert. Die Formeln ergeben sich auf dieselbe Weise zu 
\begin{equation}
    R_X = R_2 \frac{R_3}{R_4}
\end{equation}
und für die Spule $L_X$ zu 
\begin{equation}
    L_X = L_2 \frac{R_3}{R_4}
\end{equation}.

\subsection{Maxwell-Brücke}

Da die Induktivitätsmessbrücke auf Grund der zweiten Spule ähnliche starke Verluste hat, insbesondere bei niedrigen Frequenzen, benutzt man eine andere Schaltung, die anstelle der Normalinduktivität $L_2$ eine Normalkapazität benutzt. Der Kondensator sollte eine möglichst verlustarmen Kapazität $C_4$ besitzen. 
Aus den Abgleichbedingungen folgen folgende Ausdrücke:

\begin{equation}
     R_X = \frac{R_2R_3}{R_4}
\end{equation}
und 
\begin{equation}
    L_X = R_2R_3C_4
\end{equation}.

\subsection{Frequenzabhängige Brückenschaltungen}

Bei den vorherigen Schaltungen war die Frequenz nicht wichtig. Es gibt aber einen Frequenzbereich, in dem der Abgleich unter optimalen Bedingungen durchführbar ist.

\subsubsection{Wien-Robinson-Brücke}

Nach Umformung der Formeln, die sich aus der Abb. <++> ergeben, erkennt man, dass 
\begin{equation}
    \omega_0 = \frac{1}{RC}
\end{equation} ergibt und man mit $\Omega = \frac{\omega}{\omega_0}$ 

auf ein Verhältnis zwischen den Spannungen $U_B$ und $U_S$ kommt. 
Es gilt:
\begin{equation}
    (\frac{U_B}{U_S})^2 = \frac{1}{9} \frac{(\Omega^2 -1)^2}{(1- \Omega)^2 + 9 \Omega^2}
\end{equation}.

Bei der Bestimmung des Klirrfaktors wird der Anteil der Oberwellen im Verhältnis zur Grundwelle gemessen. 
Die Kleinheit des Faktors legt ein Maß für die Qualität eines Spannungsgenerators fest. 

\subsubsection{TT-Brücke}

Die Funktion ist, genau wie bei der Wien-Robinson-Brücke, die eines elektronischen Filters. 
Der Vorteil dieser Schaltung ist, dass beide Spannungen gegen Masse angeschlossen werden. 
Aus den Kirchhoffschen Gesetzen ergibt sich wieder, dass 
\begin{equation}
    \omega_0 = \frac{1}{RC}
\end{equation}
gilt und dass $U_B$ für $\omega_0$ für diese Frequenz verschwindet, aber sonst von Null verschieden ist. 
Für den Betrag des Spannungsverhältnisses gilt hierbei 

\begin{equation}
    (\frac{U_B}{U_S})^2 = \frac{(\Omega^2 -1)^2}{(1- \Omega^2)^2 + 16 \Omega^2}
\end{equation}.



