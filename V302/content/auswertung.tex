\section{Auswertung}
\label{sec:Auswertung}

\subsection{Bestimmung von Widerständen mittels Wheatstone-Brücke}
Die verschiedenen Werte für die Widerstände $R_3$, $R_4$ und $R_2$, die zur Berechnung des Widerstandes $R_X$
nötig sind, befinden sich in Tabelle \ref{taba}. Dabei beziehen sich die ersten drei Zeilen auf den Widerstand
$R_{X1}$ und die letzten drei auf den Widerstand $R_{X2}$. Für jeden Widerstand $R_X$ werden drei verschiedene
Widerstände $R_2$ verwendet.
\begin{table}\caption{Die Anzahl der Impulse, der Startwert auf der Mikrometerschraube und der Endwert auf der Mikrometerschraube.}
\label{taba}
\centering
\sisetup{round-mode = places, round-precision=2, round-integer-to-decimal=true}
\begin{tabular}{S[]S[]S[]} 
\toprule
{Anzahl} & {$d_\text{Start} / \si{\milli\meter}$} & {$d_\text{Start} / \si{\milli\meter}$}\\
\midrule
3001.0 & 6.73 & 2.0\\
3002.0 & 6.73 & 2.0\\
3000.0 & 1.82 & 6.5\\
3000.0 & 6.74 & 2.0\\
3000.0 & 1.83 & 6.5\\
3000.0 & 6.74 & 2.0\\
3001.0 & 1.84 & 6.5\\
3000.0 & 2.83 & 7.5\\
3001.0 & 7.77 & 3.0\\
3002.0 & 2.75 & 7.5\\
\bottomrule
\end{tabular}\end{table}
\noindent Die gesuchten Widerstände lassen sich daraus mit Gleichung \eqref{eqn:a_r} berechnen.
Für den ersten Widerstand ergibt sich
\begin{equation*}
    R_{X1} = \SI{319.8 \pm 1.2}{\ohm}.
\end{equation*}
Der zweite Widerstand berechnet sich zu
\begin{equation*}
    R_{X2} = \SI{899 \pm 9}{\ohm}.
\end{equation*}

\subsection{Bestimmung von Kapazitäten mittels Kapazitätsmessbrücke}
Die Werte zur Berechnung der Kapazität und des Verlustwiderstandes sind in Tabelle \ref{tabb} aufgelistet.
Für alle drei Kapazitäten $C_2$ liegen jeweils drei Messreihen vor.
\begin{table}\caption{Die Frequenzen der Sägezahnspannung.}
\label{tabb}
\centering
\sisetup{round-mode = places, round-precision=2, round-integer-to-decimal=true}
\begin{tabular}{S[]S[]} 
\toprule
{Index} & {$\nu_\text{Sä} / \si{\hertz}$}\\
\midrule
1.0 & 25.02\\
2.0 & 49.95\\
3.0 & 99.99\\
4.0 & 149.97\\
\bottomrule
\end{tabular}\end{table}
\noindent Der Verlustwiderstand lässt sich mittels Gleichung \eqref{eqn:b_r} zu
\begin{equation*}
    R_X = \SI{510 \pm 40}{\ohm}
\end{equation*}
berechnen.
Die Kapazität ergibt sich mit Gleichung \eqref{eqn:b_c} zu
\begin{equation*}
    C_X = \SI{293 \pm 26}{\nano\farad}.
\end{equation*}

\subsection{Bestimmung von Induktivitäten mittels Induktivitätsmessbrücke}
Für die Berechnung des Verlustwiderstandes sowie der Induktivität einer Spule werden zwei verschiedene
Induktivitäten $L_2$ verwendet. Die Messdaten befinden sich in Tabelle \ref{tabc}.
\begin{table}\caption{Der magnetische Fluss $B$ des gemessenen Magnetfelds gegen den Strom $I$ des erzeugenden Magnetfelds, Neukurve.}
\label{tabc}
\centering
\sisetup{round-mode = places, round-precision=1, round-integer-to-decimal=true}
\begin{tabular}{S[]S[]} 
\toprule
{$B$/ \si{\milli\tesla}} & {$I$/ \si{\ampere}}\\
\midrule
0.0 & 0.0\\
111.19999999999999 & 1.0\\
273.5 & 2.0\\
397.8 & 3.0\\
479.9 & 4.0\\
537.9000000000001 & 5.0\\
585.0999999999999 & 6.0\\
621.8000000000001 & 7.0\\
653.1 & 8.0\\
679.9 & 9.0\\
704.3000000000001 & 10.0\\
\bottomrule
\end{tabular}\end{table}
\noindent Der Verlustwiderstand wird mit Gleichung \eqref{eqn:c_r} berechnet:
\begin{equation*}
    R_X = \SI{100 \pm 4}{\ohm}.
\end{equation*}
Für die Induktivität ergibt sich mit Gleichung \eqref{eqn:c_l}
\begin{equation*}
    L_X = \SI{26.9 \pm 0.8}{\milli\henry}.
\end{equation*}

\subsection{Bestimmung von Induktivitäten mittels Maxwell-Brücke}
Für die erneute Berechnung der Induktivität und des Verlustwiderstandes mit der Maxwell-Brücke werden der Widerstand
$R_2 = \SI{1.0}{\kilo\ohm}$ und die Kapazität $C_4 = \SI{992}{\nano\farad}$ verwendet.
Die eingestellten Widerstände $R_3$ und $R_4$ befinden sich in Tabelle \ref{tabd}.
\begin{table}\caption{Kreisfrequenz $\omega$ gegen die Phasenverschiebung $\varphi$ der Kondensatorspannung $U_C$ und der Generatorspannungi $U_0$.}
\label{tabd}
\centering
\sisetup{round-mode = places, round-precision=2, round-integer-to-decimal=true}
\begin{tabular}{S[]S[]} 
\toprule
{$\omega\cdot 10^{5}$ /\si[per-mode=fraction]{\per\second}} & {$Phase \varphi$}\\
\midrule
0.5654866776461628 & 0.12440706908215582\\
0.6911503837897545 & 0.11058406140636072\\
0.8168140899333463 & 0.13069025438933538\\
0.9424777960769379 & 0.1696460032938488\\
1.0681415022205296 & 0.16022122533307945\\
1.1938052083641213 & 0.20294688542190062\\
1.319468914507713 & 0.23750440461138836\\
1.4451326206513049 & 0.26012387171723483\\
1.5707963267948966 & 0.34557519189487723\\
1.6964600329384882 & 0.4750088092227767\\
1.8221237390820801 & 0.546637121724624\\
1.8849555921538759 & 0.6785840131753952\\
1.9477874452256716 & 0.818070726994782\\
2.0106192982974673 & 0.9650972631827843\\
2.0734511513692637 & 1.1611326447667876\\
2.1362830044410592 & 1.4099467829310992\\
2.199114857512855 & 1.6713272917097701\\
2.261946710584651 & 1.9000352368911066\\
2.324778563656447 & 2.092300707290802\\
2.3876104167282426 & 2.244353791724548\\
2.450442269800039 & 2.4014334244040376\\
2.5761059759436304 & 2.5761059759436304\\
2.701769682087222 & 2.6477342884454775\\
2.827433388230814 & 2.770884720466197\\
2.9530970943744057 & 2.894035152486917\\
3.078760800517997 & 2.8940351524869175\\
3.204424506661589 & 2.948070546128662\\
3.330088212805181 & 2.9970793915246623\\
3.4557519189487724 & 2.9719466502959446\\
3.581415625092364 & 3.008389125077586\\
3.7070793312359562 & 3.0398050516134836\\
\bottomrule
\end{tabular}\end{table}
\noindent Der Verlustwiderstand, welcher mit Gleichung \eqref{eqn:d_r} berechnet wird, ergibt sich zu
\begin{equation*}
    R_X = \SI{98.9 \pm 1.8}{\ohm}.
\end{equation*}
Die mit Gleichung \eqref{eqn:d_l} errechnete Induktivität ist
\begin{equation*}
    L_X = \SI{26.1 \pm 0.5}{\milli\henry}.
\end{equation*}

\subsection{Bestimmung der Frequenzabhängigkeit der Brückenspannung mittels Wien-Robinson-Brücke}
Die doppelte Brückenspannung in Abhängigkeit von der Frequenz ist in Tabelle \ref{tabe} dargestellt.
Die Werte $\omega / \omega_0$ und $U_{Br} / U_S$ befinden sich in Tabelle \ref{tab1}, wobei die Spannung $U_S = \SI{2.5}{\volt}$ ist.
$\omega_0$ ist dabei die Frequenz, bei der das Brückenspannungsminimum liegt:
\begin{equation*}
    \omega_{0,exp} = \SI[per-mode=fraction]{2513.27}{\per\second}.
\end{equation*}
 Der mittels Gleichung \eqref{omega0} theoretisch errechnete Wert für $\omega_0$ liegt bei 
 \begin{equation*}
     \omega_{0,theo} = \SI[per-mode=fraction]{2380.95}{\per\second}.
 \end{equation*}
\begin{table}\caption{Die Spannung und die Stromstärke bei einer Heizspannung von $\SI{4.1}{\volt}$ und die Heizspannung $\SI{2.4}{\ampere}$.}
\label{tabe}
\centering
\sisetup{round-mode = places, round-precision=1, round-integer-to-decimal=true}
\begin{tabular}{S[]S[]} 
\toprule
{$U / \si{\volt}$} & {$I / \si{\micro\ampere}$}\\
\midrule
5.0 & 15.0\\
10.0 & 36.0\\
15.0 & 58.0\\
20.0 & 79.0\\
25.0 & 100.0\\
30.0 & 118.0\\
35.0 & 132.0\\
40.0 & 142.0\\
45.0 & 149.0\\
50.0 & 154.0\\
55.0 & 158.0\\
60.0 & 161.0\\
65.0 & 162.0\\
70.0 & 164.0\\
75.0 & 166.0\\
80.0 & 167.0\\
85.0 & 168.0\\
90.0 & 169.0\\
95.0 & 170.0\\
100.0 & 171.0\\
105.0 & 172.0\\
115.0 & 173.0\\
120.0 & 174.0\\
130.0 & 175.0\\
145.0 & 176.0\\
150.0 & 177.0\\
165.0 & 178.0\\
180.0 & 179.0\\
195.0 & 180.0\\
210.0 & 181.0\\
\bottomrule
\end{tabular}\end{table}
\begin{table}\caption{Erste Messung.}
\label{tab1}
\centering
\sisetup{round-mode = places, round-precision=1, round-integer-to-decimal=true}
\begin{tabular}{S[]S[]S[]} 
\toprule
{$g / \si{\centi\meter}$} & {$b / \si{\centi\meter}$} & {$B / \si{\centi\meter}$}\\
\midrule
12.700000000000003 & 38.9 & 8.4\\
13.700000000000003 & 32.2 & 6.5\\
14.700000000000003 & 28.200000000000003 & 5.3\\
15.700000000000003 & 25.299999999999997 & 4.4\\
16.700000000000003 & 22.5 & 3.7\\
17.700000000000003 & 21.299999999999997 & 3.3\\
18.700000000000003 & 19.799999999999997 & 3.0\\
19.700000000000003 & 18.5 & 2.6\\
20.700000000000003 & 17.799999999999997 & 2.4\\
21.700000000000003 & 17.400000000000006 & 2.2\\
\bottomrule
\end{tabular}\end{table}
\noindent Die Werte aus Tabelle \ref{tab1} sind in Abbildung \ref{fig:plot} gegeneinander aufgetragen.
\begin{figure}
 \centering
 \includegraphics[width= 13cm, height= 9cm]{build/plot1.pdf}
 \caption{$U_{Br}/U_S$ ist gegen $\omega / \omega_0$ aufgetragen. Es sind die Daten, ein Fit und die
 Theoriekurve eingezeichnet.}
 \label{fig:plot}
\end{figure}

\subsection{Bestimmung des Klirrfaktors}
Der minimale Wert der Brückenspannung liegt bei $U_{Br,Min} = \SI{30.4}{\milli\volt}$ bei einer Frequenz von 
$\omega_0 = 2 \pi \cdot 400$. 
\newline
Der Klirrfaktor, der mit Gleichung \eqref{eqn:k} bestimmt werden kann, ergibt sich zu
\begin{equation*}
    k = \num{0.148}. %Wie viele Nachkommastellen?
\end{equation*}
