\section{Auswertung}
\label{sec:Auswertung}

\subsection{Bestimmung von Widerständen mittels Wheatstone-Brücke}
Die verschiedenen Werte für die Widerstände $R_3$, $R_4$ und $R_2$, die zur Berechnung des Widerstandes $R_X$
nötig sind, befinden sich in Tabelle \ref{taba}. Dabei beziehen sich die ersten drei Zeilen auf den Widerstand
$R_{X1}$ und die letzten drei auf den Widerstand $R_{X2}$.
\begin{table}\caption{Die Anzahl der Impulse, der Startwert auf der Mikrometerschraube und der Endwert auf der Mikrometerschraube.}
\label{taba}
\centering
\sisetup{round-mode = places, round-precision=2, round-integer-to-decimal=true}
\begin{tabular}{S[]S[]S[]} 
\toprule
{Anzahl} & {$d_\text{Start} / \si{\milli\meter}$} & {$d_\text{Start} / \si{\milli\meter}$}\\
\midrule
3001.0 & 6.73 & 2.0\\
3002.0 & 6.73 & 2.0\\
3000.0 & 1.82 & 6.5\\
3000.0 & 6.74 & 2.0\\
3000.0 & 1.83 & 6.5\\
3000.0 & 6.74 & 2.0\\
3001.0 & 1.84 & 6.5\\
3000.0 & 2.83 & 7.5\\
3001.0 & 7.77 & 3.0\\
3002.0 & 2.75 & 7.5\\
\bottomrule
\end{tabular}\end{table}
Die gesuchten Widerstände lassen sich daraus mit Gleichung \eqref{eqn:a_r} berechnen.
Für den ersten Widerstand ergibt sich
\begin{equation*}

\end{equation*}

\subsection{Bestimmung von Kapazitäten mittels Kapazitätsmessbrücke}

\subsection{Bestimmung von Induktivitäten mittels Induktivitätsmessbrücke}

\subsection{Bestimmung von Induktivitäten mittels Maxwell-Brücke}

\subsection{Bestimmung der Frequenzabhängigkeit der Brückenspannung mittels Wien-Robinson-Brücke}

\begin{figure}
 \centering
 \includegraphics{plot1.pdf}
 \caption{Plot}
 \label{fig:plot}
\end{figure}

\subsection{Bestimmung des Klirrfaktors}