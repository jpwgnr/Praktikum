\section{Durchführung und Fragen}

\noindent Anfangs wurde gefragt, welchen Versuch wir gern besprechen würden. Wir haben uns dann für den Versuch \enquote{Geometrische Optik} entschieden, was bei den Prüfern positiv anzukommen schien, da beide meinten, dass das endlich mal was anderes sei. 
Also begannen wir zu erklären, was wir in dem Versuch genau gemacht hatten und was dabei alles zu beachten war. Wir erklärten, dass wir im ersten Versuchsteil die Brennweite bestimmen wollten und sowohl das Abbildungsgesetz als auch die Linsengleichung verifizieren wollten.
Dazu zeichneten wir dann auch den Aufbau und zeigten, wie der Plot -$g$ gegen $b$ aufgetragen- aussieht. 

\noindent An dieser Stelle wurden wir dann aber auch schon das erste mal unterbrochen. Frau Siegmann hinterfragte mit einigen einfachen Fragen, was die einzelnen Linien in dem Aufbau bedeuteten und ließ sich ein wenig zum Thema Brechung erzählen. 
In dem Zusammenhang erklärten wir dann auch Brechungsindizes und erwähnten die Fresnelschen Gesetze. 
Das war aber nicht genau das was sie hören wollte. Sie fragte nochmal genauer: \textbf{\enquote{Warum kann man annehmen, dass die Strahlen wirklich so gebrochen werden, wie Sie es hier gezeichnet haben?}} 
Worauf sie hinaus wollte, war, dass wir hier eine dünne Linse betrachteten.

\noindent \textbf{\enquote{Wie wäre das denn, wenn Sie jetzt statt Luft ein Medium mit einem höheren Brechungsindex als den der Linse außen herum hätten?}}
Wir mussten erst überlegen und skizzierten, wie die Brechung bei Luft aussieht und wie es bei einem dichteren Medium aussähe. 
Durch die Skizze und ein wenig Hilfe von Herrn Kröninger und Frau Siegmann kamen wir dann darauf, dass sich eine Sammellinse in so einem Fall wie eine Zerstreuungslinse verhalten würde.

\noindent Herr Kröninger wollte nun zum nächsten Versuch überleiten. \textbf{\enquote{Wie sieht es denn aus mit dem Doppelspaltversuch?}} 
Wir erklärten erstmal grob, wie der Aufbau aussah, wie die Beugungsbilder für Einzel- und Doppelspalt aussehen und welche Lichtquelle wir hatten. 
\textbf{\enquote{Okay. Welche Eigenschaft hat das Licht denn?}}
Wir standen an dieser Stelle sehr auf dem Schlauch. Wir erklärten, dass die Wellenlängen und die Phase übereinstimmen müssen und erzählten noch ein wenig über Amplitude etc. Worauf er hinaus wollte, war einfach der Begriff Kohärenz. 
Manchmal denkt man einfach zu kompliziert.

\noindent \textbf{\enquote{Na gut, dann ist das genug an dieser Stelle. Erzählen Sie uns doch noch etwas zu dem Versuch Spulen und Magnetfelder.}} 
Also erklärten wir erst einmal wieder den allgemeinen Aufbau, erzählten was wir so gemessen hatten und skizzierten unsere Ergebnisse. 
Vor allem bei der Hysterese-Kurve wollten die beiden dann alles ganz genau erklärt bekommen. Bei der Remanenz hinterfragte Herr Kröninger, 
was die Ursache dafür ist, dasss ein B-Feld bestehen bleibt, obwohl das äußere H-Feld nicht mehr vorhanden ist. Wir mussten das aber nicht wirklich 
genau erklären. Was er hören wollte, waren einfach die Begriffe \enquote{Spin} und \enquote{Weiß'sche Bezirke}.

\noindent Anschließend wurden wir nach draußen geschickt.

