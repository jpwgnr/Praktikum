\section{Auswertung}
\label{sec:Auswertung}

Die gemessenen Temperaturen $T_1$ und $T_2$, sowie die Drücke
$p_a$ und $p_b$ und die Leistungsaufnahme des Kompressors
zu verschiedenen Zeiten $t$ sind in Tabelle \ref{tab1}
dargestellt.
\begin{table}\caption{Erste Messung.}
\label{tab1}
\centering
\sisetup{round-mode = places, round-precision=1, round-integer-to-decimal=true}
\begin{tabular}{S[]S[]S[]} 
\toprule
{$g / \si{\centi\meter}$} & {$b / \si{\centi\meter}$} & {$B / \si{\centi\meter}$}\\
\midrule
12.700000000000003 & 38.9 & 8.4\\
13.700000000000003 & 32.2 & 6.5\\
14.700000000000003 & 28.200000000000003 & 5.3\\
15.700000000000003 & 25.299999999999997 & 4.4\\
16.700000000000003 & 22.5 & 3.7\\
17.700000000000003 & 21.299999999999997 & 3.3\\
18.700000000000003 & 19.799999999999997 & 3.0\\
19.700000000000003 & 18.5 & 2.6\\
20.700000000000003 & 17.799999999999997 & 2.4\\
21.700000000000003 & 17.400000000000006 & 2.2\\
\bottomrule
\end{tabular}\end{table}

\subsection{Temperaturverläufe}
%a & b) 
Die Temperaturverläufe der beiden Reservoire sind in Abbildung
\ref{fig:plot1} zu sehen.

\begin{figure} %Funktion in die Abbildung schreiben?
    \centering
    \includegraphics[width=14cm, height=10cm]{build/plot1.pdf}
    \caption{Temperaturverläufe. Es sind jeweils die Daten und ein Fit dargestellt.
    Die rote Kurve stellt die Temperatur in Reservoir x dar. Die grüne Kurve stellt 
    die Temperatur in Reservoir y dar. Dabei wird die Temperatur
    durch ein Polynom dritten Grades dargestellt: $T(t) = At^3+Bt^2+Ct+D$. 
    Die Fitparameter der Kurve der Temperatur im ersten Reservoir sind $A_1=\num{-1.20(17)e-8}$,
    $B_1=\num{1.83(30)e-5}$, $C_1=\num{1.95(14)e-2}$ und $D_1=\num{295.11 \pm 0.18}$.
    Die Fitparameter der Kurve der Temperatur im zweiten Reservoir sind $A_2=\num{2.61(30)e-8}$,
$B_2=\num{-3.87(52)e-5}$, $C_2=\num{-0.87(25)e-2}$ und $D_2=\num{295.99 \pm 0.32}$.} %hier ist irgendwas falsch. Der Fehler war, dass du beim ersten D1 SI statt num geschrieben hast. Dann hatte er keine Einheit und konnte das nicht. :)

    \label{fig:plot1}
\end{figure}

%c)
Die Differentialquotienten $\frac{dT}{dt}$ für vier verschiedene Temperaturen
sind im Folgenden zu sehen. Es werden die Temperaturen
\begin{align*}
    T_1 &= \SI{1}{\degreeCelsius} &= \SI{274.15}{\kelvin} \\ %celsius geht nicht, degreeCelsius auch nicht, keine Ahnung warum. Bei den oberen beiden ging es. Du hattest bei T3 und T4 num stehen :D
    T_2 &= \SI{5}{\degreeCelsius} &= \SI{278.15}{\kelvin}\\
    T_3 &= \SI{10}{\degreeCelsius} &= \SI{283.15}{\kelvin}\\
    T_4 &= \SI{15}{\degreeCelsius} &= \SI{288.15}{\kelvin}
\end{align*}
betrachtet.
%Das schon in die Theorie?:
Dabei ist \eqref{eqn:} %Gleichung einfügen (Polynom)
die Ableitung der Funktion der Temperatur $T(t)$.
Für $\frac{dT_1}{dt}$ folgt:
\begin{align*}
    dT_{1,1}(1) &= \num{0.0216 \pm 0.0015} \\
    dT_{1,2}(5) &= \num{0.0272 \pm 0.0023} \\
    dT_{1,3}(10) &= \num{0.028 \pm 0.004} \\
    dT_{1,4}(15) &= \num{0.023 \pm 0.007} %schöner angeben? Einheit? Einheit müsste K/Sekunde sein
\end{align*}
Für $\frac{dT_2}{dt}$ gilt:
\begin{align*}
    dT_{2,1}(1) &= \num{-0.0130 \pm 0.0026} \\ %0 oder 1 Grad? Lieber in Kelvin angeben? Wie meinst du das mit 0 oder 1 Grad? hab eigentlich alles in Kelvin berechnet, dachte ich.  
    dT_{2,2}(5) &= \num{-0.025 \pm 0.004} \\
    dT_{2,3}(10) &= \num{-0.027 \pm 0.007} \\
    dT_{2,4}(15) &= \num{-0.015 \pm 0.012}
\end{align*}

%d)
\subsection{Bestimmung der Güteziffern}
%c_w angeben
Die Wärmekapazität der Kupferschlange und des Eimers beträgt
\begin{equation*}
    m_\text{k} c_\text{k} = \SI{750}{\joule\per\kelvin}.
\end{equation*}
Die realen Güteziffern für die vier Temperaturen werden mittels
Gleichung \eqref{eqn:güteziffer} berechnet. %Stimmt das?
Die idealen Güteziffern werden mit Gleichung \eqref{eqn:ideal} %Stimmt das?
bestimmt.
Beide Größen sind in Tabelle \ref{tabsolution1} jeweils
gegenübergestellt.
\begin{table}
    \caption{Die Ergebnisse für die realen und idealen Gütewerte für die vier verschiedenen Temperaturwerte, berechnet mit Gleichung \ref{eqn:güteziffer} für die Werte und Gleichung \ref{eq:gütefehler} für die Fehler.}
\label{tabsolution1}
\centering
\begin{tabular}{S[table-format=1.2]  
        @{${} \pm{}$}
        S[table-format=1.2]
        @{$  $}
        S[table-format=5.1]
        @{${} \pm{}$}
    S[table-format=3.1]}
\toprule
   \multicolumn{2}{c}{$\nu_\text{real}$} &\multicolumn{2}{c}{$\nu_\text{ideal}$}\\
\midrule
    0.64 & 0.05 & 270.0 & 350.00\\
    0.67 & 0.06 & 25.8 & 3.10\\
    0.68 & 0.10 & 10.8 & 0.50\\
    0.54 & 0.16 & 7.5 & 0.23\\
\bottomrule
\end{tabular}\end{table}


%e)
\subsection{Bestimmung des Massendurchsatzes}
Das im Versuch verwendete Gas ist Dichlordifluormethan.
Die Verdampfungswärme $L$ des Gases wird durch die Dampfdruck-Kurve
in Abb. \ref{fig:plot2} bestimmt. %wie? In V203 gucken
Die Wertepaare des Drucks $p$ und der Temperatur $T$, die zur
Darstellung der Dampfdruck-Kurve nötig sind, befinden sich in
Tabelle \ref{tab2}. 
\begin{table}\caption{Die Spannung, die Stromstärke, die Anzahl der Impulse, die transportierte Ladungsmenge und die transporte Ladungsmenge in Einheiten der Elementarladung.}
\label{tab1}
\centering
\sisetup{round-mode = places, round-precision=2, round-integer-to-decimal=true}
\begin{tabular}{S[]S[] S[]@{${}\pm{}$}S[] S[]@{${}\pm{}$} S[] S[]@{${}\pm{}$} S[]} 
\toprule
{U / \si{\volt}} & {I / \si{\ampere}} & \multicolumn{2}{c}{N/second} &  \multicolumn{2}{c}{$\Delta Q / \si{\coulomb}$} &  \multicolumn{2}{c}{$\Delta Q \si{\elementarycharge}$}\\
\midrule
320.0 & 0.1     & 86.91 & 0.07 &  8.975  &  0.007  & 5.602   &  0.005e+19\\
400.0 & 0.2     & 90.92 & 0.07 & 17.157  &  0.014  & 1.0709  &  0.0009e+20\\
480.0 & 0.3     & 93.35 & 0.07 & 25.068  &  0.020  & 1.5646  &  0.0012e+20\\
540.0 & 0.35    & 94.62 & 0.07 & 28.851  &  0.023  & 1.8008  &  0.0014e+20\\
560.0 & 0.4     & 92.83 & 0.07 & 33.610  &  0.027  & 2.0977  &  0.0017e+20\\
600.0 & 0.45    & 95.03 & 0.07 & 36.935  &  0.029  & 2.3053  &  0.0018e+20\\
640.0 & 0.5     & 95.41 & 0.08 & 40.877  &  0.032  & 2.5514  &  0.0020e+20\\
660.0 & 0.55    & 96.21 & 0.08 & 44.591  &  0.035  & 2.7832  &  0.0022e+20\\
680.0 & 0.6     & 97.38 & 0.08 & 48.06   &  0.04   & 2.9997  &  0.0023e+20\\
\bottomrule
\end{tabular}\end{table}
\begin{figure}
    \centering
    \includegraphics[width=14cm, height=10cm]{build/plot2.pdf}
    \caption{Dampfdruck-Kurve.}
    \label{fig:plot2}
\end{figure}
Die Verdampfungswärme ist durch eine Ausgleichsrechnung somit %wie kommt man darauf?
\begin{equation*}
    L = \SI{-2.032(011)e4}{}. %Einheit? Einheiten stehen alle auf dem heiligen Zettel :D müsste Pa*s sein oder so, wenn ich mich richtig erinnere.
\end{equation*}
Die Massendurchsätze ergeben sich damit mit Gleichung \eqref{eqn:massendurchsatz} zu %was ist Q_2?
\begin{align*}
    m_1 &= \SI{0.0032 \pm 0.0006}{} \\
    m_2 &= \SI{0.0060 \pm 0.0010}{} \\
    m_3 &= \SI{0.00103 \pm 0.00029}{} \\
    m_4 &= \SI{0.0006 \pm 0.0005}{}. %Einheit? Anders angben? kg/Sekunde vermutlich. Ich hatte die auch nur m1 genannt. Eigentlich ist das dm/delta*t oder so.
\end{align*}

%f)
\subsection{Bestimmung der mechanischen Kompressorleistung}
Dichlordifluormethan hat bei den Werten
\begin{align*}
    T &= \SI{273.15}{\kelvin} \\
    p &= \SI{e5}{\pascal} \\
    \kappa &= \num{1.14}
\end{align*}
die Dichte 
\begin{equation*}
    \rho_0 = \SI{5.51}{\gram\per\liter},
\end{equation*}
die mit Gleichung \eqref{eqn:dichte} berechnet werden kann.
Die mechanischen Kompressorleistungen für die vier Temperaturen
ergeben sich mit Gleichung \eqref{eqn:nmech}
zu den in Tabelle \ref{tabsolution2} stehen Werten.
\begin{table}\caption{Die Ergebnisse für die mechanische und die elektrische Leistung für die vier verschiedenen Temperaturwerte.}
\label{tabsolution2}
\centering
\sisetup{round-mode = places, round-precision=2, round-integer-to-decimal=true}
\begin{tabular}{S[]S[]} 
\toprule
{$\nu_\text{real}$} & {$\nu_\text{ideal}$}\\
\midrule
 135 \pm 32 & 165\\
 160 \pm 40 &  200\\
 50 \pm 70 & 208\\
 310 \pm 140 & 212\\
\bottomrule
\end{tabular}\end{table}
 %Für die el. Leistung ist das kein Ergebnis
