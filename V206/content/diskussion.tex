\section{Diskussion}
\label{sec:Diskussion}

\subsection{Temperaturverläufe}
Die Temperaturverläufe, die in den Plots aufgetragen wurden, können mit einem 
Polynom dritten Grades ziemlich exakt gefittet werden. 

\noindent Die Differentialquotienten haben relative Fehler im Bereich von \SI{6.9}{\percent} 
bis \SI{80}{\percent}. Sie sind somit zum Teil ziemlich groß. Was dabei auffällt, ist, dass die relativen Fehler 
am Anfang klein sind und dann für spätere Zeitpunkte größer werden.

\subsection{Bestimmung der Güteziffern}
Die realen Güteziffern haben einen relativen Fehler. Bei der ersten Temperatur beträgt 
dieser \SI{7.8}{\percent}. Bei der zweiten Messung liegt der relative Fehler bei 
\SI{9.0}{\percent}, bei der dritten bei \SI{14.7}{\percent} und bei der 
vierten Messung bei \SI{29.6}{\percent}. 
Die idealen Güteziffern haben aufgrund der fehlerbehafteten Temperaturwerte einen 
relativen Fehler, der im Bereich von \SI{3.1}{\percent} bis \SI{129.6}{\percent} 
liegt. Der relative Fehler ist bei der ersten idealen Ziffer besonders hoch und nimmt bei den anderen drei Werten dann ziemlich stark ab. %im Berech "liegt"?
Die realen Gütewerte sind prozentual betrachtet für die 
erste Temperatur \SI{0.24}{\percent}, für die zweite Temperatur 
\SI{2.60}{\percent}, für die dritte Temperatur \SI{6.30}{\percent} und für die 
vierte Temperatur \SI{7.2}{\percent} der idealen Gütewerte.
%ABweichung von ... bis ... insgesamt?

%Gründe für die schlechte Güteziffer angeben

\subsection{Bestimmung des Massenduchsatzes}
Die Verdampfungswärme $L$, die mittels einer Ausgleichsrechnung bestimmt wurde, 
ergibt einen Wert von \SI{-20320}{\joule\per\mol} mit einem relativen Fehler von 
\SI{0.54}{\percent}. %"ergibt" einen Wert von..?

\noindent Der Massenduchsatz für die erste Temperatur hat einen relativen Fehler von 
\SI{18.75}{\percent}, für die zweite Temperatur \SI{16.67}{\percent}, für die dritte 
Temperatur \SI{28.16}{\percent} und für die vierte Temperatur \SI{83.33}{\percent}. 

\subsection{Bestimmung der mechanischen Kompressorleistung}
Die mechanische Leistung bei der ersten Temperatur entspricht 
\SI{4.36}{\percent} der Leistung, die verbraucht wurde, um die Wärmepumpe 
anzutreiben. Für die zweite Temperatur ergibt sich ein Wert von \SI{6.65}{\percent}, 
für die dritte Temperatur \SI{1.25}{\percent} und für die vierte Temperatur 
\SI{0.75}{\percent}. 
Somit ist das System der Wärmepumpe nicht als sonderlich effizient zu bewerten.
