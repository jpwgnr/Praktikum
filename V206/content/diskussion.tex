\section{Diskussion}
\label{sec:Diskussion}

\subsection{Temperaturverläufe}
Die Temperaturverläufe, die in den Plots aufgetragen wurden, können mit einem 
Polynom dritten Grades ziemlich exakt gefittet werden. 

\noindent Die Differentialquotienten haben relative Fehler im Bereich von \SI{<++>}{\percent}<++> 
bis \SI{<++>}{\percent}<++>.

\subsection{Bestimmung der Güteziffern}
Die realen Güteziffern haben einen relativen Fehler. Bei der ersten Temperatur beträgt 
dieser \SI{<++>}{\<++>}<++>. Bei der zweiten Messung liegt der relative Fehler bei 
\SI{<++>}{\percent}<++>, bei der dritten bei \SI{<++>}{\percent}<++> und bei der 
vierten Messung bei \SI{<++>}{\percent}<++>. 
Die idealen Güteziffern haben aufgrund der fehlerbehafteten Temperaturwerte einen 
relativen Fehler, der im Bereich von \SI{<++>}{\percent}<++> bis \SI{<++>}{\percent}<++> 
variiert. %im Berech "liegt"?
Die relative Abweichung zwischen den idealen und den realen Gütewerten beträgt für die 
erste Temperatur \SI{<++>}{\percent}<++>, für die zweite Temperatur 
\SI{<++>}{\percent}<++>, für die dritte Temperatur \SI{<++>}{\percent}<++> und für die 
vierte Temperatur \SI{<++>}{\percent}<++>.
%ABweichung von ... bis ... insgesamt?

%Gründe für die schlechte Güteziffer angeben

\subsection{Bestimmung des Massenduchsatzes}
Die Verdampfungswärme $L$, die mittels einer Ausgleichsrechnung bestimmt wurde, 
ergibt einen Wert von \SI{<++>}{\<++>}<++> mit einem relativen Fehler von 
\SI{<++>}{\percent}<++>. %"ergibt" einen Wert von..?

\noindent Der Massenduchsatz für die erste Temperatur hat einen relativen Fehler von 
\SI{<++>}{\percent}, für die zweite Temperatur \SI{<++>}{\percent}, für die dritte 
Temperatur \SI{<++>}{\percent} und für die vierte Temperatur \SI{<++>}{\percent}. 

\subsection{Bestimmung der mechanischen Kompressorleistung}
Die mechanische Leistung bei der ersten Temperatur weicht um 
\SI{<++>}{\percent} von der Leistung ab, die verbraucht wurde, um die Wärmepumpe 
anzutreiben. Für die zweite Temperatur ergibt sich ein Wert von \SI{<++>}{\percent}, 
für die dritte Temperatur \SI{<++>}{\percent} und für die vierte Temperatur 
\SI{<++>}{\percent}. 
Somit ist das System der Wärmepumpe nicht als sonderlich effizient zu bewerten.