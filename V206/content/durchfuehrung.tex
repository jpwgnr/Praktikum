\section{Durchführung}
\label{sec:Durchführung}

Die Reservoire der in Abb. xy dargestellten Apparatur werden mit einer Wassermenge von \SI{}{\liter} aufgefüllt. Anschließend werden die Temperaturen $T1$ und $T2$ in den Reservoiren, die Drücke $p_a$ und $p_b$ im Verdampfungs- bzw. Verflüssigungsbereich und die Leistungsaufnahme des Kompressors gemessen. Der Zeittakt beträgt dabei eine Minute. 
Die Messung wird abgebrochen, wenn T1 einen Wert von ca. \SI{50}{\degrees\celsius} erreicht hat. 
