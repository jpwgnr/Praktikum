\section{Auswertung}
\label{sec:Auswertung}

\subsection{Kennlinien der Hochvakuumdiode}
\label{sec:a}
Die für die Anodenspannung und den Anodenstrom aufgenommenen Werte
bei den Heizstömen von $I_\text{H} = \num{2}$ bis $\SI{2.4}{\ampere}$
sind in den Tabellen \ref{taba} bis \ref{tabe} zu sehen.

\begin{table}\caption{Die Anzahl der Impulse, der Startwert auf der Mikrometerschraube und der Endwert auf der Mikrometerschraube.}
\label{taba}
\centering
\sisetup{round-mode = places, round-precision=2, round-integer-to-decimal=true}
\begin{tabular}{S[]S[]S[]} 
\toprule
{Anzahl} & {$d_\text{Start} / \si{\milli\meter}$} & {$d_\text{Start} / \si{\milli\meter}$}\\
\midrule
3001.0 & 6.73 & 2.0\\
3002.0 & 6.73 & 2.0\\
3000.0 & 1.82 & 6.5\\
3000.0 & 6.74 & 2.0\\
3000.0 & 1.83 & 6.5\\
3000.0 & 6.74 & 2.0\\
3001.0 & 1.84 & 6.5\\
3000.0 & 2.83 & 7.5\\
3001.0 & 7.77 & 3.0\\
3002.0 & 2.75 & 7.5\\
\bottomrule
\end{tabular}\end{table}
\begin{table}\caption{Die Frequenzen der Sägezahnspannung.}
\label{tabb}
\centering
\sisetup{round-mode = places, round-precision=2, round-integer-to-decimal=true}
\begin{tabular}{S[]S[]} 
\toprule
{Index} & {$\nu_\text{Sä} / \si{\hertz}$}\\
\midrule
1.0 & 25.02\\
2.0 & 49.95\\
3.0 & 99.99\\
4.0 & 149.97\\
\bottomrule
\end{tabular}\end{table}
\begin{table}\caption{Der magnetische Fluss $B$ des gemessenen Magnetfelds gegen den Strom $I$ des erzeugenden Magnetfelds, Neukurve.}
\label{tabc}
\centering
\sisetup{round-mode = places, round-precision=1, round-integer-to-decimal=true}
\begin{tabular}{S[]S[]} 
\toprule
{$B$/ \si{\milli\tesla}} & {$I$/ \si{\ampere}}\\
\midrule
0.0 & 0.0\\
111.19999999999999 & 1.0\\
273.5 & 2.0\\
397.8 & 3.0\\
479.9 & 4.0\\
537.9000000000001 & 5.0\\
585.0999999999999 & 6.0\\
621.8000000000001 & 7.0\\
653.1 & 8.0\\
679.9 & 9.0\\
704.3000000000001 & 10.0\\
\bottomrule
\end{tabular}\end{table}
\begin{table}\caption{Kreisfrequenz $\omega$ gegen die Phasenverschiebung $\varphi$ der Kondensatorspannung $U_C$ und der Generatorspannungi $U_0$.}
\label{tabd}
\centering
\sisetup{round-mode = places, round-precision=2, round-integer-to-decimal=true}
\begin{tabular}{S[]S[]} 
\toprule
{$\omega\cdot 10^{5}$ /\si[per-mode=fraction]{\per\second}} & {$Phase \varphi$}\\
\midrule
0.5654866776461628 & 0.12440706908215582\\
0.6911503837897545 & 0.11058406140636072\\
0.8168140899333463 & 0.13069025438933538\\
0.9424777960769379 & 0.1696460032938488\\
1.0681415022205296 & 0.16022122533307945\\
1.1938052083641213 & 0.20294688542190062\\
1.319468914507713 & 0.23750440461138836\\
1.4451326206513049 & 0.26012387171723483\\
1.5707963267948966 & 0.34557519189487723\\
1.6964600329384882 & 0.4750088092227767\\
1.8221237390820801 & 0.546637121724624\\
1.8849555921538759 & 0.6785840131753952\\
1.9477874452256716 & 0.818070726994782\\
2.0106192982974673 & 0.9650972631827843\\
2.0734511513692637 & 1.1611326447667876\\
2.1362830044410592 & 1.4099467829310992\\
2.199114857512855 & 1.6713272917097701\\
2.261946710584651 & 1.9000352368911066\\
2.324778563656447 & 2.092300707290802\\
2.3876104167282426 & 2.244353791724548\\
2.450442269800039 & 2.4014334244040376\\
2.5761059759436304 & 2.5761059759436304\\
2.701769682087222 & 2.6477342884454775\\
2.827433388230814 & 2.770884720466197\\
2.9530970943744057 & 2.894035152486917\\
3.078760800517997 & 2.8940351524869175\\
3.204424506661589 & 2.948070546128662\\
3.330088212805181 & 2.9970793915246623\\
3.4557519189487724 & 2.9719466502959446\\
3.581415625092364 & 3.008389125077586\\
3.7070793312359562 & 3.0398050516134836\\
\bottomrule
\end{tabular}\end{table}
\begin{table}\caption{Die Spannung und die Stromstärke bei einer Heizspannung von $\SI{4.1}{\volt}$ und die Heizspannung $\SI{2.4}{\ampere}$.}
\label{tabe}
\centering
\sisetup{round-mode = places, round-precision=1, round-integer-to-decimal=true}
\begin{tabular}{S[]S[]} 
\toprule
{$U / \si{\volt}$} & {$I / \si{\micro\ampere}$}\\
\midrule
5.0 & 15.0\\
10.0 & 36.0\\
15.0 & 58.0\\
20.0 & 79.0\\
25.0 & 100.0\\
30.0 & 118.0\\
35.0 & 132.0\\
40.0 & 142.0\\
45.0 & 149.0\\
50.0 & 154.0\\
55.0 & 158.0\\
60.0 & 161.0\\
65.0 & 162.0\\
70.0 & 164.0\\
75.0 & 166.0\\
80.0 & 167.0\\
85.0 & 168.0\\
90.0 & 169.0\\
95.0 & 170.0\\
100.0 & 171.0\\
105.0 & 172.0\\
115.0 & 173.0\\
120.0 & 174.0\\
130.0 & 175.0\\
145.0 & 176.0\\
150.0 & 177.0\\
165.0 & 178.0\\
180.0 & 179.0\\
195.0 & 180.0\\
210.0 & 181.0\\
\bottomrule
\end{tabular}\end{table}

\noindent Die Ströme sind in Abb. \ref{fig:plot1} jeweils gegen die Spannungen
aufgetragen.
\begin{figure}
    \centering
    \includegraphics[width=15cm, height=9cm]{build/plot1.pdf}
    \caption{Die Anodenströme sind jeweils für die verschiedenen Heizströme/-Spannungen
    gegen die Anodenspannungen aufgetragen.}
    \label{fig:plot1}
\end{figure}

\noindent Aus der Abb. \ref{fig:plot1} lassen sich jeweils
die Sättigungsströme ablesen. Diese sind für die verschiedenen
Heizströme
\begin{align*}
    I_\text{H} = \SI{2.0}{\ampere} \Rightarrow I_\text{S} &= \SI{8e-6}{\ampere} \\
    I_\text{H} = \SI{2.1}{\ampere} \Rightarrow I_\text{S} &= \SI{20e-6}{\ampere} \\
    I_\text{H} = \SI{2.2}{\ampere} \Rightarrow I_\text{S} &= \SI{37e-6}{\ampere} \\
    I_\text{H} = \SI{2.3}{\ampere} \Rightarrow I_\text{S} &= \SI{80e-6}{\ampere} \\
    I_\text{H} = \SI{2.4}{\ampere} \Rightarrow I_\text{S} &= \SI{175e-6}{\ampere}.
\end{align*}


\subsection{Gültigkeitsbereich des Raumladungsgesetzes}
Die logarithmierte Stromstärke ist gegen die logarithmierte
Spannung in Abb. \ref{fig:plot2} aufgetragen.
Daraus ergibt sich mit einer linearen Regression der
Exponent $x$. Dieser ist die Steigung der Geraden.
\begin{equation*}
    x = \num{1.18 \pm 0.03}.
\end{equation*}
%erwartet wird 1.5

\begin{figure}
    \centering
    \includegraphics[width=15cm, height=9cm]{build/plot2.pdf}
    \caption{}
    \label{fig:plot2}
\end{figure}


\subsection{Anlaufstromgebiet der Diode und Bestimmung der Kathodentemperatur}
Die Werte für die Spannungen und Ströme im Anlaufstromgebiet
befinden sich in den Tabellen \ref{tabf} bis \ref{tabh}.

\begin{table}\caption{Die gemessene Gegenspannung und die dazu gehörende Stromstärke.}
\label{tabf}
\centering
\sisetup{round-mode = places, round-precision=1, round-integer-to-decimal=true}
\begin{tabular}{S[]S[]} 
\toprule
{$U / \si{\milli\volt}$} & {$I / \si{\nano\ampere}$}\\
\midrule
0.0 & 38.0\\
50.0 & 34.0\\
100.0 & 28.0\\
150.0 & 23.0\\
200.0 & 19.0\\
250.0 & 15.0\\
300.0 & 12.0\\
350.0 & 9.0\\
400.0 & 7.0\\
450.0 & 6.0\\
500.0 & 5.0\\
\bottomrule
\end{tabular}\end{table}
\begin{table}\caption{Die Gegenspannung und die dazu gehörende Stromstärke.}
\label{tabg}
\centering
\sisetup{round-mode = places, round-precision=1, round-integer-to-decimal=true}
\begin{tabular}{S[]S[]} 
\toprule
{$U / \si{\milli\volt}$} & {$I / \si{\nano\ampere}$}\\
\midrule
400.0 & 8.9\\
450.0 & 6.8\\
500.0 & 5.3\\
550.0 & 4.0\\
600.0 & 3.1\\
650.0 & 2.3\\
700.0 & 1.75\\
750.0 & 1.35\\
800.0 & 1.0\\
850.0 & 0.7\\
900.0 & 0.5\\
950.0 & 0.2\\
1000.0 & 0.1\\
\bottomrule
\end{tabular}\end{table}
\begin{table}\caption{Die Gegenspannung und die dazu gehörende Stromstärke.}
\label{tabh}
\centering
\sisetup{round-mode = places, round-precision=1, round-integer-to-decimal=true}
\begin{tabular}{S[]S[]} 
\toprule
{$U / \si{\milli\volt}$} & {$I / \si{\nano\ampere}$}\\
\midrule
850.0 & 0.69\\
900.0 & 0.52\\
950.0 & 0.37\\
1000.0 & 0.22\\
\bottomrule
\end{tabular}\end{table}

\noindent Die logarithmierte Stromstärke ist in Abb. \ref{fig:plot3}
gegen die Spannung aufgetragen.
\begin{figure}
    \centering
    \includegraphics[width=15cm, height=9cm]{build/plot3.pdf}
    \caption{}
    \label{fig:plot3}
\end{figure}

\noindent Durch Ermittlung der Steigung lässt sich
die Temperatur bestimmen, indem Gleichung \eqref{eqn:anlauf}
nach $T$ umgestellt wird. 
Die Temperatur im Anlaufstromgebiet ist
\begin{equation*}
    T = \SI{2.44(12)e3}{\kelvin}.
\end{equation*}


\subsection{Leistungsbilanz des Heizstromkreises und Abschätzung der Kathodentemperatur}
Die in Teil \ref{sec:a} verwendeten Heizleistungen, die sich 
aus Multiplikation der Heizströme mit den Heizspannungen
ergeben, sind
\begin{align*}
    P_\text{1} &= \SI{6.00}{\watt} \\
    P_\text{2} &= \SI{6.72}{\watt} \\
    P_\text{3} &= \SI{7.7}{\watt} \\
    P_\text{4} &= \SI{9.20}{\watt} \\
    P_\text{5} &= \SI{9.84}{\watt}.
\end{align*}  %P oder L?

\noindent Die Werte für die emittierende Kathodenoberfläche $f$ und den
Emissionsgrad der Oberfläche $\eta$ sind
\begin{align*}
    f &= \SI{0.32}{\centi\meter\squared} \\
    \eta &= \num{0.28}.
\end{align*}

\noindent Daraus ergeben sich mit der Gleichung \eqref{eqn:Temp} 
die Temperaturen
\begin{align*}
    T_\text{1} &= \SI{1768.87}{\kelvin} \\
    T_\text{2} &= \SI{1829.38}{\kelvin} \\
    T_\text{3} &= \SI{1903.15}{\kelvin} \\
    T_\text{4} &= \SI{2001.74}{\kelvin} \\
    T_\text{5} &= \SI{2039.71}{\kelvin}.
\end{align*}


\subsection{Austrittsarbeit für Wolfram}
Die berechneten Austrittsarbeiten $e_0 \phi$ sind
\begin{align*}
    (e_0 \phi)_\text{1} &= \SI{9.93e-19}{\electronvolt} \\
    (e_0 \phi)_\text{2} &= \SI{1.01e-18}{\electronvolt} \\
    (e_0 \phi)_\text{3} &= \SI{1.03e-18}{\electronvolt} \\
    (e_0 \phi)_\text{4} &= \SI{1.07e-18}{\electronvolt} \\
    (e_0 \phi)_\text{5} &= \SI{1.07e-18}{\electronvolt}.
\end{align*}

\noindent Der sich daraus ergebende Mittelwert ist
\begin{equation*}
    (e_0 \phi)_\text{mittel} = \SI{100000000}{\electronvolt}.
\end{equation*}