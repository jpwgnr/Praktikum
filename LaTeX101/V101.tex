%new header 
% This example is meant to be compiled with lualatex or xelatex
% The theme itself also supports pdflatex
\PassOptionsToPackage{unicode}{hyperref}
\documentclass[aspectratio=1610, 9pt]{beamer}

% Load packages you need here
\usepackage{polyglossia}
\usefonttheme{professionalfonts}
%\setmainlanguage{english}

\usepackage{csquotes}
    

\usepackage{amsmath}
\usepackage{amssymb}
\usepackage{mathtools}

\usepackage{hyperref}
\usepackage{bookmark}

% Paket float verbessern
\usepackage{scrhack}

% Warnung, falls nochmal kompiliert werden muss

% unverzichtbare Mathe-Befehle
\usepackage{amsmath}
% viele Mathe-Symbole
\usepackage{amssymb}
% Erweiterungen für amsmath
\usepackage{mathtools}

% Fonteinstellungen
\usepackage{fontspec}
% Latin Modern Fonts werden automatisch geladen
% Alternativ:
%\setromanfont{Libertinus Serif}
%\setsansfont{Libertinus Sans}
%\setmonofont{Libertinus Mono}
% sollte man das Seiten-Layout neu berechnen lassen

% deutsche Spracheinstellungen
%\usepackage{polyglossia}
\setmainlanguage{english}

\usepackage[
  math-style=ISO,    % ┐
  bold-style=ISO,    % │
  sans-style=italic, % │ ISO-Standard folgen
  nabla=upright,     % │
  partial=upright,   % ┘
  warnings-off={           % ┐
    mathtools-colon,       % │ unnötige Warnungen ausschalten
    mathtools-overbracket, % │
  },                       % ┘
]{unicode-math}

% Zahlen und Einheiten
\usepackage[
  locale=UK,                   % deutsche Einstellungen
  separate-uncertainty=true,   % immer Fehler mit \pm
  per-mode=symbol-or-fraction, % / in inline math, fraction in display math
]{siunitx}

% schöne Brüche im Text
\usepackage{xfrac}

% Standardplatzierung für Floats einstellen
\usepackage{float}
\floatplacement{figure}{H}
\floatplacement{table}{H}

% Floats innerhalb einer Section halten
\usepackage[
  section, % Floats innerhalb der Section halten
  below,   % unterhalb der Section aber auf der selben Seite ist ok
]{placeins}

%dassselbe für Subsections 
\makeatletter
\AtBeginDocument{%
  \expandafter\renewcommand\expandafter\subsection\expandafter{%
    \expandafter\@fb@secFB\subsection
  }%
}
\makeatother

% Seite drehen für breite Tabellen: landscape Umgebung
\usepackage{pdflscape}

% Captions schöner machen.
\usepackage[
  labelfont=bf,        % Tabelle x: Abbildung y: ist jetzt fett
  font=small,          % Schrift etwas kleiner als Dokument
  width=0.7\textwidth, % maximale Breite einer Caption schmaler
]{caption}
% subfigure, subtable, subref
\usepackage{subcaption}

% Grafiken können eingebunden werden
\usepackage{graphicx}
% größere Variation von Dateinamen möglich
%\usepackage{grffile}

% schöne Tabellen
\usepackage{booktabs}
\usepackage{physics}
% Verbesserungen am Schriftbild
\usepackage{microtype}

% Literaturverzeichnis
\usepackage[
  sorting=none,
  style=authortitle,
  autolang=hyphen,
  backend=biber,
]{biblatex}
% Quellendatenbank
\addbibresource{lit.bib}

% Hyperlinks im Dokument

% Trennung von Wörtern mit Strichen
\usepackage[shortcuts]{extdash}

%selbst hinzugefügt
\usepackage{physics}

% load the theme after all packages

\usetheme[
  showtotalframes, % show total number of frames in the footline
]{tu}

% Put settings here, like
\unimathsetup{
  math-style=ISO,
  bold-style=ISO,
  nabla=upright,
  partial=upright,
  mathrm=sym,
}


% traditionelle Fonts für Mathematik
\setmathfont{Latin Modern Math}
% Alternativ:
%\setmathfont{Libertinus Math}
\setmathfont{XITS Math}[range={scr, bfscr}]
\setmathfont{XITS Math}[range={cal, bfcal}, StylisticSet=1]

% title etc. 
\title{Heavy Quarks (ARGUS at DORIS and CLEO at CESR)}
\date{25.06.2020}
\author{Jan Herdieckerhoff}
%\institute[E5]{Experimentelle Physik V\\ Fakultät Physik}





\begin{document}
    \section{Ziel}
    Bei diesem Versuch soll das
    Trägheitsmoment zweier verschiedener
    Körper, sowie die Trägheitsmomente
    einer Modellpuppe in zwei verschiedenen
    Körperhaltungen gemessen werden.
    Die experimentellen Werte sollen
    jeweils mit theoretischen Ergebnissen
    verglichen und der Satz von Steiner
    verifiziert werden.

    \section{Theorie}
    %Drehmoment
    Wirkt auf einen Körper eine Kraft
    im Abstand $\vec{r}$ von der Achse, entsteht ein auf
    diesen wirkendes Drehmoment $\vec{M} = \vec{F} \times \vec{r}$.
    Das Drehmoment bei Rotationsbewegungen ist das Analogon zur
    Kraft bei Translationsbewegungen.
    Es lässt sich auch durch die Winkelrichtgröße $D$ und den
    Winkel der Auslenkung $\phi$ ausdrücken:

    $M = D \cdot \phi$

    Wenn die Kraft $\vec{F}$ senkrecht zu dem Hebelarm $\vec{r}$
    steht, kann man die Winkelrichtgröße berechnen durch

    $D = \frac{Fr}{\phi}$.

    %Schwingungsdauer
    Bei schwingungsfähigen Systemen wird
    beim Loslassen eines um den Winkel $\phi$
    ausgelenkten Körpers eine harmonische
    Schwingung ausgeführt. Die Schwingungsdauer
    wird folgendermaßen bestimmt:

    $T = 2\pi \sqrt{\frac{I}{D}}$.

    Dabei ist $I$ das Trägheitsmoment des Körpers.
    Dieses gesamte Trägheitsmoment setzt sich aus den
    Trägheitsmomenten $I_D$ der Drillachse und $I_K$ des
    Körpers zusammen. Für $I_K$ ergibt sich dann

    $I_K = \frac{T^2D}{4 \pi^2} - I_D$.

    %Trägheitsmoment
    Das Trägheitsmoment bei Rotationsbewegungen
    ist das Analogon zur Masse bei Translationsbewegungen. Es wird im
    Allgemeinen durch das folgende Integral bestimmt:


    $I = \int \vec{r}^2_{\perp} dm$.

    
    %Satz von Steiner
    Wenn die Drehachse parallel zu der
    Achse, die durch den Schwerpunkt des
    Körpers geht, verschoben ist, berechnet
    man das Trägheitsmoment bezüglich
    dieser Drehachse mit dem Satz
    von Steiner:


    $I = I_S + m \cdot a^2$.

    $I_S$ ist das Trägheitsmoment bezüglich der Drehachse durch 
    den Schwerpunkt des Körpers und $m$ die Masse des Körpers.
    $a$ ist der Abstand der beiden Achsen.

    %Formeln herleiten
    Die im Versuch verwendeten Körper haben die
    Trägheitsmomente

    $I_{Kugel} = \frac{2}{5} m r^2$

    $I_{Zylinder} = \frac{1}{2} m r^2$.

    Für die Berechnung der Puppe wird für die einfache und schwierige Variante für beide
    Körperhaltungen jeweils der Satz von Steiner verwendet:

    $I_{ges} = \sum_{i=1}^N I_i + a_i^2 \cdot m_i$.

    Dabei ist $I_i$ jeweils das Trägheitsmoment eines einzelnen Körperteils.
    $m_i$ ist seine Masse und $a_i$ sein Abstand zur Drehachse.
    Bei der Berechnung der einfachen Variante wird die Puppe als ein aus
    fünf Zylindern und einer Kugel bestehenden Körper angenommen.
    Für die Berechnung der schwierigeren Vatriante wird der Körper in
    19 Zylinder bzw. Kugeln eingeteilt.

    $I_{Pos1}$ ist im Folgenden das gesamte Trägheitsmoment der Modellpuppe
    in der ersten Körperhaltung. $I_{Pos2}$ ist das gesamte
    Trägheitsmoment der Modellpuppe in der zweiten Körperhaltung.
    $I_{Pos1exp}$ und $I_{Pos2exp}$ sind die dazugehörigen experimentell
    errechneten Werte.

    Die Massen der einzelnen Körperteile der Puppe werden durch das Verhältnis
    zum Volumen bestimmt:

    $m_{i} = \frac{V_{i}}{V_{ges}} \cdot m_{ges}$

    $V_{i}$ ist das Volumen eines einzelnen Körperteil, $V_{ges}$ das gesamte Volumen der Puppe, $m_{ges}$
    die Masse der Puppe und $m_{i}$ die Masse eines einzelnen Körperteils.
    Das Volumen der Körperteile (Kugeln und Zylinder) kann folgendermaßen bestimmt werden:

    $V_{Kugel} = \frac{4}{3} \pi r^3$

    $V_{Zylinder} = \pi r^2 h$.

    \section{Durchführung}
    Um das Trägheitsmoment $I$ eines Körpers zu
    bestimmen, wird der Körper auf einer Drillachse
    befestigt. Der Körper wird in Schwingung versetzt.
    Durch die Periodendauer $T$, die Winkelrichtgröße
    $D$ und das Eigenträgheitsmoment $I_D$ der Drillachse lässt
    sich das Trägheitsmoment des Körpers bestimmen.

    SKIZZE

    %Winkelrichtgröße D
    Zunächst muss die Winkelrichtgröße $D$ bestimmt
    werden. Um diese zu bestimmen,
    wird eine Federwaage an eine nahezu masselose
    Metallstange, die mittig an der Drillachse befestigt wird, eingehakt und um den Winkel
    $\phi$ ausgelenkt. Die Federwaage wird senkrecht
    zum Radius der Kreisbahn der Metallstange
    gehalten. Die Kraft wird für zehn verschiedene
    Auslenkwinkel $\phi$ gemessen.

    %Eigenträgheitsmoment I_D
    Um das Eigenträgheitsmoment $I_D$ der Drillachse
    zu bestimmen, werden zwei Gewichte auf
    der Metallstange befestigt. Die Stange mit
    Gewichten wird ausgelenkt und losgelassen.
    Die zehnfache Schwingungsdauer $T$ wird für zehn verschiedene
    gemessene Abstände $a$ der Gewichte zur Drillachse gemessen.
    Die Massen der Gewichte werden durch eine
    fehlerfreie Waage bestimmt.

    %Trägheitsmomente zwei Körper
    Die Trägheitsmomente der zwei Körper werden
    bestimmt, indem die Körper jeweils auf der
    Drillachse befestigt und ausgelenkt werden.
    Die zehnfache Schwingungsdauer wird zehn mal gemessen.
    Die Massen der beiden Körper werden wieder
    durch eine fehlerfreie Waage gemessen.
    Durchmesser und Höhe werden mit einer Schieblehre
    bestimmt.

    %Trägheitsmomente Puppe
    Zuletzt soll das Trägheitsmoment einer
    Modellpuppe in zwei verschiedenen
    Körperhaltungen bestimmt werden.
    Die Puppe wird durch einen an ihr befestigten Stab auf der
    Drillachse befestigt und ausgelenkt. Die
    zehnfache Schwingungsdauer wird jeweils
    zehn mal für beide Körperhaltungen gemessen.
    Das Gesamtgewicht der Puppe wird gewogen.
    Die Körperteile der Puppe werden einzeln vermessen.

    \section{Fehlerrechnung}
    %Mittelwert
    Der Mittelwert einer Stichprobe von $N$ Werten
    wird durch

    $\overline{x} = \frac{1}{N} \sum_{i=1}^N x_i$

    bestimmt.

    %Standardabweichung
    Die Standardabweichung der Stichprobe wird berechnet mit:

    $\sigma_x = \sqrt{\frac{1}{N-1} \sum_{i=1}^N (x_i - \overline{x})^2}$.

    %Gauß'sche Fehlerfortpflanzung
    Das sogenannte Gauß'sche Fehlerfortpflanzungsgesetz
    ist gegeben durch

    $\sigma_f = \sqrt{\sum_{i=1}^N (\frac{\partial f}{\partial x_i} \sigma_i)^2}$.

    Dabei ist $f$ eine von unsicheren Werten $x_i$
    abhängige Funktion mit Standardabweichungen $\sigma_i$.

    %Lineare Regression
    Bei der linearen Regression wird die Gerade
    $y(x) = mx + b$
    durch das Streudiagramm gelegt.
    Dabei ist $m$ die Steigung mit
    $m = \frac{\overline{xy} - \overline{x} \cdot \overline{y}}{\overline{x^2} - \overline{x}^2}$
    und $b$ der $y$-Achsenabschnitt mit
    $b = \frac{\overline{y} \cdot \overline{x^2} - \overline{xy} \cdot \overline{x}}{\overline{x^2} - \overline{x}^2}$.

    (Wenn im Folgenden Mittelwert, Standardabweichung
    oder die Standardabweichung von Funktionen unsicherer
    Größen berechnet werden, werden diese Formeln
    verwendet.)
    Zur Auswertung und Berechnung der Fehler
    wurde Python, im Speziellen uncertainties.unumpy
    und numpy, benutzt.

    \section{Auswertung}

    \subsection{Bestimmung der Apparatenkonstanten}
    Im Folgenden werden die Winkelrichtgröße $D$ und das Eigenträgheitsmoment
    $I_D$ der Drillachse aus den gemessenen Werten berechnet.

    \subsubsection{Bestimmung der Winkelrichtgröße}
    Die Messwerte, die zur Bestimmung der Winkelrichtgröße $D$ verwendet werden,
    sind in Tabelle 1 aufgelistet. Bei der Messung wurde die 
    Federwaage in dem Abstand $r = 25,9 cm$ gehalten.

    HIER TABELLE 1

    Die Messwerte werden gemittelt und in die Formel für $D$ eingesetzt.
    Für die Winkelrichtgröße ergibt sich
    $D = (0,014088 \pm 0,000027)Nm$.

    \subsubsection{Bestimmung des Eigenträgheitsmoments der Drillachse}
    Zunächst werden aus der Periodendauer $T$ der Schwingung der Massen
    und ihrem Abstand $a$ von der Drehachse ihre Quadrate gebildet, um
    eine lineare Regression von $T^2$ gegen $a^2$ zu bilden.
    Mit der Schwingungsgleichung und $I = I_D + I_{ges1} + I_{ges2}$, wobei
    aufgrund des Steinerschen Satzes
    $I_{ges1} = I_1 + m_1 r_1^2$ und $I_{ges2} = I_2 + m_2 r_2^2$ sind, ergibt sich
    für $I_D$ die Gleichung

    $I_D = T^2 \cdot \frac{D}{4\pi^2} -I_{ges1} -I_{ges2}$.

    $T^2$ ist der $y$-Achsenabschnitt.
    $r$ ist der Abstand von der Drehachse bis zum Schwerpunkt
    der als Punktmassen angenommenen Zylinder.
    Die lineare Regression der Messwerte aus Tabelle 2 ist in
    Abbildung 2 dargestellt.

    HIER TABELLE 2

    HIER GRAPH

    Das Eigenträgheitsmoment $I_D$ der Drillachse ist somit
    $I_D = (9,762 \pm 0,025) \cdot 10^-4 kg m^2$.

    \subsection{Bestimmung der Trägheitsmomente zweier Körper}
    Die Trägheitsmomente zweier Körper sollen experimentell bestimmt
    und mit den theoretisch berechneten Werten verglichen werden.
    Die verwendeten Körper sind ein Zylinder und eine Kugel.

    \subsubsection{Bestimmung des Trägheitsmoments eines Zylinders}
    Die Maße des Zylinders sind $h_Z = 138 mm$ und $d_Z = 77,95 mm$.
    Seine Masse ist $m_Z = 1005,28g$.
    Die gemessenen zehnfachen Periodendauern sowie die Periodendauern
    sind in Tabelle 3 dargestellt.

    HIER TABELLE 3

    Die Werte werden gemittelt. Mit der mittleren Schwingungsdauer $T_{Zyl} = (1,195 \pm 0,016) s$ und mit $I_{Zyl,exp} = \frac{DT_{Zyl}^2}{4\pi^2} -I_D$ wird das Trägheitsmoment zu
    $I_{Zyl,exp} = (-4,67 \pm 0,14) \cdot 10^-4 kg m^2$ berechnet.
    Der theoretisch berechnete Wert für das Trägheitsmoment eines
    Zylinders ist $I_{Zyl,theo} = (7,635 \pm 0,010) \cdot 10^-4 kg m^2$.


    \subsubsection{Bestimmung des Trägheitsmoments einer Kugel}
    Der Durchmesser der Kugel beträgt $d_K = 132,95 mm$. Die Masse
    der Kugel ist $m_K = 811,9 g$.
    Die gemessenen zehnfachen Periodendauern und die Periodendauern
    sind in Tabelle 4 aufgelistet.

    HIER TABELLE 4

    Mit der gemittelten Periodendauer $T_{Kug} = (1,687 \pm 0,016) s$ und mit $I_{Kug,exp} = \frac{DT_{Kug}^2}{4\pi^2} - I_D$
    ergibt sich für das Trägheitsmoment der Kugel $I_{Kug,exp} = (3,9 \pm 1,9) \cdot 10^-5 kg m^2$.
    Der theoretisch errechnete Wert ist $I_{Kug,theo} = (1,435 \pm 0,001) \cdot 10^-3 kg m^2$.

    \subsection{Bestimmung der Trägheitsmomente einer Modellpuppe}
    Das Trägheitsmoment einer Modellpuppe soll in zwei verschiedenen
    Körperhaltungen bestimmt werden und mit den theoretisch berechneten
    Werten verglichen werden.
    Die Abmessungen der einzelnen Körperteile für die einfache Variante
    sind in Tabelle 5 und die für die schwierige Variante in Tabelle 6 dargestellt.

    TABELLE 5

    TABELLE 6

    Im Folgenden wird die einfache Variante als $Var1$ und die schwierige Variante als $Var2$ bezeichnet.
    Das Volumen der Puppe beträgt also $V_{Var1} = (1,93 \pm 0,04) \cdot 10^-4 m^3$ 
    beziehungsweise $V_{Var2} = (1,314 \pm 0,023) \cdot 10^-4 $.

    \subsubsection{Bestimmung des Trägheitsmoments der Puppe in der ersten Körperhaltung}

    SKIZZE KÖRPERHALTUNG 1

    Die zehnfache Periodendauer der Rotation der Puppe in der ersten Stellung
    und die Periodendauer sind in Tabelle 7 aufgelistet.

    TABELLE 7

    Die Periodendauer wird gemittelt. Für das Trägheitsmoment ergibt sich
    $I_{Pos1,exp} = \frac{DT_{Pos1}^2}{4\pi^2} - I_D = -(9,08 \pm 0,06) \cdot 10^-4 kg m^2$.
    Das theoretisch berechnete Trägheitsmoment für die einfache
    Variante, dass die Puppe aus fünf Zylindern und einer Kugel besteht,
    ist $I_{Pos1,Var1} = (6,927 \pm 0,2784) \cdot 10^-4 kg m^2$.
    Unter der Annahme, dass die Puppe aus 19 Zylindern bzw. Kugeln
    besteht, ist das Trägheitsmoment $I_{Pos1,Var2} = (6,42 \pm 0,13) \cdot 10^-5 kg m^2$.

    \subsubsection{Bestimmung des Trägheitsmoments der Puppe in der zweiten Körperhaltung}

    SKIZZE KÖRPERHALTUNG 2

    Die zenfache Periodendauer sowie die Periodendauer der Modellpuppe
    in der zweiten Stellung sind in Tabelle 8 zu finden.

    TABELLE 8

    Mit der gemittelten Periodendauer beträgt das Trägheitsmoment $I_{Pos2,exp} = \frac{DT_{Pos2}^2}{4\pi^2} - I_D = -(6,47 \pm 0,11) \cdot 10^-4 kg m^2$.
    Für den theoretischen Wert ergibt sich mit der einfachen Annahme $I_{Pos2,Var1} = (3,8758 \pm 0,1466) \cdot 10^-5 kg m^2$.
    Die schwierigere Variante liefert für das Trägheitsmoment der Puppe in der zweiten Körperhaltung
    $I_{Pos2,Var2} = (1,23 \pm 0,60) \cdot 10^-5 kg m^2$.

    \section{Diskussion}
    Nach unserer Auswertung muss man die Messung der Trägheitsmomente durch Schwingungsdauern verschiedener Figuren auf der mit einer Spiralfeder verbundenen Drillachse als ungenau bewerten.
Die experimentell ermittelten Werte für das Trägheitsmoment des Zylinders und der Puppe in beiden Körperhaltungen sind beispielsweise negativ. Dies deutet auf einen Fehler in der Messung hin. In der Formel für die Bestimmung des Trägheitsmoments wird das Trägheitsmoment der Drillachse $I_D$ abgezogen. Zieht man diesen Wert nicht ab, entstehen deutlich konsistentere Werte, die sowohl positiv als auch nah am theoretisch berechnet Wert liegen.
    Die Ungenauigkeit entsteht hier dadurch, dass die Stange, die ein erhebliches Gewicht hat, in der Theorie als masselos angenommen wird. Dadurch ist das Trägheitsmoment natürlich deutlich höher, als es sein sollte.
    Die Messung durch Stoppuhren auf dem Handy ist ebenfalls ziemlich ungenau, denn diese wird immer nur dann gestoppt, wenn der Umkehrpunkt der Pendelbewegung vermeintlich erreicht ist, was aber nicht exakt zu erkennen ist. Daher beinhaltet die Messung vermutlich sogar einen größeren Fehler als die vorgegeben 0,5 Sekunden. 
    Auch das Direktionsmoment $D$ könnte Grund für die Ungenauigkeiten sein. Es wurde natürlich versucht, den Kraftmesser genau im 90° Winkel zur Stange zu halten. Dies wurde allerdings nicht exakt gemessen. Insofern kann es auch dabei zu Fehlern gekommen sein. Außerdem war es nicht gut möglich, die Werte für kleine Auslenkungen zu bestimmen. Da man die Feder nicht über 360° auslenken sollte, war das Spektrum der zu sammelnden Werte nicht sehr groß.
    Das theoretische Trägheitsmoment der Puppe wurde, wie in der Auswertung gezeigt, auf einfache und komplizierte Weise jeweils in zwei Positionen bestimmt. Dabei wurde von einer homogenen Massenverteilung ausgegangen, wobei zum Beispiel der Metallstab bei der einfachen Variante völlig außer Acht gelassen wurde. Bei der schwierigeren Auswertung wurde er nur als weiteres Stück Holz betrachtet, obwohl der Stab natürlich eine andere Dichte hat, als beispielsweise das Bauchstück. Auch die kleinen Schrauben und Verbindungsstücke aus Metall wurden bei der Berechnung nicht beachtet. 
    Ein weiteres Problem war, dass die Puppe nicht wirklich fixiert war. Bei der Drehung hat sie sich durch die Fliehkräfte vor allem in Position 1 stark nach außen bewegt. 
    Bei fast allen Werten wurden zehn Drehschwingungen durchgeführt, damit der Fehler kleiner wurde. Dadurch gingen einzelne Durchgänge länger als eine Minute. In dieser Zeit sind möglicherweise bereits Fehler durch Luftwiderstand und Reibung des Materials entstanden, die auch Auswirkungen auf einige der Ergebnisse haben könnten.



    \section{Literatur}
    
\end{document}