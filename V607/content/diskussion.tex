\section{Diskussion}
\label{sec:Diskussion}

\subsection{Bestimmung des Ionenstroms und der Energiedosisrate}

Die erste Ionendosisrate weicht von der zweiten um \SI{7.95}{\percent} ab. 
Der relative Fehler der gemittelten Ionendosisrate beträgt \SI{3.88}{\percent}. 
Der relative Fehler der erzeugten Ionen pro Kilogramm pro Sekunde ergibt sich zu \SI{3.83}{\percent}. Die Energiedosisrate hat somit einen relativen Fehler von \SI{3.74}{\percent}. 

\subsection{Ionenstrom als Funktion des Anodenstroms}
%Was ist bei unterschiedlichen U_K zu beobachten?
Der Fitparameter $a$ bei der kleinen Blende hat einen relativen Fehler von \SI{0.15}{\percent}. 
Der Fitparameter $a$ bei der großen Blende hat einen relativen Fehler von \SI{0.12}{\percent}. 
Es ist zu erkennen, dass sich der Ionenstrom proportional zum Anodenstrom erhöht. 
\subsection{Ionenstrom als Funktion der Beschleunigungsspannung}
%Was ist bei unterschiedlichen U_K zu beobachten?
Der Fitparameter $a$ bei der kleinen Blende hat einen relativen Fehler von \SI{2.60}{\percent}. 
Der Fitparameter $a$ bei der großen Blende hat einen relativen Fehler von \SI{3.40}{\percent}. 
Es ist zu erkennen, dass sich der Ionenstrom proportional zum Quadrat der Beschleunigungsspannung erhöht. 
