\section{Durchführung}
\label{sec:Durchführung}

Der Versuchsaufbau ist in Abb. \ref{fig:aufbau} zu sehen.
\begin{figure}
    \centering
    \includegraphics[width=10cm, height=5cm]{build/aufbau.png}
    \caption{\cite{V607}}
    \label{fig:aufbau}
\end{figure}

\subsection{Messen der Ionendosis J und der Energiedosisrate D} %anderes D
Die Röntgenröhre wird auf die Beschleunigungsspannung
$U_\text{B} = \SI{25}{\kilo\volt}$ und einen Emissionsstrom 
von $I_\text{A} = \SI{1}{\milli\ampere}$ gestellt.
Es werden die Blenden mit einem Durchmesser von
$d_1 = \SI{2}{\milli\meter}$ und $d_2 = \SI{5}{\milli\meter}$ 
verwendet.
\newline
Der Kondensatorstrom $I_\text{K}$ wird jeweils in Abhängigkeit
von der Kondensatorspannung $U_\text{K}$ in $\SI{50}{\volt}$ 
Schritten gemessen.

\subsection{Messen des Ionenstroms in Abhängigkeit von $I_\text{A}$}
Die Röntgenröhre wird wieder auf eine Beschleunigungsspannung von
$U_\text{B} = \SI{25}{\kilo\volt}$ gestellt. Es wird eine
Blende mit dem Durchmesser $d = \SI{5}{\milli\meter}$
verwendet. Die Kondensatorspannung wird auf 
$U_\text{K,1} = \SI{500}{\volt}$ und anschließend auf
$U_\text{K,2} = \SI{300}{\volt}$ eingestellt.
\newline
Es wird jeweils der Kondensatorstrom $I_\text{K}$ in
Abhängigkeit vom Anodenstrom $I_\text{A}$ gemessen.
Der Anodenstrom wird zunächst auf $I_\text{A} = \SI{1}{\milli\ampere}$
gestellt und in $\SI{0.05}{\milli\ampere}$ Schritten verringert.

\subsection{Messen des Ionenstroms in Abhängigkeit von $U_\text{B}$}
Die Röntgenröhre wird auf $I_\text{A} = \SI{1}{\milli\ampere}$
eingestellt. Es wird eine Blende mit $d = \SI{5}{\milli\meter}$
verwendet. Die Kondensatorspannung $U_\text{K,1} = \SI{500}{\volt}$
bzw. $U_\text{K,2} = \SI{300}{\volt}$ wird eingestellt.
\newline
Es wird der Kondensatorstrom $I_\text{K}$ in Abhängigkeit von
der Beschleunigungsspannung $U_\text{B}$ gemessen.
Dabei wird die Beschleunigungsspannung zunächst auf
$U_\text{B} = \SI{35}{\kilo\volt}$ gestellt und anschließend
jeweils um $\SI{5}{\kilo\volt}$ verringert.