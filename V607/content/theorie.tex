\section{Ziel}
Das Ziel dieses Versuchs ist es die Strahlendosis und die
Strahlungsleistung in einem mit Röntgenstrahlung bestrahlten
Luftvolumen zu bestimmen.

\section{Theorie}
\label{sec:Theorie}

Die Dosimetrie ist die Messung der von einem System
aufgenommenen Strahlungsenergie.
Es wird die mit der ionisierenden Strahlung
verbundene Strahlenwirkung gemessen.

\subsection{Energiedosis D}
Die Energiedosis ist das Verhältnis von absorbierter
Energie $dE$ zu der Masse $dm$ des Absorbers.
Sie wird beschrieben durch
\begin{equation*}
    D = \frac{dE}{dm} = \frac{1}{\rho} * \frac{dE}{dV}.
    \label{eqn:Energiedosis}
\end{equation*}
Dabei ist $\rho$ die Dichte des Absorbers.

\subsection{Ionendosis J}
Mit der Ionisation des bestrahlten Materials geht zumeist
eine Absorption von Röntgenstrahlung einher. %gleiche Formulierung wie in der Anleitung
Die Ionendosis ist durch die in Luft erzeugte Ladung $dQ$
relativ zur Masse $dm_\text{L}$ der bestrahlten Luft definiert:
\begin{equation*}
    J = \frac{dQ}{dm_\text{L}}.
    \label{eqn:Ionendosis}
\end{equation*}

\subsection{Äquivalenzdosis H}
Die Wirkung ionisierender Strahlung auf biologische Materie
hängt bei gleicher Energiedosis von der Art der ionisierenden
Strahlung ab. Dieser Einfluss der Strahlungsenergie und -art
auf die biologische Wirkung wird durch den Qualitätsfaktor,
den faktor der relativen biologischen Wirkung,
beschrieben.
Die Äquivalenzdosis $H$ kann durch diesen Qualitätsfaktor
berechnet werden:
\begin{equation*}
    H = Q * \frac{dE}{dm} = Q * D.
    \label{eqn:Aequivalenzdosis}
\end{equation*}

\subsection{Dosisleistung}
Die Dosisleistung ist jeweils die Dosis pro Zeiteinheit.
Kurven gleicher Dosisleistung sind sogenannte Isodosen.

\subsection{Bestrahlung eines Luftvolumens mit Röntgenstrahlung}
Wird ein Luftvolumen in einem Plattenkondensator
(siehe Abb. \ref{fig:Strahlgeometrie}) mit
Röntgenstrahlung bestrahlt und ionisiert, erzeugen die durch
den Röntgenstrahl erzeugten Ionen und Elektronen einen Strom.
Dieser Strom wächst mit steigender Kondensatorspannung 
$U_\text{K}$ an bis er den Sättigungsstrom $I_\text{S}$
erreicht. Mithilfe dieses Stroms können die dosimetrischen
Größen bestimmt werden.
\newline
Das ionisierte Luftvolumen kann folgendermaßen bestimmt werden:
\begin{equation*}
    V = \frac{1}{3} \pi (R^2 (x_0 + x_1 + x_2) - r^2 (x_0 + x_1)). 
\end{equation*}
Dabei sind die Radien
\begin{align*}
    R &= \frac{d \, x_2}{2 \, x_0} \\
    r &= \frac{d \, x_1}{2 \, x_0}. 
\end{align*}
Das Volumen ist also 
\begin{equation}
    V = \frac{1}{3} \pi \left(\frac{d^2 x_2^2}{4 x_0^2}(x_0 + x_1 + x_2) - \frac{d^2 x_1^2}{4 x_0^2}(x_0 + x_1)\right).
    \label{eqn:V}
\end{equation}

\begin{figure}
    \centering
    \includegraphics[width=12cm, height=4cm]{build/strahl.png}
    \caption{\cite{V607}}
    \label{fig:Strahlgeometrie}
\end{figure}

 
