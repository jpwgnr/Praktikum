\section{Auswertung}
\label{sec:Auswertung}

Für die Auswertung wird Python und im Speziellen
Matplotlib \cite{matplotlib}, SciPy \cite{scipy},
Uncertainties \cite{uncertainties} und NumPy \cite{numpy} verwendet.

\noindent Die Werte für die Abstände in Abb. \ref{fig:Strahlgeometrie} sind
\begin{align*}
    x_0 &= \SI{83}{\milli\meter} \\
    x_1 &= \SI{67}{\milli\meter} \\
    x_2 &= \SI{100}{\milli\meter}.
\end{align*}
Die mit Gleichung \eqref{eqn:V} ermittelten Luftvolumina
für die Blendendurchmesser $d_1 = \SI{2}{\milli\meter}$ und
$d_2 = \SI{5}{\milli\meter}$ sind
\begin{align*}
    V_1 &= \SI{27.77}{\centi\meter\cubed} \\ 
    V_2 &= \SI{173.5}{\centi\meter\cubed}.  
\end{align*}


\subsection{Bestimmung der Ionendosisrate und der Energiedosisrate}

%Ionendosis J aus V und I_S bestimmen
Der Kondensatorstrom $I_\text{K}$ in Abhängigkeit von der
Kondensatorspannung $U_\text{K}$ bei der Blende mit
$d_1 = \SI{2}{\milli\meter}$ ist in Tab. \ref{taba}
dargestellt. Die Werte für die Blende mit 
$d_2 = \SI{5}{\milli\meter}$ befinden sich in Tab. \ref{tabb}.

\begin{table}\caption{Die Anzahl der Impulse, der Startwert auf der Mikrometerschraube und der Endwert auf der Mikrometerschraube.}
\label{taba}
\centering
\sisetup{round-mode = places, round-precision=2, round-integer-to-decimal=true}
\begin{tabular}{S[]S[]S[]} 
\toprule
{Anzahl} & {$d_\text{Start} / \si{\milli\meter}$} & {$d_\text{Start} / \si{\milli\meter}$}\\
\midrule
3001.0 & 6.73 & 2.0\\
3002.0 & 6.73 & 2.0\\
3000.0 & 1.82 & 6.5\\
3000.0 & 6.74 & 2.0\\
3000.0 & 1.83 & 6.5\\
3000.0 & 6.74 & 2.0\\
3001.0 & 1.84 & 6.5\\
3000.0 & 2.83 & 7.5\\
3001.0 & 7.77 & 3.0\\
3002.0 & 2.75 & 7.5\\
\bottomrule
\end{tabular}\end{table}
\begin{table}\caption{Die Frequenzen der Sägezahnspannung.}
\label{tabb}
\centering
\sisetup{round-mode = places, round-precision=2, round-integer-to-decimal=true}
\begin{tabular}{S[]S[]} 
\toprule
{Index} & {$\nu_\text{Sä} / \si{\hertz}$}\\
\midrule
1.0 & 25.02\\
2.0 & 49.95\\
3.0 & 99.99\\
4.0 & 149.97\\
\bottomrule
\end{tabular}\end{table}

\noindent Aus den Sättigungswerten des Kondensatorstroms ergibt sich die Ionendosisrate $\dot{J}$ und die Energiedosisrate $\dot{D}$. 
Der Sättigungswert für die kleine Blende ergibt sich zu 
\begin{equation*}
    I_\text{Sättigung, 1} = \SI{0.45}{\nano\ampere}.
\end{equation*}
Für die große Blende ergibt sich ein Wert von 
\begin{equation*}
    I_\text{Sättigung, 2} = \SI{2.6}{\nano\ampere}.
\end{equation*}

\noindent Somit lässt sich die Ionendosisrate mit Gleichung \eqref{eqn:Ionendosisrate} als die Werte  
\begin{align*}
    \dot{J}_\text{1} &= \SI{1.344e-5}{\ampere\per\kilo\gram}\\
    \dot{J}_\text{2} &= \SI{1.245e-5}{\ampere\per\kilo\gram} 
\end{align*}
bestimmen.

\noindent Der Mittelwert ergibt sich damit zu 
\begin{equation*}
    \dot{J}_\text{mittel} = \SI{1.29(05)e-5}{\ampere\per\kilo\gram}.
\end{equation*}

\noindent Die Anzahl der erzeugten Ionen ergibt sich mit Gleichung \eqref{eqn:Ionenanzahl} zu 
\begin{equation*}
    n = \SI{8.09(31)e13}{\per\kilo\gram\per\second}.
\end{equation*}

%mit Ionisationsenergie eines Luftmoleküls (Wert R aus Anleitung?) Energiedosis D und Energiedosisrate D° berechnen
\noindent Mit dem Wert von 
\begin{equation*}
    \Phi_\text{Luft} = \SI{33}{\electronvolt} = \SI{52.8e-19}{\joule}
\end{equation*}
ergibt sich die mittlere Energiedosisrate von 
\begin{equation*}
    \dot{D}_\text{m} = \SI{4.27(16)e-4}{\joule\per\kilo\gram\per\second}.
\end{equation*}



\subsection{Ionenstrom als Funktion des Anodenstroms}

Die Kondensatorströme $I_\text{K}$ in Abhängigkeit vom
Anodenstrom $I_\text{A}$ für die beiden Kondensatorspannungen
$U_\text{K, 1} = \SI{500}{\volt}$ und $U_\text{K, 2} = \SI{300}{\volt}$
sind in \ref{tabc} eingetragen.

\subsubsection{Erste Kondensatorspannung}
%I_K gegen I_A plotten
Die $I_\text{K}$- und $I_\text{A}$-Werte stehen in Tab. \ref{tabc} und sind in Abb. \ref{fig:plot1} aufgetragen. 

\begin{table}\caption{Der magnetische Fluss $B$ des gemessenen Magnetfelds gegen den Strom $I$ des erzeugenden Magnetfelds, Neukurve.}
\label{tabc}
\centering
\sisetup{round-mode = places, round-precision=1, round-integer-to-decimal=true}
\begin{tabular}{S[]S[]} 
\toprule
{$B$/ \si{\milli\tesla}} & {$I$/ \si{\ampere}}\\
\midrule
0.0 & 0.0\\
111.19999999999999 & 1.0\\
273.5 & 2.0\\
397.8 & 3.0\\
479.9 & 4.0\\
537.9000000000001 & 5.0\\
585.0999999999999 & 6.0\\
621.8000000000001 & 7.0\\
653.1 & 8.0\\
679.9 & 9.0\\
704.3000000000001 & 10.0\\
\bottomrule
\end{tabular}\end{table}

\begin{figure}
    \centering
    \includegraphics[width=15cm, height=8cm]{build/plot1.pdf}
    \caption{$I_\text{K}$- und $I_\text{A}$-Werte gegeneinander aufgetragen.}
    \label{fig:plot1}
\end{figure}

\noindent Die Parameter, die sich aus der linearen Regression ergeben, betragen
\begin{align*}
    a &= \num{27.48(04)e-4}\\
    b &= \SI{-4.32e-11}{\ampere},
\end{align*}
wobei $a$ die Steigung und $b$ der $y$-Achsenabschnitt ist. 

\subsubsection{Zweite Kondensatorspannung}
%I_K gegen I_A plotten
Die $I_\text{K}$- und $I_\text{A}$-Werte stehen in Tab. \ref{tabc} und sind in Abb. \ref{fig:plot2} aufgetragen. 

\begin{figure}
    \centering
    \includegraphics[width=15cm, height=8cm]{build/plot2.pdf}
    \caption{$I_\text{K}$- und $I_\text{A}$-Werte gegeneinander aufgetragen.}
    \label{fig:plot2}
\end{figure}

\noindent Die Parameter, die sich aus der linearen Regression ergeben, betragen
\begin{align*}
    a &= \num{24.74(03)e-4}\\
    b &= \SI{-1.63e-11}{\ampere},
\end{align*}
wobei $a$ die Steigung und $b$ der $y$-Achsenabschnitt ist. 



\subsection{Ionenstrom als Funktion der Beschleunigungsspannung}

Die Kondensatorströme $I_\text{K}$ in Abhängigkeit von der
Beschleunigungsspannung $U_\text{B}$ für die beiden Kondensatorspannungen
$U_\text{K, 1} = \SI{500}{\volt}$ und $U_\text{K, 2} = \SI{300}{\volt}$
sind in \ref{tabd} eingetragen.

\subsubsection{Erste Kondensatorspannung}
%I_K gegen U_B plotten
Die $I_\text{K}$- und $U_\text{B}$-Werte stehen in Tab. \ref{tabd} und sind in Abb. \ref{fig:plot3} aufgetragen. 

\begin{table}\caption{Kreisfrequenz $\omega$ gegen die Phasenverschiebung $\varphi$ der Kondensatorspannung $U_C$ und der Generatorspannungi $U_0$.}
\label{tabd}
\centering
\sisetup{round-mode = places, round-precision=2, round-integer-to-decimal=true}
\begin{tabular}{S[]S[]} 
\toprule
{$\omega\cdot 10^{5}$ /\si[per-mode=fraction]{\per\second}} & {$Phase \varphi$}\\
\midrule
0.5654866776461628 & 0.12440706908215582\\
0.6911503837897545 & 0.11058406140636072\\
0.8168140899333463 & 0.13069025438933538\\
0.9424777960769379 & 0.1696460032938488\\
1.0681415022205296 & 0.16022122533307945\\
1.1938052083641213 & 0.20294688542190062\\
1.319468914507713 & 0.23750440461138836\\
1.4451326206513049 & 0.26012387171723483\\
1.5707963267948966 & 0.34557519189487723\\
1.6964600329384882 & 0.4750088092227767\\
1.8221237390820801 & 0.546637121724624\\
1.8849555921538759 & 0.6785840131753952\\
1.9477874452256716 & 0.818070726994782\\
2.0106192982974673 & 0.9650972631827843\\
2.0734511513692637 & 1.1611326447667876\\
2.1362830044410592 & 1.4099467829310992\\
2.199114857512855 & 1.6713272917097701\\
2.261946710584651 & 1.9000352368911066\\
2.324778563656447 & 2.092300707290802\\
2.3876104167282426 & 2.244353791724548\\
2.450442269800039 & 2.4014334244040376\\
2.5761059759436304 & 2.5761059759436304\\
2.701769682087222 & 2.6477342884454775\\
2.827433388230814 & 2.770884720466197\\
2.9530970943744057 & 2.894035152486917\\
3.078760800517997 & 2.8940351524869175\\
3.204424506661589 & 2.948070546128662\\
3.330088212805181 & 2.9970793915246623\\
3.4557519189487724 & 2.9719466502959446\\
3.581415625092364 & 3.008389125077586\\
3.7070793312359562 & 3.0398050516134836\\
\bottomrule
\end{tabular}\end{table}

\begin{figure}
    \centering
    \includegraphics[width=15cm, height=8cm]{build/plot3.pdf}
    \caption{$I_\text{K}$- und $U_\text{B}$-Werte gegeneinander aufgetragen.}
    \label{fig:plot3}
\end{figure}

\noindent Die Parameter, die sich aus der Ausgleichsrechnung ergeben, betragen
\begin{align*}
    a &= \SI{4.314(112)e-18}{\ampere\per\volt}\\
    b &= \SI{-1.8(7)e-10}{\ampere},
\end{align*}
wobei $a$ die Amplitude und $b$ der $y$-Achsenabschnitt ist. 

\subsubsection{Zweite Kondensatorspannung}
%I_K gegen U_B plotten
Die $I_\text{K}$- und $U_\text{B}$-Werte stehen in Tab. \ref{tabd} und sind in Abb. \ref{fig:plot4} aufgetragen. 

\begin{figure}
    \centering
    \includegraphics[width=15cm, height=8cm]{build/plot4.pdf}
    \caption{$I_\text{K}$- und $U_\text{B}$-Werte gegeneinander aufgetragen.}
    \label{fig:plot4}
\end{figure}

\noindent Die Parameter, die sich aus der Ausgleichsrechnung ergeben, betragen
\begin{align*}
    a &= \SI{3.825(130)e-18}{\ampere\per\volt}\\
    b &= \SI{-1.35(8)e-10}{\ampere},
\end{align*}
wobei $a$ die Amplitude und $b$ der $y$-Achsenabschnitt ist.