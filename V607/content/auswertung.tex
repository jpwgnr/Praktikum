\section{Auswertung}
\label{sec:Auswertung}

Für die Auswertung wird Python und im Speziellen
Matplotlib \cite{matplotlib}, SciPy \cite{scipy},
Uncertainties \cite{uncertainties} und NumPy \cite{numpy} verwendet.

\noindent Die Werte für die Abstände in Abb. \ref{fig:Strahlgeometrie} sind
\begin{align*}
    x_0 &= \SI{83}{\milli\meter} \\
    x_1 &= \SI{67}{\milli\meter} \\
    x_2 &= \SI{100}{\milli\meter}.
\end{align*}
Die mit Gleichung \eqref{eqn:V} ermittelten Luftvolumina
für die Blendendurchmesser $d_1 = \SI{2}{\milli\meter}$ und
$d_2 = \SI{5}{\milli\meter}$ sind
\begin{align*}
    V_1 &= \SI{193}{\nano\meter\cubed} \\ %habe 27.77 cm^3 raus
    V_2 &= \SI{1207}{\nano\meter\cubed}.  %habe 173.5 cm^3 raus
\end{align*}


\subsection{Bestimmung der Ionendosis und der Energiedosisrate}

%Ionendosis J aus V und I_S bestimmen
Der Kondensatorstrom $I_\text{K}$ in Abhängigkeit von der
Kondensatorspannung $U_\text{K}$ bei der Blende mit
$d_1 = \SI{2}{\milli\meter}$ ist in Tab. \ref{taba}
dargestellt. Die Werte für die Blende mit 
$d_2 = \SI{5}{\milli\meter}$ befinden sich in Tab. \ref{tabb}.

Aus den Sättigungswerten des Kondensatorstroms ergibt sich die Ionendosis und die Energiedosisrate. 
Der Sättigungswert für die kleine Blende ergibt sich zu 
\begin{equation*}
    I_\text{Sättigung, 1} = \SI{<++>}{\<++>}.
\end{equation*}
Für die große Blende ergibt sich ein Wert von 
\begin{equation*}
    I_\text{Sättigung, 2} = \SI{<++>}{\<++>}.
\end{equation*}

Somit lässt sich dann die Ionendosis als die Werte  
\begin{align*}
    J_\text{1} &= \SI{<++>}{\<++>}\\
    J_\text{2} &= \SI{<++>}{\<++>} 
\end{align*}
bestimmen. 

%Anzahl der erzeugten Ionen aus J bestimmen
Daraus ergibt sich dann eine Anzahl von 
\begin{align*}
    n_\text{1} &= \SI{<++>}{\<++>}\\
    n_\text{2} &= \SI{<++>}{\<++>} 
\end{align*}
erzeugten Ionen. 


%mit Ionisationsenergie eines Luftmoleküls (Wert R aus Anleitung?) Energiedosis D und Energiedosisrate D° berechnen
Mit dem Wert von 
\begin{equation*}
    f_\text{Luft} = \SI{35}{\gray\kilo\gram\per\coulomb}
\end{equation*}
ergeben  sich dann Energiedosen von 
\begin{align*}
    D_\text{1} &= \SI{<++>}{\<++>}\\
    D_\text{2} &= \SI{<++>}{\<++>} 
\end{align*}
und für die Energiedosisraten 
\begin{align*}
    Dpunkt_\text{1} &= \SI{<++>}{\<++>}\\
    Dpunkt_\text{2} &= \SI{<++>}{\<++>}. 
\end{align*}


\subsection{Ionenstrom als Funktion des Anodenstroms}

Die Kondensatorströme $I_\text{K}$ in Abhängigkeit vom
Anodenstrom $I_\text{A}$ für die beiden Kondensatorspannungen
$U_\text{K} = \SI{500}{\volt}$ und $U_\text{K} = \SI{300}{\volt}$
sind in \ref{tabc} eingetragen.

\subsubsection{Erste Kondensatorspannung}
%I_K gegen I_A plotten

\subsubsection{Zweite Kondensatorspannung}
%I_K gegen I_A plotten


\subsection{Ionenstrom als Funktion der Beschleunigungsspannung}



Die Kondensatorspannungen $I_\text{K}$

\subsubsection{Erste Kondensatorspannung}
%I_K gegen U_B plotten

\subsubsection{Zweite Kondensatorspannung}
%I_K gegen U_B plotten
