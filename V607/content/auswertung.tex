\section{Auswertung}
\label{sec:Auswertung}

Für die Auswertung wird Python und im Speziellen
Matplotlib \cite{matplotlib}, SciPy \cite{scipy},
Uncertainties \cite{uncertainties} und NumPy \cite{numpy} verwendet.

\noindent Die mit Gleichung \eqref{eqn:V} ermittelten Luftvolumina
für die Blendendurchmesser $d_1 = \SI{2}{\milli\meter}$ und
$d_2 = \SI{5}{\milli\meter}$ sind
\begin{align*}
    V_1 &= \SI{193}{\nano\meter\cubed} \\
    V_2 &= \SI{1207}{\nano\meter\cubed}.
\end{align*}


\subsection{Bestimmung des Ionenstroms und der Energiedosisrate}

\subsubsection{Kleinere Blende}
%Ionendosis J aus V und I_S bestimmen

%Anzahl der erzeugten Ionen aus J bestimmen

%mit Ionisationsenergie eines Luftmoleküls (Wert R aus Anleitung?) Energiedosis D und Energiedosisrate D° berechnen

\subsubsection{Größere Blende}
%Ionendosis J aus V und I_S bestimmen

%Anzahl der erzeugten Ionen aus J bestimmen

%mit Ionisationsenergie eines Luftmoleküls (googlen) Energiedosis D und Energiedosisrate D° berechnen


\subsection{Ionenstrom als Funktion des Anodenstroms}

\subsubsection{Erste Kondensatorspannung}
%I_K gegen I_A plotten

\subsubsection{Zweite Kondensatorspannung}
%I_K gegen I_A plotten


\subsection{Ionenstrom als Funktion der Beschleunigungsspannung}

\subsubsection{Erste Kondensatorspannung}
%I_K gegen U_B plotten

\subsubsection{Zweite Kondensatorspannung}
%I_K gegen U_B plotten