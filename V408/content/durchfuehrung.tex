\section{Durchführung}
\label{sec:Durchführung}

Es wird eine optische Bank benutzt, auf der die optischen Elemente, also die Lampe, die Linsen und  der Schirm befestigt werden und somit auf einfache Art und Weise die Abstände zwischen den Elementen bestimmt werden können. 
Die Lichtquelle ist eine Halogenlampe und der Gegenstand ein "Perl L", eine Art Filter, durch den ein L förmiges Bild durchgelassen wird. 

\subsection{Brennweitenbestimmung}
Im ersten Schritt soll bei einer festen Gegenstandsweite $g$ die Position des Schirms variiert werden bis ein scharfes Bild zu erkennen ist. Dabei sollen die Gegenstands- und Bildweite gemessen werden. 
Die Messung wird für zehn verschiedene Gegenstandsweiten durchgeführt.

\subsection{Methode von Bessel}
Bei dieser Methode wird die Brennweite einer Linse bestimmt, in dem man den Abstand zwischen Gegenstand und Bild konstant lässt und die Linsenposition sucht, bei der das Bild scharf wird. 
Die Messung soll zehn mal durchgeführt werden. Anschließend soll die chromatische Abberation für blaues und rotes Licht bestimmt werden bei fünf verschiedenen Abständen. 

\subsection{Methode von Abbe}
Eine Zerstreuungslinse und eine Sammellinse werden zwischen der Lampe und dem Schirm positioniert. Die beiden Linsen werden so dicht zusammengestellt, dass sie sich berühren und wenn die beiden verschoben werden, wird ein fester Abstand zwischen den beiden Linsen beibehalten. Ein Referenzpunkt A wird gemessen und die neue Bild- und Gegenstandsweite $b'$ und $g'$ werden gemessen. 

