\section{Auswertung}
\label{sec:Auswertung}

In Tab. \ref{tab1}  sind die Gegenstandsweite $g$, die Brennweite $b$ und die Bildhöhe $B$ bei verschiedenen Entfernungen des Schirms von der Lampe aufgelistet. Dabei ist die Gegenstandshöhe $G = \SI{3}{\centi\meter}$.

\begin{table}\caption{Erste Messung.}
\label{tab1}
\centering
\sisetup{round-mode = places, round-precision=1, round-integer-to-decimal=true}
\begin{tabular}{S[]S[]S[]} 
\toprule
{$g / \si{\centi\meter}$} & {$b / \si{\centi\meter}$} & {$B / \si{\centi\meter}$}\\
\midrule
12.700000000000003 & 38.9 & 8.4\\
13.700000000000003 & 32.2 & 6.5\\
14.700000000000003 & 28.200000000000003 & 5.3\\
15.700000000000003 & 25.299999999999997 & 4.4\\
16.700000000000003 & 22.5 & 3.7\\
17.700000000000003 & 21.299999999999997 & 3.3\\
18.700000000000003 & 19.799999999999997 & 3.0\\
19.700000000000003 & 18.5 & 2.6\\
20.700000000000003 & 17.799999999999997 & 2.4\\
21.700000000000003 & 17.400000000000006 & 2.2\\
\bottomrule
\end{tabular}\end{table}

Um die Messgenauigkeit visuell erkennen zu können, werden die zueinander gehörenden Werte $b$ und $g$ als Schnittpunkte mit der x- und y-Achse betrachtet und es wird geschaut, in welchem Punkt sich die Geraden schneiden. Dieser Schnittpunkt ergibt die Brennweite der Linse. 

% figure b-g

Aus den Werten für die Bildweite $b$ und die Gegenstandsweite $g$ ergibt sich mit Gleichung \eqref{abbildungsgesetz} der Mittelwert des Abbildungsmaßstab $V$ zu \begin{equation}
    V = \SI{<++>}{\<++>}.
\end{equation}

Aus den Werten für die Bildhöhe $B$ und die Gegenstandshöhe $G$ ergibt sich mit Gleichung \eqref{abbildungsgesetz} der Mittelwert des Abbildungsmaßstab $V$ zu \begin{equation}
    V = \SI{<++>}{\<++>}.
\end{equation}

Nach Gleichung \eqref{linsengleichung} ergibt sich für die Brennweite $f$ ein Wert von 
\begin{equation}
    f = \SI{<++>}{\<++>}.
\end{equation}

Die Herstellerangabe gibt für die Brennweite einen Wert von 
\begin{equation}
    f = \SI{10}{\centi\meter}
\end{equation}
an.

\subsection{Methode von Bessel}

In Tab. \ref{tab2} sind die Werte bei verschiedenen Abständen zwischen Schirm und Lampe gemeinsam mit den Gegenstandsweiten und den Bildweiten aufgelistet. 

\begin{table}\caption{Die Spannung, die Stromstärke, die Anzahl der Impulse, die transportierte Ladungsmenge und die transporte Ladungsmenge in Einheiten der Elementarladung.}
\label{tab1}
\centering
\sisetup{round-mode = places, round-precision=2, round-integer-to-decimal=true}
\begin{tabular}{S[]S[] S[]@{${}\pm{}$}S[] S[]@{${}\pm{}$} S[] S[]@{${}\pm{}$} S[]} 
\toprule
{U / \si{\volt}} & {I / \si{\ampere}} & \multicolumn{2}{c}{N/second} &  \multicolumn{2}{c}{$\Delta Q / \si{\coulomb}$} &  \multicolumn{2}{c}{$\Delta Q \si{\elementarycharge}$}\\
\midrule
320.0 & 0.1     & 86.91 & 0.07 &  8.975  &  0.007  & 5.602   &  0.005e+19\\
400.0 & 0.2     & 90.92 & 0.07 & 17.157  &  0.014  & 1.0709  &  0.0009e+20\\
480.0 & 0.3     & 93.35 & 0.07 & 25.068  &  0.020  & 1.5646  &  0.0012e+20\\
540.0 & 0.35    & 94.62 & 0.07 & 28.851  &  0.023  & 1.8008  &  0.0014e+20\\
560.0 & 0.4     & 92.83 & 0.07 & 33.610  &  0.027  & 2.0977  &  0.0017e+20\\
600.0 & 0.45    & 95.03 & 0.07 & 36.935  &  0.029  & 2.3053  &  0.0018e+20\\
640.0 & 0.5     & 95.41 & 0.08 & 40.877  &  0.032  & 2.5514  &  0.0020e+20\\
660.0 & 0.55    & 96.21 & 0.08 & 44.591  &  0.035  & 2.7832  &  0.0022e+20\\
680.0 & 0.6     & 97.38 & 0.08 & 48.06   &  0.04   & 2.9997  &  0.0023e+20\\
\bottomrule
\end{tabular}\end{table}

Mit Gleichung \eqref{e}, Gleichung \eqref{d} und der Gleichung \eqref{fbessel} ergibt sich für die Brennweite der Linse ein Wert von 

\begin{equation}
    f = \SI{<++>}{\<++>}.
\end{equation}

In Tab. \ref{tab3} sind die gleichen Werte für einen roten Filter und in Tab. \ref{tab4} sind die Werte für einen blauen Filter aufgelistet.

Daraus ergibt sich, dass die Brennweite für rote Strahlen bei 

\begin{equation}
    f_\text{rot} = \SI{<++>}{\<++>}
\end{equation}

und für blaue Strahlen bei 
\begin{equation}
    f_\text{blau} = \SI{<++>}{\<++>}
\end{equation}
liegt.


\subsection{Methode von Abbe}

In Tab. \ref{tab5} sind die Werte Entfernung des Schirms, die dazugehörigen Refernzwerte $A$ und die jeweilige Bildhöhe aufgetragen. 

Die Werte $g'$ sind in Abb. \ref{fig:gstrich} gegen $(1+1/V)$ aufgetragen. In Abb. \ref{fig:bstrich} sind die Werte $b'$ gegen $(1+V)$ aufgetragen.

%Figure gstrich 
%Figure hstrich 

Die Steigung ergibt dann nach Gleichung \eqref{gstrich} die Brennweite. 
Für Abb. \ref{fig:gstrich} ergibt sich die Brennweite zu 
\begin{equation}
    f = \SI{<++>}{\<++>}
\end{equation}
und die Hauptebene des Linsensystems. 

Analog ergibt die Steigung dann nach Gleichung \eqref{hstrich} die Brennweite. 
Für Abb. \ref{fig:gstrich} ergibt sich die Brennweite zu 
\begin{equation}
    f = \SI{<++>}{\<++>}
\end{equation}
und die Hauptebene des Linsensystems. 
