\section{Auswertung}
\label{sec:Auswertung}

In Tab. \ref{tab1}  sind die Gegenstandsweite $g$, die Brennweite $b$ und die Bildhöhe $B$ bei verschiedenen Entfernungen des Schirms von der Lampe aufgelistet. Dabei ist die Gegenstandshöhe $G = \SI{3}{\centi\meter}$.

\begin{table}\caption{Der maximale Drehimpuls $L$, der Gesamtspin $S$ und der Gesamtdrehimpuls $J$ ergeben sich zum Landé-Faktor $g_\text{J}$ für die vier verschiedenen Elemente.}
\label{tab1}
\centering
\sisetup{round-mode = places, round-precision=2, round-integer-to-decimal=true}
\begin{tabular}{S[]S[]S[]S[]} 
\toprule
{$L$} & {$S$} & {$J$} & {$g_\text{J}$}\\
\midrule
5.0 & 1.0 & 4.0 & 0.8\\
0.0 & 3.5 & 3.5 & 2.0\\
6.0 & 1.5 & 4.5 & 0.7272727272727273\\
5.0 & 2.5 & 7.5 & 1.3333333333333333\\
\bottomrule
\end{tabular}\end{table}

Um die Messgenauigkeit visuell erkennen zu können, werden die zueinander gehörenden Werte $b$ und $g$ als Schnittpunkte mit der x- und y-Achse betrachtet und es wird geschaut, in welchem Punkt sich die Geraden schneiden. Dieser Schnittpunkt ergibt die Brennweite der Linse. 

% figure b-g

Aus den Werten für die Bildweite $b$ und die Gegenstandsweite $g$ ergibt sich mit Gleichung \eqref{abbildungsgesetz} der Mittelwert des Abbildungsmaßstab $V$ zu \begin{equation}
    V = \SI{<++>}{\<++>}.
\end{equation}

Aus den Werten für die Bildhöhe $B$ und die Gegenstandshöhe $G$ ergibt sich mit Gleichung \eqref{abbildungsgesetz} der Mittelwert des Abbildungsmaßstab $V$ zu \begin{equation}
    V = \SI{<++>}{\<++>}.
\end{equation}

Nach Gleichung \eqref{linsengleichung} ergibt sich für die Brennweite $f$ ein Wert von 
\begin{equation}
    f = \SI{<++>}{\<++>}.
\end{equation}

Die Herstellerangabe gibt für die Brennweite einen Wert von 
\begin{equation}
    f = \SI{10}{\centi\meter}
\end{equation}
an.

\subsection{Methode von Bessel}

In Tab. \ref{tab2} sind die Werte bei verschiedenen Abständen zwischen Schirm und Lampe gemeinsam mit den Gegenstandsweiten und den Bildweiten aufgelistet. 

\begin{table}\caption{Das Verhältnis des magnetischen Feldes durch die Beschleunigungsspannung aufgetragen gegen die Höhe.}
\label{tab2}
\centering
\sisetup{round-mode = places, round-precision=2, round-integer-to-decimal=true}
\begin{tabular}{S[]S[]S[]} 
\toprule
{$B_1 / \si{\henry}$} & {$B_2 / \si{\henry}$} & {$\frac{D}{(L^2 + D^2)} / \si{\per\meter}$}\\
\midrule
0.0 & 0.0 & 0.0\\
3.5649278338607584e-07 & 3.862005153349155e-07 & 0.29289724188430566\\
8.912319584651897e-07 & 8.912319584651897e-07 & 0.5827222842713544\\
1.4259711335443034e-06 & 1.396263401595464e-06 & 0.8665094112549946\\
1.9250610302848096e-06 & 1.8418793808280586e-06 & 1.1414982164090373\\
2.3885016486867084e-06 & 2.3172030920094934e-06 & 1.4052180429996723\\
2.923240823765822e-06 & 2.822234535139767e-06 & 1.6555530006898145\\
3.4223307205063282e-06 & 3.3272659782700412e-06 & 1.8907846756403912\\
\bottomrule
\end{tabular}\end{table}

Mit Gleichung \eqref{e}, Gleichung \eqref{d} und der Gleichung \eqref{fbessel} ergibt sich für die Brennweite der Linse ein Wert von 

\begin{equation}
    f = \SI{<++>}{\<++>}.
\end{equation}

In Tab. \ref{tab3} sind die gleichen Werte für einen roten Filter und in Tab. \ref{tab4} sind die Werte für einen blauen Filter aufgelistet.

Daraus ergibt sich, dass die Brennweite für rote Strahlen bei 

\begin{equation}
    f_\text{rot} = \SI{<++>}{\<++>}
\end{equation}

und für blaue Strahlen bei 
\begin{equation}
    f_\text{blau} = \SI{<++>}{\<++>}
\end{equation}
liegt.


\subsection{Methode von Abbe}

In Tab. \ref{tab5} sind die Werte Entfernung des Schirms, die dazugehörigen Refernzwerte $A$ und die jeweilige Bildhöhe aufgetragen. 

Die Werte $g'$ sind in Abb. \ref{fig:gstrich} gegen $(1+1/V)$ aufgetragen. In Abb. \ref{fig:bstrich} sind die Werte $b'$ gegen $(1+V)$ aufgetragen.

%Figure gstrich 
%Figure hstrich 

Die Steigung ergibt dann nach Gleichung \eqref{gstrich} die Brennweite. 
Für Abb. \ref{fig:gstrich} ergibt sich die Brennweite zu 
\begin{equation}
    f = \SI{<++>}{\<++>}
\end{equation}
und die Hauptebene des Linsensystems. 

Analog ergibt die Steigung dann nach Gleichung \eqref{hstrich} die Brennweite. 
Für Abb. \ref{fig:gstrich} ergibt sich die Brennweite zu 
\begin{equation}
    f = \SI{<++>}{\<++>}
\end{equation}
und die Hauptebene des Linsensystems. 
