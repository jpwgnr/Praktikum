\section{Theorie}
\label{sec:Theorie}

Linsen bestehen im Allgemeinen aus einem Material, das dichter als Luft ist.
Nach Fresnells Brechungsgesetzen wird das Licht beim Ein- und Austritt gebrochen.

In diesem Versuch werden Sammellinsen und Zerstreuungslinsen verwendet. Sammellinsen sind in der Mitte dicker als am Rand und bündeln das einstrahlende Licht.Die Brennweite $f$ einer Sammellinse, also die Entfernung von der Linse und dem Brennpunkt $F$, und die Bildweite $b$ sind, wie in Abb. \ref{sammellinse} zu erkennen, positiv und es entsteht ein reales Bild auf einem Schirm, während bei Zerstreuungslinsen (Abb. \ref{streulinse}) die Brennweite $f$ und die Bildweite $b$ negativ sind und ein virtuelles Bild erzeugen.  
%Skizze dünne Sammellinse 
%Skizze dünne Streulinse
%Skizze breite Sammellinse

Für die Bildkonstruktion in den Skizzen werden jeweils drei ausgezeichnete Strahlen eingezeichnet. Dabei gibt es den Parallelstrahl $P$, den Mittelpunktsstrahl $M$ und den Brennpunktsstrahl $B$. Die Wortbedeutung ist selbsterklärend.

Aus den Brechungsgesetzen und einigen geometrischen Überlegungen folgt dann das sogenannte Abbildungsgesetz

\begin{equation}
    V = \frac{B}{G} = \frac{b}{g}.
    \label{eqn:abbildungsgesetz}
\end{equation}

Dabei ist $V$ der Abbildungsmaßstab, also das Verhältnis zwischen Bild- und Gegenstandsgröße, $B$ die Bildgröße und $G$ die Gegenstandsgröße. Im selben Verhältnis stehen $b$, die Bildweite, und $g$, die Gegenstandsweite, zueinander. 

Für dünne Linsen folgt aus denselben Überlegungen die Linsengleichung 

\begin{equation}
    \frac{1}{f}= \frac{1}{b} + \frac{1}{g}.
    \label{eqn:linsengleichung}
\end{equation}

Bei Systemen aus mehreren Linsen muss die Rechnung anders gemacht werden. 
Die Mittelebene muss dann durch zwei Hauptebenen $H$ und $H'$ ersetzt werden.
Brennweite, Gegenstandsweite und die Bildweite werden dann in Bezug zu der jeweiligen Hauptebene bestimmt, die man stellvertretend für das ganze Linsensystem betrachtet. 

Es gibt Abbildungsfehler, die die Messung und vor allem die Exaktheit der Messungen beeinflussen können. So liegt zum Beispiel bei der so genannten spährischen Abberration der Brennpunkt achsenferner Strahlen näher an der Linse als von achsennahen Strahlen. Bei der chromatischen Abberation liegt der Brennpunkt von blauem Licht näher an der Linse als der von rotem Licht, da blaues Licht aufgrund der Dispersionsrelation stärker gebrochen wird.  

Die sogenannte reziproke Brennweite definiert die Brechkraft als $D= 1/f$ mit der Einheit Dioptrie. Bei einem Linsensystem aus verschiedenen gekrümmten Linsen summieren sich die Brechkräfte $D_i$, der einzelnen Linsen, sodass gilt 

\begin{equation}
    D= \sum{i}^N D_i.
    \label{eqn:brechungskraft}
\end{equation}

\subsection{Mathematische Hintergründe bei der Methode von Bessel}

Die Brennweite wird mit einer Linse bestimmt, indem der Abstand zwischen Gegenstand und Bild konstant bleibt und zwei Linsenpositionen gesucht werden, bei denen das Bild scharf ist. aus der Summe 
\begin{equation}
    e = g_1 + b_1 = g_2 + b_2
    \label{eqn:e}
\end{equation}
und der Differenz 
\begin{equation}
    d = g1- b1 = g2 - b2
    \label{eqn:d}
\end{equation}

ergibt sich dann die Brennweite zu
\begin{equation}
    f = \frac{e^2 - d^2}{4 e}.
    \label{eqn:fBessel}
\end{equation} 

\subsection{Mathematische Hintergründe bei der Methode von Abbe}
Ein Linsensystem wird zwischen Schirm und Lampe hin und her bewegt und es wird nach einem scharfen Bild gesucht wie in Abb. \ref{abbe} zu erkennen ist.
Die Gegenstandsweite $g'$ und die Bildweite $b'$ lassen sich bei einem System aus Linsen von einem Referenzpunkt gut messen. 
Wenn der Abbildungsmaßstab bekannt ist, lässt sich daraus dann die Brennweite und der Abstand zur Hauptebene bestimmen. 
Dafür kann der lineare Zusammenhang 
\begin{equation}
    g' = f \cdot \left(1 + \frac{1}{V} \right) + h
    \label{eqn:gstrich}
\end{equation}

beziehungsweise der Zusammenhang 
\begin{equation}
    b' = f \cdot \left(1 + V \right) + h'
    \label{eqn:gstrich}
\end{equation}
mit einer linearen Regression gelöst werden. 
