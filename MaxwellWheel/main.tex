% This example is meant to be compiled with lualatex or xelatex
% The theme itself also supports pdflatex
\PassOptionsToPackage{unicode}{hyperref}
\documentclass[aspectratio=1610, 9pt]{beamer}

% Load packages you need here
\usepackage{polyglossia}
\setmainlanguage{german}

\usepackage{csquotes}
    

\usepackage{amsmath}
\usepackage{amssymb}
\usepackage{mathtools}

\usepackage{hyperref}
\usepackage{bookmark}

% load the theme after all packages

\usetheme[
  showtotalframes, % show total number of frames in the footline
]{tudo}

% Put settings here, like
\unimathsetup{
  math-style=ISO,
  bold-style=ISO,
  nabla=upright,
  partial=upright,
  mathrm=sym,
}

\title{Das Maxwell'sche Rad}
\date{Vortragsdatum: 01.04.2019}
\author{
   Karina Overhoff \&
  Jan Herdieckerhoff 
}


\begin{document}

\maketitle

\begin{frame}{Einführung}
  \tableofcontents
\end{frame}

\section{Theorie}
\begin{frame}{Theorie}
    <++>
\end{frame}

\section{Durchführung}
\begin{frame}{Durchführung}
    <++>
\end{frame}

\section{Auswertung}
\begin{frame}{Auswertung}
    <++>
\end{frame}

\section{Diskussion}
\begin{frame}{Diskussion}
    <++>
\end{frame}

\end{document}
