\section{Theorie}
\begin{frame}{Theorie}
    
    \begin{equation}
        v= \frac{s}{t}.
    \end{equation}
    
    \begin{equation}
        v = \omega \cdot r
    \end{equation}

    \begin{equation}
        \abs{\vec{M}} = \abs{ \vec{r} \cross \vec{F} } = I \cdot \dot{\omega} = r m g
        \label{eqn:drehmoment}
    \end{equation}

    \begin{equation}
        I_\text{S}= \frac{1}{2} m R^2 
        \label{eqn:trägheit}
    \end{equation}

    \begin{equation}
        I_\text{M} = I_\text{S}+ m \cdot r^2
        \label{eqn:steiner}
    \end{equation}

    Mit Gleichung \ref{eqn:drehmoment}, Gleichung \ref{eqn:trägheit} und Gleichung \ref{eqn:steiner} ergibt sich
    \begin{equation}
        (\frac{R^2}{2 r^2} +1) \dot{v} = g.
    \end{equation}
    
    \begin{equation}
        \dotdot{s} = \frac{1}{1 + \frac{R^2}{2r^2}} \cdot g
    \end{equation}

    $r$ ergibt sich mit der abgerollten Länge von 10 Umdrehungen $\Delta s$ zu

    \begin{equation}
        r = \frac{\Delta s}{10 \cdot 2\pi}.
    \end{equation}

    Die wirkliche Beschleunigung ergibt sich zu 
    
    \begin{equation}
        a = \frac{2s}{t^2}.
    \end{equation}

\end{frame}
