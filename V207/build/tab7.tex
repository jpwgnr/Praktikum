\begin{table}\caption{Die Temperatur und die Reynoldszahlen der erste und zweite Messung.}
\label{tab7}
\centering
\sisetup{round-mode = places, round-precision=2, round-integer-to-decimal=true}
\begin{tabular}{S[]S[]S[]} 
\toprule
{$T /\si{\kelvin}$} & {$Re_1$} & {$Re_2$}\\
\midrule
326.15 & 60.8192940846345 & 61.58075867442081\\
328.15 & 65.29196107212914 & 67.15376369132561\\
330.15 & 68.31551907276062 & 72.56044280868363\\
331.15 & 73.44990716651733 & 74.5672655167899\\
333.15 & 77.9633113577189 & 80.516146521398\\
335.15 & 83.57229092824743 & 89.76242107453724\\
337.15 & 92.72562511274035 & 98.9476442151419\\
339.15 & 104.75113452403667 & 115.98891142199052\\
341.15 & 125.74458914652912 & 138.2887489038005\\
343.15 & 166.74813886555594 & 165.69035598392597\\
\bottomrule
\end{tabular}\end{table}