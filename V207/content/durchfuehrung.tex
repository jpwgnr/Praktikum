\section{Durchführung}
\label{sec:Durchführung}

Als erstes wird die Dichte der großen und kleinen Glaskugel aus der Masse und dem Volumen bestimmt. 
Auch die Dichte der Flüßigkeit $\rho_{Fl}$ wird mit einer Mohr-Westphalschen Waage uned einem Aräometer bestimmt. 

Mithilfe der Libelle wird überprüft, ob das Viskosimeter gerade steht und falls nicht, muss es nach justiert werden. 

Das Viskosimeter wird mit destilliertem Wasser gefüllt, wobei darauf geachtet wird, dass sich keine Luftblasen an der Rohrwand bzw. an der Kugel befinden. 
Die Fallzeit wird mit einer Stopuhr gemessen. Beim ersten mal mit der großen Kugel bei Raumtemperatur. Nachdem die untere Markierung überschritten 
wurde, wird das Viskosimeter um 180° gedreht und die Messung wiederholt. Diese Messung wird mit der kleinen und großen Kugel jeweils zehn mal durchgeführt. Somit wird für die große Kugel die Apparaturkonstante $K_{gr}$ bestimmt. Die Apparaturkonstante für die kleine Kugel ist bereits gegeben. 

Als nächstes wird die Temperaturabhängigkeit von destilliertem Wasser bestimmt. Dazu wird das Wasserbad auf 70° aufgeheizt und anschließend die Fallzeit jeweils zwei mal für zehn verschiedene Temperaturen mit der großen Kugel gemessen. 
