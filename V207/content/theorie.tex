\section{Ziel}

\section{Theorie}
\label{sec:Theorie}

Ein Körper, der sich in einer Flüssigkeit bewegt, wird von verschiedenen Kräften beeinflusst. 
Es wirken die Reibungskraft, die Schwerkraft und die Auftriebskraft. 
Die Reibungskraft hängt dabei von verschiedenen Faktoren ab:
Von der Berührungsfläche $A$, der Geschwindigkeit $v$ und
der sogenannten dynamischen Viskosität $\eta$. 
Diese ist eine Materialkonstante der Flüssigkeit und hängt stark von der Temperatur dieser Flüssigkeit ab. 
%Zur Durchführung:
% Mit dem Kugelfallviskosimeter lässt sich diese Viskosität bestimmen. 
% Dafür wird eine Kugel in einer Flüssigkeit, deren Ausdehnung hinreichend groß ist, 
% damit sich keine Wirbel bilden, fallen gelassen. 
Die Stokes'sche Reibung lässt sich folgendermaßen beschreiben:
\begin{equation}
    F_R = 6\pi \eta v r.
    \label{eq:Stokes}
\end{equation}
Beim Fallen nimmt die Reibung mit zunehmender Geschwindigkeit immer weiter zu, 
bis sich ein Kräftegleichgewicht einstellt. 
Die Reibungs- und Auftriebskraft stehen der Schwerkraft gegenüber. 
%Zur Durchführung:
% Beim Kugelfallviskosimeter nach Höppler lässt man die Kugel in einem Rohr fallen, 
% dessen Radius nur geringfügig größer ist als der Radius der Kugel. 
% Da beim senkrechten Fall evtl Wirbel entstehen würden und die Kugel unkontrolliert an die Rohrwand stoßen würde, 
% wird das Fallrohr um einen kleinen Winkel gekippt, sodass die Kugel an der Rohrwand heruntergleiten kann und 
% sich keine Wirbel bilden. 
Die Viskosität $\eta$ lässt sich aus der Fallzeit $t$, 
der Dichte der Flüssigkeit $\rho_Fl$ und der Dichte der Kugel $\rho_K$ bestimmen.
Der Proportionalitätsfaktor K ist eine Apparaturkonstante und enthält sowohl die Höhe,
als auch die Kugelgeometrie.
Es gilt:
\begin{equation}
    \eta = K (\rho_K -\rho_Fl) \cdot t.
    \label{eq:nu}
\end{equation}
Die Temperaturabhängigkeit der Viskosität lässt sich mit der Andradeschen Gleichung für viele Flüssigkeiten 
beschreiben: 
\begin{equation}
    \eta(T) = A exp(\frac{B}{T}).
    \label{eq:Temperatur}
\end{equation}
