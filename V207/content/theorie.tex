\section{Ziel}
Das Ziel dieses Versuchs ist es, die Temperaturabhängigkeit
der dynamischen Viskosität von destilliertem Wasser mittels
Kugelfall-Viskosimeter zu bestimmen.

\section{Theorie}
\label{sec:Theorie}

Ein Körper, der sich in einer Flüssigkeit bewegt, wird von verschiedenen Kräften beeinflusst. 
Es wirken die Reibungskraft, die Schwerkraft und die Auftriebskraft. 
Die Reibungskraft hängt dabei von verschiedenen Faktoren ab:
Von der Berührungsfläche $A$, der Geschwindigkeit $v$ und
der sogenannten dynamischen Viskosität $\eta$. 
Diese ist eine Materialkonstante der Flüssigkeit und hängt stark von der Temperatur dieser Flüssigkeit ab. 
\newline
Mit dem Kugelfallviskosimeter lässt sich diese Viskosität bestimmen. 
Dafür wird eine Kugel in einer Flüssigkeit, deren Ausdehnung hinreichend groß ist, 
damit sich keine Wirbel bilden, fallen gelassen. 
Die Stokes'sche Reibung lässt sich folgendermaßen beschreiben:
\begin{equation}
    F_R = 6\pi \eta v r.
    \label{eqn:stokes}
\end{equation}
Beim Fallen nimmt die Reibung mit zunehmender Geschwindigkeit immer weiter zu, 
bis sich ein Kräftegleichgewicht einstellt. 
Die Reibungs- und Auftriebskraft wirken entgegen der Schwerkraft. 
Die Viskosität $\eta$ lässt sich aus der Fallzeit $t$, 
der Dichte der Flüssigkeit $\rho_Fl$ und der Dichte der Kugel $\rho_K$ bestimmen.
Der Proportionalitätsfaktor $K$ ist eine Apparaturkonstante und enthält sowohl die Höhe,
als auch die Kugelgeometrie.
Es gilt:
\begin{equation}
    \eta = K (\rho_K -\rho_{Fl}) \cdot t.
    \label{eqn:eta}
\end{equation}
Die Temperaturabhängigkeit der Viskosität lässt sich mit der Andradeschen Gleichung 
beschreiben: 
\begin{equation}
    \eta(T) = A exp(\frac{B}{T}). %höhere Klammern
    \label{eqn:Temperatur}
\end{equation}
$A$ und $B$ sind hier Konstanten.

%Reynoldszahlen
Die Reynoldszahlen geben das Verhältnis von Trägheits- zu
Zähigkeitskräften an:
\begin{equation*}
    Re = \frac{\rho v x}{\eta}.
\end{equation*}
Mit der Fallgeschwindigkeit
\begin{equation*}
    v = \frac{x}{t}
\end{equation*}
werden die Reynoldszahlen mittels

berechnet.
