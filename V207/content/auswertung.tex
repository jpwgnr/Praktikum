\section{Auswertung}
\label{sec:Auswertung}

\subsection{Überprüfung der Strömung auf Laminarität} %Laminarität ist kein Wort, Überschrift ändern
Die Falldauern der kleinen und der großen Kugel im Kugelfall-Viskosimeter
bei Raumtemperatur befinden sich in Tabelle \ref{tab1}.
%Daraus besser zwei Tabellen machen?:
\begin{table}\caption{Erste Messung.}
\label{tab1}
\centering
\sisetup{round-mode = places, round-precision=1, round-integer-to-decimal=true}
\begin{tabular}{S[]S[]S[]} 
\toprule
{$g / \si{\centi\meter}$} & {$b / \si{\centi\meter}$} & {$B / \si{\centi\meter}$}\\
\midrule
12.700000000000003 & 38.9 & 8.4\\
13.700000000000003 & 32.2 & 6.5\\
14.700000000000003 & 28.200000000000003 & 5.3\\
15.700000000000003 & 25.299999999999997 & 4.4\\
16.700000000000003 & 22.5 & 3.7\\
17.700000000000003 & 21.299999999999997 & 3.3\\
18.700000000000003 & 19.799999999999997 & 3.0\\
19.700000000000003 & 18.5 & 2.6\\
20.700000000000003 & 17.799999999999997 & 2.4\\
21.700000000000003 & 17.400000000000006 & 2.2\\
\bottomrule
\end{tabular}\end{table}
\noindent Die Fallstrecke beträgt 
\begin{equation*}
    x = \SI{10}{\centi\meter}.
\end{equation*}
Die Masse, der Durchmesser und die Apparaturkonstante der kleinen Kugel sind:
\begin{align*}
    m_\text{klein} &= \SI{3.71}{\gram} \\
    d_\text{klein} &= \SI{1.56}{\centi\meter} \\
    K_\text{klein} &= \SI{0.07640}{\milli\pascal\cubic\centi\meter\per\gram}.
\end{align*}
Die Masse und der Durchmesser der großen Kugel sind:
\begin{align*}
    m_\text{groß} &= \SI{4.21}{\gram} \\
    d_\text{groß} &= \SI{1.58}{\gram}. \\
\end{align*}
Die Apparaturkonstante der großen Kugel lässt sich mit %...
bestimmen:
\begin{equation*}
    K_\text{groß} = \SI{}{}.
\end{equation*}

\subsection{Bestimmung der Temperaturabhängigkeit der Viskosität von destilliertem Wasser}
Die Falldauern der großen Kugel für verschiedene Temperaturen sind in den Tabellen
\ref{tab2} und \ref{tab3} dargestellt.
%Daraus besser eine Tabelle machen?:
\begin{table}\caption{Die Spannung, die Stromstärke, die Anzahl der Impulse, die transportierte Ladungsmenge und die transporte Ladungsmenge in Einheiten der Elementarladung.}
\label{tab1}
\centering
\sisetup{round-mode = places, round-precision=2, round-integer-to-decimal=true}
\begin{tabular}{S[]S[] S[]@{${}\pm{}$}S[] S[]@{${}\pm{}$} S[] S[]@{${}\pm{}$} S[]} 
\toprule
{U / \si{\volt}} & {I / \si{\ampere}} & \multicolumn{2}{c}{N/second} &  \multicolumn{2}{c}{$\Delta Q / \si{\coulomb}$} &  \multicolumn{2}{c}{$\Delta Q \si{\elementarycharge}$}\\
\midrule
320.0 & 0.1     & 86.91 & 0.07 &  8.975  &  0.007  & 5.602   &  0.005e+19\\
400.0 & 0.2     & 90.92 & 0.07 & 17.157  &  0.014  & 1.0709  &  0.0009e+20\\
480.0 & 0.3     & 93.35 & 0.07 & 25.068  &  0.020  & 1.5646  &  0.0012e+20\\
540.0 & 0.35    & 94.62 & 0.07 & 28.851  &  0.023  & 1.8008  &  0.0014e+20\\
560.0 & 0.4     & 92.83 & 0.07 & 33.610  &  0.027  & 2.0977  &  0.0017e+20\\
600.0 & 0.45    & 95.03 & 0.07 & 36.935  &  0.029  & 2.3053  &  0.0018e+20\\
640.0 & 0.5     & 95.41 & 0.08 & 40.877  &  0.032  & 2.5514  &  0.0020e+20\\
660.0 & 0.55    & 96.21 & 0.08 & 44.591  &  0.035  & 2.7832  &  0.0022e+20\\
680.0 & 0.6     & 97.38 & 0.08 & 48.06   &  0.04   & 2.9997  &  0.0023e+20\\
\bottomrule
\end{tabular}\end{table}
\begin{table}\caption{Die Zeit des Durchschallungsverfahrens gegen die Länge des Zylinders.}
\label{tab3}
\centering
\sisetup{round-mode = places, round-precision=2, round-integer-to-decimal=true}
\begin{tabular}{S[]S[]} 
\toprule
{t/ \si{\second}} & {l/ \si{\meter}}\\
\midrule
8.95e-05 & 0.1208\\
7.8e-05 & 0.1023\\
5.93e-05 & 0.0805\\
3.08e-05 & 0.0404\\
2.47e-05 & 0.0311\\
\bottomrule
\end{tabular}\end{table}

%Plot ln(eta) gegen 1/T
%A und B bestimmen

%Reynoldszahlen berechnen
%Strömung laminar?