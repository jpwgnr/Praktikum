\section{Durchführung}
\label{sec:durchfuehrung}

Im ersten Teil wird die Zeitabhängigkeit der Amplitude einer gedämpften Schwingung untersucht und im Anschluss kann daraus in der Auswertung ein Wert für den effektiven Dämpfungswiderstand ermittelt werden. 
Dafür wird mit einem Nadelimpulsgenerator der Schwingkreis zu einer gedämpften Schwingung angeregt. 
Die Werte, die dem Graphen entnommen werden müssen, werden dabei von einem Speicheroszillographen ausgegeben und ein Thermodruck wird angefertigt, der bei der Auswertung hilft. 


Durch Variation des Widerstands kann danach der Wert $R_{aperiodisch}$ ermittelt werden, bei dem der aperiodische Grenzfall erreicht wird. 
Der Widerstand wird am Anfang auf seinen Maximalwert eingestellt. Nach und nach wird der Widerstand dann verringert. Sobald ein Kurvenbereich auftritt in dem die Steigung des Graphen $\frac{dU_{C}}{dt}$ größer 0 ist, wurde $R_{aperiodisch}$ überschritten.


Die Kondesatorspannung und deren Frequenzabhängigkeit wird an einem Serienresonanzkreis gemessen. Hierfür wird eine Wechselspannung an den Schwingkreis angeschloßen, welche mit einem Frequenzmesser abgenommen wird. Auf der anderen Seite wird die Kondensatorspannung mit einem AC-Milli-Voltmeter gemessen.Der Dämpfunswiderstand soll hierbei der größere sein. Auch der Dämpfungswiderstand des Sinusgenerators ist mit zu kalkulieren.  

Anschließend wird die Frequenzabhängigkeit der Phase zwischen Erreger- und Kondensatorspannung verglichen. Dafür wird der Abstand zwischen den beiden Spannungen im Nullpunkt und die Periodendauer der Spannung gemessen.

Im letzten Schritt wird der Scheinwiderstand eines Serienresonanzkreises gemessen. Da aber kein Amperemeter zur Verfügung steht, welches Hochfrequenzströme messen kann, muss die effektive Stromstärke $I_{eff}$ als Spannungsabfall an einem niederohmigen Widerstand $R_{M}$ der deutlich kleiner ist als der Widerstand $R$ des RLC-Kreises. Das Millivoltmeter muss dabei mit der Gehäuseerde an den Sinusgerator angeelgt werden, wie in Abb. \ref{fig: Abb1} zu erkennen ist. 
Alternativ kann der Scheinwiderstand mit einem Impedanzmeter gemessen werden. 


