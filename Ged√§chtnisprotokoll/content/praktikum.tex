\section{Durchführung und Fragen}

\noindent Anfangs wurde gefragt, welchen Versuch wir gerne besprechen würden. Wir hatten uns für den Versuch Geometrische Optik entschieden, was bei den Prüfern positiv anzukommen schien, da beide meinten, dass das endlich mal was anderes sei. 
Also begannen wir zu erklären, was in dem Versuch genau gemacht werden sollte. Wir erklärten, dass wir im ersten Versuchsteil die Brennweite bestimmen wollten und sowohl das Abbildungsgesetz als auch die Linsengleichung verifizieren wollten.
Dazu zeichneten wir dann auch den Aufbau und zeigten, wie der Graph von g gegen b geplottet aussieht. 

\noindent An dieser Stelle wurden wir dann aber auch das erste mal unterbrochen. Frau Siegmann hinterfragte mit einigen einfachen Fragen, was die einzelnen Linien in dem Aufbau bedeuteten und ließ sich ein wenig zur Brechung erzählen. 
In dem Zusammenhang erklärten wir dann auch ein wenig zu Brechungsindizes und erwähnten die Fresnelschen Gesetze. 
Das war aber nicht genau das was sie hören wollte. Sie fragte nochmal genauer: \textbf{"Warum kann man annehmen, dass die Strahlen wirklich so gebrochen werden, wie Sie es hier gezeichnet haben?"} 
Worauf sie hinaus wollte, war, dass wir hier eine dünne Linse betrachten.

\noindent \textbf{"Wie wäre das denn, wenn Sie jetzt statt Luft ein Medium mit einem höheren Brechungsindex als den der Linse hätten?"}
Wir mussten erst überlegen und skizzierten, wie die Brechung bei Luft aussieht und wie es bei einem anderen Medium aussähe. 
Durch die Skizze und ein wenig Hilfe von Herrn Kröninger und Frau Siegmann kamen wir dann darauf, dass sich eine Sammellinse in so einem Fall wie eine Zerstreuungslinse verhalten würde.

\noindent Herr Kröninger wollte nun weiter gehen zu einem anderen Versuch. \textbf{"Wie sieht es denn aus mit dem Doppelspaltversuch?"} 
Wir erklärten erstmal grob, wie der Aufbau aussah, wie die Beugungsbilder für Einzel- und Doppelspalt aussehen und welche Lichtquelle wir hatten. 
\textbf{"Okay. Welche Eigenschaft hat das Licht denn?"}
Wir standen irgendwie sehr auf dem Schlauch an dieser Stelle. Wir erklärten, dass die Wellenlängen und die Phase übereinstimmen mussten und erzählten noch ein wenig über Amplitude etc. Worauf er hinaus wollte, war einfach der Begriff Kohärenz. 
Manchmal denkt man einfach zu kompliziert.

\noindent \textbf{"Na gut, dann ist das genug an dieser Stelle. Erzählen Sie uns doch noch etwas zu dem Versuch Magnetfelder und Spulen."} 
Also erklärten wir erst einmal wieder den allgemeinen Aufbau, erzählten was wir so gemessen hatten und skizzierten unsere Ergebnisse. 
Vor allem bei der Hysterse-Kurve wollten die beiden dann alles ganz genau erklärt bekommen. Bei der Remanenz hinterfragte Herr Kröninger, 
was da genau dazu führt, dass ein B-Feld bestehen bleibt, obwohl das äußere H-Feld nicht mehr vorhanden ist. Wir mussten das nicht wirklich 
genau erklären. Was er dabei hören wollte, war der Spin und die Weiß'schen Bezirke.

\noindent Anschließend wurden wir nach draußen geschickt.

