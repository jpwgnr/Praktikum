\section{Durchführung}
\label{sec:Durchführung}

Es wird die Bragg-Bedingung überprüft, indem der LiF-Kristall auf einen festen Kristallwert von \SI{14}{\degree} eingestellt wird. Das Geiger-Müller Zählrohr misst den Winkelnbereich von \num{26} bis \SI{30}{\degree} mit einem Abstand von \SI{0.1}{\degree}.

Anschließend wird Emessionspektrum einer Cu-Röntgenröhre in einem Winkelbereich von \num{4} bis \SI{26}{\degree} gemessen. Es wird in \SI{0.2}{\degree}-Schritten gemessen und die Integrationszeit pro Winkel beträgt \SI{5}{\second}. 

Aus der K-Kante wird die Abschirmungskonstante $\sigma_K$ für 5 verschiedene Materialien bestimmt. Aus der L-Kante ergibt sich die Abschirmungskonstante $\sigma_L$ für ein Material. 

Vor ein Geiger-Müller Zählrohr wird ein Germaniumabsorber gesetzt und das Absorbtionsspektrum wird gemessen. Ein geeigneter Messbereich muss selbstständig bestimmt werden. 


