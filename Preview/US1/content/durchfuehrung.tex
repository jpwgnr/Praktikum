\section{Durchführung}
\label{sec:Durchführung}

%Gerät und Programmeinstellung 

% Schallgeschwindigkeit in Acryl mittels Impuls-Echo-Verfahren und Durschallungs-Verfahren bestimmen 
Als erstes wird ein Acrylzylinder auf einem Papiertuch mit einer der 2 MHz-Sonden auf einer Schicht aus bidestiliertem Wassser gekoppelt. Damit wird dann ein A-Scan mittels des Puls-Echo-Verfahrens durchgeführt und für die ersten beiden Messungen wird die Laufzeit und Amplitude des ersten Pulses festgestellt. 
Die Länge des Zylinders wird gemessen und somit kann auf die Schallgeschwindigkeit geschlossen werden. 

Die Verstärkung am Echoskop muss anschließend so verändert werden, dass der zweite Puls eine Amplitude von 1 bis 1.2 \si{\Volt} hat. Die Graphiken sollen exportiert werden.
% Materialspezifischen Schwächungskoeffizienten von Acryl mittel Impuls-Echo-Verfahren bestimmen

Der Acrylzylinder wird wieder auf dieselbe Art und Weise vorbereitet wie beim ersten Teil des Versuchs. Anschließend wird die Amplitude des ausgesendeten und des reflektierten Pulses bestimmt. 
Diese Messung wird für sechs weitere Zylinder wiederholt. 

%Dicke zweier Acrylplatten mit Impuls-Echo-Verfahren aus der FFT und dem Cepstrum bestimmen und vergleichen 
Als nächstes wird die Abmessung aller Zylinder bestimmt. Es wird wieder ein A-Scan durchgeführt. Dieses mal sieben verschiedene Messungen, wobei die Zylinder auch gestapelt werden können. 

Nachdem erneut die Abmessungen der Acrylzylinder bestimmt wurden, werden sie in die schwarze Halterung gespannt und an beiden Seite wird eine Sonde mit Koppelgel gekoppelt. 
Mit einem A-Scan wird die Laufzeit bestimmt und dadurch die Schallgeschwindigkeit berechnet. Die Messung wird für alle Zylinder wiederholt. 

Mithilfe zweier Acrylscheiben und eines Zylinders werden Mehrfachimpulse im Impuls-Echo-Verfahren aufgenommen. Dafür werden die Scheieben aufeinander gelegt und der Zylinder oben drauf. Zur Kopplung wird bidestiliertes Wasser verwendet. 
Eine Mehrfachreflexion wird aufgenommen. 

Die Marker werden so gesetzt, dass mit der FFT-Funktion ein Spektrum und das Cepstrum der Sonde erhalten wird. 

Mit der FFT und dem Cepstrum sollen die Mehrfachechos analysiert werden. Aus dem Sprektum sollen die Laufzeiten erkannt werden und somit mit der Schallgeschwindigkeit die Dicke der Platte bestimmt werden. 



