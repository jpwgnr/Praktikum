\section{Durchführung}
\label{sec:Durchführung}

Durch Variation der Heizleistung wird eine Kennlinienschar einer Hochvakkumdiode aus mindestens 5 Kennlinien erstellt. Daraus wird der Sättigungsstrom $I_S$ abgelesen. 

Es wird versucht für die maximal mögliche Heizleistung den ungefähren Gültigkeitsbereich des Langmuir-Schottkyschen Raumladungsgesetzes zu finden. 
Aus den gemessenen Wertepaaren wird der Exponent der Strom-Spannungsbeziehung festgestellt. 

Für die maximal mögliche Heizleistung wird das Anlaufstromgebiet der Diode untersucht und die Kathodentemperatur T bestimmt. 

Aus einer Leistungsbilanz des Heizstromkreises wird die Kathodentemperatur bei den am Anfang verwendeten Heizleistungen abgeschätzt. 

Aus den Wertepaaren $T$ und $I_S$ wird die Austrittsarbeit für das Kathodenmaterial berechnet. 
