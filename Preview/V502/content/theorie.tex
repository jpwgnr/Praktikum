\section{Theorie}
\label{sec:Theorie}

\subsection{Theoretische Grundlage}
Elektronische Felder haben auf ruhende Ladungen ein ausübendes Kraftfeld, magnetostatische Felder im Gegensatz dazu nur auf Ladungen, die sich relativ zum Feld bewegen. 
Eine Ladung $q$, die sich mit der Geschwindigkeit $\vec{v}$ in einem homogenen Magnetfeld $\vec{B}$ erfärt die Lorentz-Kraft 
\begin{equation}
    \vec{F_\text{L}}= q \vec{v} \cross \vec{B}.
    \label{eqn:Lorentz}
\end{equation}

Die Lorentz-Kraft ist nur nicht null, wenn es eine Geschwindigkeitskomponente $\vec{v}$ gibt, die senkrecht zu $\vec{B}$ ausgerichtet ist. 

Das  Magnetfeld ändert allerdings nur die Richtung und ändert nicht die Geschwindigkeit. Also ist die Energie konstant innerhalb des Systems der Ladung. 

Der Radius der Krümmungsbahn lässt aus dem Gleichgewicht der Lorentz- und der Zentrifugalkraft bestimmen. Es ergibt sich 
\begin{equation}
    r= \frac{m_\text{0}v_\text{0}}{e_\text{0} B}.
    \label{eqn:radius}
\end{equation}
Die rechte Seite der Gleichung ist konstant, insofern ist die Krümmungsbahn iene Kreisbahn. 

\subsection{Praktische Grundlage}

Mit Gleichung \ref{eqn:radius} lässt sich die spezifische Ladung der Elektronen $e_\text{0}/m_\text{0}$ bestimmen. 
Mit der Beschleunigungsspannung $U_\text{B}$ ergibt sich die konstante Geschwindigkeit $v_\text{0}$ zu 
\begin{equation}
    v_\text{0}= \sqrt{2 U_\text{B} \frac{e_\text{0}}{m_\text{0}}}.
    \label{eqn:v0}
\end{equation}
In einem feldfreien Raum bewegen sich die Elektronen eines Kathodenstrahls in Richtung Mittelpunkt des Leuchtschirms und erzeugen einen Leuchtfleck. 
Wenn das Magnetfeld dann eingeschaltet wird, verschiebt sich der Leuchtfleck aufgrund der Krümmung auf der vertikalen Achse um das Stück $D$. Zwischen dem Wirkungsbereich L, der Weite zwischen der Quelle und dem Schirm, dem Stück $D$ und dem Radius $r$ ergibt sich über den Satz des Pythagoras die Verbindung

\begin{equation}
    r = \frac{L^2 + D^2}{2D}.
    \label{eqn:radius2}
\end{equation}

Dies kann man in \ref{eqn:radius} einsetzen und erhält den Zusammenhang 
\begin{equation}
    \frac{D}{L^2 + D^2}= \frac{1}{\sqrt{8 U_\text{B}}}\sqrt{\frac{e_\text{0}}{m_\text{0}}} B.
    \label{eqn:Ende}
\end{equation}
