\section{Durchführung}
\label{sec:Durchführung}


Die spezifische Ladung des Elektrons wird bestimmt und die Intensität des lokalen Erdmagnetfeldes. 

Dafür erzeugt man mittels eines großen Helmholtz-Spulenpaars ein nahezu homogenes Magnetfeld, das senkrecht zum Elektronenstrahl einer Kathodenstrahlröhre ausgerichtet ist. Nach der korrekten Ausrichtung misst man bei konstanter Beschleungigungsspannung $U_B = \SI{250}{\volt}$ und $\SI{500}{\volt}$ die Strahlverschiebung $D$ in Abhängigkeit von den beiden Magnetfeldstärken. 

Die Veränderung des angezeigte Leuchtfleck im XY-Koordinatensystem wird beobachtet während die Ausrichtung von der Nord-Süd-Richtung zur Ost-West-Richtung geändert wird. Die Elektronen werden nun in Y-Richtung abgelenkt. Das Helmholtz-Spulenpaar wird eingeschaltet und die Auslenkung wird kompentisiert. Somit ergibt sich der Wert des Erdmagnetfelds. 
