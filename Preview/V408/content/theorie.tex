\section{Theorie}
\label{sec:Theorie}

Die meisten Linsen bestehen aus Material, dass dichter ist als Das Umgebungsmaterial "Luft". Wenn ein Lichtstrahl auf eine Linse trifft, dann wird er nach dem Brechungsgesetz beim Ein- und Austritt aus dem Medium der Linse gebrochen. 

Es gibt verschiedene Arten von Linsen. So gibt es zum Beispiel Sammellinsen und Zerstreuungslinsen. Sammellinsen sind dicker in der Mitte als am Rand und bündeln parallel eingehendes Licht in einem Punkt, dem Brennpunkt. Die Brennweite $f$, also die Entfernung von Linse und Brennpunkt, und die die Bildweite $b$, die Entfernung von Linse und Bild, sind beide positiv und es entsteht ein reeles Bild, das auf einen Schirm abgebildet werden kann. Bei einer Zerstreungslinse ist das sind Brennweite $f$ und Bildweite $b$ negativ und es entsteht ien virtuelles Bild. 
%Skizze dünne Sammellinse 
%Skizze dünne Streulinse
%Skizze breite Sammellinse

Für die geometrische Bildkonstruktion verwendet man drei ausgezeichnete Strahlen. Den Parallelstrahl P, den Mittelpunktsstrahl M und den Brennpunktsstrahl B. 
Der Parallelstrahl verläuft vom Gegenstand parallel zur optischen Achse und wird dort zum Brennpunktstrahl. 
Der Mittelpunkstrahl geht durch die Mitte der Linse. 
Der Brennpunktstrahlt geht durch den Brennpunkt der Linse und wird an der optischen Achse gebrochen, sodass er zum Parallelstrahl wird. 

Aus den Strahlensätzen und der Bildkonstruktion folgt das Abbildungsgesetz 

\begin{equation}
    V = \frac{B}{G} = \frac{b}{g}.
    \label{eqn:abbildungsgesetz}
\end{equation}

Dabei ist $V$ der Abbildungsmaßstab, also das Verhältnis zwischen Bild- und Gegenstandsgröße, $B$ die Bildgröße und $G$ die Gegenstandsgröße.

Für dünne Linsen folgt die Linsengleichung 

\begin{equation}
    \frac{1}{f}= \frac{1}{b} + \frac{1}{g}.
    \label{eqn:linsengleichung}
\end{equation}

Bei dicken Linsen und Linensystemen gilt die Brechung des Lichtstrahts an der Linsenoberfläche nicht mehr, also keine Brechung mehr an der Mittelebene. Die Mittelebene muss in diesem Fall durch zwei Hauptebenen ersetzt werden. Brennweite, Gegenstandsweite und Bildweite werden im Verhältnis zu der jeweiligen Hauptebene bestimmt, sodass die Linsengleichung noch immer gilt. 

Exakt bemessen gilt die Reduktion auf die Mittel- bzw. Hauptebene nur für achsenferne Strahlen. Achsenferne Strahlen werden eigentlich stärker gebrochen und es treten Abbildungsfehler auf. Das Bild eines Gegenstandes wird dann nicht mehr scharf abgebildet. 

Bei der sphärischen Aberration liegt der Brennpunkt der achsenfernen Strahlen näher an der Linse als der von den achsennahen Strahlen. Für ein scharfes Bild müssen dann die achsenfernen Strahlen zum Beispiel mit einer Irisblende herausgefiltert werden. Bei der chromatischen Abberration liegt der Brennpunkt von blauem Licht näher an der Linse als der von rotem Licht auf Grund der Dispersion. 

Die reziproke Brennweite definiert die Brechkraft als $D= 1/f$ mit der Einheit Dioptrie. Setzt man ein Linsensystem aus mehreren dünnen Linsen zusammen, summieren sich die Brechkräfte $D_i$, also es gilt 

\begin{equation}
    D= \sum{i}^N D_i.
    \label{eqn:brechungskraft}
\end{equation}


