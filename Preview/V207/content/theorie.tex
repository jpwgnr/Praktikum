\section{Theorie}
\label{sec:Theorie}

Ein Körper der sich in Flüßigkeit bewegt, wird von verschiedenen Kräften beeinflusst. Es wirken die Reibungskraft, die Schwerkraft und die Auftriebskraft. 
Die Reibungskraft hängt dabei von verschiedenen Faktoren ab. Zum einen von der Berührungsfläche $A$ und von der Geschwindigkeit $v$. 
Außerdem aber noch von der sogenannten dynamischen Viskosität $\nu$. Diese ist eine Materialkonstante der Flüßigkeit und hängt stark von der Temperatur derselbigen ab. 
Mit dem Kugelfallviskosimeter lässt sich eben diese Viskosität bestimmen. Dafür wird eine Kugel in einer Flüßigkeit, deren Ausdehnung hinreichend groß ist, damit sich keine Wirbel bilden, fallen gelassen. 
Die Stock'sche Reibung lässt sich folgendermaßen beschreiben: 

\begin{equation}
    F_R = 6\pi \nu v r.
    \label{eq:Stokes}
\end{equation}

Beim Fallen nimmt die Reibung bei zunehmender Geschwindigkeit immer weiter zu, bis sich ein Kräftegleichgewicht gebildet hat. Die Reibungs- und Auftriebskraft stehen dann der Schwerkraft gegenüber. 
Beim Kugelfallviskosimeter nach Höppler lässt man die Kugel in einem Rohr fallen, dessen Radius nur geringfügig größer ist als der Radius der Kugel. Da beim senkrechten Fall evtl Wirbel entstehen würden und die Kugel unkontrolliert an die Rohrwand stoßen würde, wird das Fallrohr um einen kleinen Winkel gekippt, sodass die Kugel an der Rohrwand heruntergleiten kann und sich keine Wirbel bilden. Die Viskositätskonstante $\nu$ lässt sich aus der Fallzeit $t$, der Dichte der Flüssigkeit $\rho_Fl$ und der Dichte der Kugel $\rho_K$ bestimmen. Der Proportionalitätsfaktor K ist eine Apparaturkonstante und enthält sowohl die Höhe, als auch die Kugelgeometrie. 
Es gilt: 

\begin{equation}
    \nu = K (\rho_K -\rho_Fl) \cdot t.
    \label{eq:nu}
\end{equation}

Die Temperaturabhängigkeit der Viskosität lässt sich mit der Andradeschen Gleichung für viele Flüssigkeiten beschreiben: 

\begin{equation}
    \nu(T) = A exp(\frac{B}{T}).
    \label{eq:Temperatur}
\end{equation}
