\section{Durchführung}
\label{sec:Durchführung}

Der experimentelle Aufbau lässt sich Skizze \ref{<++>} entnehmen. 

Eine \beta-Quelle wird vor das Zählrohr gestellt und die Zählrate wird in Abhängigkeit von der Betriebsspannung $U$ gemessen. 

Die Strahlintensität der \beta-Quelle wird soweit abgesenkt, dass während der Laufzeit des Kathodenstrahls von links nach rechts kein weiterer Impuls mehr auf dem Bildzuschirm zu sehen ist. 

Dann wird die Strahlintensität wieder hochgedreht. Die Zeitablenkung des Oszillographen wird durch die Anstiegsflanke getriggert, sodass die Totzeit abgelesen werden kann. 

Es wird die Totzeit mit der Zwei-Quellen-Methode gemessen. Dafür wird die Zählrate des ersten Präparats gemessen, dann die Zählrate des zweiten und anschließend die Zählrate von beiden gemeinsam. Somit ergibt sich, dass die Differenz aus der Addition der beiden ersten Zählraten und der Summe der beiden Zählraten der Rate während der Totzeit bestimmen und daraus dann die Totzeit. 

Mithilfe eines Strommessgeräts wird der mittlere Zählrohrstrom gemessen. 


