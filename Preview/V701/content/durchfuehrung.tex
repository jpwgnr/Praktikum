\section{Durchführung}
\label{sec:Durchführung}

Es wird der Energieverlust von \alpha-Strahlung in Luft untersucht. 
Der Aufbau des Experiments ist in Abb. \ref{<++>} zu erkennen. In einem Glaszylinder befindet sich ein \alpha-Präparat und ein Detektor. Ein Americium-Präparat dient als Strahlungsquelle. Der Abstand zwischen Präparat und Detektor lässt sich mittels eines verschiebbaren Halters ändern. Der Detektor ist ein Halbleiter Sperrschicht-Zähler, der ähnlich einer Diode aufgebaut ist. 

Zur Messung wird das Programm Multichannal Analyzer benutzt. 
Es soll die Energieverteilung und die Zählrate der \alpha-Strahlung in Abhängigkeit des Drucks. Es wird die Zählrate als Funktion des effektiven Abstands bestimmt. Außerdem wird die Zählrate als Funktion des Drucks bestimmt. 

Im letzten Schritt wird die Statistik des radioaktiven Zerfalls überprüft, indem bei evakuiertem Gaszylinder die Zerfälle mindestens 100 mal bestimmt werden. 
