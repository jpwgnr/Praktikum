\section{Theorie}
\label{sec:Theorie}

Die Energie von $\alpha$-Strahlung kann durch Messung der Reichweite bestimmt werden. 
Während sich die Strahlung durch Materie bewegt, kann durch elastische Stöße mit dem Material Energie abgegeben werden. Die elastischen Stöße können als Rutherfordsche Streuung beschrieben werden. Die Energie kann auch durch Anregung oder Dissoziation von Molekülen abgegeben werden. Dieser Energieverlust hängt von der Ausgangsenergie der Strahlung und der Dichte des Material ab, wobei für kleine Geschwindigkeiten die Wahrscheinlichkeit zu nimmt, dass es zu Wechselwirkungen kommt. Die Bethe-Bloch-Gleichung beschreibt diesen Energieverlust für hinreichend große Energien mit 

\begin{equation}
    - \frac{dE_\text{\alpha}}{dx} = \frac{z^2 e^4}{4 \pi epsilon_\text{0} m_\text{e}} \frac{n Z}{v^2} ln(\frac{2 m_\text{e}v^2}{I}).
    \label{eqn:bethebloch}
\end{equation}
$z$ ist dabei die Ladung und $v$ die Geschwindigkeit der $\alpha$-Stahlung, $Z$ die Ordnungszahl, $n$ die Teilchendichte und $I$ die Ionisierungsenergie des Gases. Bei kleinen Energien treten Ladungsaustauschprozesse statt, wodurch Gleichung \ref{eqn:bethebloch} ihre Gültigkeit verliert. 

Um die Reichweite $R$ eines $\alpha$-Teilchens, die Wegstrecke bis zur kompletten Abbremsung, zu berechnen bildet man das Integral 

\begin{equation}
    R = \int[0, E_\text{\alpha}] \frac{dE_\text{\alpha}}{-dE_\text{\alpha}/dx}.
    %Keine Ahnung wie das mit Integralgrenzen nochmal geht
    \label{eqn:reichweite}
\end{equation}

Man verwendet zur Bestimmung der mittleren Reichweite empirisch gewonnene Kurven, wobei man für die mittlere Reichweite von $\alpha$-Strahlung in Luft mit Energien, die kleiner als \SI{2.5}{\mega\electon\volt} sind, die Beziehung $R_\text{m} = 3.1 \cdot E^{3/2}_\text{\alpha}$ verwenden kann. Dabei ist die Reichweite $R_\text{m}$ in \si{\milli\meter} angegeben und $E_\text{\alpha}$ in \si{\mega\electron\volt}. 

Die Reichweite von von $\alpha$-Teilchen ist proportional zum Druck $p$, wenn Temperatur und Volumen konstant sind. Damit kann man eine Absorptionsmessung durchführen, in dem man den Druck $p$ varriert. Bei feste, Abstand $x_\text{0}$ zwischen Detektor und $\alpha$-Strahler gilt für die effektive Länge die Beziehung 

\begin{equation}
    x = x_\text{0} \frac{p}{p_\text{0}}.
    \label{eqn:abstand}
\end{equation}
Dabei ist $p_\text{0}= \SI{1013}{\milli\bar}$ als Normaldruck. 
