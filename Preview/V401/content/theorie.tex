\section{Theorie}
\label{sec:Theorie}

\subsection{Interferenz}
Licht kann als ebene, elektromagnetische Welle beschrieben werden
\begin{equation*}
    \var{E}= \var{E}_0 cos(kx - \omega t - \delta).
    \label{eqn:efeld}
\end{equation*}
Dabei ist $x$ die Ortskoordinate, $k$ die Wellenzahl, $\omega$ die Kreisfrequenz und $\delta$ 
die Phasenverschiebung. 

%\noindent Da sich Licht als Maxwellgleichung beschreiben lässt, gilt das Superpositionsprinzip. 
%Daher ist das elektrische Feld eines Lichtstrahls meist die Summe der Felder mehrerer 
%Lichtwellen.
%Dies lässt sich aber nicht unmittelbar nachprüfen. 
%Doch man kann die Intensität $I$ bestimmen. 

\noindent Die Lichtintensität, also der zeitliche Mittelwert der auf eine Flächeneinheit 
treffenden Lichtleistung, lässt sich aus den Maxwellgleichungen als
\begin{equation}
    I = const \, \abs{\vec{E}}^{2}
    \label{eqn:intensität}
\end{equation} 
formulieren.

\noindent Aus der Kombination des Superpositionsprinzips von Wellen und der Gleichung 
\ref{eqn:intensität} folgt, dass die Lichtintensität insgesamt an einem bestimmten Ort 
$x$, durch
\begin{equation}
    I_\text{ges}= \frac{const}{t_2-t_1} int[t_1]{t_2} \l(\abs{\vec{E}_1 + \vec{E}_2} \r)^2 (x, t) dt
    \label{eqn:iges}
\end{equation}
beschrieben wird. Dabei entsprechen die beiden $E$-Felder den in Gleichung \ref{eqn:efeld} 
beschriebenen Zusammenhängen. 
Durch das Einsetzen dieser beiden Terme entsteht zusätzlich zu der Summe der 
Einzelintensitäten noch ein Teil 
\begin{equation}
    2 const \vec{E}_{0}^2 cos(\delta_2 - \delta_1).
    \label{eqn:zusatz}
\end{equation}
Die Gesamtintesität kann um diesen Faktor von der Summe der Einzelintensitäten 
abweichen. Im Fall, dass 
\begin{equation}
    \delta_2 - \delta_1 = (2n +1) \pi , n= 0, 1, 2, ...
\end{equation}
ist, verschwindet die Gesamtintensität sogar komplett. 

\subsection{Kohärenz}
%inkohärent
Inkohärentes Licht ist nicht interferenzfähiges Licht. Dieses Licht wird von zwei
verschiedenen Punkten einer Lichtquelle oder von zwei Lichtquellen emittiert.
\newline
%kohärent
Kohärentes Licht kann beispielsweise durch Laser (= light amplification by stimulated emisson of radiation)
erzeugt werden. Atome emittieren hierbei in konstantem Abstand Licht, sodass dieses 
kohärent ist.
\newline
%kohärentes Licht erzeugen
Eine andere Möglichkeit kohärentes Licht zu erzeugen ist, das Licht aus einer Quelle
mit einem Strahlteiler in zwei räumlich getrennte Strahlbündel aufzuteilen.
Durch einen Spiegel können die Bündel wieder zusammengeführt werden.
%
Je nach Phasenverschiebung an diesem Punkt 
verschwindet dann zum Beispiel die Lichtintensität an diesem Punkt, vorausgesetzt, dass 
der Gangunterschied beträgt $\lambda/2$ und die Feldstärke beträgt in beiden Bündeln 
dieselbe Größe.
\newline
%Kohärenzlänge
Der Emissionsakt dauert nur eine endliche Zeit $\tau$, weshalb der Wellenzug auch nur 
eine endliche Länge besitzt. Ist der Gangunterschied zwischen den Wellen deutlich größer 
als diese Länge, so verschwindet die Interferenzerscheinung. Daher dürfen die 
Wegunterschiede zwischen den Teilbündeln nicht zu groß werden. Die Länge, bei der die 
Erscheinungen so eben verschwinden, nennt man Kohärenzlänge $l$. 
%
Die Kohärenzlänge ist der Wegunterschied, bei dem die Interferenzerscheinungen
gerade verschwinden:
\begin{equation}
    l = N \lambda.
\end{equation}
Dabei entspricht $N$ der Anzahl der am Schnittpunkt der Strahlen entstehenden Intensitätsmaxima 
und $\lambda$ der Wellenlänge des Lichts. 
Der Gangunterschied darf also nicht größer als $l$ sein.
\newline
%Fouriersches Theorem
Aus dem Fouriersche Theorem folgt, dass ein Wellenzug endlicher Länge nicht 
monochromatisch sein kann, sondern ein Frequenzspektrum besitzen muss. Somit muss 
entweder der Gangunterschied so klein sein oder das Frequenzspektrum so schmal, dass 
Maximum- und Minimumsbedingung für zwei Wellenlängen nicht am selben Ort realisiert werden 
können. 

Der Zusammenhang zwischen Kohärenzlänge und der Wellenlängenverteilung, bzw. zwischen 
der Zeitdauer und der Frequenzverteilung wird im folgenden betrachtet. 
Das Frequenzspektrum $g(\omega)$ der elektrische Feldstärke lässt sich mittels einer 
Fouriertransformation zu 
\begin{equation}
    g(\omega) = 2 E_0 \frac{sin(\omega - \omega_0) \frac{\tau}{2}}{\omega- \omega_0}
\end{equation}
bestimmen. 
Die Intensität $G(\omega)$ ist somit das Quadrat dieses Terms. 

Die Funktionen $E(t)$, $g(\omega)$ und $G(\omega)$ sind in Abb. \ref{abb:funktionen} zu 
erkennnen. 

Es ist zu erkennen, dass der größte Teil der Energie im Bereich $\Delta \omega =\pm 2 \frac{\pi}{\tau}$ um $\omega_0$ verteilt ist. Man bezeichnet $\Delta \omega$ als Breite der Verteilungsfunktion $G(\omega)$. Die Breite der Wellenlängenverteilung ergibt sich zu $\Delta \lambda = \frac{\lambda^{2}_0}{l}$.

%Kohärenzbedingung
Eine weitere Bedingung für Interferenzerscheinungen ist, dass bei ausgedehnten 
Lichtquellen die Richtungsänderung $\zeta$ wie in Abb. \ref{ausgedehnt} dargestellt, 
klein gegenüber $\pi$ sein muss. Es gilt also die Bedingung 
\begin{equation}
    a sin(\zeta) << \frac{\lambda}{2}.
    \label{eqn:zeta}
\end{equation}

%Polarisation
Interferenzerscheinungen bei linear polarisierten Teilbündeln
können nur bei nicht senkrecht zueinander polarisierten Teilbündeln auftreten. 

\subsection{Aufbau des Michelson-Interferometers}

Ein Interferometer ist ein Gerät, das unter Ausnutzung von Interferenzeffekten die 
Messung optischer Größen erlaubt. Dafür spaltet man einen Lichtstrahl in mindestens zwei 
Teilbündel, unterwirft die beiden Bündel einer Veränderung, also fügt einem von beiden 
zum Beispiel einen Gangunterschied hinzu und führt sie anschließend wieder zusammen. 
Das Michelson Interferometer nutzt die Methode der Teilung mittels einer semipermeablen 
Platte. 
Um die Phasenverschiebung durch den Brechungsindex des Spiegels auszugleichen muss eine 
Kompensationsplatte auf dem zweiten Strahlweg aufgestellt werden. 
Mit der in Abb. \ref{abb:michelson} dargestellten Apparatur kann die Intensität am 
Detektor $D$ gemessen werden. Dadurch lässt sich durch Verschiebung des Detektors 
feststellen, an welchen Stellen die Maxima liegen. Dadurch lässt sich dann die 
Wellenlänge berechnen mit 
\begin{equation}
    \Delta d = z \cdot \frac{\lambda}{2}.
\end{equation}

Alternativ setzt man ein Medium mit geändertem Brechungsindex ein mit einer Breite $b$. 
Entweder kann dafür die Breite oder der Gasdruck geändert werden, um verschiedene Maxima 
festzustellen. 
Dann gilt 
\begin{equation}
    b \cdot \Delta n = \frac{z \lamdba}{2}.
\end{equation}

Da $\lambda$ im allgemeinen deutlich kleiner als $b$ ist, lässt sich damit ein 
Unterschied des Brechungsindex in der Größenordnung von $10^{-5}$ bestimmen. 

Da divergente Strahlbündel genutzt werden, kann der Wegunterschied auch noch vom Winkel 
zwischen der Spiegelnormalen und der Einfallsrichtung des Strahls abhängen. Für eine 
Veränderung des Winkels $\alpha$ können ebenso mehrere Maxima festgestellt werden. 
Diese befinden sich bei 
\begin{equation}
    cos (a_k)= \frac{k \lambda}{2 d} (k= 1, 2, ..., \frac{2d}{\lambda}).
\end{equation}
Dabei wird ein System konzentrischer Kreise am Detektor betrachtet. Diese werden 
Interferenzkurven gleicher Neigung genannt. 

\subsection{Fourier-Spektroskopie}

Mit dem Fourierschen Theorem 
\begin{equation}
    G(k)= \frac{1}{2\pi} \int{-\inf}{\inf} L(x) e^{-i k x} dx
    \label{eqn:fourier}
\end{equation}
ergibt sich, dass die Intensität $G(k)$ mittles der Wellenzahl $k$ und der Strecke 
$L(x)$ bestimmt werden kann.