\section{Theorie}
\label{sec:Theorie}

\subsection{Interferenz}

Die Ausbreitung des Lichts lässt sich recht gut als elektromagnetische Welle beschreiben. Im Fall einer ebenen Welle ergibt sich die Orts- und Zeitabhängigkeit der elektrischen Komponente zu 
\begin{equation}
    \var{E}= \var{E}_0 cos(kx - \omega t - \delta).
\end{equation}

Dabei ist x die Ortskoordinate, k die Wellenzahl, \omega die Kreisfrequenz und \delta die Phasenverschiebung. 

Da sich Licht als Maxwellgleichung beschreiben lässt, gilt das Superpositionsprinzip. Daher ist das elektrische Feld eines Lichtstrahls meist die Summe der Felder mehrerer Lichtwellen.
Dies lässt sich aber nicht unmittelbar nachprüfen. 
Doch man kann die Intensität $I$ bestimmen. 

Die Lichtintensität 
