\section{Durchführung}
\label{sec:Durchführung}

Das Experiment zur Bestimmung der Wärmeleitfähigkeit ist in Abb. xy dargestellt. 
Auf einer Grundplatte liegen vier verschiedene Proben, aus drei verschiedenen Materialien. Die vor Stäbe werden von einem Peltierelement geheizt und gekühlt. 
Dabei wird die dynamische Methode angewandt. Die Temperaturen werden an zwei Stellen an jedem Stab mit Thermoelementen gemessen und an einen Datenlogger weitergegeben. Der Abstand zwischen den Thermoelementen muss bestimmt werden. 

Als erstes sollen der zeitliche Verlauf der Temperaturänderung für die verschiedenen Materialien untersucht werden. 
Dafür wird jeweils an zwei verschiedenen Stellen einens Stabes die Temperatur als Funktion der Zeit aufgestellt und damit die Wärmeleitfähigkeit gemessen. 

Die Wellenlänge und die Frequenz der Temperaturwelle sollen nach periodischer Anregung untersucht werden. 

Außerdem wird die Wärmeleitfähigkeit von Aluminium, Messing und Edelstahl nach der Angström-Methode bestimmt. 
Die Angström-Methode ist ein dynamisches Verfahren bei der der Stab periodisch geheizt wird. Die Wärmeleitfähigkeit wird dann aus der Ausbreitsungsgeschwindigkeit der Temperaturwelle bestimmt. 

