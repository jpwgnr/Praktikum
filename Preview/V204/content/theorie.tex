\section{Ziel}
\label{sec:Ziel}

Es soll die Wärmeleitung von Aluminium, Mesing und Edelstahl untersucht werden. 

\section{Theorie}
\label{sec:Theorie}

In einem Körper, der sich nicht in einem Temperaturgleichgewicht befindet, kommt es zu einem Wärmetransport entlang des Temperaturgefälles. Dies kann z.B. durch Wärmeleitung geschehen. In festen Körpern erfolgt der Wärmetransport über Phonenen und frei bewegliche Elektronen. In Metallen ist der Gitterbeitrag zu vernachlässigen. 

/noindent Ein Stab hat die Länge $L$ und eine Querschnittsfläche $A$. Er hat eine Dichte $\rho$ und eine spezifische Wärme $c$.
In der Zeit $dt$ durchfließt die Querschnittsfläche $A$ die Wärmemenge $dQ$. Es gilt:

\begin{equation}
dQ = -\kappa A \frac{\partial T}{\partial x} dt.
\label{eq:dQ}
\end{equation}

$\kappa$ ist die Wärmeleitfähigkeit. Für die Wärmestromdichte $j_\omega$ gilt:

\begin{equation}
    j_{\omega} = -\kappa \frac{\partial T}{\partial x}.
    \label{eq:jomega}
\end{equation}

Mit Hilfe der Kontinuitätsgleichung kann hieraus die eindimensionale Wärmeleitungsgleichung aufgestellt werden:

\begin{equation}
    \frac{\partial T}{\partial t} = \frac{\kappa}{\roh c} \frac{\partial^2 T}{\partial x^2}.
    \label{eq:warmeleitung}
\end{equation}

Dabei ist $\frac{\kappa}{rho c}$ die Temperaturleitfähig. Sie gibt die Schnelligkeit an, mit der sich ein Temperaturunterschied ausgleicht. Die Lösung hängt von der Geometrie des Stabes und den Anfangsbedingungen ab. 

Wird ein Stab abwechselnd erwärmt und abgekühlt, so pflanz sich aufgrund der periodischen Wechsel  eine räumliche und zeitliche Temperaturwelle aus. Diese wird folgendermaßen beschrieben: 

\begin{equation}
    T(x,t) = T_{max} e^{-\sqrt{\frac{\omega \rho c}{2 \kappa}} x} cos(\omega t - \sqrt{\frac{\omega \rho c}{2 \kappa}} x).
    \label{eq:Wellengleichung}
\end{equation}

Die Phasengeschwindigkeit $v$ der Welle lässt sich beschreiben als:

\begin{equation}
    v= \frac{\omega}{k} = \sqrt{\frac{2\kappa \omega}{\rho c}}.
    \label{eq:phasengesch}
\end{equation}
Für die Wärmeleitfähigkeit ergibt sich nach einigen kleinen Umformungen folgendermaßen:

\begin{equation}
    \kappa = \frac{\rho c (\Delta x)^2}{2 \Delta t ln(A_{nah}/A_{fern})}.
    \label{eq:Wärme}
\end{equation}
Dabei sind $A_{nah}$ und $A_{fern}$ die Amplituden an verschiedenen Stellen $x_{nah}$ und $x_{fern}$. $\Delta x$ ist der Abstand zwischen diesen beiden Messstellen und $\Delta t$ die Phasendifferenz der Temperaturwelle zwischen beiden Messstellen. 
