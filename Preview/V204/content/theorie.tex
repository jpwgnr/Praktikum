\section{Ziel}
\label{sec:Ziel}

Es soll die Wärmeleitung von Aluminium, Mesing und Edelstahl untersucht werden. 

\section{Theorie}
\label{sec:Theorie}

In einem Körper, der sich nicht in einem Temperaturgleichgewicht befindet, kommt es zu einem Wärmetransport entlang des Temperaturgefälles. Dies kann z.B. durch Wärmeleitung geschehen. In festen Körpern erfolgt der Wärmetransport über Phonenen und frei bewegliche Elektronen. In Metallen ist der Gitterbeitrag zu vernachlässigen. 

/noindent Ein Stab hat die Länge $L$ und eine Querschnittsfläche $A$. Er hat eine Dichte $\rho$ und eine spezifische Wärme $c$.
In der Zeit $dt$ durchfließt die Querschnittsfläche $A$ die Wärmemenge $dQ$. Es gilt:

\begin{equation}
dQ = -\kappa A \frac{\partial T}{\partial x} dt.
\label{eq:dQ}
\end{equation}

$\kappa$ ist die Wärmeleitfähigkeit. Für die Wärmestromdichte $j_\omega$ gilt:

\begin{equation}
    j_{\omega} = -\kappa \frac{\partial T}{\partial x}.
    \label{eq:jomega}
\end{equation}


