\section{Ziel}
\label{sec:Ziel}

Die Funktionsweise und das physikalische Verhalten einer Wärmepumpe soll bei diesem Versuch übermittelt werden.

\section{Theorie}
\label{sec:Theorie}

\subsection{Theoretische Grundlagen einer Wärmepumpe}
Die thermische Energie geht in einem abgeschlossenen System immer vom heißeren zum kälteren Körper über. Es ist möglich die Richtung des Wärmeflusses mit der Aufwendung zusätzlicher Energie umzukehren. Eine Vorrichtung, die diesen Prozess durchführt, ist eine sogenannte Wärmepumpe. 

/noindent Das Verhältnis aus der transportierten Wärmemenge und der dafür aufgebrachten Arbeit nennt man Güteziffer $\nu$ der Wärmepumpe. Aus dem ersten Hauptsatz der Thermodynamik lässt sich ableiten, dass sich die abgegebene Wärmemenge $Q_1$ aus der Addition der entnommenen Wärmemenge $Q_2$ und der aufgewandten Energie $W$ bestimmt. Somit gilt

\begin{equation}
Q_1 = Q_2 + W.
\end{equation} 
Die Güteziffer ergibt sich zu 
\begin{equation}
\nu = \frac{Q_1}{W}.
\end{equation}

Aus dem zweiten Hauptsatz der Thermodynamik ergibt sich, dass zwischen den Wärmemengen $Q_1$ und $Q_2$ und den Temperaturen $T_1$ und $T_2$ in einem idealen System folgende Beziehung besteht: 

\begin{equation}
    \frac{Q_1}{T_1} - \frac{Q_2}{T_2} = 0.
\end{equation}

Für die Richtigkeit dieser Gleichung muss aber die Voraussetzung gelten, dass der Prozess der Wärmeübertragung reversibel, also umkehrbar, ist. 

/noindent Aus diesen Gleichungen folgt, dass 
\begin{equation}
    Q_1 = W+ \frac{T_2}{T_1} Q_1 
\end{equation}
gilt und sich die Güteziffer $\nu$ zu folgender Gleichung ergibt: 
\begin{equation}
    \nu_{id}= \frac{T_1}{T_1-T_2}.
\end{equation} 

Dies gilt aber natürlich nur im Idealfall. Für die reale Wärmepumpe gilt die Ungleichung: 
\begin{equation}
    \nu_{real} < \frac{T_1}{T_1-T_2}.
\end{equation}

\subsection{Funktionsweise einer Wärmepumpe}
Die Wärme wird innerhalb der Pumpe als Phasenumwandlungsenergie eines Gases transportiert, dass beim verdampfen Wärme aufnimmt und bei der Verflüssigung wieder abgibt. Der schematische Aufbau der hier verwendeten Apparatur ist in Abb. xy zu erkennen. 

\subsection{Messungen}

Die Temperaturen $T_1$ und $T_2$ und die Drücke $p_1$ und $p_2$ werden in Abhängigkeit zur Zeit $t$ gemessen. 

Aus dem Quotienten aus $\Delta T_1$ und $\Delta t$ ergibt sich die pro Zeiteinheit gewonnene Wärmemenge zu 
\begin{equation}
    \frac{\Delta Q_1}{\Delta t} = (m_1c_w + m_kc_k)\frac{\Delta T_1}{\Delta t}.
\label[eq:Q1]
\end{equation}
$m_1c_w$ ist dabei die Wärmekapazität des Wassers in Reservoir 1. $m_kc_k$ ist die Wärmekapazität der Kupferschlange und des Eimers. Für die Güteziffer ergibt sich dann mit $N$ als die vom Wattmeter angetzeigte und über das Zeitinterall $\Delta t$ gemittelte Leistungsaufnahme des Kompressors:

\begin{equation}
    \nu = \frac{\Delta Q_1}{\Delta t N}. 
\end{equation}

Für $Q_2$ lässt sich die Gleichung \ref{eq:Q1} analog anwenden. Es gilt für die Verdampfungswärme $L$ und den Massendurchsatz:

\begin{equation}
    \frac{\Delta Q_2}{\Delta t} = L \frac{\Delta m}{\Delta t}.
\end{equation}

Die mechanische Kompressorleistung $N_{mech}$ ergibt sich mit dem Verhältnis $\kappa$ aus $C_P$ und $C_V$ zu der Dichte $\rho$ des Transportmediums im gasförmigen Zustand, also unter dem Druck $p_1$, zu:

\begin{equation}
    N_{mech} = \frac{1}{\kappa - 1} (p_b \sqrt[\uproot{\kappa}]{\frac{p_a}{p_b}} - p_a)\frac{1}{\rho}\frac{\Delta m}{\Delta t}. 
\end{equation}


