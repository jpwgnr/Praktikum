\section{Durchführung}
\label{sec:Durchführung}

%Die Funktion eines phasenempfindlichen Gleichrichters für fünf verschiedene Phasen wird verifiziert. 
Als erstes werden sich die Signale des Funktionsgenerators angeschaut. Dabei werden die beiden Ausgänge verglichen und ihre Eigenschaften überprüft. 

Mit der Schaltung in Abb. xy wird die Funktionsweise eines Lock-In-Verstärkers getestet. Dabei sollen die Ausgangssignale für fünf verschiedene Ausgangssignale skizziert werden. Im nächsten Schritt wird ein Tiefpass in den Schaltkreis integriert. Das neue Ausgangssignal soll überprüft werden. 
Danach werden die Ausgangsspannungen für verschiedene Phasenverschiebungen gemessen. 

Dieselben Messungen wie zuvor werden noch einmal mit einem Rauschsignal von der Größenordnung der Signalspannung durchgeführt.

Im letzten Schritt wird eine Photodetektorschaltung wie in Abb. xy gebaut. Die Leuchtdiode wird mit einer Rechteckspannung gespeist und mit einer Frequenz von 50 bis 500Hz zum blinken gebracht. Mit einer Photodiode wird das ausgesendete Licht anschließend gemessen. Dabei soll die Lichtintensität als Funktion des Abstands $r$ zwischen LED und Photodiode gemessen werden. 


